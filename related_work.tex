\chapter{Related Work}
\label{ch:related_work}
\label{sec:related}
Relative to the traditional closed-form equilibrium approaches, agent-based
simulation of (electricity) markets is a new field of research.  For
comprehensive reviews and surveys of the many different techniques that have
been applied in recent years the interested reader is directed to
\cite{anke:2008,tesfatsi:handbook,visud:thesis}.  This section will focus on
reviewing literature from the field in which reinforcement learning techniques
were applied in combination with explicit power system models.  A short review
is also provided of some more general applications of reinforcement learning
with connectionist systems and policy-gradient methods.

% Game theoretic models are commonly associated with economics and attempt to
% capture behaviour in strategic situations mathematically.  They have been
% applied to electric energy problems of many forms, including but not limited
% to analysis of market structure, market liquidity, pricing methodologies,
% regulatory structure, plant positioning and network congestion.  More
% recently, agent-based simulation has received a certain degree of attention
% from researchers and has been applied in some of these fields also.
%
% While popular and seemingly promising, agent-based simulation is still centred
% around abstracted models.  The assumptions made is this abstraction must be
% subjected to the same verification and validation as with equation-based
% models.  Verification of assumptions and model validation are often overlooked
% in agent-based simulations of energy markets, yet they are possibly the most
% important steps in the model building process.  Techniques used to develop,
% debug and maintain large computer programs can often be used to verify that a
% model does what it is intended to do.
%
% Validation of an energy market model is more difficult.  It can be accomplished
% using the intuition of experts or through comparison of simulation results
% with either historical market data or theoretical results from more abstract
% representations of the model.  Finding verifyable trends in existing markets
% is a very large challenge.  To then prove that a computational model
% replicates these characteristics with suitable fidelity is yet more
% challenging still.  Only when a model is suitably verified and validated can
% any conclusions be drawn from results obtained through implementation and
% simulation of suitable scenarios.

\section{Learning algorithm comparison}

\section{Q-learning}
Krause et al.\ have published agent-based energy market research in which
Q-learning methods were applied while considering physical system properties.
In a comparison between Nash equilibrium analysis and agent-based simulation,
the suitability of bottom-up modelling for the assessment of market evolution
was assessed\cite{krause:nash}.  This is built upon in subsequent publications
which evaluate the influence on market power and social welfare of three
congestion management schemes and which analyse strategic behavior in combined
gas and electricity markets\cite{krause:cong,krause:gas}.  Power Transmission
Distribution Factors (PTDF) are used in place of explicit power flow equations
in determining line flows.  The action domain of generating agents is limited
to 0\%, 5\% and 10\% markups on true marginal costs.  The implementation of the
Q-learning method used does not differentiate between environment states when
selecting actions. This is a modification to the traditional formulation that
still results in convincing conclusions, as with the popular Roth-Erev method.

There are similar applications of Q-learning in which states \textit{are}
defined, but none model the AC transmission system.  A common approach is to
use categorised market price from the previous period as state
information\cite{bakirtzis:psce,xiong:discrim}.

\section{Roth-Erev}
The AMES (Agent-based Modeling of Electricity Systems) power market test bed is
a software package that models core features of the Wholesale Power Market
Platform (WPMP) -- a market design proposed by the US Federal Energy Regulatory
Commission (FERC) in April 2003 for common adoption in regions of the
US\cite{tesfatsi:wpmp}. The design features:
\begin{itemize}
  \item a centralised structure managed by an independent market operator,
  \item parallel day-ahead and real-time markets and
  \item locational marginal pricing.
\end{itemize}
Learning agents may represent load serving entities or generating companies and
learn using implementations of the Roth-Erev method (See sections
\ref{sec:rotherev} and \ref{sec:variant}, above) built on the Repast agent
simulation toolkit\cite{gieseler:thesis}.  The permissive license under which
the source code for these algorithms has been released allowed direct
translation of them for use in this study.  Agents learn from the solutions of
hourly bid/offer based DC-OPF problems formulated as quadratic programs and
solved using QuadProgJ\cite{tesfatsi:dcopf}.

The ability of generator agents to learn particular supply offers has been
demonstrated along with the plotting and data handling capabilities of AMES
using a 5-bus network model\cite{tesfatsi:pes09}.  The same network has been
used to investigate strategic capacity withholding in FERC wholesale power
market design\cite{tesfasi:psce}.  Generator agents report linear marginal cost
functions to the market operator and supply functions are formed through linear
interpolation between the prices at minimum and maximum production limits.
Load serving agents submit combinations of fixed demand bids and
price-sensitive bid functions, the ratio between which is varied between 0.0 and
1.0 to test physical and economic capacity withholding potential.  Comparing
results from a benchmark case (in which true production costs are reported,
but higher than marginal cost functions may be reported) and cases in which
reported production limits may be less than the true values, the authors find,
that with sufficient capacity reserve, no evidence to suggest potential for
inducing higher net earnings through capacity withholding in the WPMP.
