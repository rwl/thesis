% \chapter{Background Theory}
% \label{ch:background}
% This chapter provides an introduction to optimal power flow and reinforcement
% learning.  The methods described are used in Chapter \ref{ch:method} below to
% model electricity markets and market participant behaviour.  Optimal power
% flow is one of the most widely studied subjects in electric power Engineering
% and a comprehensive literature review is available in
% \cite{momoh:part1,momoh:part2}.  For detailed definitions and analysis
% reinforcement learning methods the interested reader is referred to
% \cite{suttonbarto:1998,bertsekas:96}.

% \chapter{MATPOWER OPF Formulation}
% \label{sec:power_system_model}
% Power systems are modelled as three-phase AC circuits operating in the
% steady-state, under balanced conditions that can be represented by an
% equivalent single phase nodal graph of busbars connected by branches
% \cite{grainger:psa}.
%
% \section{Branches}
% Following \cite[p.11]{pserc:mp_manual}, each branch is modelled as a medium
% length transmission line in series with a regulating transformer at the ``from'' end.  A nominal-$\pi$ model with total
% series admittance $y_s = 1/(r_s+jx_s)$ and total shunt capacitance $b_c$
% represents the transmission line.  The transformer is assumed to be ideal,
% phase-shifting and tap-changing, with the ratio between primary winding
% voltage $v_{f}$ and secondary winding voltage $N = \tau e^{j\theta_{ph}}$
% where $\tau$ is the tap ratio and $\theta_{ph}$ is the phase shift angle.
% Figure X diagrams the branch model.  From Kirchhoff's Current Law the current
% in the series impedance is
% \begin{equation}
% \label{eq:iseries}
% i_s = \frac{b_c}{2}v_t - i_t
% \end{equation}
% and from Kirchhoff's Voltage Law the voltage across the secondary winding of
% the transformer is
% \begin{equation}
% \frac{v_{f}}{N} = v_t + \frac{i_s}{y_s}
% \end{equation}
% Substituting $i_s$ from equation (\ref{eq:iseries}), gives
% \begin{equation}
% \label{eq:vfrom}
% \frac{v_{f}}{N} = v_t - \frac{i_t}{y_s} + v_t\frac{b_c}{2y_s}
% \end{equation}
% and rearranging in terms if $i_t$, gives
% \begin{equation}
% \label{eq:ito}
% i_t = v_s \left( \frac{-y_s}{\tau e^{\theta_{ph}}} \right) +
% v_r \left( y_s + \frac{b_c}{2} \right)
% \end{equation}
% The current through the secondary winding of the transformer is
% \begin{equation}
% N^*i_f = i_s + \frac{b_c}{2}\frac{v_{f}}{N}
% \end{equation}
% Substituting $i_s$ from equation(\ref{eq:iseries}) again, gives
% \begin{equation}
% N^*i_f = \frac{b_c}{2}v_t - i_t + \frac{b_c}{2}\frac{v_{f}}{N}
% \end{equation}
% and substituting $\frac{v_{f}}{N}$ from equation (\ref{eq:vfrom}) and
% rearranging, gives
% \begin{equation}
% \label{eq:ifrom}
% i_s = v_s \left( \frac{1}{\tau^2} \left(y_s + \frac{b_c}{2}\right) \right) +
% v_r \left(\frac{y_s}{\tau e^{-j\theta}}\right)
% \end{equation}
% Equations (\ref{eq:ito}) and (\ref{eq:ifrom}) are used in section
% \ref{sec:acpf} below to define the system admittance matrices that describe
% the electrical network.
%
% \section{Generators}
% \label{sec:generators}
% Each generator $k$ is modelled as an apparent power injection $s^k_g = p^k_g +
% jq^k_g$ at a bus $i$, where $p^k_g$ is the active power injection,
% $q^k_g$ is the reactive power injection and each are expressed in per-unit to
% the system base MVA.  Upper and lower limits on $p^k_g$ are specified by
% $p^k_{max}$ and $p^k_{min}$, respectively, where $p^k_{max} > p^k_{min} \geq
% 0$.  Similarly, upper and lower limits on $q^k_g$ are specified by $q_{max}^k$
% and $q_{min}^k$, respectively, where $q^k_{max} > q^k_{min}$.
%
% \section{Buses and Loads}
% At each bus $i$, constant active power demand is specified by $p^i_d$ and
% reactive power demand by $q^i_d$.  Upper and lower limits on the voltage
% magnitude at the bus are defined by $v_m^{i,max}$ and $v_m^{i,min}$,
% respectively.  One generator bus $i \in \mathcal{I}_{ref}$ in the circuit is
% designated the \textit{reference} bus and has voltage angle $\theta^{ref}_k$.
% Dispatchable loads are modelled as generators with negative $p^i_g$ and
% $p^i_{min} < p^i_{max} = 0$. %TODO: Constant power factor.
%
% \section{AC Power Flow Equations}
% \label{sec:acpf}
% Following \cite[p.13]{pserc:mp_manual}, for a network of $n_b$ buses, $n_l$
% branches and $n_g$ generators, let $Cg$ be the $n_b \times n_g$ bus-generator connection matrix such that the $(i,j)^{th}$
% element of $C_{g}$ is $1$ if generator $j$ is connected to bus $i$.  The
% $n_b \times 1$ vector of complex power injections from generators at all buses
% is
% \begin{equation}
% S_{g,bus} = C_g \cdot S_g
% \end{equation}
% where $S_g = P_g + jQ_g$ is the $n_g \times 1$ vector with the $i^{th}$ element
% is equal to $s^i_g$.
%
% Combining equations (\ref{eq:ito}) and (\ref{eq:ifrom}), the \textit{from}
% and \textit{to} end complex current injections for branch $l$ are
% \begin{equation}
% \label{eq:ybranch}
% \begin{bmatrix}
% i_f^l\\
% i_t^l
% \end{bmatrix}
% =
% \begin{bmatrix}
% y_{ff}^l& y_{ft}^l\\
% y_{tf}^l& y_{tt}^l
% \end{bmatrix}
% \begin{bmatrix}
% v_f^l\\
% v_t^l
% \end{bmatrix}
% \end{equation}
% where
% \begin{eqnarray}
% \label{eq:yff}
% y_{ff}^l& =& \frac{1}{\tau^2} \left(y_s + \frac{b_c}{2}\right)\\
% \label{eq:yft}
% y_{ft}^l& =& \frac{y_s}{\tau e^{-j\theta_{ph}}}\\
% \label{eq:ytf}
% y_{tf}^l& =& \frac{-y_s}{\tau e^{j\theta_{ph}}}\\
% \label{eq:ytt}
% y_{tt}^l& =& y_s + \frac{b_c}{2}
% \end{eqnarray}
% Let $Y_{ff}$, $Y_{ft}$, $Y_{tf}$ and $Y_{tt}$ be $n_l \times 1$ vectors where
% the $l^{th}$ element of each corresponds to $y_{ff}^l$, $y_{ft}^l$,
% $y_{tf}^l$ and $y_{tt}^l$, respectively.  Furthermore, let $C_f$ and $C_t$ be the
% $n_l \times n_b$ branch-bus connection matrices, where $C_{f_{i,j}} = 1$ and
% $C_{t_{i,k}} = 1$ if branch $i$ connects from bus $j$ to bus $k$.  The $n_l
% \times n_b$ branch admittance matrices are
% \begin{eqnarray}
% Y_f& =& \diag(Y_{ff})C_f + \diag(Y_{ft})C_t\\
% Y_t& =& \diag(Y_{tf})C_f + \diag(Y_{tt})C_t
% \end{eqnarray}
% and relate the complex bus voltages $V$ to the branch ``from'' and
% ``to'' end current vectors
% \begin{eqnarray}
% I_{f}& =& Y_{f}V\\
% I_{t}& =& Y_{t}V
% \end{eqnarray}
% The $n_b \times n_b$ bus admittance matrix
% \begin{eqnarray}
% Y_{bus}& =& C_f^\mathsf{T} Y_f + C_t^\mathsf{T} Y_t
% \end{eqnarray}
% relates the complex bus voltages to the nodal current injections
% \begin{eqnarray}
% I_{bus}& =& Y_{bus}V
% \end{eqnarray}
% The complex bus power injections are expressed as a non-linear function of $V$
% \begin{eqnarray}
% S_{bus}(V)& =& \diag(V)I_{bus}^* \nonumber \\
% \label{eq:sbus}
% &= & \diag(V)Y_{bus}^*V^*
% \end{eqnarray}
% As are the complex power injections at the ``from'' and ``to'' ends of all
% branches
% \begin{eqnarray}
% S_{f}(V)& =& \diag(C_fV)I_f^* \nonumber \\
% \label{eq:sf_loss}
% & =& \diag(C_fV)Y_f^*V^*\\
% S_{t}(V)& =& \diag(C_tV)I_t^* \nonumber \\
% \label{eq:st_loss}
% & =& \diag(C_tV)Y_t^*V^*
% \end{eqnarray}
% The net complex power injection (generation - load) at each bus must equal the
% sum of complex power flows on each branch connected to the bus.  Hence the AC
% power balance equations are
% \begin{equation}
% \label{eq:mismatch}
% S_{bus}(V) + S_d - S_g = 0
% \end{equation}
%
% \section{DC Power Flow Equations}
% Following \cite[p.14]{pserc:mp_manual}, the same power system model is used in
% the formulation of the linearised DC power flow equations, but the following additional assumptions are made:
% \begin{itemize}
%   \item The resistance $r_s$ and shunt capacitance $b_c$ of all branches can be
%   considered negligible.
%   \begin{equation}
%   \label{eq:lossless}
%   y_s \approx \frac{1}{jx_s}, \quad b_c \approx 0
%   \end{equation}
%   \item Bus voltage magnitudes $v_{m,i}$ are all approximately 1 per-unit.
%   \begin{equation}
%   \label{eq:oneperunit}
%   v_i \approx 1e^{j\theta_i}
%   \end{equation}
%   \item The voltage angle difference between bus $i$ and bus $j$ is small enough
%   that
%   \begin{equation}
%   \label{eq:busangdiff}
%   \sin\theta_{ij} \approx \theta_{ij}
%   \end{equation}
% \end{itemize}
% Applying the assumption that branches are lossless from equation
% (\ref{eq:lossless}), the quadrants of the branch admittance matrix in equations
% (\ref{eq:yff}), (\ref{eq:yft}), (\ref{eq:ytf}) and (\ref{eq:ytt}), approximate
% to
% \begin{eqnarray}
% y_{ff}^l& =& \frac{1}{jx_s \tau^2}\\
% y_{ft}^l& =& \frac{-1}{jx_s \tau e^{-j\theta_{ph}}}\\
% y_{tf}^l& =& \frac{-1}{jx_s \tau e^{j\theta_{ph}}}\\
% y_{tt}^l& =& \frac{1}{jx_s}
% \end{eqnarray}
% respectively.  Applying the uniform bus voltage magnitude assumption from
% equation (\ref{eq:oneperunit}) to equation (\ref{eq:ybranch}), the branch
% ``from'' end current approximates to
% \begin{eqnarray}
% i_f& \approx& \frac{e^{j\theta_f}}{jx_s\tau^2} -
% \frac{e^{j\theta_t}}{jx_s \tau e^{-j\theta_{ph}}}\\
% & =& \frac{1}{jx_s\tau} ( \frac{1}{\tau}e^{j\theta_f} -
% e^{j(\theta_t + \theta_{ph})} )
% \end{eqnarray}
% % ToDo: Branch to end current derivation.
% and the branch ``from'' end complex power flow $s_f = v_f \cdot i_f^*$
% approximates to
% \begin{eqnarray}
% s_f& \approx& e^{j\theta_f} \cdot \frac{j}{x_s\tau}
% (\frac{1}{\tau}e^{-j\theta_f} - e^{j(\theta_t + \theta_{ph})})\\
% & =& \frac{1}{x_s\tau} \left[ \sin(\theta_f-\theta_t-\theta_{ph}) +
% j\left( \frac{1}{\tau} - \cos(\theta_f-\theta_t-\theta_{ph}) \right) \right]
% \end{eqnarray}
% Applying the voltage angle difference assumption from equation
% (\ref{eq:busangdiff}) yields the approximation
% \begin{equation}
% p_f \approx \frac{1}{x_s\tau}(\theta_f-\theta_t-\theta_{ph})
% \end{equation}
% Let $B_{ff}$ and $P_{f,ph}$ be the $n_l \times 1$ vectors where
% $B_{ff_i} = 1 / (x_s^i\tau^i)$ and $P_{f,ph_i} =
% -\theta_{ph}^i / (x_s^i\tau^i)$.  Then if the system $B$ matrices are
% \begin{eqnarray}
% B_f& =& \diag(B_{ff})(C_f-C_t)\\
% B_{bus}& = &(C_f-C_t)^\mathsf{T}B_f
% \end{eqnarray}
% then the real power bus injections are
% \begin{equation}
% \label{eq:bbus}
% P_{bus}(\Theta) = B_{bus}\Theta + P_{bus,ph}
% \end{equation}
% where $\Theta$ is the $n_b \times 1$ vector of bus voltage angles and
% \begin{equation}
% P_{bus,ph} = (C_f-C_t)^\mathsf{T} + P_{f,ph}
% \end{equation}
% The active power flows at the branch ``from'' ends are
% \begin{equation}
% \label{eq:pf_loss}
% P_f(\Theta) = B_f\Theta + P_{f,ph}
% \end{equation}
% and $P_t = -P_f$ since all branches are assumed lossless.
%
% \section{AC OPF Formulation}
% Following \cite[p.26]{pserc:mp_manual}, generator active and, optionally,
% reactive power output costs are defined by a convex $n$-segment piecewise
% linear cost function
% \begin{equation}
% c^{(i)}(x) = m_ip + c_i
% \end{equation}
% for $p_i \leq p \leq p_{i+1}$ with $i = 1,2,\dotsc n$ where $m_{i+1} \geq m_i$
% and $p_{i+1} > p_i$ as diagramed in Figure X.  Since these costs are
% non-differentiable the constrained cost variable approach from
% \cite{zimmerman:ccv} is used to make the optimisation problem smooth.  For
% each generator $i$ a helper cost variable $y_i$ added to the objective
% function.  The inequality constraints
% \begin{equation}
% y_i \geq m_{i,j}(p-p_j) + c_j, \quad j = 1\dotsc n
% \end{equation}
% require that $y_i$ lies on the epigraph\footnote{Informally, the epigraph of a
% function is a set of points lying on or above its graph.} of $c^{(i)}(x)$. The objective of
% the optimal power flow problem is to minimise the sum of the cost variables
% for all generators.
% \begin{equation}
% \min_{\theta, V_m, P_g, Q_g, y} \sum_{i=1}^{n_g}y_i
% \end{equation}
% Equation (\ref{eq:mismatch}) forms an equality constraint which enforce the
% balance between the net complex power injection and injections into the
% network.  Branch complex power flow limits $S_{max}$ are enforced by the
% inequality constraints
% \begin{eqnarray}
% \abs{S_f(V)} - S_{max}& \leq &0\\
% \abs{S_f(V)} - S_{max}& \leq &0
% \end{eqnarray}
% and the reference bus voltage angle $\theta_i$ is fixed with the equality
% constraint
% \begin{equation}
% \label{eq:refbusang}
% \theta_i^{ref} \leq \theta_i \leq \theta_i^{ref}, \quad i \in \mathcal{I}_{ref}
% \end{equation}
% Upper and lower limits on the optimisation variables $V_m$, $P_g$ and $Q_g$ are
% enforced by the inequality constraints
% \begin{eqnarray}
% v_m^{i,min} \leq v_m^i \leq v_m^{i,max},& \quad i= 1 \dotsc n_b&\\
% \label{eq:pglim}
% p_g^{i,min} \leq p_g^i \leq p_q^{i,max},& \quad i= 1 \dotsc n_g&\\
% q_g^{i,min} \leq q_g^i \leq q_q^{i,max},& \quad i= 1 \dotsc n_g&
% \end{eqnarray}
%
% \section{DC OPF Formulation}
% Piecewise linear cost functions are also used to define generator active power
% costs in the DC optimal power flow formulation.  Since the power flow equations
% are linearised, following the assumptions in equations (\ref{eq:lossless}),
% (\ref{eq:oneperunit}) and (\ref{eq:busangdiff}), the optimal power flow
% problem simplifies to a linear program.  The voltage magnitude variables $V_m$
% and generator reactive power set-point variable $Q_g$ are eliminated following
% the assumption in equation (\ref{eq:busangdiff}) since branch reactive power
% flows depend on bus voltage angle differences.  The objective function reduces to
% \begin{equation}
% \min_{\theta, P_g, y} \sum_{i=1}^{n_g}y_i
% \end{equation}
% Combining the nodal real power injections, expressed as a function of $\Theta$,
% from equation (\ref{eq:bbus}), with active power generation $P_g$ and active
% demand $P_d$, the power balance constraint is
% \begin{equation}
% B_{bus}\Theta + P_{bus,ph} + P_d - C_gP_g = 0
% \end{equation}
% Limits on branch active power flows $B_f\Theta$ and $B_t\Theta$ are enforced by
% the inequality constraints
% \begin{eqnarray}
% B_f\Theta + P_{f,ph} - F_{max}& \leq& 0\\
% -B_f\Theta + P_{f,ph} - F_{max}& \leq& 0
% \end{eqnarray}
% The reference bus voltage angle equality constraint from
% equation (\ref{eq:refbusang}) and the $p_g$ limit constraint from
% (\ref{eq:pglim}) are also applied.
%
% \section{Optimal Power Flow Solution}
% \label{sec:opfsol}
% % Generator dispatch points are used with the associated cost functions to
% % compute the objective function value -- the total system cost.  The power
% % balance Lagrangian multipliers are the shadow prices or system nodal prices and
% % equal the cost to the system of supplying one more unit of load at that bus.
%
% \section{Unit De-commitment}
% The optimal power flow formulation defined in Section \ref{sec:opf} above
% requires generators are dispatched within their upper and lower power limits.
% Expensive generators can not be completely shutdown, even if doing so would
% result in a lower total system cost.  Algorithm \ref{alg:ud} defines the unit
% de-commitment algorithm from \cite[p.20]{pserc:mp_manual} which allows a least
% cost commitment and dispatch to be determined using the optimal power flow
% formulation. The algorithm finds the least cost dispatch by solving repeated
% optimal power flow problems with different combinations of generating units
% that are at their minimum active power limit deactivated.  The lowest cost
% solution is returned when no further improvement can be made and no candidate
% generators remain.
% \begin{algorithm}[H]
% \caption{Unit de-commitment}
% \label{alg:ud}
% \begin{algorithmic}[1]
% \STATE $\text{initialise}~N \leftarrow 0$
% \STATE $\text{solve initial OPF}$
% \STATE $L_{tot} \leftarrow \text{total load capacity}$
% \WHILE{$\text{total min gen.\ capacity} > L_{tot}$}
% 	\STATE $N \leftarrow N + 1$
% \ENDWHILE
%
% \REPEAT
% 	\FOR{c in candidates}
% 		\STATE $\text{solve OPF}$
% 	\ENDFOR
% \UNTIL{$\text{done} = \text{True}$}
% \end{algorithmic}
% \end{algorithm}
