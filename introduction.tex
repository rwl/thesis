\chapter{Introduction}
This thesis describes the application of value-function and policy gradient
reinforcement learning algorithms to the electric energy trade problem.  This
chapter introduces the problem and explains the motivation for the research.
The goals of the research are stated along with the principle contributions.
Finally, a reading guide is provided that provides an overview of the remaining
chapters.

\section{Motivation/Setting the scene}
Free market democracy has underpinned the transformation of large western
economies in the post war era and continues to be relied upon.  Towards the
end of the nineteenth century the principals of competitive trade were
successfully applied to electric power industries, beginning in the UK in March
1990 with the creation of The Electricity Pool.

\section{Problem statement/Aims \& Objectives/Problem Description}

% \begin{quotation}
%  ``Every scientist knows the rule called Occam's Razor:  Faced with several
%  competing hypotheses, prefer the simplest one.  There is also an unspoken
%  corollary that might be called Occam's Castle:  Faced with several competing
%  places to build a new science, prefer the simplest one\ldots  Where the
%  foundation is firmest, the castle will rise highest.  Where the ground is
%  solid, build there, and the universe is so constructed that you will have a
%  view.''\cite{weiner:tlm}
% \end{quotation}

Engineers must strive for complexity in their work.  Rarely will a simple
solution will perform a function to a higher degree than a more complex one.
Certainly, where a function is either performed or not performed, prefer the
simpler one, but most often problems can be solved to varying degrees.

The broad aim of the research presented in this thesis is to prove that the
above conjecture applies to reinforcement learning algorithms for power trade.
Previous research in this field (See Chapter~\ref{ch:related_work} below) has
used very simple algorithms in relation to those from the latest advances in
artificial intelligence (See Sections~\ref{sec:enac}~and~\ref{sec:reinforce}
below).  The goal is to prove that policy gradient methods, using artificial
neural networks for policy function approximation, are better suited to
learning the complex dynamics of a power system.

\section{Research contributions}
The research presented in this thesis pertains to the academic fields of power
engineering, artificial intelligence and economics.  The principle
contributions in these areas are

\begin{itemize}
  \item The proof that policy gradient reinforcement learning algorithms
  outperform value-function algorithms when applied to the power trade problem,
  \item A novel coupling of power system models and optimal power flow
  algorithm results with agents capable of handling discrete and continuous
  sensor and action spaces,
  \item Implementations of Roth-Erev reinforcement learning algorithms and
  continuous versions of Q-learning and Q($\lambda$) for the open source
  PyBrain library,
  \item Open source implementations of power flow and optimal power flow
  algorithms in the Python programming language, preserving sparsity throughout
  the optimisation using the open source CVXOPT library.
\end{itemize}

\section{Thesis structure/Overview/Reading guide}
This thesis is focussed on the application of standard and advanced
reinforcement learning algorithms to a particular problem domain.  The reader
will require a certain degree of prior knowledge, or must be willing to read
much of the referenced material, to fully understand the methodology taken.
The intended audience is engineering and economics researchers interested in
the application of reinforcement learning algorithms to the problem of trading
energy in electric power systems.

% Industrialised societies have become increasingly reliant on the supply of
% electric energy since the connection of large power stations began in 1938.
% The extent to which this is true can be seen in the financial impact that loss
% of supply has on society.
%
%
% In June 2004 the United Kingdom (UK) became a net importer of natural gas for
% the first time in 8 years[EIA DOE].  Since energy industry privatisation by
% the Conservatives in the 1980s, use of domestic gas reserves for electricity
% generation has been encouraged and exploited.
%
% % Insert dash for gas pie-chart here.
%
% As UK natural gas production has now peaked and consumption continues to grow,
% concern over reliance on imports from less stable regions has increased.
%
% The UK is hugely reliant on fossil fuels.  More than three quarters of UK
% electricity is generated from a relatively small number of large coal and gas
% fired power stations[Energy Digest].  Of the 298 stations with capacity over
% 1MW in the UK, 63 gas fuelled (including CCGT) and 13 coal fired stations
% supply over 290 TWh of the UK's 393TWh annual electric energy
% production[DUKES].  Much of the remainder is generated through nuclear
% fission.
%
%
% Concerns over climate change and security of supply have caused the UK
% government to pursue self-sustainable sources of electric energy.  This is
% illustrated by the government?s recent decision to make a legally binding
% commitment to an 80\% reduction in carbon dioxide emissions by 2050, relative
% to 1990 levels[].
%
%
% Relative to most other commodities, trade of electric energy is still in its
% infancy.  Liberalisation and unbundling of electricity supply industries costs
% many millions of pounds to implement[].  Countries, having made this
% investment, continue to restructure and adjust their energy markets in the
% hope of further reducing costs to the consumer and promoting innovation and
% efficiency through competition.
%
%
% \section{Conventional national power systems}
% Economies of scale prompted construction of the first large-scale power
% stations in the UK at the beginning of the twentieth century.  Following the
% introduction of the Electricity (Supply) Act 1926 the largest and most
% efficient of these were connected by a series of regional high-voltage
% three-phase AC grids synchronised at 50 Hertz.  Integration was completed and
% a national transmission system made operational in the UK for the first time
% in 1938.  This approach to electricity supply is largely the same as that
% still employed throughout the UK to the present day.  Alternating current,
% mainly from rotating synchronous machines, is transformed to high voltages for
% bulk transmission over long distances with high efficiency.  Power from the
% transmission system is fed through distribution networks in a uni-directional
% fashion at lower voltages before final usage.
%
% While maintenance and extension of the transmission system and of distribution
% networks is an everyday activity for energy utilities, many power system
% components have extremely long operational lifetimes[].  This and the
% magnitude of the capital investment made post-war in construction of the
% electricity networks suggests that there is likely to be little change in the
% topology of the system of wires in the foreseeable future.  This is further
% compounded by the fact that distribution networks are often radial in their
% structure.  Reliability and protection are major issues in the operation of
% power systems and the task of detecting and isolating a fault is often more
% difficult in systems that are meshed.
%
% Large-scale thermal power stations operate steam and gas turbines around 13
% metres in length, approximately 400 tons in weight and rotate at up to 3600
% revolutions per minute[SIEMENS].  The kinetic energy stored in these turbines
% and the connected synchronous machines is vast and the associated flywheel
% effect plays an important role in smoothing short-term imbalances in supply
% and demand[].
%
% Synchronous machines used in large thermal power stations invariably use a
% rotor winding that is excited and a magnetic flux created by a DC current.
% Controlling the magnitude of the rotor field current allows the reactive power
% provided or absorbed by the machine to be adjusted.  Reactive power is
% essential in maintaining system voltage within permissable limits[].
%
% \section{Highly distributed national power systems}
% The highly distributed power system is a conceptual future for the UK's
% national energy network.  It is capable of meeting current targets for reduced
% greenhouse gas emissions[] and reduces reliance on foreign fuel imports.  It
% is a system in which all but the least polluting fossil fuelled power stations
% have been decommissioned.  Their output supplanted by distributed generation,
% supported with demand-side management measures and advances in energy storage
% technology.
%
% Distributed (or embedded) generation is most easily defined as electricity
% production plant connected to the national electricity network at the
% distribution level[DG Book].  The transmission system being defined as that
% which operates at of above 275kV in England and Wales or at or above 132kV in
% Scotland.  This encompasses most small-scale plant, but there exist exceptions
% such as large-scale wind farms.
%
% Hydro-electric dams, biomass fuelled power stations and wind farms are the
% three sources of renewable energy currently available in the UK with
% capacities equivalent to that of fossil fuelled plant.  Hydro-electric power
% stations are often large-scale and transmission connected for bulk transport
% of power to load centres.  The low energy density of biomass fuels, relative
% to that of fossil fuels, often dictates that related generating plant is
% located close the fuel source origin and connected to a distribution network.
% The time varying nature of wind energy necessitates the use of induction
% machines for generation.  These are typically sinks of reactive power and
% require support for operation and maintenance of voltage.
%
% \subsection{Increased source granularity}
% In terms of demand the UK power system is already highly distributed.  The
% number of generators, controllable loads and storage systems is expected to be
% much greater in a HDPS.  However, there are at present already around 260
% generators supplying energy to consumers and the mechanisms currently in place
% to facilitate the trade of their power may be adaptable.
%
% Consumers and owners of small-scale generation will continue to desire
% protection from the risks associated with the wholesale marketplace.  Today?s
% typical domestic and commercial supply contracts between consumers and energy
% retailers offer such protection.  As the penetration of DER grows the role of
% energy retailers offering aggregation services will likely grow as more plant
% owners come to desire representation in the marketplace.  While there are a
% great many ways in which this aggregation may be arranged it remains
% essentially a contractual arrangement. A much larger challenge lies in the
% operation and control of a system in which the state of plant is principally
% determined by a decentralised free market.  The network may be divided
% into ?cells? for the provision of a single point of control, but each cell may
% contain individual items of plant being aggregated by different companies.
% Controlling the DER, so as to adhere to system constraints, must be done in
% the most economically efficient, in the least environmentally damaging manner
% and according to contracted access arrangements.  Also, the details of any
% control measures undertaken must be fed back to the marketplace such that they
% may be taken into consideration during the settlement process.  This interface
% between the technical domain and the commercial mechanisms remains an open
% research topic.
%
% \subsection{Inversion of control}
% Consumers have grown accustomed to using power from the grid at will.  System
% loads are by and large passive and, with the exception of dual rate white
% metered loads, only the largest make adjustments to their consumption
% according to price signals or system conditions.  It is likely that the
% fluctuating nature of the power output from generators exploiting renewable
% energy sources will necessitate an increase in the adoption of active loads.
% As control shifts from being almost purely supply-side, an opportunity opens
% for the energy marketplace to offer appropriate consumer choice, not only in
% terms of cost, but Quality of Service (QoS) also. Advances in smart metering
% promise to support such a migration.
%
% \subsection{Reduced network reticulation}
% The connection of controllable energy resources to lower voltage and less
% reticulated areas of the network may also offer new possibilities for
% structuring the relationship between the market and the system for management
% of network constraints.  There are generally three options for power system
% constraint management.
%
% \begin{itemize}
%   \item Only permit the formation of energy trades if delivery is physically
%   feasible.
%   \item Impose delivery charges which increase as network constraints are
%    approached.
%   \item Request extended bids and offers which include costs associated with
%   the adjustment of participant?s desired position.
% \end{itemize}
%
% The third option most closely describes the method currently used in the United
% Kingdom.  In a highly reticulated network it is difficult to determine the
% direction of each participant?s energy flows.  In turn, this poses
% difficulties in determining if a particular delivery is feasible and which
% participants are responsible for congesting the network.  Distribution
% networks are typically less reticulated than the transmission network to which
% the majority of generation is connected at present.  Therefore, there may be
% opportunities to utilise options one or two in an energy marketplace for an
% HDPS.
%
% Furthermore, the lower voltage of distribution networks may open up
% opportunities for more widespread use of power electronics technologies such
% as FACTS and phase shifting devices.  These would provide a limited ability to
% direct the flow of electrical energy.  How this might be managed on a wide
% scale and how efficient interaction with the marketplace might be achieved is
% open to investigation
%
% \subsection{Dual objective optimisation}
% Along with concern over the UK's dependence on natural gas imports, concern
% over the environmental impact of electricity generation is a primary motivator
% for a move to HDPS.  The energy market is expected to simultaneously minimise
% costs to the consumer and encourage reduced emission of greenhouse gasses.
%
% The European Union Greenhouse Gas Emission Trading Scheme (EU ETS) is an
% example of how a second marketplace running parallel to electricity trade, in
% this case trading allowances, may give weight to this new objective.  There
% are other ways in which the greenhouse gas output of certain technologies may
% be taken into consideration and this remains an open and important research
% topic.
