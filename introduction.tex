\chapter{Introduction}
This thesis compares methods from the field of artificial intelligence in their
ability to learn from reinforcement when trading electricity in a simulated
competitive marketplace.  An introduction to electricity supply and the
associated markets is provided in the present chapter.  The motivation for the
research presented in this thesis follows, along with a statement of the
principle research contributions that have been made.  Finally, a reading guide
and outline of the remaining chapters is provided.

\section{Motivation}
The average total demand for electricity in the UK is approximately 45.7GW and
the cost of buying 1MW for one hour is around \pounds40 \cite{decc:dukes09}.
This equates to yearly transaction values of \pounds16 billion.  Supply
failures highlight the value of electricity to societies also.  The New York
black-out in August 2003 involved 61.8GW of lost power supply to approximately
50 million consumers. The majority of supplies were restored within two days,
but the event is estimated to have cost more than \$6~billion and to have
contributed to 11~deaths \cite{minkel:2008,icf:2003}.

Quality of life for a person is directly proportional to that person's
electricity consumption\cite{alam:qol}.  The world population is currently
6.7~billion and forecast to pass 9~billion by the year 2050 \cite{un:pop}.
Electricity production currently demands over $1/3$ of the annual primary energy extracted. As people endevour to improve their quality of life,
finite primary energy fuel resources are becoming increasingly scarce.
Competitive markets are an economic device for efficient allocation of scarce
resources.

Commercialisation of electricity supply industries is a relatively new
practice, having begun in the early 1990s.  The inability to store electricity,
once generated, in a commercially viable quantity prevents trade as a
conventional commodity.  Trading mechanisms must be created to allow shortfalls
in electric energy to be purchased at short notice from quickly dispatchable
generators.  Various mechanisms have been implemented in countries and states
around the world.  How best to structure electricity markets is an active field
of research.  Electricity market designs are also complicated by the need to
manage complex dynamic constraints in the power systems that deliver the traded
product.

\section{Electric Power Systems}
Generation and bulk movement of electricity in the UK takes place in a
three-phase alternating current (AC) power system.  The phases are high voltage
electrical waveforms, $120^\circ$ offset in time and oscillating 50 times per
second.  Alternators or synchronous generators, typically rotating at 3600rpm
or 1800rpm, generate apparent power $S$ and inject current $I_l$ into a line at
a voltage $V_l$, typically between 11kV and 25kV.  One of the prinicpal reasons
that alternating current, and not direct current (DC), systems are used to
supply electricity is that the power can be transformed between voltages with
very high efficiency.  The apparent power conducted by a line is the product of
the line voltage and the line current,
\begin{equation}
S = \sqrt{3} V_l I_l
\end{equation}
and the ohmic heating losses are proportional to the square of the line
current,
\begin{equation}
P_{r} = 3 I^2 R
\end{equation}
where $R$ is the resistance of the line.  Transmitting power at ultra-high
voltages reduces the current flow, resulting in substantial losses reductions.
One drawback of higher voltages is the larger extent and integrity of conductor
insulation required between one another, neutral and earth.  This results in
the need for large transmission towers and high cable costs when undergrounding
systems.

UK transmission systems operate at 400kV, 275kV or 132kV, but systems upto
and beyond 1000kV are in operation in larger countries.  For transmission over
very long distances or undersea, high voltage DC systems have become
econmically viable in recent years.  The ability to transform power between
voltages and transmit large amounts of power over long distances allows
generation at high capacity stations, located away from large load centres,
which offer economies of scale and lower operating costs.  It also allows
electricity to be transmitted across country borders and from plant generating
power from renewable energy sources remote locations.

For delivery to typical consumers electric energy is transferred from the
transmission system, at a substation, to the grid supply point of a
distribution system.  Distribution networks are also three-phase AC systems,
but operate at lower voltages and differ in their general structure or topology
from transmission networks.  Transmission networks are typically highly
interconnected, providing several paths for power flow, whereas distribution
networks in rural areas typically consist of long radial feeders (usually
overhead lines) or in urban areas consist of many ring circuits.  Three-phase
transformers that step the voltage down to levels more convenient for general
use (typically from 11kV or 33kV to 400V) are spaced along the branches/rings.
All three-phases at 400V may be provided for industrial and commercial loads or
individual phases at 230V supply typical domestic and other commercial loads.
Splitting of phases is usually planned so that each is loaded equally.  This
produces a balanced symetrical system that may be analysed as a single phase
circuit (See Section \ref{sec:power_system_model}).

\section{Electricity Markets}
The United Kingdom was the first large country to privatise its electricity
supply industry and the market structures that have since been adopted
encapsulate the main principles behind electricity markets.

The England \& Wales Electricity Pool was created in 1990 to break apart the
monolithic Central Electricity Generating Board and gradually introduce
competition in generation and retail supply (See Section \ref{sec:thepool},
below).  Early adoption of electrcity markets by the United Kindom has lead to
it hosting the main European power and gas exchanges and the UK boasts a
enviably high degree of consumer switching, deemed essential to a competitive
marketplace.  The Pool has since been replaced by trading arrangements in which
market outcomes are no longer centrally determined, but arise largely from
bilateral agreements between producers and suppliers.

\subsection{British Electricity Transmission and Trading Arrangements}
\label{sec:betta}
Concerns over exploitation of market power in The England \& Wales Electricity
Pool and its effectiveness in reducing consumer electricity prices prompted the
introduction of the New Electricity Trading Arrangements (NETA) in March 2001
\cite{martoccia:2005}.  The Scottish electricity industry was integrated into
the nationwide British Electricity Transmission and Trading Arrangments (BETTA)
in April 2005 by The Energy Act 2004.  The aim was to improve efficiency and
provide greater choice to participants.  While The Pool operated a single daily
auction and plant was dispatched centrally, under the new arrangements,
participants became self-dispatching and market positions became determined
through continuous bilateral trading between generators, suppliers, traders and
consumers.

Under BETTA, the majority of power is traded through long-term contracts
that are customised to the requirements of each party.  These suit participants
responsible for large power plants or those purchasing large volumes of
power for many customers.  A relatively large amount of time is required for
long-term contracts to be agreed upon and this has an associated transaction
cost.  However, they reduce risk for large players and a degree of flexibility
can be provided through option contracts.

Power is also traded directly between participants through over the counter
(OTC) contracts that are usually of a standardised form.  Such contracts
typically concern smaller volumes of power and have much lower associated
transaction costs.  Often they are used by participants to refine their market
position ahead of delivery time.

Trading facilities, such as power exchanges, provide a means for participants
to fine-tune their positions further through short-term transactions for
relatively small quantities of energy.  Modern exchanges are computerised and
accept anonymous offers and bids submitted electronically.  A submitted
offer/bid will be paired with any outstanding bids/offers in the system with
compatible prices and quantities.  The details are then displayed for traders
to observe and use to educate their decisions.

All bilateral trading must be completed before ``gate-closure'', which is point
set before real-time that gives the system operator the opportunity to balance
supply and demand and mitigate breaches of system limits.  In keeping with the
Uk's free market philosphy, a competitive market is used in the balancing
process also.  A generator that is not fully loaded may offer a price at which
it is willing to increase its output by a specified quantity.  The speed at
which it is capable of doing so must be stated with the offer.  Certain loads
may also offer demand reductions at a price and can typically be implemeted
very quickly.  Longer-term contracts for balancing services are also struck
between the system operator and generators/suppliers in order to avoid the
price volatility often associated with spot markets.

\subsection{The England \& Wales Electricity Pool}
\label{sec:thepool}
The Electric Lighting Act 1882 began the development of the UK's electricity
supply industry by allowing persons, companies and local authorities to set up
supply systems, principally at the time for the purposes of street lighting and
trams.  Under The Electricity Supply Act 1926 the Central Electricity Board
started operating the first grid of regional networks interconnected and
synchronised at 132kV, 50Hz in 1933.  This began operation as a national system
five years later in 1938 and was nationalised under The Electricity Act 1947
with the merger of over 600 electricity companies and the creation of the
British Electricity Authority.  This was then dissolved and replaced with the
Central Electricity Generating Board (CEGB) and the Electricity Council under
The Electricity Act 1957.  The CEGB was responsible for planning the network
and generating sufficient electricity until the start of privatisation in 1990.

The UK electricity supply industry was privatised under Prime Minister
Margaret Thatcher and The England \& Wales Electricity Pool was created in
March 1990.  Control of the transmission system was transferred from the
CEGB to The National Grid Company which was originally owned by twelve regional
electricity companies and is now publically listed.  The Pool was a
multilateral contractual arrangement between generators and suppliers and did
not itself buy or sell electricity.  Competition in generation was introduced
gradually, by only entitling customers with consumption greater than or equal
to 1MW (approximately 45\% of the non-domestic market \cite{decc:dukes09}) to
purchase electricity form any listed supplier.  This limit was lowered in April
1994 to included customers with peak loads of 100kW or more.  Finally, between
September 1998 and March 1999 the market was opened to all customers.

Scheduling of generation was on a merit order basis (cheapest first) at a day
ahead stage and set a wholesale electricity price for each half-hour period of
the schedule day.

Forecasts of total demand in MW, based on historic data and adjusted for
factors such as the weather, for each settlement period were used by generating
companies and organisations with interconnects to the England \& Wales grid to
formulate bids that had to be submitted to the grid operator by 10AM on the day
before the schedule day.

% TODO: Insert Pool bids figure.

Bids consisted of five price parameters as illustrated in Figure X and
represented the avoidable cost of generation.  A start-up price was also
included, representing the cost of turning on the generator from cold.  A
no-load price $c_{noload}$ would equal the cost in pounds of keeping the
generator running regardless of output. Three incremental prices $c_1$, $c_2$
and $c_3$ specify the cost per MWh of generation between set-points $p_1$,
$p_2$ and $p_3$.

A settlement computer program was used to calculate an unconstrained schedule
(not accounting for the physical limitations of the transmission system),
meeting the forecast demand and requirements for reserve while minimising cost.
Cheapest bids up to the marginal point would get accepted first and the bid
price from the marginal generator would generally determine the system marginal
price for each settlement period.  The system marginal price determined the
prices paid by consumers and paid to generators that get adjusted such that
that the costs of transmission are covered by the market and that the
availability of capacity is encouraged at certain times.

Variations in demand and changes in plant availability would be adjusted for by
the grid operator, producing a constrained schedule.  Generators having
submitted bids would get instructed to increase or reduce production as
appropriate.  Alternatively, the grid operator could instruct large customers
with contracts to curtail their demand to do so or instruct generators
contracted to provide ancillary services to adjust production.

\section{Electricity Market Simulation}
The previous sections have illustrated the dependence of modern societies on
electric energy and explained how its supply is trusted to unadministered
bilateral trading arrangements.  Electricity supply involves technology, money,
people, natural resources and the environment.  These aspects are all changing
and the discipling must be constantly researched in order that systems such as
electricity markets are fit for purpose.  The value of electricity to society
makes it infeasible to experiment with radical changes to trading arrangements
on real systems.  A practical alternative is to create an abstract methematical
model with a set of simplifying approximations and assumptions and find
analytical solutions by simulating the model using a computer program.

Game theory is a branch of applied mathematics in which behaviour in strategic
situations is captured mathematically.  A common approach to doing this is to
model the system and players as a numerical optimisation problem.  Section
\ref{sec:opf} defines the optimal power flow problem, which is a classic
optimisation problem from the field of Power Engineering.  Electricity markets
are commonly modelled using variations on the optimal power flow problem with
player strategies integrated[ref].  The present thesis reports electricity
market research using \textit{agent-based} modelling, which is an alternative
approach to the mathematics of games.

\subsection{Agent-Based Modelling}
Social systems, such as electricity markets, are inherently complex and involve
interactions between different types of individuals and/or between individuals
and collective entities, such as organisations or groups, the behaviour of which
is itself the product of individual interactions.  This level of complexity
drives classical monolithic equilibrium models to their limits.  Models are
often highly stylised and limited to small numbers of players with strong
constraining assumptions made on their behaviour.

Agent-based simulation involves modelling simultaneous operations and
interactions between adaptive agents and assessing their effect on the system
as a whole.  Macro-level system properties arise from agent interactions, even
those with simple behavioural rules, that could not be deduced by simply
aggregating the agent's properties [Life].

Following \cite{tesfatsi:handbook}, the objectives of agent-based modelling are
roughly in four strands: empirical, normative, heuristic and methodological.
The \textit{empirical} objectives are to understand how and why macro-level
regularities have evolved from micro-level interactions when little or no
top-down control is present.  \textit{Normative} research aims to relate
agent-based models to an ideal standard or optimal design.  The objective being
to evaluate proposed designs for social policy, institutions or processes in
their ability to produce socially desirable system performance.  The heuristic
strand aims to generate theories on the fundamental causal mechanisms in social
systems that can be observed, even in simple systems, when there are
alternative initial conditions.  The research in this thesis has the general
goal of providing \textit{methodologial} advancement to the field.
Improvements in the tools and methods available aid research with the
previously stated goals.

\section{Problem Statement \& Hypothesis}%/Aims \& Objectives}
In an electricity market environment, as in most operational settings, the
state and action spaces are high-dimensional and continuous in nature.
Furthermore, certain state information, such as demand forecasts, exhibits a
degree of uncertainty and other data, such as competitor bids, are hidden.

Traditional value-function based reinforcement learning methods (See Section
\ref{sec:rl}, below) offer few convergence guarantees in Partially Observable
Markov Decsison Processes[].  Without the use of value function approximation
techniques these methods are restricted by Bellmans's Curse of Dimentionality
and can not be applied to complex problems with high-dimensional state and
actions space.  With value function approximation, feedback between policy
updates and value function changes can result in oscillations or divergence in
these methods \cite{peters:enac}.

Policy gradient reinforcement learning methods (See Section
\ref{sec:policygradient}, below) do not suffer in the same way from many of the
problems that marr value-function based methods in high dimensional domains.
They offer strong convergence guarantees, do not require that all states are
visited and work with state and action spaces that are continuous, discrete or
mixed.  Policy performance may be degraded by uncertainty in state data, but
the learning methods need not be altered.  Policy gradient methods have beeb
sucessfully applied in many operational settings
\cite{barto:policy,shaal:robots,moody:direct,peshkin:routing}.

It is proposed that agents learning using policy gradient methods may
outperform those using value function based based methods in a simulated
competitive energy trade environment.  Policy methods may learn faster, achieve
greater profitability and exploit constraints in the power system.

% Methods including learning classifier systems, genetic algorithms and
% reinforcement learning have been successfully used to research the
% charateristics of energy markets in the past\cite{anke:2008}.  This an
% alternative to the traditional closed-form equilibrium approaches to game
% theory research in which behavior emerges from the interactions of many
% separable, self-serving agents. Typically in these studies, an agent is
% associated with a portfolio of generating units and/or dispatchable
% loads. Interaction with the environment involves submission of offers to sell
% or bids to buy\footnote{Beware that certain authors may use the term ``bid''
% to refer to an offer to sell when discussing single-sided auctions.} a
% quantity of power at a specified price in a particular time period.  The
% learning algorithms typically use revenue or earnings as a reward signal and
% adjust the policy used to select offer/bid price and quantity values.

% Research into electricity markets using reinforcement learning typically
% involves discretization the action domain, often into incremental markups
% on marginal cost.  Also, either sensor domains are discretized or state
% information is disregarded altogether.  Despite these simplifications, authors
% have been drawn many practical conclusions from this approach \cite{anke:2008}.
% If learning algorithms are to deliver on their potential for application in
% operational electricity market settings, modelling continuous domains will be
% necessary.

% Traditional reinforcement learning methods such as Sarsa and Q-learning (See
% Section \ref{sec:rl}, below) can be applied to systems with continuous domains
% by using connectionist systems for value function approximation
% \cite{barto:neuron}. However, feedback between policy updates and value
% function changes can result in oscillations or divergence in these algorithms
% even when applied to simple systems \cite{peters:enac}.  In response to this,
% policy-gradient methods, pioneered by Williams \cite{williams:reinforce}, which
% search directly in the policy space have been developed and applied in many
% real-life settings
% \cite{barto:policy,shaal:robots,moody:direct,peshkin:routing}.
%
% The proposal is to combine policy-gradient reinforcement learning methods with
% artificial neural networks for continuous policy function approximation and
% apply to simulations electricity trade.  The hypothesis being that superior use
% of sensor data results in improved performance over previously applied
% value-function methods.

\section{Reader's Guide \& Thesis Outline}
This thesis is written for several kinds of readers.  The combination of this
introduction and the background material in Chapter \ref{ch:background} should
be sufficient for newcomers to the field to understand methods and results.  A
student who has taken an energy economics class or two may appreciate this
thesis as an introduction to electricity markets and their simulation.
Research students embarking upon postgraduate study of electricity markets may
find the ideas for further work in Chapter \ref{ch:future} of particular
interest.  Researchers experieinced in adpative control and machine learning,
looking for new application domains for their methods may find the electricity
market model definition in Chapter \ref{ch:method} to be of value.

The presentation is organised into nine chapters.  Chapter \ref{ch:background}
provides a simple introduction to the theories of optimal power flow and
reinforcement learning which underpin the later research.  A comprehensive
review of closely related work from the fields of Power Enginneering, Machine
Learning and Computer Science is given in Chapter \ref{ch:related_work}.
Chapter \ref{ch:method} defines the electricity market model based on optimal
power flow and the multi-agent system used to coordinate electricity trade.
Several ideas for building upon the tools developed for this research are
explained in Chapter \ref{ch:future}.  Finally, a summary of the overall
conclusions that can be made from this thesis is given in \ref{ch:conclusion}.

% Free market democracy has underpinned the transformation of large western
% economies in the post war era and continues to be relied upon.  Towards the
% end of the nineteenth century the principals of competitive trade were
% successfully applied to electric power industries, beginning in the UK in March
% 1990 with the creation of The Electricity Pool.

% Two trends characterise modern power systems Engineering:
% Increased liberalisation of the industry through competitive energy trade and
% increased presence of renewable energy generation on the network.  As the
% number and variety of electricity sources becomes greater, the necessity for
% automated trade of their power increases.  Control algorithms may draw sensor
% input from data networks and other sources and use some relevant measure of
% performance, such as profitability, to learn from trading decisions.

% \IEEEPARstart{E}{nergy} consumers will often not switch suppliers despite
% apparent benefits of doing so, or mistakenly switch to more expensive
% contracts[].  Energy companies are thus unlikely to reduce prices as this
% can not be expected to result in new contracts.  One possible solution to this
% problem is to give the responsibility for buying energy to autonomous,
% learning control algorithms.  Drawing information from the internet and using
% standardised messaging [CIM] to form contracts, energy bots have the potential
% to introduce greater liquidity to energy markets and promote reduction of
% costs to the consumer.

% \IEEEPARstart{D}{ynamic} competitive markets will be key in the transition
% to a low carbon economy\cite{decc:transition}.  Fluctuations in global oil
% prices have recently raised questions over the functioning of the energy
% market.  Energy industries in the UK contribute 4.8\% of 2.13
% trillion dollar gross domestic product.  The UK electricity industry supplies
% \~26 million customers with \~350TWh of energy each year at a cost of 3.5--9.0
% p/kWh. The New York blackout in August, 2003 resulted in the loss of 61.8
% GW of electric load associated with 50 million consumers.  Supply by was
% largely restored within two days and the event is estimated to have cost
% \$7--\$10 billion.  Small improvements in the design of markets associated with
% this industry can have an impact greatly upon the welfare of society and
% conversely also. To understand the complex dynamics of these systems it is
% possible to simulate them computationally.  One alternative to the game
% theoretic models typically employed in computational economics is the study
% of emergent behaviour in collections of individual autonomous actors.
%
% Representing competitive behaviour is key to the simulation of markets in this
% way. Simple heurisitc approaches have been taken in the study of
% \ldots[ConzelmannEMCAS].  While more complex machine learning techniques such
% as state vector machines have been employed in the study of market efficiency
% and market power[Bunn, Bagnall].  Reinforcement Learning (RL) is an
% unsupervised machine learning technique, most commonly applied to control
% problems, and as such is well suited to this
% application [Acrobot]\cite{suttonbarto:reinforcement}. [TD()]. In this paper
% RL is used to control the output and price of generating units and
% despatchable loads connected in a balanced three-phase high-voltage power
% system. RL has been applied before in a very similar manner to the energy
% trading problem using the Roth-Erev algorithm[TesfatsoRE].  The software
% implementation for this algorithm used in the simulations for this paper was
% translated from[TesfatsoAMES].

% Game theoretic models are commonly associated with economics and attempt to
% capture behaviour in strategic situations mathematically.  They have been
% applied to electric energy problems of many forms, including but not limited to
% analysis of market structure, market liquidity, pricing methodologies,
% regulatory structure, plant positioning and network congestion.  More recently,
% agent-based simulation has received a certain degree of attention from
% researchers and has been applied in some of these fields.
% \begin{quotation}
%  ``Every scientist knows the rule called Occam's Razor:  Faced with several
%  competing hypotheses, prefer the simplest one.  There is also an unspoken
%  corollary that might be called Occam's Castle:  Faced with several competing
%  places to build a new science, prefer the simplest one\ldots  Where the
%  foundation is firmest, the castle will rise highest.  Where the ground is
%  solid, build there, and the universe is so constructed that you will have a
%  view.''\cite{weiner:tlm}
% \end{quotation}

% Engineers must strive for complexity in their work.  Rarely will a simple
% solution will perform a function to a higher degree than a more complex one.
% Certainly, where a function is either performed or not performed, prefer the
% simpler one, but most often problems can be solved to varying degrees.

% The broad aim of the research presented in this thesis is to prove that the
% above conjecture applies to reinforcement learning algorithms for power trade.
% Previous research in this field (See Chapter~\ref{ch:related_work} below) has
% used very simple algorithms in relation to those from the latest advances in
% artificial intelligence (See Sections~\ref{sec:enac}~and~\ref{sec:reinforce}
% below).  The goal is to prove that policy gradient methods, using artificial
% neural networks for policy function approximation, are better suited to
% learning the complex dynamics of a power system.

\section{Research Contributions}
The research presented in this thesis pertains to the academic fields of power
engineering, artificial intelligence and economics.  The principle
contributions in these areas are:

\begin{itemize}
  \item The proof that policy gradient reinforcement learning algorithms
  outperform value-function algorithms when applied to the power trade problem,
  \item A novel coupling of power system models and optimal power flow
  algorithm results with agents capable of handling discrete and continuous
  sensor and action spaces,
  \item Implementations of Roth-Erev reinforcement learning algorithms and
  continuous versions of Q-learning and Q($\lambda$) for the open source
  PyBrain library,
  \item Open source implementations of power flow and optimal power flow
  algorithms in the Python programming language, preserving sparsity throughout
  the optimisation using the open source CVXOPT library.
\end{itemize}

% This paper compares policy-gradient reinforcement learning algorithms
% REINFORCE and ENAC with value-function methods Roth-Erev, Q($\lambda$) and
% Sarsa in their relative ability to trade electricity competitively.  Power
% systems are modelled as balanced three-phase AC networks in the steady-state.
% Offers/bids for active and reactive power from agent participants are cleared
% using AC optimal power flow with an auction interface that returns single
% period revenue and earnings values. Through individual and multi-player
% experiments the methods are compared in their ability to learn quickly, compete
% in large systems and exploit characteristics of the power system. It is shown
% that, in electricity trade, policy-gradient methods:
% \begin{itemize}
%   \item converge successfully on an optimal policy,
%   \item are slower to converge than value-function algorithms,
%   \item can learn more complicated characteristics of the power system than
%   value-function algorithms,
%   \item scale better when applied to larger systems and reactive power markets.
% \end{itemize}

\section{Outline}%Thesis structure/Overview/Reading guide}
This thesis is focussed on the application of standard and advanced
reinforcement learning algorithms to a particular problem domain.  The reader
will require a certain degree of prior knowledge, or must be willing to read
much of the referenced material, to fully understand the methodology taken.
The intended audience is engineering and economics researchers interested in
the application of reinforcement learning algorithms to the problem of trading
energy in electric power systems.

% Section~\ref{sec:opf} of this paper presents the power system model, the
% optimal power flow formulation and the auction interface from MATPOWER.
% Reinforcement learning methods Sarsa, Q($\lambda$), Roth-Erev, REINFORCE and
% ENAC are defined in Section~\ref{sec:rl}. Section~\ref{ch:method} introduces
% the three experiments used to compare the aforementioned methods. Numerical
% results from these experiments are reported in Section \ref{ch:results} and
% an interpretation and critical analysis of them is given in
% Section~\ref{sec:discuss}.  Finally, a review of related research is presented
% in Section~\ref{sec:related} and Section~\ref{ch:conclusion} provides a
% conclusion.

% Industrialised societies have become increasingly reliant on the supply of
% electric energy since the connection of large power stations began in 1938.
% The extent to which this is true can be seen in the financial impact that loss
% of supply has on society.
%
%
% In June 2004 the United Kingdom (UK) became a net importer of natural gas for
% the first time in 8 years[EIA DOE].  Since energy industry privatisation by
% the Conservatives in the 1980s, use of domestic gas reserves for electricity
% generation has been encouraged and exploited.
%
% % Insert dash for gas pie-chart here.
%
% As UK natural gas production has now peaked and consumption continues to grow,
% concern over reliance on imports from less stable regions has increased.
%
% The UK is hugely reliant on fossil fuels.  More than three quarters of UK
% electricity is generated from a relatively small number of large coal and gas
% fired power stations[Energy Digest].  Of the 298 stations with capacity over
% 1MW in the UK, 63 gas fuelled (including CCGT) and 13 coal fired stations
% supply over 290 TWh of the UK's 393TWh annual electric energy
% production[DUKES].  Much of the remainder is generated through nuclear
% fission.
%
%
% Concerns over climate change and security of supply have caused the UK
% government to pursue self-sustainable sources of electric energy.  This is
% illustrated by the government?s recent decision to make a legally binding
% commitment to an 80\% reduction in carbon dioxide emissions by 2050, relative
% to 1990 levels[].
%
%
% Relative to most other commodities, trade of electric energy is still in its
% infancy.  Liberalisation and unbundling of electricity supply industries costs
% many millions of pounds to implement[].  Countries, having made this
% investment, continue to restructure and adjust their energy markets in the
% hope of further reducing costs to the consumer and promoting innovation and
% efficiency through competition.
%
%
% \section{Conventional national power systems}
% Economies of scale prompted construction of the first large-scale power
% stations in the UK at the beginning of the twentieth century.  Following the
% introduction of the Electricity (Supply) Act 1926 the largest and most
% efficient of these were connected by a series of regional high-voltage
% three-phase AC grids synchronised at 50 Hertz.  Integration was completed and
% a national transmission system made operational in the UK for the first time
% in 1938.  This approach to electricity supply is largely the same as that
% still employed throughout the UK to the present day.  Alternating current,
% mainly from rotating synchronous machines, is transformed to high voltages for
% bulk transmission over long distances with high efficiency.  Power from the
% transmission system is fed through distribution networks in a uni-directional
% fashion at lower voltages before final usage.
%
% While maintenance and extension of the transmission system and of distribution
% networks is an everyday activity for energy utilities, many power system
% components have extremely long operational lifetimes[].  This and the
% magnitude of the capital investment made post-war in construction of the
% electricity networks suggests that there is likely to be little change in the
% topology of the system of wires in the foreseeable future.  This is further
% compounded by the fact that distribution networks are often radial in their
% structure.  Reliability and protection are major issues in the operation of
% power systems and the task of detecting and isolating a fault is often more
% difficult in systems that are meshed.
%
% Large-scale thermal power stations operate steam and gas turbines around 13
% metres in length, approximately 400 tons in weight and rotate at up to 3600
% revolutions per minute[SIEMENS].  The kinetic energy stored in these turbines
% and the connected synchronous machines is vast and the associated flywheel
% effect plays an important role in smoothing short-term imbalances in supply
% and demand[].
%
% Synchronous machines used in large thermal power stations invariably use a
% rotor winding that is excited and a magnetic flux created by a DC current.
% Controlling the magnitude of the rotor field current allows the reactive power
% provided or absorbed by the machine to be adjusted.  Reactive power is
% essential in maintaining system voltage within permissable limits[].
%
% \section{Highly distributed national power systems}
% The highly distributed power system is a conceptual future for the UK's
% national energy network.  It is capable of meeting current targets for reduced
% greenhouse gas emissions[] and reduces reliance on foreign fuel imports.  It
% is a system in which all but the least polluting fossil fuelled power stations
% have been decommissioned.  Their output supplanted by distributed generation,
% supported with demand-side management measures and advances in energy storage
% technology.
%
% Distributed (or embedded) generation is most easily defined as electricity
% production plant connected to the national electricity network at the
% distribution level[DG Book].  The transmission system being defined as that
% which operates at of above 275kV in England and Wales or at or above 132kV in
% Scotland.  This encompasses most small-scale plant, but there exist exceptions
% such as large-scale wind farms.
%
% Hydro-electric dams, biomass fuelled power stations and wind farms are the
% three sources of renewable energy currently available in the UK with
% capacities equivalent to that of fossil fuelled plant.  Hydro-electric power
% stations are often large-scale and transmission connected for bulk transport
% of power to load centres.  The low energy density of biomass fuels, relative
% to that of fossil fuels, often dictates that related generating plant is
% located close the fuel source origin and connected to a distribution network.
% The time varying nature of wind energy necessitates the use of induction
% machines for generation.  These are typically sinks of reactive power and
% require support for operation and maintenance of voltage.
%
% \subsection{Increased source granularity}
% In terms of demand the UK power system is already highly distributed.  The
% number of generators, controllable loads and storage systems is expected to be
% much greater in a HDPS.  However, there are at present already around 260
% generators supplying energy to consumers and the mechanisms currently in place
% to facilitate the trade of their power may be adaptable.
%
% Consumers and owners of small-scale generation will continue to desire
% protection from the risks associated with the wholesale marketplace.  Today?s
% typical domestic and commercial supply contracts between consumers and energy
% retailers offer such protection.  As the penetration of DER grows the role of
% energy retailers offering aggregation services will likely grow as more plant
% owners come to desire representation in the marketplace.  While there are a
% great many ways in which this aggregation may be arranged it remains
% essentially a contractual arrangement. A much larger challenge lies in the
% operation and control of a system in which the state of plant is principally
% determined by a decentralised free market.  The network may be divided
% into ?cells? for the provision of a single point of control, but each cell may
% contain individual items of plant being aggregated by different companies.
% Controlling the DER, so as to adhere to system constraints, must be done in
% the most economically efficient, in the least environmentally damaging manner
% and according to contracted access arrangements.  Also, the details of any
% control measures undertaken must be fed back to the marketplace such that they
% may be taken into consideration during the settlement process.  This interface
% between the technical domain and the commercial mechanisms remains an open
% research topic.
%
% \subsection{Inversion of control}
% Consumers have grown accustomed to using power from the grid at will.  System
% loads are by and large passive and, with the exception of dual rate white
% metered loads, only the largest make adjustments to their consumption
% according to price signals or system conditions.  It is likely that the
% fluctuating nature of the power output from generators exploiting renewable
% energy sources will necessitate an increase in the adoption of active loads.
% As control shifts from being almost purely supply-side, an opportunity opens
% for the energy marketplace to offer appropriate consumer choice, not only in
% terms of cost, but Quality of Service (QoS) also. Advances in smart metering
% promise to support such a migration.
%
% \subsection{Reduced network reticulation}
% The connection of controllable energy resources to lower voltage and less
% reticulated areas of the network may also offer new possibilities for
% structuring the relationship between the market and the system for management
% of network constraints.  There are generally three options for power system
% constraint management.
%
% \begin{itemize}
%   \item Only permit the formation of energy trades if delivery is physically
%   feasible.
%   \item Impose delivery charges which increase as network constraints are
%    approached.
%   \item Request extended bids and offers which include costs associated with
%   the adjustment of participant?s desired position.
% \end{itemize}
%
% The third option most closely describes the method currently used in the United
% Kingdom.  In a highly reticulated network it is difficult to determine the
% direction of each participant?s energy flows.  In turn, this poses
% difficulties in determining if a particular delivery is feasible and which
% participants are responsible for congesting the network.  Distribution
% networks are typically less reticulated than the transmission network to which
% the majority of generation is connected at present.  Therefore, there may be
% opportunities to utilise options one or two in an energy marketplace for an
% HDPS.
%
% Furthermore, the lower voltage of distribution networks may open up
% opportunities for more widespread use of power electronics technologies such
% as FACTS and phase shifting devices.  These would provide a limited ability to
% direct the flow of electrical energy.  How this might be managed on a wide
% scale and how efficient interaction with the marketplace might be achieved is
% open to investigation
%
% \subsection{Dual objective optimisation}
% Along with concern over the UK's dependence on natural gas imports, concern
% over the environmental impact of electricity generation is a primary motivator
% for a move to HDPS.  The energy market is expected to simultaneously minimise
% costs to the consumer and encourage reduced emission of greenhouse gasses.
%
% The European Union Greenhouse Gas Emission Trading Scheme (EU ETS) is an
% example of how a second marketplace running parallel to electricity trade, in
% this case trading allowances, may give weight to this new objective.  There
% are other ways in which the greenhouse gas output of certain technologies may
% be taken into consideration and this remains an open and important research
% topic.
