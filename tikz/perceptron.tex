
% \newcommand{\neuron}[4]{
% \node[minimum size=20pt,inner sep=1pt,rotate=-90,name=s,shape=circle
% split,draw,#4] (#1) at (#2,#3)
% {\begin{sideways}\tiny$\hspace{0.5mm}f$\end{sideways}\nodepart{lower}\begin{sideways}\tiny$\sum$\end{sideways}};
% }

\def\layersep{3.5cm}
% \begin{figure}
% \centering
% \begin{tikzpicture}[shorten >=1pt,->,draw=black!50, node distance=\layersep]
%     \tikzstyle{every pin edge}=[<-,shorten <=1pt]
%     \tikzstyle{neuron}=[circle,draw=black!25,minimum size=20pt,inner sep=0pt]
%     \tikzstyle{input neuron}=[neuron];
%     \tikzstyle{output neuron}=[neuron];
%     \tikzstyle{hidden neuron}=[neuron];
%     \tikzstyle{annot} = [text width=4em, text centered]
%
%     % Input layer nodes.
%     \foreach \name / \y in {1,...,4}
% %        \node[input neuron, pin=left:Input \y] (I-\name) at (0,-\y) {\tiny$f$};
%         \neuron{I-\name}{0}{-\y}{pin=south:Input \y};
%
%     % Hidden layer nodes.
%     \foreach \name / \y in {1,...,5} {
%         \path[yshift=0.5cm]
%             node[minimum size=20pt,inner sep=1pt,rotate=-90,name=s,shape=circle
%             split,draw] (H-\name) at (\layersep,-\y cm)
%             {\begin{sideways}\tiny$\hspace{0.5mm}g$\end{sideways}\nodepart{lower}\begin{sideways}\tiny$\sum$\end{sideways}};
%
% %             node[hidden neuron] (H-\name) at (\layersep,-\y cm)
% %             {\scriptsize$\sum$};
% %        \draw[-,yshift=0.5cm] (\layersep-1cm,-\y-1) cos (\layersep,-\y) sin
% %        (\layersep+1cm,-\y+1);
%     }
%
%     % Output layer nodes.
% %    \node[output neuron,pin={[pin edge={->}]right:Output}, right of=H-3] (O){};
%     \node[minimum size=20pt,inner sep=1pt,rotate=-90,name=s,shape=circle
%     split,draw,pin={[pin edge={->}]north:Output},above of=H-3] (O)
%     {\begin{sideways}\tiny$\hspace{0.5mm}h$\end{sideways}\nodepart{lower}\begin{sideways}\tiny$\sum$\end{sideways}};
%
%
%     % Input layer - hidden layer connections.
%     \foreach \source in {1,...,4}
%         \foreach \dest in {1,...,5}
%             \path (I-\source) edge (H-\dest);
%
%     % Hidden layer - output layer connections.
%     \foreach \source in {1,...,5}
%         \path (H-\source) edge (O);
%
%     % Annotate the layers
%     \node[annot,above of=H-1, node distance=1cm] (hl) {Hidden Layer};
%     \node[annot,left of=hl] {Input Layer};
%     \node[annot,right of=hl] {Output Layer};
% \end{tikzpicture}
% \caption{Multi-layer feed-forward perceptron with bias nodes.}
% \label{fig:perceptron}
% \end{figure}

\begin{figure}
\centering
\begin{scriptsize}
\begin{tikzpicture}[shorten >=1pt,->,draw=black, node distance=\layersep]
	\tikzstyle{neuron}=[minimum size=30pt,inner sep=1pt,rotate=-90,name=s,
  		shape=circle split,draw,drop shadow,fill=white];
	\tikzstyle{every pin edge}=[<-,shorten <=1pt];
    \tikzstyle{annot} = [text width=4em, text centered];
    \tikzstyle{conn} = [draw=black];

% Input layer bias node.
\node[neuron] (I-0) at (0, 0)
{\begin{sideways}\hspace{1.0mm}1\end{sideways}};

% Input layer neurons.
\node[neuron,pin=south:\small Input 1] (I-1) at (0,-1.4 cm)
{\begin{sideways}\hspace{1.0mm}$f$\end{sideways}};
\node[neuron,pin=south:\small Input 2] (I-2) at (0,-2.8 cm)
{\begin{sideways}\hspace{1.0mm}$f$\end{sideways}};
\node[neuron,pin=south:\small Input 3] (I-3) at (0,-4.2 cm)
{\begin{sideways}\hspace{1.0mm}$f$\end{sideways}};

% Hidden layer bias node.
\node[neuron,xshift=-7mm] (H-0) at (\layersep, 0)
{\begin{sideways}\hspace{1.0mm}1\end{sideways}};

% Hidden layer neurons.
\node[neuron] (H-1) at (\layersep,-0.7 cm)
{\begin{sideways}$\hspace{1.0mm}g$\end{sideways}\nodepart{lower}\begin{sideways}$\sum$\end{sideways}};
\node[neuron] (H-2) at (\layersep,-2.1 cm)
{\begin{sideways}$\hspace{1.0mm}g$\end{sideways}\nodepart{lower}\begin{sideways}$\sum$\end{sideways}};
\node[neuron] (H-3) at (\layersep,-3.5 cm)
{\begin{sideways}$\hspace{1.0mm}g$\end{sideways}\nodepart{lower}\begin{sideways}$\sum$\end{sideways}};
\node[neuron] (H-4) at (\layersep,-4.9 cm)
{\begin{sideways}$\hspace{1.0mm}g$\end{sideways}\nodepart{lower}\begin{sideways}$\sum$\end{sideways}};

% Output layer neurons.
\node[neuron,xshift=-7mm,pin={[pin edge={->}]north:\small Output 1},above
of=H-2] (O-1)
{\begin{sideways}$\hspace{1.0mm}h$\end{sideways}\nodepart{lower}\begin{sideways}$\sum$\end{sideways}};

\node[neuron,xshift=-7mm,pin={[pin edge={->}]north:\small Output 2},above
of=H-3] (O-2)
{\begin{sideways}$\hspace{1.0mm}h$\end{sideways}\nodepart{lower}\begin{sideways}$\sum$\end{sideways}};


% Layer annotations.
\node[annot,above of=H-0, node distance=1.2cm] (hl) {\small Hidden Layer};
\node[annot,left of=hl] {\small Input Layer};
\node[annot,right of=hl] {\small Output Layer};


% Input layer - hidden layer connections.
\foreach \source in {0,...,3}
    \foreach \dest in {0,...,4}
        \path[conn] (I-\source) edge (H-\dest);

% Hidden layer - output layer connections.
\foreach \source in {0,...,4}
    \foreach \dest in {1,...,2}
    	\path[conn] (H-\source) edge (O-\dest);

\end{tikzpicture}
\end{scriptsize}
\caption{Multi-layer feed-forward perceptron with bias nodes.}
\label{fig:perceptron}
\end{figure}

% \begin{tikzpicture}
%   \node[rotate=-90,name=s,shape=circle split,draw] (neuron) at (2,5)
%   {\begin{sideways}$~g$\end{sideways}\nodepart{lower}\begin{sideways}$\sum$\end{sideways}};
% \end{tikzpicture}

% \begin{tikzpicture}
%   \matrix (network)
%     [matrix of nodes,%
%      nodes in empty cells,
%      nodes={outer sep=0pt,circle,minimum size=4pt,draw},
%      column sep={1cm,between origins},
%      row sep={1cm,between origins}]
%   {
%                   &                 &                & \\
%                   &                 &                & \\
%     |[draw=none]| & |[xshift=1mm]| & |[xshift=-1mm]|   \\
%   };
%   \foreach \a in {1,...,4}{
%     \draw (network-3-2) -- (network-2-\a);
%     \draw (network-3-3) -- (network-2-\a);
%     \draw [-stealth] ([yshift=5mm]network-1-\a.north) -- (network-1-\a);
%     \foreach \b in {1,...,4}
%       \draw (network-1-\a) -- (network-2-\b);
%   }
%   \draw [stealth-] ([yshift=-5mm]network-3-2.south) -- (network-3-2);
%   \draw [stealth-] ([yshift=-5mm]network-3-3.south) -- (network-3-3);
% \end{tikzpicture}
