\begin{figure}
  \label{fig:cls_experiment}
  \centering
  \begin{tikzpicture}[thick]

    \node (Agent) [final, rectangle split, rectangle split parts=3,
    		       text width=5.8cm] {
      \textbf{LearningAgent}
        \nodepart[text justified]{second}%
        name: string\newline
        history: ReinforcementDataSet\newline
        module: Module\newline
        learner: RLLearner
		\nodepart[text justified]{third}%
		integrateObservation(obs)\newline
		getAction()\newline
		giveReward(r)\newline
		learn(episodes=1)
    };

    \node (AuxNode01) [text width=4mm,text height=6cm,right=of Agent] {};


    \node (Task) [final, rectangle split, rectangle split parts=3,
    		      text width=4.5cm,right=of AuxNode01] {
      \textbf{ProfitTask}
        \nodepart[text justified]{second}%
        name: string\newline
        sensor\_limits: list = [ ]\newline
        actor\_limits: list = [ ]
		\nodepart[text justified]{third}%
		getObservation()\newline
		performAction(action)\newline
		getReward()
    };


    \node (Expr) [final, rectangle split, rectangle split parts=3,
    		      text width=5cm,above=of AuxNode01] {
      \textbf{MarketExperiment}
        \nodepart[text justified]{second}%
        market: SmartMarket
		\nodepart[text justified]{third}%
		doInteractions(number=1)
    }
    edge [->] node[near end,right,text width=6mm] {agents 1..*} (Agent)
    edge [->] node[near end,right,text width=6mm] {tasks 1..*} (Task);

  \end{tikzpicture}
  \caption{Class diagram for Pyreto market experiment.}
\end{figure}
