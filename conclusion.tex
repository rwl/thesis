\chapter{Further Work}
\label{sec:furtherwork}

\section{UK Transmission System}
Some of the more ambitious agent-based electricity market simulations have used
stylised models of national transmission systems
\cite{cincotti:09,weidlich:06}.  This work has often been motivated by recent
or expected changes to the arrangements in the associated regions.  The drop in
oil prices around the time of the global economic crash in 2008 was not
refledted quickly in energy prices and this amplified concerns over liquiduty
levels in the UK electricity markets.  Ofgem found competition to be
sufficient[ref], but the concerns persists and the market arrangements have
been re-examined under Project Discovery[ref].

Several of the the UK's largest power stations are due to be decommissioned
around 2015 in accordance with EU regulations[ref]  The ability of the market
to sufficiently incentivise new investment in generation that will cover the
resulting shortfall is in question.  The concern extends to the need for
long-term investment in new nuclear power plant that is deemed neccessary for
the UK to meet the legally binding obligations, made in the Climate Change
Bill, to cut greenhouse gas emissions by 80\% by 2050, compared to 1990 levels.
Calls have been made for a radical overhaul of the exising arrangments and for
a more interventionist strategy from the government.

Ofgem's project discovery makes many large assumptions in its model of the UK
energy industry.  Future examinations could be enhanced by the advanced
participant behavioural models and accurate electric power system simulations
presented in this thesis.  To this end the transmission systems data
for winter peak load provided by National Grid in [ref] has been converted
into the PSS/E version 30 data file in Appendix \ref{sec:nget_pf}.  Generator
set-points were determined using the state estimator from Pylon [Crow].  The
power flow results are shown in Figure X and correlate accurately with those in
[SYS], having a mean variance of X.XX.  Cost data for each of the aggregated generating units has
been estimated from public sources and stored in the PSS/E version 30 optimal
power flow data file in Appendix \ref{sec:nget_opf}.  Execution times for
solving this case using DC, AC and unit de-commitment optimal power flow are
shown in Table X.  The problem is currently too computationally expensive to be
solved repeated in an agent-based simulation, but the profiling output in
Table~Y shows that much of the execution time is spent constructing matrices in
linked list format.  Significant improvements in speed should be possible
through more efficient construction of the Hessian matrices in the AC optimal
power flow solver.  Agent-based simulation lends itself to parallelisation and
the artificial neural networks could be processed in multiple threads on
multi-core processors or on distributed memory architectures.

\section{Security Constrained Optimal Power Flow}
To the best of the author's knowledge this is the first application of security
constrained optimal power flow in agent-based electricity market simulation.
AC optimal power flow is more difficult to implement and more computationally
expensive to solve due to the non-linear sets of constraints involved.  The
additional complexity does not always add sufficient value.  For locational
marginal price calculation, linearised DC forumulations have been found to
provide suitable results under most circumstances \cite{overbye:acdc}.
However, the option to use an AC formulation offers some interesting
possibilities for further work.

The inclusion of the cost associated with producing (or absorbing) reactive
power in the objective function of an AC optimal power flow problem means that
a parallel auction for voltage support may be run.  This could be open to
agents associated with reactive compensation equipment such as that commonly
needed for wind farm developments.  Reactive power markets have traditionally
been purely of academic interest, but as the UK becomes more dependant upon
offshore wind power the topic could become of increasing interest.

Bus voltage magnitudes are not all assumed to be 1~per-unit in AC optimal power
flow problems, but are part of the optimisation variable.  Generator reactive
power and bus voltage constraint limits are usually determined by system
security constraint requirements.  The policy gradient methods used in this
thesis can operate with large state spaces that may include information on
these constraints.  The additional constraints that are part of an AC
formulation represent further opportunities for agents to exploit system
conditions and receive greater reward.  However, bus voltages are typically
regulated by tap changing transformers and this is a difficult feature to
implement.  Tap positions are typically restricted to discrete intervals,
making the optimisation a mixed integer problem and thus much more difficult
to solve.  A formulation extended to includ automatic tap changing might also
implement variable phase shifting transformers.  These offer a degree of
control in directing power flows and the formulation would allow advanced
constraint management schemes to be researched.

\section{Multi-Market Simulation}
Policy gradient method's superior use of sensory data and their ability to
operate in large action domains opens opportunities for more detailed study of
inter-market relationships.  The global economy is a holistic system of
systems and the anaylsis of markets independently must be of limited value.
Recent agent-based electricity market simulations studies have investigated the
interaction between electricity, gas and emission allowances markets
\cite{krause:gas,wang:09}.  Non-linear models [ref] have been published for gas
flows in pipelines such as the UK gas network in Figure X and could be used in a
similar way to the UK transmission system model.  As in \citeA{krause:gas},
actions in the gas market would constrain the generators options to sell power
in subsequent electricity auctions.  In the same way that agents using policy
gradient learning methods can better exploit conditions in electricity markets,
these methods could be used to learn complex strategies for buying and selling
allowances while avioding penalties for exceeding quotas.


% \subsection{Common Information Model}
% Many tools exist for steady-state analysis of balanced three-phase AC networks
% and most are centred around bespoke models that describe the power system
% data.  Several attempts have been made in the past to standardise the format
% in which power system data is stored [CDF, UKGDS, ODF] and latest and most
% popular is the Common Information Model.
%
% The Common Information Model (CIM) is an abstract ontological model that
% describes the elements of national electric power systems and the associations
% between them.  CIM is an evolving international standard approved by the
% International Electrotechnical Commission (IEC).
%
% Unlike many tool specific models the CIM does not simplify the power system
% into a graph of buses connected by branches.  Instead it describes each of the
% components in the system and the electrical connectivity between them.
% Conventional numerical techniques for steady-state analysis of AC power
% systems require a simplified bus-branch model such that when the voltage angle
% and magnitude at each bus is determined the power flows on each branch may be
% calculated.

% market power, constraint management
%\section{Decentralised Trade}
% distribution level, renewables
%\section{Standarisation}
% CIM for markets
%\section{Blackbox optimisation}
% periodic

\chapter{Summary Conclusions}
\label{ch:conclusion}

% restate contributions