\chapter{Further Work}
\label{sec:furtherwork}
% \subsection{Common Information Model}
% Many tools exist for steady-state analysis of balanced three-phase AC networks
% and most are centred around bespoke models that describe the power system
% data.  Several attempts have been made in the past to standardise the format
% in which power system data is stored [CDF, UKGDS, ODF] and latest and most
% popular is the Common Information Model.
%
% The Common Information Model (CIM) is an abstract ontological model that
% describes the elements of national electric power systems and the associations
% between them.  CIM is an evolving international standard approved by the
% International Electrotechnical Commission (IEC).
%
% Unlike many tool specific models the CIM does not simplify the power system
% into a graph of buses connected by branches.  Instead it describes each of the
% components in the system and the electrical connectivity between them.
% Conventional numerical techniques for steady-state analysis of AC power
% systems require a simplified bus-branch model such that when the voltage angle
% and magnitude at each bus is determined the power flows on each branch may be
% calculated.

\section{AC Optimal Power Flow}
% market power, constraint management
\section{Decentralised Trade}
% distribution level, renewables
\section{Standarisation}
% CIM for markets
\section{Blackbox optimisation}
% periodic

\chapter{Summary Conclusions}
\label{ch:conclusion}

% restate contributions