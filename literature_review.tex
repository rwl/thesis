\section{Literature Review}
Game theoretic models are commonly associated with economics and attempt to
capture behaviour in strategic situations mathematically.  They have been
applied to electric energy problems of many forms, including but not limited
to analysis of market structure, market liquidity, pricing methodologies,
regulatory structure, plant positioning and network congestion.  More
recently, agent-based simulation has received a certain degree of attention
from researchers and has been applied in some of these fields also.

While popular and seemingly promising, agent-based simulation is still centred
around abstracted models.  The assumptions made is this abstraction must be
subjected to the same verification and validation as with equation-based
models.  Verification of assumptions and model validation are often overlooked
in agent-based simulations of energy markets, yet they are possibly the most
important steps in the model building process.  Techniques used to develop,
debug and maintain large computer programs can often be used to verify that a
model does what it is intended to do.

Validation of an energy market model is more difficult.  It can be accomplished
using the intuition of experts or through comparison of simulation results
with either historical market data or theoretical results from more abstract
representations of the model.  Finding verifyable trends in existing markets
is a very large challenge.  To then prove that a computational model
replicates these characteristics with suitable fidelity is yet more
challenging still.  Only when a model is suitably verified and validated can
any conclusions be drawn from results obtained through implementation and
simulation of suitable scenarios.