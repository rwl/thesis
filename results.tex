\chapter{Learning to Trade Power}
\label{ch:learningtotrade}
To the best of the author's knowledge, this thesis presents the first case of
policy gradient reinforcement learning methods being applied to electricity
trading problems.  It must first be proven that these methods are capable of
learning a basic power trading policy.  This section describes the method used
to compare methods in their ability to do so.

\section{Aims \& Objectives} The purpose of this first experiment is to
compare the relative abilities of value-function and policy gradient methods in
learning a basic policy for trading power.  The objective of the exercise is to
examine:
\begin{itemize}
  \item Speed of convergence to an optimal policy,
  \item Magnitude and variance of profit and,
  \item Sensitivity to algorithm parameter changes.
\end{itemize}

\section{Method of Simulation}
Each learning method is tested individually using a range of parameter
configurations.  A power system model with one bus, one generator $k$ and
one dispatchable load $l$, as illustrated in Figure X is used.  In this
context, the market clearing process is equivalent to creating offer and bids
stacks and finding the point of intersection.  A passive agent is associated
with the dispatchable load.  This agent bids for $-p_{g,l}^{min}$ at marginal
cost each period regardless of environment state or reward signal.  A
dispathcable load is used instead of a constant load to allow a price to be
set. Generator $k$ is given sufficient capacity to supply the demand
of the dispatchable load, $p_{g,k}^{max} > -p_{g,l}^{min}$, and the marginal of
the $k$ is half that of the load $l$.  The generator and dispatchable load
attributes are given in Table X.  A price cap for the market is set to twice the
marginal cost of the $l$ at full capacity, $p_{g,l}^{min}$.  The DC optimal
power flow formulation (See Section \ref{sec:opf}, above) is used to clear the
market and reactive power trade is omitted.  The Python code used to conduct the
simulations is provided in Listing X.

\section{Results}
\section{Discussion}
\section{Critical Analysis}


\chapter{Competitive Power Trade}
Having compared the learning methods in a one-player context, this section
describes the method used to pit them against one and other and compare their
performance.

\section{Aims \& Objectives}
Competition is fundamental to markets and this experiment aims to compare
learning methods in a complex dynamic market environment with multiple
competing participants.  The objective is to compare:
\begin{itemize}
  \item Performance, in terms of profitability, over a finite number of
  periods,
  \item Profitability when trading both active and reactive power.
  \item Consistency of profit making and,
  \item Sensitivity to algorithm parameter changes.
\end{itemize}

%\begin{figure}
\label{fig:case6ww}
\centering
\begin{scriptsize}
\begin{tikzpicture}[thick]

\coordinate (c1) at (0,4);
\coordinate (c2) at (5,6.5);
\coordinate (c3) at (10,4);
\coordinate (c4) at (0,0);
\coordinate (c5) at (5,-2.5);
\coordinate (c6) at (10,0);
\coordinate (over) at (0,3.75);

\busbar{b1}{c1}{20mm}
\busbar{b2}{c2}{30mm}
\busbar{b3}{c3}{20mm}
\busbar{b4}{c4}{20mm}
\busbar{b5}{c5}{30mm}
\busbar{b6}{c6}{20mm}

% Branch 1-2.
\draw[line] ([xshift=5mm] b1.north) node[rotate=90,above right]{$15.41$}
node[rotate=90,below right]{$-9.58$} -- ++(0,3.75) -| ([xshift=-8mm] b2.north)
node[rotate=90,above right]{$-15.14$} node[rotate=90,below right]{$5.70$};
% Branch 1-4.
\draw[line] ([xshift=-5mm] b1.south) node[rotate=90,above left]{$33.95$}
node[rotate=90,below left]{$22.50$} -- ([xshift=-5mm] b4.north)
node[rotate=90,above right]{$-33.15$} node[rotate=90,below right]{$-23.46$};
% Branch 1-5.
\draw[line] ([xshift=5mm] b1.south) node[rotate=90,above left]{$27.86$}
node[rotate=90,below left]{$12.80$} -- ([xshift=5mm,yshift=-15mm] b1.south)
-- ([xshift=-8mm,yshift=15mm] b5.north) -- ([xshift=-8mm] b5.north)
node[rotate=90,above right]{$-27.11$} node[rotate=90,below right]{$-16.20$};
% Branch 2-3.
\draw[line] ([xshift=8mm] b2.north) node[rotate=90,above right]{$0.29$}
node[rotate=90,below right]{$-11.76$} -- ++(0,1.25) -| ([xshift=-5mm] b3.north)
node[rotate=90,above right]{$-0.25$} node[rotate=90,below right]{$5.18$};
% Branch 2-4.
\draw[line,ultra thick] ([xshift=-8mm] b2.south) node[rotate=90,above
left]{$41.74$} node[rotate=90,below left]{$43.11$} --
([xshift=-8mm,yshift=-15mm] b2.south) -- node[sloped,above]{$\vert S_{max}^5
\vert = 60.0$} ([xshift=5mm,yshift=15mm] b4.north) -- ([xshift=5mm] b4.north)
node[rotate=90,above right]{$-40.06$} node[rotate=90,below right]{$-41.83$};
% Branch 2-5.
\draw[line] (b2.south) node[rotate=90,above left]{$17.35$}
node[rotate=90,below left]{$14.93$} -- (b5.north) node[rotate=90,above
right]{$-16.81$} node[rotate=90,below right]{$-17.46$};
% Branch 2-6.
\draw[line] ([xshift=8mm] b2.south) node[rotate=90,above left]{$25.03$}
node[rotate=90,below left]{$12.67$} -- ([xshift=8mm,yshift=-15mm] b2.south)
-- ([xshift=-5mm,yshift=15mm] b6.north) -- ([xshift=-5mm] b6.north)
node[rotate=90,above right]{$-24.49$} node[rotate=90,below right]{$-16.38$};
% Branch 3-5.
\draw[line] ([xshift=-5mm] b3.south) node[rotate=90,above left]{$23.18$}
node[rotate=90,below left]{$21.57$} -- ([xshift=-5mm,yshift=-15mm] b3.south)
-- ([xshift=8mm,yshift=15mm] b5.north) -- ([xshift=8mm] b5.north)
node[rotate=90,above right]{$-21.99$} node[rotate=90,below right]{$-24.28$};
% Branch 3-6.
\draw[line] ([xshift=5mm] b3.south) node[rotate=90,above left]{$47.50$}
node[rotate=90,below left]{$59.90$} -- ([xshift=5mm] b6.north)
node[rotate=90,above right]{$-46.45$} node[rotate=90,below right]{$-56.82$};
% Branch 4-5.
\draw[line] ([xshift=5mm] b4.south) node[rotate=90,above left]{$3.21$}
node[rotate=90,below left]{$-4.71$} -- ++(0,-3.75) -| ([xshift=-8mm] b5.south)
node[rotate=90,above left]{$-3.19$} node[rotate=90,below left]{$-3.03$};
% Branch 5-6.
\draw[line] ([xshift=8mm] b5.south) node[rotate=90,above left]{$-0.90$}
node[rotate=90,below left]{$-9.03$} -- ++(0,-1.25) -| ([xshift=-5mm] b6.south)
node[rotate=90,above left]{$0.94$} node[rotate=90,below left]{$3.21$};

% Generator 1.
\genset{g1}{$(c1)+(-5mm,15mm)$}
\draw[line] ([xshift=-5mm] b1.north) -- node[sloped,above]{$77.22$}
node[sloped,below]{$25.72$} (g1.south);
% Generator 2.
\genset{g2}{$(c2)+(0,15mm)$}
\draw[line] (b2.north) -- node[sloped,above]{$69.27$}
node[sloped,below]{$64.65$} (g2.south);
% Generator 3.
\genset{g3}{$(c3)+(5mm,15mm)$}
\draw[line] ([xshift=5mm] b3.north) -- node[sloped,above]{$70.42$}
node[sloped,below]{$86.64$} (g3.south);

% Load 1.
\loadd{l1}{$(c4)-(5mm,15mm)$}
\draw[line] (l1.south) -- node[sloped,above]{$70.00$}
node[sloped,below]{$70.00$} ([xshift=-5mm] b4.south);
% Load 2.
\loadd{l2}{$(c5)-(0mm,15mm)$}
\draw[line] (l2.south) -- node[sloped,above]{$70.00$}
node[sloped,below]{$70.00$} (b5.south);
% Load 3.
\loadd{l3}{$(c6)+(5mm,-15mm)$}
\draw[line] (l3.south) -- node[sloped,above]{$70.00$}
node[sloped,below]{$70.00$} ([xshift=5mm] b6.south);

\end{tikzpicture}
\end{scriptsize}
\caption{One line diagram for six bus power system model from [].}
\end{figure}


\section{Method of Simulation}
Figure X illustrates the structure of the six bus power system model, from
\cite{wood:pgoc}, with three generators and fixed demand at three of the buses
used to provide a dynamic environment with typical system values.  Bus,
branch and generator attribute values are stated in Tables X, Y, Z,
respectively.  Three learning methods are compared in six simulations
encapsulating all method--generator combinations.

A price cap $c_{cap}$ of twice the marginal cost of the most expensive generator
at full capacity is set by the market.  The simulations are repeated for with agents
actions composing both price and quantity and with just price.  For the
value-function methods, the state is defined by the market clearing price from
the previous period, divided equally into $x_s$ discrete states between $0$ and
$c_{cap}$.  The state vector $s_t$ for the policy gradient methods consists of
the market clearing price and generator set-point from the previous period.
\begin{equation}
s_t =
\begin{bmatrix}
c_{mcp}\\
p_g
\end{bmatrix}
\end{equation}
The script used to conduct the simulation is provided in Listing X.

\section{Results}
\section{Discussion}
\section{Critical Analysis}

\chapter{System Constraint Exploitation}
\label{ch:results}
One of the main features of agents using policy gradient learning methods and
artifical neural networks for policy function approximation is their ability to
accept many signals of continuous sensor data.  This section describes an
experiment in which the power system is severly constrained for certain
periods, resulting in elevated nodal marginal prices in particular areas.  The
methods are tested in their ability to exploit these constraints and improve
their total accumulated reward.

\section{Aims \& Objectives}

% \section{Experiment \Rmnum{1}: Basic Learning}
% \section{Experiment \Rmnum{2}: Competitive Trade}
% \section{Experiment \Rmnum{3}: Constraint Exploitation}

%\begin{figure}
\label{fig:ieee30}
\centering
\begin{tikzpicture}[thick]

\coordinate (c1) at (0, 16);
\coordinate (c2) at (2, 16);
\coordinate (c3) at (0, 13);
\coordinate (c4) at (2, 13);
\coordinate (c5) at (4, 16);
\coordinate (c6) at (5, 13);
\coordinate (c7) at (6, 16);
\coordinate (c8) at (7.5, 13);
\coordinate (c9) at (7, 11.5);
\coordinate (c10) at (7, 10);
\coordinate (c11) at (9.5, 12);
\coordinate (c12) at (1.5, 10);
\coordinate (c13) at (-1, 10);
\coordinate (c14) at (-1, 6);
\coordinate (c15) at (0, 5);
\coordinate (c16) at (4, 10);
\coordinate (c17) at (5, 9);
\coordinate (c18) at (3, 7.5);
\coordinate (c19) at (4, 5);
\coordinate (c20) at (5, 7);
\coordinate (c21) at (8, 8);
\coordinate (c22) at (7, 7);
\coordinate (c23) at (1, 2);
\coordinate (c24) at (5, 2);
\coordinate (c25) at (7, 2);
\coordinate (c26) at (8, 0);
\coordinate (c27) at (10, 8);
\coordinate (c28) at (10, 15);
\coordinate (c29) at (9, 1);
\coordinate (c30) at (10, 3);

\busbar{b1}{c1}{10mm}
\busbar{b2}{c2}{20mm}
\busbar{b3}{c3}{15mm}
\busbar{b4}{c4}{15mm}
\busbar{b5}{c5}{15mm}
\busbar{b6}{c6}{25mm}
\busbar{b7}{c7}{15mm}
\busbar{b8}{c8}{15mm}
\busbar{b9}{c9}{15mm}
\busbar{b10}{c10}{25mm}
\busbar{b11}{c11}{15mm}
\busbar{b12}{c12}{25mm}
\busbar{b13}{c13}{15mm}
\busbar{b14}{c14}{15mm}
\busbar{b15}{c15}{20mm}
\busbar{b16}{c16}{15mm}
\busbar{b17}{c17}{15mm}
\busbar{b18}{c18}{15mm}
\busbar{b19}{c19}{15mm}
\busbar{b20}{c20}{15mm}
\busbar{b21}{c21}{10mm}
\busbar{b22}{c22}{15mm}
\busbar{b23}{c23}{15mm}
\busbar{b24}{c24}{15mm}
\busbar{b25}{c25}{15mm}
\busbar{b26}{c26}{15mm}
\busbar{b27}{c27}{20mm}
\busbar{b28}{c28}{15mm}
\busbar{b29}{c29}{15mm}
\busbar{b30}{c30}{15mm}

% Branch 1-2.
\draw[line] ([xshift=2mm] b1.south) -- ++(0,-5mm) -| ([xshift=-8mm] b2.south);
% Branch 1-3.
\draw[line] ([xshift=-2mm] b1.south) -- ([xshift=-2mm] b3.north);
% Branch 2-4.
\draw[line] ([xshift=0mm] b2.south) -- (b4.north);
% Branch 3-4.
\draw[line] ([xshift=2mm] b3.north) -- ++(0,5mm) -| ([xshift=-5mm] b4.north);
% Branch 2-5.
\draw[line] ([xshift=8mm] b2.south) -- ++(0,-5mm) -| ([xshift=-5mm] b5.south);
% Branch 2-6.
\draw[line] ([xshift=5mm] b2.south) -- ++(0,-14mm) -| ([xshift=-5mm] b6.north);
% Branch 4-6.
\draw[line] ([xshift=5mm] b4.north) -- ++(0,5mm) -| ([xshift=-10mm] b6.north);
% Branch 5-7.
\draw[line] ([xshift=5mm] b5.south) -- ++(0,-5mm) -| ([xshift=-5mm] b7.south);
% Branch 6-7.
\draw[line] ([xshift=0mm] b6.north) -- ++(0,15mm) -| ([xshift=0mm] b7.south);
% Branch 6-8.
\draw[line] ([xshift=8mm] b6.south) -- ++(0,-5mm) -| ([xshift=-4mm] b8.south);
% Branch 6-9.
\draw[line] (b6.south) -- ++(0,-10mm) -| (b9.north);
% Branch 6-10.
\draw[line] ([xshift=-5mm] b6.south) -- ++(0,-20mm) -| ([xshift=-5mm]
b10.north);
% Branch 9-11.
\draw[line] ([xshift=2mm] b9.south) -- ++(0,-5mm) -| (b11);
% Branch 9-10.
\draw[line] (b9.south) -- (b10.north);
% Branch 4-12.
\draw[line] ([xshift=-2mm] b4.south) -- ([xshift=3mm] b12.north);
% Branch 12-13.
\draw[line] ([xshift=-10mm] b12.south) -- ++(0,-5mm) -| (b13.south);
% Branch 12-14.
\draw[line] ([xshift=-5mm] b12.south) -- ++(0,-20mm) -| (b14.north);
% Branch 12-15.
\draw[line] (b12.south) -- ++(0,-30mm) -| (b15.north);
% Branch 12-16.
\draw[line] ([xshift=10mm] b12.north) -- ++(0,5mm) -| ([xshift=-2mm]
b16.north);
% Branch 14-15.
\draw[line] ([xshift=2mm] b14.south) -- ++(0,-5mm) -| ([xshift=-2mm] b15.north);
% Branch 16-17.
\draw[line] ([xshift=5mm] b16.south) -- ([xshift=-5mm] b17.north);
% Branch 15-18.
\draw[line] ([xshift=5mm] b15.north) -- ++(0,10mm) -| ([xshift=-5mm] b18.south);
% Branch 18-19.
\draw[line] ([xshift=5mm] b18.south) -- ([xshift=-5mm] b19.north);
% Branch 19-20. TODO: Check!
% Branch 10-20.
\draw[line] (b10.south) -- ++(0,-15mm) -| ([xshift=2mm] b20.north);
% Branch 10-17.
\draw[line] ([xshift=-5mm] b10.south) -- ++(0,-5mm) -| ([xshift=2mm] b17.north);
% Branch 10-21.
\draw[line] ([xshift=8mm] b10.south) -- ([xshift=-2mm] b21.north);
% Branch 10-22.
\draw[line] ([xshift=4mm] b10.south) -- ++(0,-18mm) -| ([xshift=-2mm]
b22.north);
% Branch 21-22.
\draw[line] (b21.south) -- ++(0,-5mm) -| ([xshift=5mm] b22.north);
% Branch 15-23.
\draw[line] ([xshift=5mm] b15.south) -- ([xshift=-5mm] b23.north);
% Branch 22-24. TODO: Check!
\draw[line] ([xshift=-2mm] b22.south) -- ++(0,-30mm) -| (b24.north);
% Branch 23-24.
\draw[line] ([xshift=2mm] b23.south) -- ++(0,-5mm) -| ([xshift=-5mm] b24.south);
% Branch 24-25.
\draw[line] ([xshift=5mm] b24.south) -- ++(0,-5mm) -| (b25.south);
% Branch 25-26.
\draw[line] ([xshift=5mm] b25.south) -- ([xshift=-5mm] b26.north);
% Branch 25-27.
\draw[line] ([xshift=2mm] b25.north) -- ++(0,10mm) --
([xshift=-5mm,yshift=-10mm] b27.south) -- ([xshift=-5mm] b27.south);
% Branch 28-27.
\draw[line] ([xshift=5mm] b28.south) -- ([xshift=5mm] b27.north);
% Branch 27-29.
\draw[line] (b27.south) -- ++(0,-35mm) -| ([xshift=-2mm] b29.north);
% Branch 27-30.
\draw[line] ([xshift=4mm] b27.south) -- ([xshift=4mm] b30.north);
% Branch 29-30.
\draw[line] ([xshift=4mm] b29.north) -- ++(0,10mm) -| (b30.south);
% Branch 8-28.
\draw[line] (b8.north) -- ++(0,8mm) -| (b28.south);
% Branch 6-28.
\draw[line] ([xshift=8mm] b6.north) -- ++(0,12mm) -| ([xshift=-5mm] b28.south);

% Generator 1 @ Bus 1.
\sgen{g1}{$(c1)+(0mm,8mm)$}
\draw[line] (b1.north) -- (g1.south);
% Generator 2 @ Bus 2.
\sgen{g2}{$(c2)+(0mm,8mm)$}
\draw[line] (b2.north) -- (g2.south);
% Generator 3 @ Bus 22.
\sgen{g3}{$(c22)+(5mm,-8mm)$}
\draw[line] ([xshift=5mm] b22.south) -- (g3.north);
% Generator 4 @ Bus 27.
\sgen{g4}{$(c27)+(-5mm,8mm)$}
\draw[line] ([xshift=-5mm] b27.north) -- (g4.south);
% Generator 5 @ Bus 23.
\sgen{g5}{$(c23)+(5mm,8mm)$}
\draw[line] ([xshift=5mm] b23.north) -- (g5.south);
% Generator 6 @ Bus 13.
\sgen{g6}{$(c13)+(0mm,8mm)$}
\draw[line] ([xshift=0mm] b13.north) -- (g6.south);

% Load @ Bus 2.
%\sload{l1}{$(c2)-(5mm,8mm)$}
%\draw[line] (l1.south) -- ([xshift=-5mm] b2.south);
\draw[loadline] ([xshift=-5mm] b2.south) -- ++(0,-1);
% Load @ Bus 3.
% \sload{l2}{$(c3)-(0mm,8mm)$}
% \draw[line] (l2.south) -- (b3.south);
\draw[loadline] (b3.south) -- ++(0,-1);
% Load @ Bus 4.
% \sload{l3}{$(c4)-(-5mm,8mm)$}
% \draw[line] (l3.south) -- ([xshift=5mm] b4.south);
\draw[loadline] ([xshift=5mm] b4.south) -- ++(0,-1);
% Load @ Bus 7.
% \sload{l4}{$(c7)-(-5mm,8mm)$}
% \draw[line] (l4.south) -- ([xshift=5mm] b7.south);
\draw[loadline] ([xshift=5mm] b7.south) -- ++(0,-1);
% Load @ Bus 8.
% \sload{l5}{$(c8)-(-5mm,8mm)$}
% \draw[line] (l5.south) -- ([xshift=5mm] b8.south);
\draw[loadline] ([xshift=5mm] b8.south) -- ++(0,-1);
% Load @ Bus 10.
% \sload{l6}{$(c10)-(-10mm,8mm)$}
% \draw[line] (l6.south) -- ([xshift=10mm] b10.south);
\draw[loadline] ([xshift=10mm] b10.south) -- ++(0,-1);
% Load @ Bus 12.
% \sload{l7}{$(c12)-(-5mm,8mm)$}
% \draw[line] (l7.south) -- ([xshift=5mm] b12.south);
\draw[loadline] ([xshift=5mm] b12.south) -- ++(0,-1);
% Load @ Bus 14.
% \sload{l8}{$(c14)-(5mm,8mm)$}
% \draw[line] (l8.south) -- ([xshift=-5mm] b14.south);
\draw[loadline] ([xshift=-5mm] b14.south) -- ++(0,-1);
% Load @ Bus 15.
% \sload{l9}{$(c15)-(5mm,8mm)$}
% \draw[line] (l9.south) -- ([xshift=-5mm] b15.south);
\draw[loadline] ([xshift=-5mm] b15.south) -- ++(0,-1);
% Load @ Bus 16.
% \sload{l10}{$(c16)-(5mm,8mm)$}
% \draw[line] (l10.south) -- ([xshift=-5mm] b16.south);
\draw[loadline] ([xshift=-5mm] b16.south) -- ++(0,-1);
% Load @ Bus 17.
% \sload{l11}{$(c17)-(5mm,8mm)$}
% \draw[line] (l11.south) -- ([xshift=-5mm] b17.south);
\draw[loadline] ([xshift=-5mm] b17.south) -- ++(0,-1);
% Load @ Bus 18.
% \sload{l12}{$(c18)-(0mm,8mm)$}
% \draw[line] (l12.south) -- ([xshift=0mm] b18.south);
\draw[loadline] ([xshift=0mm] b18.south) -- ++(0,-1);
% Load @ Bus 19.
% \sload{l13}{$(c19)-(-0mm,8mm)$}
% \draw[line] (l13.south) -- (b19.south);
\draw[loadline] (b19.south) -- ++(0,-1);
% Load @ Bus 20.
% \sload{l14}{$(c20)-(0mm,8mm)$}
% \draw[line] (l14.south) -- ([xshift=0mm] b20.south);
\draw[loadline] ([xshift=0mm] b20.south) -- ++(0,-1);
% Load @ Bus 21.
% \sload{l15}{$(c21)-(-2mm,8mm)$}
% \draw[line] (l15.south) -- ([xshift=2mm] b21.south);
\draw[loadline] ([xshift=2mm] b21.south) -- ++(0,-1);
% Load @ Bus 23.
% \sload{l16}{$(c23)-(2mm,8mm)$}
% \draw[line] (l16.south) -- ([xshift=-2mm] b23.south);
\draw[loadline] ([xshift=-2mm] b23.south) -- ++(0,-1);
% Load @ Bus 24.
% \sload{l17}{$(c24)-(0mm,8mm)$}
% \draw[line] (l17.south) -- ([xshift=0mm] b24.south);
\draw[loadline] ([xshift=0mm] b24.south) -- ++(0,-1);
% Load @ Bus 26.
% \sload{l18}{$(c26)-(0mm,8mm)$}
% \draw[line] (l18.south) -- ([xshift=0mm] b26.south);
\draw[loadline] ([xshift=0mm] b26.south) -- ++(0,-1);
% Load @ Bus 29.
% \sload{l19}{$(c29)-(-4mm,8mm)$}
% \draw[line] (l19.south) -- ([xshift=4mm] b29.south);
\draw[loadline] ([xshift=4mm] b29.south) -- ++(0,-1);
% Load @ Bus 30.
% \sload{l20}{$(c30)-(-4mm,8mm)$}
% \draw[line] (l20.south) -- ([xshift=4mm] b30.south);
\draw[loadline] ([xshift=4mm] b30.south) -- ++(0,-1);

\end{tikzpicture}
\caption{One line diagram for IEEE 30 bus test system.}
\end{figure}


\section{Results}
\section{Discussion}
\label{sec:discuss}
\section{Critical Analysis}
