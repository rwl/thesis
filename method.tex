\chapter{Modelling Power Trade}
\label{ch:method}
The present chapter defines the model used in this thesis to simulate electric
power trade. The first section describes how optimal power flow solutions are
used to clear offers and bids submitted to a simulated power exchange auction.
The second section defines how market participants are modelled as agents that
use reinforcement learning algorithms to adjust their bidding behaviour. It
explains the modular structure of a multi-agent system that coordinates
interactions between the auction model and market participants.

% Societies reliance on secure energy supplies and the high volumes of
% electricity typically consumed render it impractical to experiment with
% radically new approaches to energy trade on real systems.  This section
% explains the approach taken modelling real systems in software such that they
% may be simulated computationally.  The method by which the physical power
% systems, that deliver electricity to consumers, were modeled is given, as well
% as for the mechanisms that facilitate trade and participants that utilise
% these mechanisms.
%
% \subsection{Energy market model}
% Mechanisms for facilitating competitive trade between electricity producers and
% consumers differ greatly in the specifics of their implementations in coutries
% throughout the world.  However, fundamentally they either provide a
% centralised pool through which all electricity is bought and sold or they
% permit producers and suppliers to trade directly.
%
% The UK transmission network is frequently congested[].  The thermal limits of
% transmission lines between particular areas are often reached.  The balancing
% mechanism is the financial instrument used by the system operator to resolve
% constraint issues and energy imbalances.  Should the market not be suitably
% effective in this function the system operator may choose to contract outwith
% the balancing mechanism.  By way of incentive to match demand and avoid
% congestion, imbalance charges are imposed on responsible participants.  There
% is some evidence to suggest that centralised resolution by a system operator
% and socialisation of the incurred costs leads to inefficient despatch of
% generators[Neuhoff].
%
% There are a number of alternative approaches to congestion
% resolution.
% %\cite{neuhoff:power}
%
% \subsection{Transmission capacity rights}
% One approach is to issue contracts for transmission capacity rights or
% equivalent financial rights.  The maximum available transmission capacity
% being auctioned for certain periods of time and firm contracts made entitling
% owners to full compensation upon curtailment or
% withdrawl.
% %\cite{efet:principles}.
%
% The states of Pensylvania, New Jersey and Maryland (PJM) operate a
% non-compulsory power pool with nodal market-clearing prices based on
% competitive bids.  This is complemented by daily and monthly capacity markets
% plus the monthly auction of Financial Transmission Rights to provide a hedging
% mechanism against future congestion charges.
%
% \subsection{Transmission charging}
% Impose delivery charges which increase as network constraints are approached.
%
%
% \subsection{Extended bids/offers}
% Request extended bids and offers which include costs associated with the
% adjustment of participant's desired position.
%
% \subsection{Software agents}
% Participants are modeled in software also.  The nature of a highly distributed
% power system dictates that a very large number of entities may be interacting
% in the marketplace.  Economic studies regularly integrate participant logic
% into the same optimisation problem as the market.  However, this does not
% scale to large numbers of individual participants.  Separating participant
% logic into individual software agents allows their action selection procedures
% to be processed in simultaneously.  The definition of an agent in this context
% emerges from the machine learning technique employed to implement the
% competitive decision making process.

\section{Electricity Market Model}
A model of a power exchange auction market is used in this thesis to compare
reinforcement learning algorithms.  It accepts offers of and bids for blocks of
power at specified prices.  The auction withholds invalid offers and bids and
determines the cleared quantities and prices.

\subsection{Optimal Power Flow}
Bespoke implementations of the optimal power flow formulations from \matpower
are used in the clearing process \cite[\S5]{pserc:mp_manual}.
Both the DC and AC formulations are used!  The trade-offs between DC and AC
models have been examined by \citeA{overbye:acdc}.  DC models were found suitable for most nodal marginal
price calculations and are considerably less computationally expensive.  The
AC optimal power flow formulation is used in this thesis to examine the
exploitation of voltage constraints that are not part of a DC formulation.
A class diagram in the Unified Modelling Language (UML) for the
object-orientated power system model that is used to compute optimal power
flow solutions is shown in Figure \ref{fig:cls_pylon}.

\ifthenelse{\boolean{includefigures}}{\begin{figure}
  \centering
  \begin{tikzpicture}[node distance=2cm,thick]

%     \node (Case) [final, rectangle split, rectangle split parts=3]{
%       \textbf{Case}
%       \nodepart[text justified]{second}%
%       name\newline
%       baseMVA\hfill{\color{gray} $S_{base}$}
%       \nodepart[text justified]{third}%
%       makeYbus()\newline
%       makeBdc()\newline
%       dSbus\_dV()\newline
%       dIbr\_dV()\newline
%       dSbr\_dV()\newline
%       dAbr\_dV()\newline
%       d2Sbus\_dV2()\newline
%       d2Ibr\_dV2()\newline
%       d2Sbr\_dV2()\newline
%       d2ASbr\_dV2()\newline
%       d2AIbr\_dV2()
%     };

    \node (Bus) at (0,0) [final, rectangle split, rectangle split parts=2,
            text width=5.5cm]{
      \textbf{Bus}
        \nodepart[text justified]{second}%
        name: string~{\color{gray} $i$}\newline
        type: string = PQ\newline
        v\_mag\_guess: float = 1.0~{\color{gray} $V_{m0}$}\newline
        v\_angle\_guess: float = 0.0~{\color{gray} $V_{a0}$}\newline
        v\_max: float = 1.1~{\color{gray} $V_{max}$}\newline
        v\_min: float = 0.9~{\color{gray} $V_{min}$}\newline
        p\_demand: float~{\color{gray} $P_d S_{b}$}\newline
        q\_demand: float~{\color{gray} $Q_d S_{b}$}\newline
        g\_shunt: float~{\color{gray} $G_{sh}$}\newline
        b\_shunt: float~{\color{gray} $B_{sh}$}};

    \node (Branch) [final, rectangle split, rectangle split parts=2,
    				text width=5cm, below=of Bus]{
      \textbf{Branch}
        \nodepart[text justified]{second}%
        name~{\color{gray} $l$}\newline
%        online: bool = true\newline
        r: float = 0.001~{\color{gray} $r_s$}\newline
        x: float = 0.001~{\color{gray} $x_s$}\newline
        b: float = 0.001~{\color{gray} $b_c$}\newline
        s\_max: float = 200~{\color{gray} $S_{max} S_{b}$}\newline
        ratio: float = 1.0~{\color{gray} $\tau$}\newline
        phase\_shaft: float~{\color{gray} $\theta_{ph}$}\newline
        angle\_max: float~{\color{gray} $\theta_{max}$}\newline
        angle\_min: float~{\color{gray} $\theta_{min}$}}

    edge [->] node[near end,left,text width=12mm]
    {to\_bus 1..1} ([xshift=-6mm]Bus.south)
    edge [->] node[near end,right,text width=16mm]
    {from\_bus 1..1} ([xshift=6mm]Bus.south);

    \node (Gen) [final, rectangle split, rectangle split parts=3,
    		     text width=5cm, right=of Bus]{
      \textbf{Generator}
        \nodepart[text justified]{second}%
        name: string~{\color{gray} $k$}\newline
%        online\newline
        v\_magnitude~{\color{gray} $V_m$}\newline
        p: float = 100~{\color{gray} $P_g S_{b}$}\newline
        p\_max: float = 200~{\color{gray} $P_g^{max} S_{b}$}\newline
        p\_min: float~{\color{gray} $P_g^{min} S_{b}$}\newline
        q: float~{\color{gray} $Q_g S_{b}$}\newline
        q\_max: float = 100~{\color{gray} $Q_g^{max} S_{b}$}\newline
        q\_min: float = -100~{\color{gray} $Q_g^{min} S_{b}$}\newline
        p\_cost: list\newline
        q\_cost: list
        \nodepart[text justified]{third}
        is\_load(): bool\newline
        total\_cost(p): float\newline
        offers\_to\_pwl(offers)\newline
        bids\_to\_pwl(bids)}

    edge [->] node[near end,below,text width=6mm] {bus 1..1} (Bus);

    \node (Case) [final, rectangle split, rectangle split parts=3,
    		      text width=4.5cm, right=of Branch] {
      \textbf{Case}
        \nodepart[text justified]{second}%
        name: string\newline
        base\_mva: float = 100~{\color{gray} $S_{b}$}
		\nodepart[text justified]{third}%
		makeYbus()
    }

    edge [->] node[near end,text width=6mm] {buses 0..*} (Bus)
    edge [->] node[near end,above,text width=6mm] {branches 0..*} (Branch)
    edge [->] node[near end,right,text width=6mm] {generators 0..*} (Gen);

  \end{tikzpicture}
  \caption{Class diagram for the power system model.}
  \label{fig:cls_pylon}
\end{figure}
}{}

As in \textsc{Matpower} \cite[p.26]{pserc:mp_manual}, generator active
power, and optionally reactive power, output costs may be defined by convex
$n$-segment piecewise linear cost functions
\begin{equation}
c^{(i)}(x) = m_ip + c_i
\end{equation}
for $p_i \leq p \leq p_{i+1}$ with $i = 1,2,\dotsc n$ where $m_{i+1} \geq m_i$
and $p_{i+1} > p_i$, as diagramed in Figure \ref{fig:ccv_function}
\cite[Figure5-3]{pserc:mp_manual}. Since these costs are non-differentiable,
the constrained cost variable approach from \cite{zimmerman:ccv} is used to
make the optimisation problem smooth.  For each generator $i$ a helper cost
variable $y_i$ added to the vector of optimisation variables.  The additional
inequality constraints
\begin{equation}
y_i \geq m_{i,j}(p-p_j) + c_j, \quad j = 1\dotsc n
\end{equation}
ensure that $y_i$ lies on or above $c^{(i)}(x)$. The
objective of the optimal power flow problem becomes the minimisation of the
sum of cost variables for all generators:
\begin{equation}
\min_{\theta, V_m, P_g, Q_g, y} \sum_{i=1}^{n_g}y_i
\end{equation}

The extensions to the optimal power flow formulations defined in
\textsc{Matpower} for user-defined cost functions and generator P-Q capability
curves are not utilised.

%
% \section{DC OPF Formulation}
% Piecewise linear cost functions are also used to define generator active power
% costs in the DC optimal power flow formulation.  Since the power flow equations
% are linearised, following the assumptions in equations (\ref{eq:lossless}),
% (\ref{eq:oneperunit}) and (\ref{eq:busangdiff}), the optimal power flow
% problem simplifies to a linear program.  The voltage magnitude variables $V_m$
% and generator reactive power set-point variable $Q_g$ are eliminated following
% the assumption in equation (\ref{eq:busangdiff}) since branch reactive power
% flows depend on bus voltage angle differences.  The objective function reduces to
% \begin{equation}
% \min_{\theta, P_g, y} \sum_{i=1}^{n_g}y_i
% \end{equation}
% Combining the nodal real power injections, expressed as a function of $\Theta$,
% from equation (\ref{eq:bbus}), with active power generation $P_g$ and active
% demand $P_d$, the power balance constraint is
% \begin{equation}
% B_{bus}\Theta + P_{bus,ph} + P_d - C_gP_g = 0
% \end{equation}
% Limits on branch active power flows $B_f\Theta$ and $B_t\Theta$ are enforced by
% the inequality constraints
% \begin{eqnarray}
% B_f\Theta + P_{f,ph} - F_{max}& \leq& 0\\
% -B_f\Theta + P_{f,ph} - F_{max}& \leq& 0
% \end{eqnarray}
% The reference bus voltage angle equality constraint from
% equation (\ref{eq:refbusang}) and the $p_g$ limit constraint from
% (\ref{eq:pglim}) are also applied.

\subsection{Unit De-commitment}
\label{sec:decommit}
The optimal power flow formulations constrain generator set-points between
upper and lower power limits.  The output of expensive generators can be
reduced to the lower limit, but they can not be completely shutdown.  The
online status of generators could be incorporated into the vector of
optimisation variables, but as they are Boolean the problems would become
mixed-integer non-linear programs which are typically very difficult to
solve.

To compute a least cost commitment and dispatch the unit de-commitment
algorithm from \citeA[p.57]{pserc:mp_manual} is used.  Algorithm~\ref{alg:ud}
shows how this involves shutting down the most expensive units until the
minimum generation capacity is less than the total load capacity and then
solving repeated optimal power flow problems with candidate generating units,
that are at their minimum active power limit, deactivated.  The lowest cost
solution is returned when no further improvement can be made and no candidate
generators remain.

\begin{algorithm}%[H]
\caption{Unit de-commitment}
\label{alg:ud}
\begin{algorithmic}[1]
%\STATE $\text{initialise}~N \leftarrow 0$
%\STATE $P_{d} \leftarrow \text{total load capacity}$
%\STATE $P_{g}^{min} \leftarrow \text{total min.\ gen.\ capacity}$
\WHILE{$\sum P_{g}^{min} > \sum P_{d}$}
%	\STATE $N \leftarrow N + 1$
	\STATE shutdown most expensive unit
\ENDWHILE

%\STATE $\text{solve initial OPF}$
\STATE $f \leftarrow \text{initial total system cost}$

\REPEAT
	\STATE $c \leftarrow \text{generators at } P_{min}$
	\FOR{$g$ in $c$}
		\STATE $d \leftarrow \text{true}$
		\STATE shutdown $g$
		\STATE $f^\prime \leftarrow \text{new total system cost}$
		\IF{$f^\prime < f$}
			\STATE $f \leftarrow f^\prime$
			\STATE $g_{c} \leftarrow g$
			\STATE $d \leftarrow \text{false}$
		\ENDIF
		\STATE startup $g$
	\ENDFOR
	\STATE shutdown $g_c$
\UNTIL{$d = \text{true}$}
\end{algorithmic}
\end{algorithm}

\subsection{Power Exchange}
To simulate electric power trade a model is used in which agents representing
market participants do not provide cost functions for the generators in their portfolio, but submit
offers to sell and/or bids to buy blocks of active or reactive power.  The
offers/bids are submitted to a power exchange auction market model based on
SmartMarket from \citeA[p.92]{pserc:mp_manual}.

The clearing process begins by withholding
offers/bids outwith maximum offer and minimum bid price limits, along with
those specifying non-positive quantities. Valid offers/bids for each generator
are then sorted into non-decreasing/non-increasing order and are converted
into corresponding generator/dispatchable load capacities and piecewise linear
cost functions. The newly configured units are used in a unit de-commitment
optimal power flow problem, the solution of which holds generator set-points
and nodal marginal prices which are used to determine the proportion of each
offer/bid block that should be cleared and the cleared price for each.

% Pricing may be uniform,
% where each offer/bid is cleared at the price of the marginal unit, or
% discriminatory, where the offer/bid is cleared at the price at which it
% offered/bid (pay-as-bid).

A basic nodal marginal pricing scheme is used in which the price of each
offer/bid is cleared at the value of the Lagrangian multiplier on the power
balance constraint for the bus at which the associated generator is connected.
Alternatively, a discriminatory pricing scheme may be used in which offer/bids
are cleared at the price at which they were submitted (pay-as-bid). Cleared
offers/bids are returned to the agents and used to determine revenue values
from which each agent's earnings or losses are derived.

%\subsection{Auction Example}
% This example demonstrates how a set of offers and a set of bids are cleared
% using the auction mechanism based on SmartMarket by \citeA{pserc:mp_manual}.
% The six bus power system model used in this example is adapted from
% \citeA[pp.~104, 112, 119, 123-124, 549]{wood:pgoc} and a one-line diagram for
% the case is given in Figure \ref{fig:case6ww}.  The model has 3 generators
% with a total capacity of 530MW and the total system load is 210MW.  The
% initial generator costs are defined by quadratic functions of the form $a + bx
% + cx^2$, where $x$ is the generator set-point, with the parameters are given in
% Table \ref{tbl:ex_coeffs}.
% \begin{table}
% \begin{center}
% \begin{tabular}{c|c|c|c}
% \hline
% Generator bus & $a$ & $b$ & $c$ \\
% \hline
% 1 & 215.0 & 10.0 & 0.005 \\
% 2 & 200.0 & 12.0 & 0.008 \\
% 3 & 240.0 & 15.0 & 0.010 \\
% \hline
% \end{tabular}
% \end{center}
% \caption{Generator cost function coefficients.}
% \label{tbl:ex_coeffs}
% \end{table}
%
% Suppose each offers half of its capacity with a markup of 10\% and the
% remainder marked up by 20\%.  This correlates to a set of offers with
% quantities and prices given in Table \ref{tbl:ex_offers}.
% \begin{table}
% \begin{center}
% \begin{tabular}{c|cc|cc}
% \hline
% Generator bus & \multicolumn{2}{c}{Offer 1} & \multicolumn{2}{|c}{Offer 2}\\
%  & MW & \$/MWh & MW & \$/MWh \\
% \hline
% 1 & 100 & 20 & 100 & 30 \\
% 2 & 75  & 25 & 75  & 40 \\
% 3 & 90  & 30 & \sout{90}  & \sout{50} \\
% \hline
% \end{tabular}
% \end{center}
% \caption{Offered quantities and prices.}
% \label{tbl:ex_offers}
% \end{table}
% Setting a price
% cap of \$45 causes the second offer from the generator at bus 3 to be withheld
% and ignored in the conversion to piecewise linear cost functions.
%
% Table
% \ref{tbl:ex_pwl} lists the points of the resulting piecewise linear cost
% functions and Figure X plots the original marginal cost function and the cost
% function corresponding to the submitted offers for the generator at bus 1.
% \begin{table}
% \begin{center}
% \begin{tabular}{c|c|c|c}
% \hline
% Generator bus & $(P_g, C)$ & $(P_g, C)$ & $(P_g, C)$ \\
% \hline
% 1 & (0, 215) & (100, 10.0) & (200, 0.005) \\
% 2 & (0, 200) & (75, 12.0) & (150, 0.008) \\
% 3 & (0, 240) & (90, 15.0) & (180, 0.010) \\
% \hline
% \end{tabular}
% \end{center}
% \caption{Piecewise linear cost function points.}
% \label{tbl:ex_pwl}
% \end{table}
% Also plotted is the generator set-point from the optimal power flow
% solution and the elevation of the nodal marginal price caused by network
% congestion and branch losses.  The diagram indicates the difference between the
% original marginal cost function and the cleared price that is the earnings from
% that generator and would be used as part of the reward for the responsible
% agent.

\newpage
\section{Multi-Agent System}
\label{sec:mas}
Market participants are modelled with software agents from PyBrain that use
reinforcement learning algorithms to adjust their behaviour \cite{schaul:2010}.
Their interaction with the market is coordinated in multi-agent experiments,
the structure of which is derived from PyBrain's single player design.

% In PyBrain, agents do not interact directly with their environment, but are
% associated with a particular \textit{task}.
This section describes the environment of each agent, their tasks and the
modules used for policy function approximation and storing state-action values
in tables. The process by which each agent's policy is updated by a learning
algorithm is explained and the sequence of interactions between multiple
agents and the market is described and illustrated.

\subsection{Environment}
In each experiment, agents are endowed with a portfolio of generators from the
electric power system model (See Figure \ref{fig:cls_pylon}).  As
illustrated by the UML class diagram in Figure \ref{fig:cls_pyreto},
the generators are contained within each agent's \textit{environment}.  The
environment also holds an association to an instance of the auction market
that allows the submission of offers/bids. Each environment is responsible for \begin{inparaenum}[(i)]
\item returning a vector representation of its current state and \item
accepting an action vector which transforms the environment into a new state.
\end{inparaenum}  To facilitate testing of value function based and policy
gradient learning methods, both discrete and continuous representations of an
electric power trading environment are defined.

\ifthenelse{\boolean{includefigures}}{\begin{figure}
  \label{fig:cls_pyreto}
  \centering
  \begin{tikzpicture}[node distance=2cm,thick]

    \node (Mkt) [final, rectangle split, rectangle split parts=3,
    		      		 text width=6.2cm] {
      \textbf{SmartMarket}
        \nodepart[text justified]{second}%
        price\_cap: float = 500\newline
        auction\_type: string = ``first price"\newline
        period: float = 1\newline
        limits: dict = $\lbrace \rbrace$\newline
        offers: list = [ ]\newline
        bids: list = [ ]
		\nodepart[text justified]{third}%
		run()\newline
		get\_offbids(generator)
    };

    \node (Gen) [final, rectangle split, rectangle split parts=3,
    		     text width=4.5cm, below=of Mkt]{
      \textbf{Generator}
        \nodepart[text justified]{second}%
        name: string\newline
%        online\newline
        v\_magnitude\newline
        p: float = 100\newline
        p\_max: float = 200\newline
        p\_min: float\newline
        q: float\newline
        q\_max: float = 100\newline
        q\_min: float = -100\newline
        p\_cost: list\newline
        q\_cost: list
        \nodepart[text justified]{third}
        is\_load(): bool\newline
        total\_cost(p): float\newline
        offers\_to\_pwl(offers)\newline
        bids\_to\_pwl(bids)};

    \node (Env) [final, rectangle split, rectangle split parts=3,
    		      		 text width=4.5cm,left=of Mkt] {
      \textbf{MarketEnvironment}
        \nodepart[text justified]{second}%
        name: string\newline
        n\_offbids: int = 2\newline
        offbid\_qty: bool = False\newline
        markups: tuple
		\nodepart[text justified]{third}%
		getSensors()\newline
		performAction(action)\newline
		reset()
    }

    edge [->] node[near end,text width=10mm] {market 1..1} (Mkt)
    edge [->] node[near end,text width=10mm] {generators 1..*} (Gen);

    \node (Task) [final, rectangle split, rectangle split parts=3,
    		      		 text width=4.5cm,below=of Env] {
      \textbf{ProfitTask}
        \nodepart[text justified]{second}%
        name: string\newline
        sensor\_limits: list = [ ]\newline
        actor\_limits: list = [ ]
		\nodepart[text justified]{third}%
		getObservation()\newline
		performAction(action)\newline
		getReward()
    }

    edge [->] node[near end,text width=6mm] {env 1..1} (Env);

  \end{tikzpicture}
  \caption{Class diagram for the Pyreto.}
\end{figure}
}{}

\subsubsection{Discrete Environment}
For operation with learning methods that use look-up tables to store
state-action values, an environment with $n_s$ discrete states and $n_a$
discrete actions is defined.  An agent can not observe offers/bids submitted
by competitor agents, but may sense other aspects of the power system
model.  To ensure that the size of the environment state space is
kept reasonable, the agent is limited to observing a demand forecast.  The
initial demand at each bus $P_{d0}$, as defined in the original power system
model, is assumed to be peak and the demand at each bus can follow a profile at each
step $t$ of the simulation (See Chapter \ref{ch:exploitation}).  The state
space is divided into discrete steps of size $P_{step} = (P_{d0} - P_d^{min}) /
n_s$, where $P_d^{min}$ is the total demand at the lowest point of the profile.  The
environment computes the total system demand $P_{dt}$ at step $t$ and returns
an integer representation of the state
\begin{equation}
s_t = \frac{(P_{dt} - P_d^{min})}{P_{step}} + 1.
\end{equation}

To define the action space, a vector of percentage markups on marginal cost
$m_e$ is defined for each environment $e$ along with a variable $n_o \in
\mathbb{Z}^+$ which denotes the number of offers/bids to be submitted by the
agent.  (A similar vector of percentage markdowns on total capacity has also
been implemented, but is not used in this thesis.)  A set of all unique
permutations of markup for $n_o$ offers/bids of length $n_a$ is formed, from
which the agent must select select.  The action vector that the discrete
environment is passed consists of a single integer value, corresponding to the
column index in the agent's action value table.  The quantity and price for
each offer/bid submitted to the market is taken from the vector of
permutations using the $a_t$ as the index.  An example of the possible
permutations of 0, 10 and 20\% markups for a portfolio of two generators is
given in Table \ref{tbl:example_actions}.  It should be clear how quickly the
number of possible actions can grow as the number of possible markups and the
size of the portfolio increases.

\begin{table}
\begin{center}
\begin{tabular}{c|c|c}
\hline
$a_i$ &$m_1$ &$m_2$ \\
\hline\hline
 1 &0 &0 \\
 2 &0 &10 \\
 3 &0 &20 \\
 4 &10 &0 \\
 5 &10 &10 \\
 6 &10 &20 \\
 7 &20 &0 \\
 8 &20 &10 \\
 9 &20 &20 \\
\hline
\end{tabular}
\caption{Example discrete action domain.}
\label{tbl:example_actions}
\end{center}
\end{table}

\subsubsection{Continuous Environment}
% For a power system with $n_b$ buses and $n_l$ branches, the visible state of
% the environment is a vector $s_e$ of length $n_s = 2n_b + 2n_l$.
%$s^i_g$ represents the  for the agent associated with generator $i$.
% $s^i_e$ is composed of sensor values for all buses, branches and generators.
% \begin{equation}
% s^i_{e,l} =
% \begin{bmatrix}
% P_f\\
% Q_f\\
% P_t\\
% Q_t\\
% \mu_{S_f}\\
% \mu_{S_t}
% \end{bmatrix}, \quad
% s^i_{e,b} =
% \begin{bmatrix}
% V_m\\
% V_a\\
% \lambda_P\\
% \lambda_Q\\
% \mu_{v_{min}}\\
% \mu_{v_{max}}
% \end{bmatrix}, \quad
% s^i_{e,g} =
% \begin{bmatrix}
% P_g\\
% \mu_{p_{min}}\\
% \mu_{p_{max}}\\
% \mu_{q_{min}}\\
% \mu_{q_{max}}
% \end{bmatrix}\ \quad
% s^i_e =
% \begin{bmatrix}
% s^i_{e,b}\\
% s^i_{e,b}\\
% s^i_{e,g}
% \end{bmatrix}
% \end{equation}
% Not all of the values are used by each agent and they are filtered according
% to the agent's task.

For agents operating policy gradient methods, continuous environments that
output $n_s$ sensors and accept $n_a$ actions are defined.  Each environment
may be configured for actions that specify just price or price and quantity.  If
$q_e^i = 0$ where $q_e^i \in (0,1)$ then the agent's action is price selection
and the offer/bid quantity is determined by the maximum rated capacity of the
generator in question divided by the number of offers being submitted for it.
The environment accepts an action vector $a_e$ of length $n_a$ if $q_e^i = 0$,
otherwise of length $2n_a$.  If $q_e^i = 0$, the $i$-th element of $a_e$ is
the offered/bid price in \$/MWh, where $i = 1,2,\dotsc n_{in}$.  If $q_e^i =
1$, the $i$-th element of $a_e$ is the offered/bid price in \$/MWh, where $i =
1,3,5,\dotsc n_{in}-1$ and the $j$-th element of $a_e$ is the offered/bid
quantity in MW where $j = 2,4,6,\dotsc n_{in}$.  The action vector passed to
the environment is converted into sets of offers/bids that are submitted to
the market model.
%If $q_e^i = 0$, then $qty = p_{max}/n_{in}$.

\subsection{Task}
To allow alternative goals, such a profit maximisation or the meeting some
target level for plant utilisation, to be associated with a single type of
environment, an agent does not interact directly with its environment, but is paired with a particular \textit{task}. A task defines the
reward returned to the agent and thus defines the agent's purpose.  For all
experiments in this thesis the goal of each agent is to maximise
financial profit and the rewards are thus defined as the sum of earnings from
the previous period $t$ as determined by the revenue from the market and any
incurred costs.  As explained in Section \ref{sec:moody}, utilising some
measure of risk adjusted return might be of interest in the context of
simulated electricity trade and this would simply involve the definition of a
new task without any need for modification of the environment.

Sensor data from the environment is filtered according to the task
being performed.  Agents with value-function learning methods use a table to
store state-action values, with one row per environment state.  Thus, observations
consist of a single value $s_v$, where $s_v \leq n_s$ and $s_v \in
\mathbb{Z}^+$.

Agents with policy-gradient learning methods approximate their policy
functions using artificial neural networks that are presented with input vector
$w$ of length $n_s$ where $w_i \in \mathbb{R}$.  To condition
the environment state before input to the connectionist system, where possible,
each sensor $i$ in the state vector $s$ is associated with a minimum value
$s_{i,min}$ and a maximum value $s_{i,max}$.   The state vector is normalised
to:
\begin{equation}
w = 2\left(\frac{s - s_{min}}{s_{max} - s_{min}}\right) - 1
\end{equation}
such that $-1 \leq w_i \leq 1$.

The output from the policy function approximator, $y$, is denormalized using
minimum and maximum action limits, $a_{min}$ and $a_{max}$ respectively, giving
an action vector
\begin{equation}
a = \left(\frac{y + 1}{2}\right)(a_{max} - a_{min}) + a_{min}
\end{equation}
with valid values for price, and optionally quantity.

\subsection{Agent}
Each agent $i$ is defined as an entity capable of producing an action $a_i$
based on previous observations of its environment $s_i$, where $a_i$ and $s_i$
are vectors of length $n_a$ and $n_s$ respectively, where
$n_s$ is the total number of states and $n_a$ is the total number of actions.
%Figure X shows in UML that
In PyBrain each agent is associated with a \textit{module}, a \textit{learner},
a \textit{dataset} and an \textit{explorer}.  The UML class diagram in Figure
\ref{fig:cls_agent} illustrates the associations.

\ifthenelse{\boolean{includefigures}}{\begin{figure}
  \centering
  \begin{small}
  \begin{tikzpicture}[thick]

  \tikzstyle{final}=[rectangle, draw, fill=white, text=black, drop shadow,
    text width=3cm, text centered, rounded corners=2pt]

    \node (Agt) [final, rectangle split, rectangle split parts=3,
    		      		 text width=4.2cm] {
      \textbf{LearningAgent}
        \nodepart[text justified]{second}%
        learning: bool\newline
        logging: bool
		\nodepart[text justified]{third}%
		getAction()\newline
		giveReward(reward)\newline
		integrateObservation(obs)\newline
		newEpisode()\newline
		learn()\newline
		reset()
    };

%    \node (AuxNode01) [text width=4mm,text height=1cm,below of= Agt] {};


    \node (Mod) [final, rectangle split, rectangle split parts=3,
    		     text width=5.2cm,below of=Agt,node distance=6cm] {
      \textbf{ActionValueTable}
        \nodepart[text justified]{second}%
        numActions: int\newline
        numStates: int
		\nodepart[text justified]{third}%
		updateValue(row, column, value)\newline
		getValue(row, column)\newline
		getMaxAction(state)\newline
		initialize(value)
    }
    edge [<-] node[near start,right,text width=15mm,yshift=2mm] {module 1..1}
    (Agt);


    \node (Lrn) [final, rectangle split, rectangle split parts=3,
    		     text width=4.2cm] at (7,-2.8) {
      \textbf{VariantRothErev}
        \nodepart[text justified]{second}%
        batchMode: bool\newline
        offPolicy: bool\newline
        experimentation: float\newline
        recency: float
		\nodepart[text justified]{third}%
		learn()
    }
    edge [<-] node[near start,below,text width=15mm] {learner 0..1} (Agt);


    \node (Dat) [final, rectangle split, rectangle split parts=3,
    		     text width=5.2cm,above of=Lrn,node distance=4cm] {
      \textbf{ReinforcementDataSet}
        \nodepart[text justified]{second}%
        learning: bool\newline
        logging: bool
		\nodepart[text justified]{third}%
		addSample(state, action, reward)\newline
		learn()
    }
    edge [<-] node[near start,above,text width=15mm] {history 1..1} (Agt);


    \node (Exp) [final, rectangle split, rectangle split parts=3,
    		     text width=4.2cm,below of=Lrn,node distance=4cm] {
      \textbf{BoltzmannExplorer}
        \nodepart[text justified]{second}%
        epsilon: float = 2.0\newline
        decay: float = 0.9995
		\nodepart[text justified]{third}%
		activate()
    }
    edge [<-] node[near start,right,text width=15mm,yshift=2mm] {explorer 0..1}
    (Lrn);

  \end{tikzpicture}
  \end{small}
  \caption{Learning agent UML class diagram.}
  \label{fig:cls_agent}
\end{figure}
}{}

The module is used to determine the agent's policy for action selection and
returns an action vector $a_m$ when activated with observation~$s_t$.  When
using value function based methods the module is a $n_s \times n_a$ table:
\begin{equation}
\bordermatrix[{[]}]{%
 & a_0 & a_1 & & a_n \cr
s_0 & v_{1,1}& v_{1,2}& \dotsb & v_{1,m} \cr
s_1 & v_{2,1}& \ddots& & \vdots \cr
    & \vdots & & \ddots & \vdots \cr
s_n & v_{n,1} & \dotsb & \dotsb & v_{n,m}
}
\end{equation}
When using a policy gradient method, the module is a multi-layer feed-forward
artificial neural network.

The learner can be any reinforcement learning algorithm that modifies the
values/parameters of the module to increase expected future reward.  The
dataset stores state-action-reward triples for each interaction between the
agent and its environment.  The stored history is used by value-function
learners when computing updates to the table values.  Policy gradient learners
search directly in the space of the policy network parameters.

Each learner has an association with an explorer that returns an explorative
action $a_e$ when activated with the current state $s_t$ and action $a_m$ from
the module.

% For example, the $\epsilon$-greedy explorer
% has a randomness parameter $\epsilon$ and a decay parameter $d$.  When the
% $\epsilon$-greedy explorer is activated, a random number $x_r$ is drawn where
% $0 \leq x_r \leq 1$.  If $x_r < \epsilon$ then a random vector of the same
% length as $a_e$ is returned, otherwise $a_e = a_m$.

\subsection{Simulation Event Sequence}
\ifthenelse{\boolean{includefigures}}{\begin{figure}
  \centering
  \begin{tikzpicture}[thick]

    \node (Agent) [final, rectangle split, rectangle split parts=3,
    		       text width=5.8cm] {
      \textbf{LearningAgent}
        \nodepart[text justified]{second}%
        name: string\newline
        history: ReinforcementDataSet\newline
        module: Module\newline
        learner: RLLearner
		\nodepart[text justified]{third}%
		integrateObservation(obs)\newline
		getAction()\newline
		giveReward(r)\newline
		learn(episodes=1)
    };

    \node (AuxNode01) [text width=4mm,text height=6cm,right=of Agent] {};


    \node (Task) [final, rectangle split, rectangle split parts=3,
    		      text width=4.5cm,right=of AuxNode01] {
      \textbf{ProfitTask}
        \nodepart[text justified]{second}%
        name: string\newline
        sensor\_limits: list = [ ]\newline
        actor\_limits: list = [ ]
		\nodepart[text justified]{third}%
		getObservation()\newline
		performAction(action)\newline
		getReward()
    };


    \node (Expr) [final, rectangle split, rectangle split parts=3,
    		      text width=5cm,above=of AuxNode01] {
      \textbf{MarketExperiment}
        \nodepart[text justified]{second}%
        market: SmartMarket
		\nodepart[text justified]{third}%
		doInteractions(number=1)
    }
    edge [->] node[near end,right,text width=6mm] {agents 1..*} (Agent)
    edge [->] node[near end,right,text width=6mm] {tasks 1..*} (Task);

  \end{tikzpicture}
  \caption{Class diagram for Pyreto market experiment.}
  \label{fig:cls_experiment}
\end{figure}
}{}
Each experiment consists one or more agent-task pairs. At the beginning of each
simulation step (trading period) the market is initialised and all existing
offers/bids are removed.  From each task-agent tuple an observation
$s_t$ is retrieved from the task and integrated into the agent.  When an
action is requested from the agent its module is activated with $s_t$ and the
action $a_e$ is returned.  Action $a_e$ is performed on the environment
associated with the agent's task.  Figure \ref{fig:seq_action} provides
a UML sequence diagram that illustrates the process of performing an action
and Figure \ref{fig:cls_experiment} shows the class associations for an
experiment.

\ifthenelse{\boolean{includefigures}}{\begin{figure}
  \centering
  \begin{small}
  \begin{sequencediagram}
    \newthread{exp}{:Experiment}
    \newinst{tsk}{:Task}
    \newinst{env}{:Environment}
    \newinst{agt}{:Agent}
    \newinst{mod}{:Module}
    \newinst{epl}{Explorer}

    \begin{call}{exp}{getObservation()}{tsk}{$s_t$}
      \begin{call}{tsk}{getSensors()}{env}{$s_e$}
      \end{call}
      \begin{callself}{tsk}{normalise($s_e$)}{$s_t$}
      \end{callself}
    \end{call}

    \begin{call}{exp}{integrateObservation($s_t$)}{agt}{}
      \begin{callself}{agt}{setLastObs($s_t$)}{}
      \end{callself}
    \end{call}

    \begin{call}{exp}{getAction()}{agt}{$a_e$}
      \begin{callself}{agt}{getLastObs()}{$s_t$}
      \end{callself}
      \begin{call}{agt}{activate($s_t$)}{mod}{$a$}
      \end{call}
      \begin{call}{agt}{activate($a$)}{epl}{$a_e$}
      \end{call}
      \begin{callself}{agt}{setLastAction()}{$a_e$}
      \end{callself}
    \end{call}

    \begin{call}{exp}{performAction($a_e$)}{tsk}{}
      \begin{callself}{tsk}{denormalise($a_e$)}{$a_t$}
      \end{callself}
      \begin{call}{tsk}{performAction($a_t$)}{env}{}
      \end{call}
    \end{call}

  \end{sequencediagram}
  \end{small}
  \caption{Action selection sequence diagram.}
  \label{fig:seq_action}
\end{figure}
}{}
\ifthenelse{\boolean{includefigures}}{\begin{figure}
  \centering
  \begin{small}
  \begin{sequencediagram}
    \newthread{exp}{:Experiment}
    \newinst{tsk}{:Task}
    \newinst{env}{:Environment}
    \newinst{agt}{:Agent}
    \newinst{mkt}{:Market}
    \newinst{dat}{:Dataset}

    \begin{call}{exp}{getReward()}{tsk}{$r_t$}
      \begin{call}{tsk}{getGenerators()}{env}{$g$}
      \end{call}
      \begin{call}{tsk}{getClearedOffers($g$)}{mkt}{$\sigma_g$}
      \end{call}
      \begin{callself}{tsk}{getEarnings($\sigma_g$)}{$r_t$}
      \end{callself}
    \end{call}

    \begin{call}{exp}{giveReward($r_t$)}{agt}{}
      \begin{callself}{agt}{getLastObs()}{$s_t$}
      \end{callself}
      \begin{callself}{agt}{getLastObs()}{$a_e$}
      \end{callself}
      \begin{call}{agt}{addSample($s_t$,$a_e$,$r_t$)}{dat}{}
      \end{call}
    \end{call}

  \end{sequencediagram}
  \end{small}
  \caption{Sequence diagram for the reward process.}
  \label{fig:seq_reward}
\end{figure}
}{}

When all actions have been performed the offers/bids are cleared by the
market using the solution of an optimal power flow problem.  Each task returns
a reward $r_t$.  The cleared offers/bids associated with the generators in the task's environment are retrieved from the market
and $r_t$ is computed from the difference between revenue and cost values.
% \begin{equation}
% r_t = \mbox{revenue} - (c_{fixed} + c_{variable})
% \end{equation}
The reward $r_t$ is given to the associated agent and the value is stored,
along with the previous state $s_t$ and selected action $a_e$, under a new
sample is the dataset.  The reward process is illustrated by the UML sequence
diagram in Figure \ref{fig:seq_action}.

\ifthenelse{\boolean{includefigures}}{\begin{figure}
  \centering
  \begin{small}
  \begin{sequencediagram}
    \newthread{exp}{:Experiment}
    \newinst[1.5]{agt}{:Agent}
    \newinst{lrn}{:Learner}
    \newinst[1]{dat}{:Dataset}
    \newinst{mod}{:Module}

    \begin{call}{exp}{learnEpisodes($1$)}{agt}{}
      \begin{call}{agt}{learn()}{lrn}{}
        \begin{sdblock}{batchMode}{}
          \begin{call}{lrn}{getSample()}{dat}{$s$, $a$, $r$}
          \end{call}
          \begin{call}{lrn}{getValue($s_{-1}$,$a_{-1}$)}{mod}{$q_v$}
          \end{call}
          \begin{call}{lrn}{getValue($s$,$a$)}{mod}{$q_{+1}$}
          \end{call}
          \begin{callself}{lrn}{getQ($q_v$,$\alpha$,$r$,$\gamma$,$q_{+1}$)}{$q_u$}
          \end{callself}
          \begin{call}{lrn}{updateValue($s_{-1}$,$a_{-1}$,$q_u$)}{mod}{}
          \end{call}
        \end{sdblock}
      \end{call}
    \end{call}

  \end{sequencediagram}
  \end{small}
  \caption{Sequence diagram for the SARSA learning process.}
  \label{fig:seq_learn}
\end{figure}
}{}

Each agent learns from its actions using $r_t$, at which point the
values/parameters of the module associated with the agent is updated according
to the output of the learner's algorithm.  Each agent is then reset and the
history of states, actions and rewards is cleared.  The learning process is
illustrated by the UML sequence diagram in  Figure \ref{fig:seq_learn}.

All of this constitutes one step of the simulation and the process is
repeated until a set number of steps are complete.

\section{Summary}
The power exchange auction market model defined in this chapter provides a
layer of abstraction over the underlying optimal power flow problem and
presents agents with a simple interface for selling and buying power.  The
modular nature of the simulation framework described allows the type of
learning algorithm, policy function approximator, exploration technique or the
task to be easily changed.  The framework can simulate competitive electric
power trade using any conventional bus-branch power system model, requiring
little configuration, but provides the facility to adjust all of the
simulation's main aspects. The modular framework and its support for easy
configuration is intended to allow transparent comparison of learning methods
in the domain of electricity trade under a number of different scenarios.
