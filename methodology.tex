\section{Methodology}
Societies reliance on secure energy supplies and the high volumes of
electricity typically consumed render it impractical to experiment with
radically new approaches to energy trade on real systems.  This section
explains the approach taken modelling real systems in software such that they
may be simulated computationally.  The method by which the physical power
systems, that deliver electricity to consumers, were modeled is given, as well
as for the mechanisms that facilitate trade and participants that utilise
these mechanisms.


\subsection{Electricity network model}
High voltage transmission and distribution networks are the mechanisms by which
traded electric energy is delivered to consumers.  Limits to line/cable power
flows, outages and reactive power availability can impose constraints on
particular trades.  As such, certain technical characteristics of the networks
are fundamental to energy market operation and must be duly modeled.

\subsubsection{Power Flow}
The problem to be solved is finding the steady-state operating point of the
network when given levels of generation and load are present.  The primary
constraints in a power system are the branch flow limits and the voltage
limits at each bus.  The system must be operated such that these constraints
are not violated.

\subsubsection{Common Information Model}
Many tools exist for steady-state analysis of balanced three-phase AC networks
and most are centred around bespoke models that describe the power system
data.  Several attempts have been made in the past to standardise the format
in which power system data is stored [CDF, UKGDS, ODF] and latest and most
popular is the Common Information Model.

The Common Information Model (CIM) is an abstract ontological model that
describes the elements of national electric power systems and the associations
between them.  CIM is an evolving international standard approved by the
International Electrotechnical Commission (IEC).

Unlike many tool specific models the CIM does not simplify the power system
into a graph of buses connected by branches.  Instead it describes each of the
components in the system and the electrical connectivity between them.
Conventional numerical techniques for steady-state analysis of AC power
systems require a simplified bus-branch model such that when the voltage angle
and magnitude at each bus is determined the power flows on each branch may be
calculated.


\subsubsection{Energy market model}
Mechanisms for facilitating competitive trade between electricity producers and
consumers differ greatly in the specifics of their implementations in coutries
throughout the world.  However, fundamentally they either provide a
centralised pool through which all electricity is bought and sold or they
permit producers and suppliers to trade directly.

The UK transmission network is frequently congested[].  The thermal limits of
transmission lines between particular areas are often reached.  The balancing
mechanism is the financial instrument used by the system operator to resolve
constraint issues and energy imbalances.  Should the market not be suitably
effective in this function the system operator may choose to contract outwith
the balancing mechanism.  By way of incentive to match demand and avoid
congestion, imbalance charges are imposed on responsible participants.  There
is some evidence to suggest that centralised resolution by a system operator
and socialisation of the incurred costs leads to inefficient despatch of
generators[Neuhoff].

There are a number of alternative approaches to congestion resolution
[neuhoff:power].


\subsubsection{Transmission capacity rights}
One approach is to issue contracts for transmission capacity rights or
equivalent financial rights.  The maximum available transmission capacity
being auctioned for certain periods of time and firm contracts made entitling
owners to full compensation upon curtailment or withdrawl [efet:principles].

The states of Pensylvania, New Jersey and Maryland (PJM) operate a
non-compulsory power pool with nodal market-clearing prices based on
competitive bids.  This is complemented by daily and monthly capacity markets
plus the monthly auction of Financial Transmission Rights to provide a hedging
mechanism against future congestion charges.


\subsubsection{Transmission charging}
Impose delivery charges which increase as network constraints are approached.


\subsubsection{Extended bids/offers}
Request extended bids and offers which include costs associated with the
adjustment of participant's desired position.


\section{Market participant model}
Without competition between market participants there is no driver for
individuals to improve efficiency and reduce costs paid by the consumers.
Traders are typically resposible for this, but it is not feasible to use
humans for this project.  In a highly distributed power system, a very large
number of items of plant may be supplying the demand and, depending on the
levels of aggregation, this could require many traders to be used.  Also, this
project requires that experiments be repeated numerous times under a variety
of scenarios.

\subsection{Software agents}
Participants are modeled in software also.  The nature of a highly distributed
power system dictates that a very large number of entities may be interacting
in the marketplace.  Economic studies regularly integrate participant logic
into the same optimisation problem as the market.  However, this does not
scale to large numbers of individual participants.  Separating participant
logic into individual software agents allows their action selection procedures
to be processed in simultaneously.  The definition of an agent in this context
emerges from the machine learning technique employed to implement the
competitive decision making process.

\subsection{Reinforcement learning}
While there is a wealth of data available on past energy market activity
involving conventional transmission connected plant, there exists no such
resource for trade performed in highly distributed power systems.
Consequently, reactive machine learning techniques that use new data to
influence the decision making policy are used.

Reinforcement learning is a sub-area of machine learning and can be applied to
a wide variety of problems [suttonbarto:reinforcement].  To allow the same
learning algorithms developed for traditional, academic reinforcement learning
problems (chess, backgammon, lift scheduling etc.) to be applied to models of
energy markets (and vice versa) a modular machine learning library is used.
