\chapter{Background}
This chapter provides an introduction to optimal power flow and reinforcement
learning.  Different formulations of the optimal power flow problem are
explained along with the principles behind interior-point methods, common used
to find their solutions.  Reinforcement learning is also a generic term and
Section \ref{sec:rl} introduces computational approaches to learning to maximise
long term reward.  Definitions are provided for the valued-based and direct
search algorithms compared in subsequent chapters.

\section{Optimal Power Flow}
\label{sec:opf}
Optimal power flow is a term used to describe a broad class of problems in
which an objective function is optimised while constraints on the optimisation
variables, which represent physical characterisitcs of the electric power
system, are satisfied.  Optimisation techniques typically solve problems of the
form
\begin{equation}
\min_x f(x)
\end{equation}
subject to
\begin{eqnarray}
g(x)& =& 0\\
h(x)& \leq& 0\\
x_{min}\leq& x& \leq x_{max}
\end{eqnarray}

\subsection{Interior-Point Methods}

\section{Reinforcement Learning}
\label{sec:rl}
This section describes agent's policies that represent a store of experience
and the learning algorithms that its modify parameters. Together these
components form models of individual behavior which are used to determine the
actions to be performed in the agent's environment and to learn from received
rewards.
% This section provides an introduction to the reinforcement learning problem and
% some of the associated terminology.  Definitions for the value-function and
% policy gradient algorithms, that are later applied to power trade
% implementations of the problem, are given.

For a comprehensive introduction to reinforcement learning with evaluations of
algorithm designs through mathematical analysis and computational experiments
the intersted reader is directed to the seminal work by Barto and Sutton
%\cite{suttonbarto:98}.

\subsection{Introduction}
The problem of learning how best to interact with an environment so as to
maximise some long-term reward is one that arises in many aspect of life.
Reinforcement learning is a term that is typically applied to
understanding, automating and solving this problem through computational
approaches. Unlike with the majority of Machine Learning techinques, the
algorithms are not instructed as to which actions to take, but must learn to
maximise the long-term reward through trial-and-error.

Reinforcement learning starts with an interactive, goal-seeking individual and
an associated environment.  The individuals require the ability to sense
aspects of their environment, perform actions that influence the state of their
environment and be assigned rewards as a response to their chosen action.  An
agent is said to follow a particular \textit{policy} when mapping the
perceived state of its environment to an action choice.

Value-based methods attempt to find the optimal policy by
approximating a \textit{value-function} which returns the total reward an
agent can expect to accumulate, given an initial state and following the
current policy thereafter.

Policy-gradient methods are an alternative to this which
represent a policy using a learned function approximator with its own
parameters %\cite{sutton:99}.
The function approximator is updated according to the gradient of expected
reward with respect to these parameters.

% \subsection{Markov Decision Processes}
% \subsection{Dynamic Programming}
% \subsection{Monte-Carlo Methods}
% \subsection{Temporal-Difference Learning}
% \subsection{Artificial Neural-Networks}

\subsection{Value Function Methods}

\subsubsection{Basic Roth-Erev}
\label{sec:rotherev}
The Roth-Erev reinforcement learning algorithm uses a stateless policy to
select actions from a discrete domain\cite{roth:games,roth:aer}.  The dataset
stored by each agent, $j$, contains an array of length $K$, where $K$ is the
number of feasible actions $k$. Each value in the array represents the
propensity for selection of the associated action in all states of the
environment.  Following interaction $t$ in which agent $j$ performed on the
environment action $k^\prime$, for arbitrary positive $t$, a reward,
$r_{jk^\prime}(t)$, is calculated.  The propensity for agent $j$ to select
action $k$ for interaction $t+1$ is

\begin{equation}
q_{jk}(t+1) =
\begin{cases}
(1-\phi)q_{ik}(t) + r_{jk^\prime}(t)(1-\epsilon), & \text{$k = k^\prime$} \\
(1-\phi)q_{ik}(t) + r_{jk^\prime}(t)(\frac{\epsilon}{K-1}), & \text{$k \ne
k^\prime$}
\end{cases}
\end{equation}

where $\phi$ and $\epsilon$ denote \textit{recency} and
\textit{experimentation} parameters, respectively. The recency (forgetting)
parameter degrades the propensity for all actions and prevents the value from
going unbounded.  It is intended to represent the tendency for players to
forget older action choices and to prioritise more recent experience.  The
experimentation parameter prevents the probability of choosing an action from
going to zero and thus encourages exploration of the action space.

Erev and Roth proposed that actions be selected according to a discrete
probability distribution function where action $k$ is selected for interaction
$t+1$ with probability:

\begin{equation}
p_{jk}(t+1) = \frac{q_{jk}(t+1)}{\sum_{l=0}^K q_{jl}(t+1)}
\end{equation}

Since $\sum_{l=0}^K q_{jl}(t+1)$ increases with $t$, a reward $r_{jk}(t)$ for
performing action $k$ will have a greater effect on the probability
$p_{jk}(t+1)$ during early interactions while $t$ is small.  This is intended
to represent Psychology's \textit{Power Law of Practice} in which it is
qualitatively stated that, with practice, learning occurs at a decaying
expoential rate and that a learning curve will eventually flatten out.

This algorithm may alternatively use a form of the \textit{softmax} method
\cite{suttonbarto:1998} using the Gibbs, or Boltzmann, distribution to select
action $k$ for the $t+1$th interaction with probability

\begin{equation}
p_{jk}(t+1) = \frac{e^{q_{jk}(t+1)/\tau}}{\sum_{l=0}^K e^{q_{jl}(t+1)/\tau}}
\end{equation}

where $\tau$ is the \textit{temperature} parameter.  This parameter may be
decreased in value over the course of an experiment since high values give all
actions similar probability and encourage exploration of the action space,
while low values promote exploitation of past experience.

\subsubsection{Variant Roth-Erev}
\label{sec:variant}
Two shortcomings of the basic Roth-Erev algorithm (\S\ref{sec:rotherev}) have
been identified and a variant formulation proposed\cite{nicolaisen:2001}. The
problems are that the values by which propensities are updated can be zero or
very small for certain combinations of the experimentation parameter
$\epsilon$ and the total number of feasible actions $K$.  Also, all
propensity values are decreased by the same amount when the reward,
$r_{jk^\prime}(t)$ is zero.  Under the variant algorithm the propensity of
agent $j$ to select action $k$ for interaction $t+1$ becomes:

\begin{equation}
q_{jk}(t+1) =
\begin{cases}
(1-\phi)q_{ik}(t) + r_{jk^\prime}(t)(1-\epsilon), & \text{$k = k^\prime$} \\
(1-\phi)q_{ik}(t) + q_{jk}(t)(\frac{\epsilon}{K-1}), & \text{$k \ne
k^\prime$}
\end{cases}
\end{equation}

As with the basic Roth-Erev algorithm, the propensity for the action that the
reward is associated with is adjusted by the experimentation parameter.  All
other action propensities are adjusted by a small proportion of their current
value.

\subsubsection{SARSA}
\label{sec:sarsa}
% Sarsa is an on-policy Temporal Difference control method.  The
% policy is represented by a $M \times N$ table, where $M$ and $N$ are
% arbitrary positive numbers equal to the total number of feasible states and
% actions. The action-value update for agent $j$ is defined by
%
% \begin{equation}
% Q_j(s_{jt},a_{jt}) + \alpha [r_{jt+1} + \gamma Q_j(s_{jt+1},a_{jt+1}) -
% Q_j(s_{jt},a_{jt})].
% \end{equation}
%
% While the Q-learning algorithm updates action-values using a greedy policy,
% which is different to that being followed, Sarsa uses the discounted future
% reward of the next state-action observation following the original policy.
The SARSA algorithm is an on-policy Temporal Difference control method, similar
to Q-learning.  The action-value update for agent $j$ is defined by

\begin{equation}
Q_j(s_{jt},a_{jt}) + \alpha [r_{jt+1} + \gamma Q_j(s_{jt+1},a_{jt+1}) -
Q_j(s_{jt},a_{jt})] \text{.}
\end{equation}

While the Q-learning algorithm updates action-values using a greedy policy,
which is a different policy to that being followed, SARSA uses the discounted
future reward of the next state-action observation following the original
policy.

\subsubsection{Q-Learning}
\label{sec:qlearning}
The formulation of the Q-learning algorithm used is that of the original
off-policy Temporal Difference algorithm developed by
Watkins\cite{watkins:1989}.  The action-value function, $Q(s,a)$, returns
values from a $M \times N$ matrix where $M$ and $N$ are arbitrary positive
numbers equal to the total number of feasible states and actions, respectively.
Each value represents the \textit{quality} of taking a particular action, $a$,
in state $s$.  Actions are selected using either the $\epsilon$-greedy or
softmax (See section \ref{sec:rotherev}) methods.  The $\epsilon$-greedy method
either selects the action (or one of the actions) with the highest estimated
value or it selects an action at random, uniformly, independently of the
estimated values with, typically small, probability $\epsilon$.

Agent $j$ will observe a reward, $r_{jt}$, and a new state, $s_{jt+1}$,
after taking action $a_{jt}$ at step $t$ when in state $s_{jt}$.  The
state-action value, $Q_j(s_{jt},a_{jt})$, is updated according to the
maximum value of available actions in state $s_{t+1}$ and becomes

\begin{equation}
\label{eq:qlearning}
Q_j(s_{jt},a_{jt}) + \alpha [r_{jt+1} + \gamma\max_{a} Q_j(s_{jt+1},a_{jt}) -
Q_j(s_{jt},a_{jt})]
\end{equation}

where $\alpha$ and $\gamma$ are the learning rate, $0\leq\alpha\leq1$, and
discount factor, $0\leq\gamma\leq1$, respectively.  The learning rate determines
the extent to which new rewards will override the effect of older rewards.
The discount factor allows the balance between maximising immediate rewards and
future rewards to be set.

\subsubsection{Q($\lambda$)}
\label{sec:qlambda}
With the Q-learning formulation, described in equation \ref{eq:qlearning}, only
the quality associated with the previous state, $s_{jt}$, is updated.  However,
the preceding states can also, in general, be said to be associated with the
reward $r_{jt+1}$.  Eligibility traces are a mechanism for facilitating this
effect and in algorithms such as Q($\lambda$), the $\lambda$ refers to it. The
eligibility trace for a state $e(s)$ represents how eligible the state $s$ is
to receive credit or blame for the error.  The term ``trace'' refers to fact
that only recently visited states become eligible.  The eligibility value for
the current state is increased, while for all other states it is attenuated by
a factor $\lambda$.

The off-policy nature of Q-learning requires special care to be taken when
implementing eligibility traces.  While the algorithm may learn a greedy
policy, in which the action with the maximum value would always be taken,
typically a policy with some degree of exploration will be followed when
choosing actions.  If an exploratory (pseudo-random) step is taken the
preceding states can no longer be considered eligible for credit or blame.
Setting $\lambda$ to $0$ for non-greedy actions removes much of the benefit of
using eligibility traces if exploratory actions are frequent.  A solution to
this has been developed, but requires a very complex implementation
\cite{peng:1996}.  A na\"ive approach can be taken, where the effect of
exploratory actions is ignored, but the results of this are unexplored.

\subsection{Policy Gradient Methods}

\subsubsection{REINFORCE}
\label{sec:reinforce}
The previously defined learning methods typically rely upon discretisation of
the sensor and action spaces so the associated values may be stored in tables.
The memory requirements for this restrict the application of these methods to
only small environments.  Many environments, particularly from real
applications, exhibit continuous sensor and/or action spaces and require
generalisation techniques to be employed to provide a more compact policy
representation.

REINFORCE is an associative reinforcement learning algorithm that determines
a policy by modifying the parameters of a policy function approximator, rather
than approximating a value function \cite{williams:reinforce}.  Commonly,
feedforward artificial neural networks are used to represent the policy, where
the input is a representation of the state and the output is action selection
probabilities.  In learning, a \textit{policy gradient} approach is taken where
the weights of the network are adjusted in the direction of the gradient of
expected reinforcement.

Defining the network, let $\mathbf{x}^i$ denote the vector of inputs to the
$i$th unit and $y_i$ denote output of the unit.  In the input layer of the
network the elements $x_j$ of $\mathbf{x}^i$ are normalised sensor values from
the environment and in the output layer, or in any hidden layers, they are
outputs from the $j$ unit in the preceding layer.  Let $\mathbf{w}^i$ denote
the vector of the weights, $w_{ij}$, on the connections to the $i$th unit.  The
output of the $i$th unit is dependant on the vector of inputs, $\mathbf{x}^i$,
and the associated weights, $\mathbf{w}^i$.

For each interaction of the agent with the environment, each parameter $w_{ij}$
is incremented by

\begin{equation}
\label{eq:reinforce}
\Delta w_{ij} = \alpha_{ij}(r - b_{ij})\frac{\partial\ln\rho_i}{\partial
w_{ij}}
\end{equation}

where $\alpha_{ij}$ is the \textit{learning factor}, $b_{ij}$ is the
\textit{reinforcement baseline} and $\rho_i$ is the performance of the policy
(e.g., the average reward per interaction).

\subsubsection{ENAC}
\label{sec:enac}
ToDo: Episodic Natural Actor Critic\cite{peters:enac}.
