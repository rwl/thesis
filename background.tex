\chapter{Background}
\label{ch:background}
This chapter provides an introduction to electricity supply and wholesale
electricity markets in the UK.  It explains how electricity markets can be
simulated and how the power system dynamics are captured in the associated
models.

\section{Electric Power Supply}

%

\begin{figure}
  \label{fig:powersystem}
  \centering
  \begin{tikzpicture}[thick]

  % Power Station.
  \begin{scope}[very thick]
    \draw[line join=round] (0.7mm,0) -- ++(1mm,25mm) -- ++(2mm,0) --
    ++(1mm,-25mm) -- cycle;

    \draw[ultra thick] (6mm,0) rectangle ++(8mm,15mm);
    \foreach \y in {0.5,0.6,...,1.5}
      \draw (6mm,\y) -- ++(8mm,0);

    \draw (6mm,1mm) -- ++(-1.4mm,2mm);
    \draw (6mm,3mm) -- ++(-1.5mm,2mm);

    \begin{scope}[ultra thick,xshift=2cm,yshift=0cm]
      \filldraw[fill=black!70,draw=black!70] (4.3mm,15mm) rectangle
      ++(7.4mm,1.6mm);

      \draw[line join=round] (0,0) .. controls (3mm,3mm) and
      (5mm,10mm) .. (4mm,20mm)-- (12mm,20mm) --
      (12mm,20mm) .. controls (11mm,10mm) and (13mm,3mm) .. (16mm,0mm);
%      \draw (4mm,20mm) -- (12mm,20mm);
    \end{scope}

    \filldraw[ultra thick,fill=white] (14mm,0) rectangle ++(14mm,8mm);
    \filldraw[fill=black] (14mm,5mm) rectangle ++(14mm,1.5mm);

    \filldraw[ultra thick,fill=white] (28mm,0) rectangle ++(10mm,4mm);

%     \begin{scope}[xshift=2cm,yshift=3cm]
%       \draw (0,0) to[out=30,in=270]
%             (3mm,15mm) to[out=0,in=180]
%             (9mm,15mm) to[out=-90,in=150]
%             (12mm,0);
%     \end{scope}
  \end{scope}

  % Transformer.
  \begin{scope}[very thick,xshift=4.5cm]
    \draw (0,0) rectangle (6mm,8mm);
    \filldraw[fill=black] (-0.1mm,0) rectangle (6.2mm,-0.1mm);
    \filldraw[fill=black] (-0.2mm,8mm) rectangle (6.2mm,8.1mm);

    \draw[thick] (0,0.5mm) rectangle (-2.5mm,7mm);
    \foreach \x in {-0.05,-0.09,...,-0.25}
      \draw[thin] (\x,0.5mm) -- ++(0,6.5mm);

    \draw[line join=round,thick] (0.5mm,8mm) -- ++(-1mm,4mm) -- (2mm,8mm);
    \draw[thick] (-0.05mm,8.8mm) -- ++(1.9mm,0);
    \draw[thick] (-0.15mm,9.4mm) -- ++(1.65mm,0);
    \draw[thick] (-0.26mm,10mm) -- ++(1.4mm,0);
    \draw[thick] (-0.39mm,10.6mm) -- ++(1.15mm,0);

    \draw[thick] (6mm,0.5mm) rectangle (8.5mm,7mm);
    \foreach \x in {0.65,0.69,...,0.85}
      \draw[thin] (\x,0.5mm) -- ++(0,6.5mm);

    \draw[line join=round,thick] (5.5mm,8mm) -- ++(1mm,4mm) -- (4mm,8mm);
    \draw[thick] (6.05mm,8.8mm) -- ++(-1.9mm,0);
    \draw[thick] (6.15mm,9.4mm) -- ++(-1.65mm,0);
    \draw[thick] (6.26mm,10mm) -- ++(-1.4mm,0);
    \draw[thick] (6.39mm,10.6mm) -- ++(-1.15mm,0);
  \end{scope}

  % Transmission tower.
  \begin{scope}[xscale=0.1,yscale=0.7,xshift=65cm,yshift=30mm,very thick]
	\draw[line join=round] (-2,0) -- (-2,1.8) -- ( 2,1.8) -- ( 2,0) -- cycle;
	\draw[line join=bevel] (-2,0) -- ++(4,0.9) -- ++(-4,0.9);% -- ++(4,0.75);
	\draw[line join=bevel] (2,0) -- ++(-4,0.9) -- ++(4,0.9);% -- ++(-4,0.75);
    \draw[line join=round] (-2,1.8) -- (0,2.05) -- (2,1.8);

    \draw[line join=round] (-2,0) -- ++(-6,0) -- (-2,0.35) -- (-4,0) --
    (-5.5,0.15);
    \draw[line join=round] (-2,0.7) -- ++(-8,0) -- (-2,1.14) -- (-4.5,0.7) --
    ++(-2,0.2);
    \draw[line join=round] (-2,1.5) -- ++(-5,0) -- (-2,1.8) -- (-3.5,1.5) --
    ++(-1.3,0.15);

    \draw[line join=round] (2,0) -- ++(6,0) -- (2,0.35) -- (4,0) --
    (5.5,0.15);
    \draw[line join=round] (2,0.7) -- ++(8,0) -- (2,1.14) -- (4.5,0.7) --
    ++(2,0.2);
    \draw[line join=round] (2,1.5) -- ++(5,0) -- (2,1.8) -- (3.5,1.5) --
    ++(1.3,0.15);

    \draw[ultra thick] (-7.8,0) -- ++(0,-0.4);
    \draw[ultra thick] (-9.8,0.7) -- ++(0,-0.4);
    \draw[ultra thick] (-6.8,1.5) -- ++(0,-0.4);

    \draw[ultra thick] (7.8,0) -- ++(0,-0.4);
    \draw[ultra thick] (9.8,0.7) -- ++(0,-0.4);
    \draw[ultra thick] (6.8,1.5) -- ++(0,-0.4);

    \draw (-2,0) -- (-5,-3)
          ( 2,0) -- ( 5,-3)
          (-2,0) -- (2,0)

          (-2, 0) -- ( 3,-1)
          ( 2, 0) -- (-3,-1)
          ( 3,-1) -- (-3,-1)

          ( 3,-1) -- (-4.5,-2.5)
          (-3,-1) -- ( 4.5,-2.5)
          (-4.5,-2.5) -- (4.5,-2.5)

          (0,-2.5) -- (-5,-3)
          (-4.5,-2.5) -- (-3,-2.75)
          (0,-2.5) -- ( 5,-3)
          (4.5,-2.5) -- (3,-2.75);

  \end{scope}

  % Transmission customer.
  \begin{scope}[xshift=8cm,yshift=0cm,line join=round,ultra
  thick,xscale=0.5,yscale=0.5]
    \draw[draw=gray!70,xshift=2.5pt,yshift=-2.5pt]
    (0,0) -- (0,1) -- (0.5,1.5) -- (1,1) -- (1.5,1.5) -- (2,1) -- (2,3) --
    (2.3,3) -- (2.3,1.5) -- (2.6,1.5) -- (2.6,3) -- (2.9,3) -- (2.9,1.5) --
    (3.2,1.5) -- (3.2,3) -- (3.5,3) -- (3.5,0) -- (2.9,0) -- (2.9,0.8) --
    (2.2,0.8) -- (2.2,0) -- cycle;

    \filldraw[fill=white] (0,0) -- (0,1) -- (0.5,1.5) -- (1,1) -- (1.5,1.5) --
    (2,1) -- (2,3) -- (2.3,3) -- (2.3,1.5) -- (2.6,1.5) -- (2.6,3) -- (2.9,3) -- (2.9,1.5) --
    (3.2,1.5) -- (3.2,3) -- (3.5,3) -- (3.5,0) -- (2.9,0) -- (2.9,0.8) --
    (2.2,0.8) -- (2.2,0) -- cycle;
  \end{scope}

  % Distribution customer.
  \begin{scope}[xscale=0.5,yscale=0.5,xshift=18cm,yshift=-8cm,ultra thick,line
  join=round]
    \draw[draw=gray!70,xshift=2pt,yshift=-2pt] (2,2) rectangle ++(0.3,2.5);
	\filldraw[fill=black] (2,2) rectangle ++(0.3,2.5);

	\draw[draw=gray!70,xshift=2pt,yshift=-2pt] (0,0) -- (0,2) -- (3,2) -- (3,3) --
	(4,3) -- (4,0) -- cycle;
	\filldraw[fill=white] (0,0) -- (0,2) -- (3,2) -- (3,3) -- (4,3) -- (4,0) --
	cycle;

	\foreach \x in {0.5,1.1,1.7,2.3,2.9}
	  \filldraw[fill=black!80] (\x,1.2) rectangle ++(2.5mm,2.5mm);

  \end{scope}

  % House.
  \begin{scope}[xscale=0.6,yscale=0.6,xshift=15cm,yshift=-10cm,ultra thick,line
  join=round]
    \draw (0,0) rectangle ++(3,1);
    \draw (3,0) rectangle ++(0.18,2.6);
    \filldraw[fill=white] (-0.1,1) rectangle ++(3.2,1.2);

    \draw (0.2,0.4) rectangle ++(0.9,0.4);
    \draw (1.3,0) rectangle ++(0.4,0.8);
    \draw (1.9,0.4) rectangle ++(0.9,0.4);

    \draw (0.4,1.3) rectangle ++(0.4,0.4);
    \draw[thick] (0.5,1.4) rectangle ++(0.2,0.2);
    \draw (0.35,1.7) -- ++(0.5,0) -- ++(-0.25,0.2) -- cycle;
    \draw[thick] (0.6,1.4) -- ++(0,0.2);
    \draw[thick] (0.5,1.5) -- ++(0.2,0);
  \end{scope}

  \node[text width=33mm,text centered] at (2,-1) {Generating Station
  \newline 11kV--25kV};
  \node[text width=3cm,text centered] at (5,-1) {Step-Up Transformer};
  \node[text width=8cm,text centered] at (8,4) {Transmission Lines 400kV,
  275kV or 138kV};
  \node[text width=4cm,text centered] at (9,-1) {Transmission Customer \newline
  275kV and 132kV};
  \node[text width=4cm,text centered] at (6.5,-3) {Primary Customer 33kV, 11kV
  or 6.6kV};
  \node[text width=4cm,text centered] at (6.5,-5) {Secondary Customer 400V and
  220V};

  % Line key.
  \begin{scope}[yshift=-6cm]
    \draw (0,0) rectangle ++(4,1.75);
%    \node[above right] at (0.1,1.5) {\textbf{Line Key}};

    \node[above right] at (0.1,1.1) {Generation:};
    \draw[dashed] (2.8,1.3) -- ++(1,0);

    \node[above right] at (0.1,0.6) {Transmission:};
    \draw (2.8,0.8) -- ++(1,0);

    \node[above right] at (0.1,0.1) {Distribution:};
    \draw[densely dotted] (2.8,0.3) -- ++(1,0);
  \end{scope}

  % Pole.
  \begin{scope}[xshift=13cm,yshift=-5.5cm,thick]
    \draw (-0.63,-0.49) -- ++(1.15,0.6) -- ++(0,0.1) -- ++(-1.15,-0.6) -- cycle;
    \filldraw[fill=white] (-0.6,-0.5) -- ++(1.15,0.6) -- ++(0,0.1) --
    ++(-1.15,-0.6) -- cycle;

    \node[draw,cylinder,rotate=90,minimum width=1mm,minimum
    height=0.75mm,inner sep=0.8pt,cylinder uses custom fill,cylinder end
    fill=white, cylinder body fill=white] at (-0.57,-0.35) {};
    \node[draw,cylinder,rotate=90,minimum width=1mm,minimum
    height=0.75mm,inner sep=0.8pt,cylinder uses custom fill,cylinder end
    fill=white, cylinder body fill=white] at (0.49,0.2) {};
    \node[draw,cylinder,rotate=90,minimum width=1mm,minimum
    height=0.75mm,inner sep=0.8pt,cylinder uses custom fill,cylinder end
    fill=white, cylinder body fill=white] at (-0.5mm,-0.9mm) {};

    \node[draw,cylinder,rotate=90,aspect=0.5,minimum width=1mm,minimum
    height=2cm,inner sep=2pt,left,cylinder uses custom fill,cylinder end
    fill=white, cylinder body fill=white] at (0.2mm,-0.6mm) {};

%    \filldraw[fill=white] (0,-1) rectangle ++(-3mm,4mm);
    \node[draw,cylinder,rotate=90,aspect=0.75,minimum width=2.2mm,minimum
    height=4mm,inner sep=2pt,cylinder uses custom fill,cylinder end
    fill=black!30, cylinder body fill=black!50] at (-1.2mm,-7mm) {};
  \end{scope}

  \end{tikzpicture}
  \caption{Basic structure of a three phase AC power system.}
\end{figure}

Generation and bulk movement of electricity in the UK takes place in a
three-phase alternating current (AC) power system.  These phases are
high voltage, sinusoidal electrical waveforms, offset in time from each
other by 120 degrees and oscillating at a frequency of almost exactly 50Hz.
Synchronous generators (or alternators), typically rotating at 3000rpm or
1500rpm, generate apparent power $S$ at a line voltage $V_l$, typically
between 11kV and 25kV.  One of the principal reasons that alternating current,
and not direct current (DC), systems are common in electricity supply is that
they allow power to be transformed between voltages with very high efficiency.
The apparent power conducted by a three-phase transmission line $l$ is the
product of the line current $I_l$ and the line voltage
\begin{equation}
S = \sqrt{3} V_l I_l .
\end{equation}
For a constant quantity of transmitted power, increasing the line voltage has
an inverse effect on the line current.  Ohmic heating losses are proportional to the
square of line current
\begin{equation}
P_{r} = 3 I_l^2 R
\end{equation}
where $R$ is the resistance of the transmission line.  Hence reducing the line
current causes a large reduction in energy wasted through heating losses.  A
consequence of higher voltages is the larger extent and integrity of the
insulation required between conductors, neutral and earth.  This results in
the need for large transmission towers and in high cable costs when
undergrounding systems.

The UK transmission system operates at 400kV and 275kV (also 132kV in
Scotland), but systems with voltages upto and beyond 1000kV are used in
larger countries such as Canada and China.  For transmission over very long
distances or undersea, high voltage DC (HVDC) systems have become economically
viable in recent years, largely because the reactive power component of the
transmitted power is zero, allowing more active power flow for the same
line/cable diameter.

The ability to transform power between voltages and
transmit large volumes over long distances allows for generation to take place at high capacity power stations, which offer economies of scale and lower operating costs. It allows electricity to
be transmitted across country borders and from renewable energy plant such as
hydro power stations located in remote areas.
% A HVDC interconnector between
% Folkstone in the UK and Sangatte in France allows upto 2GW of
% electricity to be imported/exported.  The Moyle HVDC interconnector can export
% upto 500MW from Auchencrosh in Scotland to Ballycronan More in Northern
% Ireland or import upto 80MW.  Further HVDC interconnectors are planned between
% England and the Netherlands and between Wales and The Republic of Ireland.
Figure \ref{fig:uk_gen} shows the UK's existing HVDC interconnectors and
how larger power stations are located away from load centres and close to
sources of fuel, such as the coal fields in northern England and gas supply
terminals near Cardiff and London.

%\begin{figure}
\label{fig:uk_gen}
\centering
\begin{tikzpicture}[xscale=0.9,yscale=0.9]
%% UK
\draw[uk] (-7.0029232622551705, 63.9837806219489309) -- (-7.0258060967827030, 63.9828204421218985) -- (-7.0323004758755712, 63.9833082879295887) -- (-7.0326088308650583, 63.9890557764767038) -- (-7.0220958181545639, 64.0125222257489099) -- (-7.0193128308847310, 64.0177854183140198) -- (-7.0146752608982936, 64.0206672099693463) -- (-7.0075619454365272, 64.0211414821389013) -- (-6.9992129836270438, 64.0134827506359159) -- (-6.9964311095521179, 64.0086940327863516) -- (-6.9908651350124549, 63.9986312143424527) -- (-6.9899367304591715, 63.9952760643323018) -- (-6.9979762240843524, 63.9866612283415321) -- (-7.0029232622551705, 63.9837806219489309);
\draw[uk] (-1.1796771342243169, 65.3312023218514639) -- (-1.3005834595647072, 65.1814307246090436) -- (-1.3262481681670295, 65.1668460228547133) -- (-1.3528412813226902, 65.1556788253213028) -- (-1.3822173817481276, 65.1474224436747988) -- (-1.4276724694237668, 65.1411126953723141) -- (-1.5222940365980271, 65.2105816789109412) -- (-1.6397984382996860, 65.2844913944839647) -- (-1.6626801596323102, 65.2956884443617440) -- (-1.6688650705407289, 65.2981214176328706) -- (-1.6833978300638686, 65.2986013647504961) -- (-1.6957676518807836, 65.2961701287888872) -- (-1.7121561073153138, 65.2893549958920261) -- (-1.7467887140960758, 65.2786526525282511) -- (-1.7477160054543839, 65.2840097788230338) -- (-1.7263793986540594, 65.3098020983365046) -- (-1.6957676518807836, 65.3448401326821511) -- (-1.4570485698491926, 65.4778623406687785) -- (-1.4437525698688107, 65.4793211599138232) -- (-1.4375676589603255, 65.4783433443759293) -- (-1.2446142459836651, 65.4188713350837503) -- (-1.2399755628022537, 65.4149770690533217) -- (-1.2177116646436212, 65.3891533085663639) -- (-1.1796771342243169, 65.3312023218514639);
\draw[uk] (-1.0445475175452477, 65.4890718676173833) -- (-1.0692860479841699, 65.4837012852078715) -- (-1.0828915161489963, 65.4866349593902157) -- (-1.1382417933611999, 65.5037018683566288) -- (-1.1373145020028919, 65.5100519498169547) -- (-1.1342220465486772, 65.5144337237093737) -- (-1.0862923265927509, 65.5817915926744632) -- (-1.0807263520530916, 65.5827583437858124) -- (-1.0736152629811373, 65.5808143068277332) -- (-1.0566078711777789, 65.5681159279046568) -- (-1.0535154157235662, 65.5627524449294867) -- (-1.0420739984597387, 65.4944305123412676) -- (-1.0445475175452477, 65.4890718676173833);
\draw[uk] (-5.1782131309936190, 66.1624314043866235) -- (-5.1915080177790376, 66.1604734784666562) -- (-5.1976940418824871, 66.1624314043866235) -- (-5.2023316118689351, 66.1658764772080019) -- (-5.2035683714116256, 66.1727652906445485) -- (-5.2001675609678122, 66.2081763889622152) -- (-5.1980023968719067, 66.2219617782211429) -- (-5.1936731818749333, 66.2254093770570051) -- (-5.1862526246186658, 66.2254093770570051) -- (-5.1834685241539926, 66.2204754228513366) -- (-5.1723365750746755, 66.1712922287497634) -- (-5.1729555114434422, 66.1648903854284498) -- (-5.1782131309936190, 66.1624314043866235);
\draw[uk] (1.0071319234947749, 66.5148046748936110) -- (1.0009481257811286, 66.5133129436408979) -- (0.9854869617049711, 66.5152936273477025) -- (0.8766409899971805, 66.5350462178021047) -- (0.8695287877301947, 66.5370274329331863) -- (0.8528308641112826, 66.5464184115659236) -- (0.8454091936599760, 66.5553070508684215) -- (0.8296396745943401, 66.5795231799295664) -- (0.8246915232284789, 66.5899085685546908) -- (0.8181982573305763, 66.6042485021037578) -- (0.8243831682391907, 66.6423220726047703) -- (0.8336594214068608, 66.6670541160774803) -- (0.8463375982133157, 66.6690367826504513) -- (0.8806607368097014, 66.6606254801249491) -- (0.9932169871455656, 66.6274875414197396) -- (1.0065141003207876, 66.6230311247177127) -- (1.0117706066761263, 66.6195683574217554) -- (1.0358890875513136, 66.5874384479331098) -- (1.0426929348285683, 66.5780427120904506) -- (1.0507324284538633, 66.5627182570993483) -- (1.0547521752663838, 66.5513580052493410) -- (1.0550616434507036, 66.5444369667400082) -- (1.0541332388973641, 66.5380180586695644) -- (1.0491862007265356, 66.5345589245572455) -- (1.0071319234947749, 66.5148046748936110);
\draw[uk] (-5.0888469569797321, 69.9300095581513688) -- (-5.0956496910620874, 69.9289760259954107) -- (-5.1284277151261763, 69.9624789975047747) -- (-5.1924364223322428, 70.0037443621610151) -- (-5.2063513586814576, 70.0073568501536272) -- (-5.2131540927638129, 70.0171643343502978) -- (-5.2252144463963326, 70.0465773706661849) -- (-5.2230492823004377, 70.0517379973752270) -- (-5.2112995100471853, 70.0770368224225848) -- (-5.2082070545929806, 70.0827237147293403) -- (-5.2014043205106280, 70.0837592602865698) -- (-5.1503821451003597, 70.0868511045808873) -- (-5.1448161705606967, 70.0837592602865698) -- (-5.1104930319643840, 70.0388361337554528) -- (-5.0767877165421078, 69.9666131037944723) -- (-5.0755509569992947, 69.9593901611832649) -- (-5.0755509569992947, 69.9521678786899201) -- (-5.0770971847264903, 69.9444377422045704) -- (-5.0801896401806390, 69.9387737860242709) -- (-5.0888469569797321, 69.9300095581513688);
\draw[uk] (-4.6234702597438675, 69.9057909760740870) -- (-4.6955184605751130, 69.8589054918475370) -- (-4.8151869131779046, 69.7564987295826455) -- (-4.8439440772344966, 69.7184525718326285) -- (-4.9005322271843044, 69.7210230822816186) -- (-4.9858764279958532, 69.8151549613515527) -- (-5.0031932879837226, 69.8388231323254445) -- (-5.0845177419827694, 69.9944640898785053) -- (-5.0863734378942835, 70.2077994553168736) -- (-5.0838999188088447, 70.2140078896493947) -- (-5.0761687801732736, 70.2238261832096384) -- (-4.9812388580094167, 70.2647178425278724) -- (-4.9305261507836553, 70.2755786533527385) -- (-4.9237234167013000, 70.2766167329784679) -- (-4.9159933912606499, 70.2766167329784679) -- (-4.8022003813819047, 70.2522934781159165) -- (-4.7783891423012905, 70.2440066879090921) -- (-4.7684939527645982, 70.2362460637690731) -- (-4.7610733955083289, 70.2269384201685085) -- (-4.7533422568727595, 70.2088366432553386) -- (-4.7001560305615593, 70.0842705310542726) -- (-4.5940941593184412, 70.0636215436610286) -- (-4.5888376529631616, 70.0662014424396062) -- (-4.5817254506964353, 70.0672219011699724) -- (-4.5752321847984634, 70.0667125995359612) -- (-4.5195713262068518, 70.0543193664251760) -- (-4.5047290984993955, 70.0501918181619772) -- (-4.5044207435099093, 70.0434871027729713) -- (-4.5529671734448556, 69.9604240911487238) -- (-4.5656453502512573, 69.9470137517094201) -- (-4.6234702597438675, 69.9057909760740870);
\draw[uk] (-7.0724979440010349, 73.7302121538367743) -- (-6.9630341491192604, 73.6798873069975571) -- (-6.8665579992283856, 73.6928699654771862) -- (-6.8455308606123761, 73.6977419466528261) -- (-6.8374902537923701, 73.6961237605069641) -- (-6.7948181533866157, 73.6766497707508705) -- (-6.7889438238574771, 73.6734123703486290) -- (-6.7533839257184081, 73.6458367050812370) -- (-6.7468895466255940, 73.6344788169192555) -- (-6.7209142566439137, 73.5820658320416499) -- (-6.7165850416468853, 73.5691173504016973) -- (-6.7162766866574666, 73.5610188485555767) -- (-6.7264802311835776, 73.4676597268901617) -- (-6.7342113698191373, 73.4374616953618613) -- (-6.7193691421116259, 73.3803434692123631) -- (-6.5060064136928819, 73.0674569230986890) -- (-6.4469447446576362, 72.9811214649410687) -- (-6.4364317319471525, 72.9880847309429726) -- (-6.4293195296802930, 72.9891594719649248) -- (-6.3974699101694723, 72.9854046966363086) -- (-6.3857190247212676, 72.9789722207710412) -- (-6.3807719865504389, 72.9752198682372040) -- (-6.3569618606647316, 72.9457591220409540) -- (-6.3541788733948996, 72.9404020727488813) -- (-6.3371725947863213, 72.9055883468396502) -- (-6.3319149752361987, 72.8938105887541354) -- (-6.3302140134169340, 72.8266319852239548) -- (-6.3300592793246766, 72.8146294135640630) -- (-6.3312960388673671, 72.8071388556083718) -- (-6.3346979625060316, 72.7975067262687787) -- (-6.3377904179602469, 72.7894885635974447) -- (-6.3411912284040595, 72.7836150058597013) -- (-6.3575807970335525, 72.7649118803198149) -- (-6.3925840977184913, 72.7385100494529411) -- (-6.3959236824423993, 72.7365886930866026) -- (-6.4001905585244288, 72.7359424818155986) -- (-6.4198573729628965, 72.7359424818155986) -- (-6.4407598337492056, 72.7275056873325099) -- (-6.4747746173559886, 72.7098715758086769) -- (-6.4806500600800341, 72.7066743680499741) -- (-6.5261051477556737, 72.6741041908976655) -- (-6.5356919823027564, 72.6655569033496533) -- (-6.5434220077434206, 72.6559557783055823) -- (-6.5548634250071203, 72.6410213468755188) -- (-6.5582642354509328, 72.6346173652211888) -- (-6.5944430699586567, 72.5626252458072258) -- (-6.5956798295014698, 72.5546178732590903) -- (-6.5771273231658185, 72.5066753328296016) -- (-6.5731075763532960, 72.5018777198215076) -- (-6.5659953740864365, 72.4997417076361756) -- (-6.5536255522695210, 72.5040137907153053) -- (-6.5489879822830730, 72.5082707677635767) -- (-6.5455860586444192, 72.5141216153487704) -- (-6.5397106159203719, 72.5354422948133788) -- (-6.5186845904994062, 72.5695510510068971) -- (-6.4747746173559886, 72.6042008622737285) -- (-6.4599323896485439, 72.6154085883551943) -- (-6.3904200468174919, 72.6380305794546075) -- (-6.3876359463527628, 72.6393136948495766) -- (-6.3831842799159171, 72.6399514187717585) -- (-6.2663611534978916, 72.6500923171150248) -- (-6.2329641930649791, 72.6495505648691449) -- (-6.2082267758209522, 72.6474258567264997) -- (-6.2023513330968507, 72.6442158518950123) -- (-6.1658630304045650, 72.6020773945240876) -- (-6.1618432835920425, 72.5972719671244135) -- (-6.1596792326910430, 72.5908715918637881) -- (-6.1534954349774544, 72.5551589637357353) -- (-6.0996902722974893, 72.3757465641085957) -- (-6.0505237927987459, 72.2859314726269417) -- (-6.0452672864435879, 72.2816866482608305) -- (-6.0443399950852790, 72.2747711397402384) -- (-6.0505237927987459, 72.2307037489979535) -- (-6.0517605523415599, 72.2227457613507227) -- (-6.0792820700503132, 72.0875327480730732) -- (-6.1235627370980730, 72.0476009365361563) -- (-6.1269023218219161, 72.0460072920672303) -- (-6.1315398918083623, 72.0456946670712313) -- (-6.1382190612559269, 72.0498256119554554) -- (-6.1423011469833275, 72.0561740066328724) -- (-6.1729140069515118, 72.1178432526730830) -- (-6.1742131054089571, 72.1216574618557331) -- (-6.1747697028630339, 72.1257981604609597) -- (-6.1751403967672642, 72.1353340532135690) -- (-6.1734706044054670, 72.1436133074174109) -- (-6.1719856023982960, 72.1477418120683609) -- (-6.1703169232312813, 72.1512368014109740) -- (-6.1584424731483516, 72.1662015573961639) -- (-6.1735941690402463, 72.1946327776669108) -- (-6.1689554858589020, 72.2084177023827039) -- (-6.1677187263160773, 72.2158507702469024) -- (-6.1711206499547426, 72.2620391868148317) -- (-6.1782317390266934, 72.3045196465163684) -- (-6.1803969031226016, 72.3108964894909292) -- (-6.1890553331164693, 72.3364090888652669) -- (-6.1912193840175345, 72.3422489316950390) -- (-6.1983315862842732, 72.3528773459226926) -- (-6.2335831294338000, 72.3863795566852986) -- (-6.2778014575666923, 72.4193694634128775) -- (-6.2997558875409965, 72.4331942638630011) -- (-6.3235660134267171, 72.4459695915053743) -- (-6.3371725947863213, 72.4502253069638300) -- (-6.3436647474895080, 72.4481041265556058) -- (-6.3458299115854162, 72.4411757237025284) -- (-6.3504674815718518, 72.3704286986644689) -- (-6.3183095070715014, 72.3502191493704458) -- (-6.3133624689006052, 72.3475629572975691) -- (-6.2840086323732827, 72.2921174758321854) -- (-6.2042114817879277, 72.0712883172326997) -- (-6.1984607168936368, 72.0519054644885131) -- (-6.1906683525381077, 72.0096577900071679) -- (-6.1925229352547237, 72.0061615234343009) -- (-6.1884363967477034, 71.9070889747031430) -- (-6.1918372071913934, 71.8922823172074317) -- (-6.2190492567157643, 71.8579459492132315) -- (-6.2431688507859864, 71.8288920508400395) -- (-6.2873871789188698, 71.7861341803120894) -- (-6.2932626216429703, 71.7829727875736410) -- (-6.2991369511721098, 71.7845534679413078) -- (-6.3031578111795410, 71.7892938010795945) -- (-6.3084143175348766, 71.8009055083681744) -- (-6.3278952284236780, 71.8162175129677962) -- (-6.3340801393321078, 71.8193916566658430) -- (-6.3418101647728928, 71.8209730746332156) -- (-6.3792257588232975, 71.8246720451384562) -- (-6.3965426188111651, 71.8257308269297283) -- (-6.4815773514383430, 71.8109395968941158) -- (-6.4871433259780051, 71.8082998224468838) -- (-6.5183751223148896, 71.7877130247355097) -- (-6.5276524886776572, 71.7798001242809960) -- (-6.5378560332037576, 71.7597564320121677) -- (-6.5387833245620648, 71.7439200073418561) -- (-6.5366192736610653, 71.7296815514854273) -- (-6.5298165395787233, 71.7112299139116800) -- (-6.5288892482204144, 71.7038447854441898) -- (-6.5276524886776572, 71.6801301669785715) -- (-6.5291976032098891, 71.6632665591996982) -- (-6.5511520331840716, 71.5758443567992515) -- (-6.5638302099905959, 71.5437453400720358) -- (-6.5718697036156968, 71.5337455058998444) -- (-6.5904244363410411, 71.5179699585191031) -- (-6.6439190176418492, 71.4764358974764349) -- (-6.7280264589105858, 71.4149532828921707) -- (-6.7546206852611528, 71.4081160387483038) -- (-6.7626601788862013, 71.4075960847200264) -- (-6.7703890911320759, 71.4091691925818708) -- (-6.7997663047523549, 71.4244046387076139) -- (-6.8693409864982069, 71.4869469920574829) -- (-6.8764520755700351, 71.4974594992190191) -- (-6.8860389101171187, 71.5211187542864053) -- (-6.8931511123839799, 71.5321698910731953) -- (-6.8980981505547438, 71.5358552129627441) -- (-6.9049008846372200, 71.5384839042752958) -- (-6.9763646581417023, 71.5444821208588735) -- (-6.9769501986632401, 71.5484843444543088) -- (-6.9825150600079962, 71.5542654852893634) -- (-6.9893177940904714, 71.5584709087092108) -- (-7.0004497431697894, 71.5611003604397524) -- (-7.0564189567507638, 71.5742697881845231) -- (-7.0684793103832959, 71.5742697881845231) -- (-7.0743536399125579, 71.5716379988097628) -- (-7.0799207276471190, 71.5668975909155165) -- (-7.0839393612647328, 71.5611003604397524) -- (-7.0857950571762576, 71.5553206085579632) -- (-7.0867234617294717, 71.5495242363596020) -- (-7.0864139935450234, 71.5437453400720358) -- (-7.1773241688963028, 71.4659110778357132) -- (-7.3109075578481750, 71.4517212755582847) -- (-7.3535807714488906, 71.4396411600454684) -- (-7.3560542905342743, 71.4354421204430992) -- (-7.3693514037096186, 71.4259761932989647) -- (-7.3749162650543738, 71.4296565156330843) -- (-7.3832663400587659, 71.4412149122176885) -- (-7.4089310486610884, 71.4790625924865424) -- (-7.4970593499373903, 71.7033228611207534) -- (-7.6272408152507865, 71.9007431920575613) -- (-7.6260040557080275, 71.9091935671282698) -- (-7.6476501306927345, 71.9985731638512618) -- (-7.6544528647750907, 72.0107552880336925) -- (-7.7002174206351119, 72.0647499265752458) -- (-7.7110399015301114, 72.0753557818916022) -- (-7.7639166596569291, 72.1182725565363683) -- (-7.8205036964118291, 72.1474211436751460) -- (-7.8266886073203805, 72.1490073198631734) -- (-7.8328724050338359, 72.1479594092299834) -- (-7.8573003540935344, 72.1341699132984928) -- (-7.9714028321566603, 71.9943439693719114) -- (-7.9877924007862227, 71.9039093591162839) -- (-7.9537776171793064, 71.8463257538275570) -- (-7.9509957431043832, 71.8405096505689897) -- (-7.9497578703667955, 71.8341711810092391) -- (-7.9547060217325241, 71.7824370353836230) -- (-8.1049873343034999, 71.5921527652666043) -- (-8.1117900683858526, 71.5863707173507180) -- (-8.1411661688113437, 71.5684720110599670) -- (-8.1600281433313810, 71.5695273260646161) -- (-8.4151356807975013, 71.5958409156538522) -- (-8.4714143625630385, 71.6253296331827300) -- (-8.4825463116423538, 71.6321877905362498) -- (-8.4865649452599676, 71.6374536407771529) -- (-8.4871838816288001, 71.6500840255570637) -- (-8.4859471220860438, 71.6643231549286952) -- (-8.4874933498131853, 71.6706345727760947) -- (-8.4912036284413244, 71.6769597813185442) -- (-8.5681999806382976, 71.7391824278839465) -- (-8.5743837783518853, 71.7391824278839465) -- (-8.6247870173932526, 71.7349539774515819) -- (-8.7002371418630950, 71.7328521791006466) -- (-8.7058031164027501, 71.7328521791006466) -- (-8.7401262549990726, 71.7576406714824628) -- (-8.7518771404471547, 71.7687120138182735) -- (-8.7580609381607424, 71.7803206601960682) -- (-8.7599166340722672, 71.7935270953576747) -- (-8.7602261022566488, 71.8352300916679809) -- (-8.7661004317859099, 71.8912226320808543) -- (-8.7670288363391258, 71.8975657377167607) -- (-8.7698118236089471, 71.9039093591162839) -- (-8.7945492408529731, 71.9134333705635953) -- (-8.8403137967129961, 71.9234648105225745) -- (-8.9392656920791929, 72.0335089879938550) -- (-8.9559636156981615, 72.0414495724894550) -- (-8.9677133879513455, 72.0509884924155841) -- (-8.9785358688463468, 72.0631635581097356) -- (-9.0799612832978571, 72.1887887214968629) -- (-9.0830526255571655, 72.1951560903659697) -- (-9.0966569805270208, 72.2370688139292838) -- (-8.9578193116096205, 72.3199290132569104) -- (-8.8508279226183877, 72.3677718164255168) -- (-8.7203358759258887, 72.4092567141278209) -- (-8.6297351687590496, 72.4869709600998249) -- (-8.6578733964468313, 72.5492896764626494) -- (-8.7379622040980731, 72.5610152620585325) -- (-8.7444554699960442, 72.5642218299825430) -- (-8.8013519749353168, 72.6212828739032119) -- (-8.8097009367448109, 72.6340738002136135) -- (-8.8248537458316161, 72.6874477576348141) -- (-8.8223802267462332, 72.6938560986459379) -- (-8.7308500018312802, 72.7542115639699034) -- (-8.7160066609288585, 72.7579551601936885) -- (-8.7101323313997767, 72.7558274826312470) -- (-8.7042568886756762, 72.7494140227861550) -- (-8.6152012960410111, 72.7098715758086769) -- (-8.4083329467151451, 72.8103421308553180) -- (-8.3298582716804948, 72.9397406628976483) -- (-8.3022699622771867, 72.9934431480368460) -- (-8.2865004432113576, 73.0411719854131150) -- (-8.2447545209689483, 73.1780552643971305) -- (-8.1556989283343526, 73.3560981568168415) -- (-8.0734538621464296, 73.4052055254939120) -- (-8.0762290570519308, 73.3604021916643774) -- (-8.0719009552499319, 73.3577092336918213) -- (-7.8981178717827270, 73.3399485301507497) -- (-7.8743077458970188, 73.3415592705125334) -- (-7.8637947331864035, 73.3469422167251253) -- (-7.8538995436498444, 73.3544851748558528) -- (-7.8452422268508739, 73.3636420084030050) -- (-7.8248329114087358, 73.3884230512439757) -- (-7.8143198986982414, 73.4045828096504067) -- (-7.7858710896311782, 73.4822204164690618) -- (-7.7555676978474439, 73.5712675034377099) -- (-7.6696045606670520, 73.5939488497813556) -- (-7.5972457784564593, 73.5944991958542687) -- (-7.4921111985717168, 73.6182708900636271) -- (-7.3619297332584415, 73.6707084761627584) -- (-7.2530837615505854, 73.7210117505796063) -- (-7.0981648789931722, 73.7329161862808036) -- (-7.0724979440010349, 73.7302121538367743);
\draw[uk] (-6.8715050373992055, 73.8379695210110185) -- (-6.8807812905670085, 73.7832600602456523) -- (-6.8848021505744290, 73.7762111475643678) -- (-6.8888218973869506, 73.7729695658108824) -- (-6.9401524277866375, 73.8347097355013062) -- (-6.9716925791131965, 73.8325522273712949) -- (-6.9883916159270738, 73.8336387768109006) -- (-6.9936470090873781, 73.8363464388214368) -- (-6.9924102495446885, 73.8444738975947672) -- (-6.9716925791131965, 73.8591058736825801) -- (-6.9602522750443372, 73.8666990499032323) -- (-6.9451005791524985, 73.8683227714137871) -- (-6.8919143528411642, 73.8563974120414315) -- (-6.8776899483076344, 73.8509788231123423) -- (-6.8736702014951128, 73.8466317394497622) -- (-6.8715050373992055, 73.8379695210110185);
\draw[uk] (-5.6813148904644057, 74.1224448039558865) -- (-5.7573828381081613, 74.1045051935063412) -- (-5.8148982794162780, 74.1055818828855166) -- (-5.8238661775945939, 74.1061339367365690) -- (-5.8297405071237227, 74.1093895688870532) -- (-5.8813805057079147, 74.1387610877897743) -- (-5.8918935184184091, 74.1469342381647181) -- (-5.9447713897400662, 74.1937355339982787) -- (-5.9469354406411314, 74.1980885397213257) -- (-5.9636333642600468, 74.2509224264682217) -- (-6.0115619710211901, 74.4402062352270093) -- (-6.0121809073898991, 74.4565901109882873) -- (-6.0115619710211901, 74.4740711477487309) -- (-6.0072327560242273, 74.4959241063293405) -- (-5.9902253642208096, 74.5741306596869578) -- (-5.9868245537770530, 74.5801401530546002) -- (-5.9787839469570994, 74.5905411749573517) -- (-5.9333299724763666, 74.6354196239535099) -- (-5.9283818211106398, 74.6398095599809608) -- (-5.8946776188831365, 74.6688284865043670) -- (-5.8615901266346091, 74.6699249460014158) -- (-5.8464384307427695, 74.6677438747256588) -- (-5.8322140262092379, 74.6628100707128368) -- (-5.8201536725766401, 74.6562361024135583) -- (-5.7867589385335307, 74.6343354844502755) -- (-5.7722250658156176, 74.6217349649787991) -- (-5.7530536231111338, 74.6047759187345179) -- (-5.7443951931172670, 74.5949147529115777) -- (-5.7382113954036775, 74.5834316876255485) -- (-5.6609066882172856, 74.3539575057537547) -- (-5.6593593472951813, 74.3468698871487703) -- (-5.6491546895741722, 74.1708769880291214) -- (-5.6488463345846975, 74.1627012938232753) -- (-5.6503925623118381, 74.1550925857417980) -- (-5.6537933727555281, 74.1491005721521361) -- (-5.6698734732007061, 74.1284367001547935) -- (-5.6813148904644057, 74.1224448039558865);
\draw[uk] (-5.5842186910097880, 74.6792395677309884) -- (-5.5916403614609766, 74.6781410027926000) -- (-5.5996798550860767, 74.6797957592112738) -- (-5.6046268932569072, 74.6836182055192950) -- (-5.6983222822678226, 74.7883519927787290) -- (-5.7017230927114468, 74.7943782960279009) -- (-5.7870672935231404, 74.9800569936915053) -- (-5.7892324576189802, 74.9866588012657900) -- (-5.7951067871481188, 75.0152626494628691) -- (-5.7960351917013249, 75.0229709983748876) -- (-5.7957257235170072, 75.0301213791431962) -- (-5.7682053190030951, 75.0670151985259935) -- (-5.7629488126478261, 75.0708708918512997) -- (-5.7558366103809666, 75.0686625172509565) -- (-5.7447046613016495, 75.0615105170241748) -- (-5.6553396004826588, 74.9855558201984707) -- (-5.6064825891684311, 74.9080463499832234) -- (-5.5758697292002370, 74.7998657846959105) -- (-5.5721594505721086, 74.7861517497251640) -- (-5.5718510955826215, 74.7696973561821210) -- (-5.5737056782992385, 74.6962308683747551) -- (-5.5761791973846204, 74.6891115798903087) -- (-5.5842186910097880, 74.6792395677309884);
\draw[uk] (-6.8146085324598671, 74.9894074120952183) -- (-6.7938908620283085, 74.8553019687044952) -- (-6.7926541024855505, 74.8476243138152029) -- (-6.7877070643147199, 74.8267558128109727) -- (-6.7830683811333765, 74.8135802982891533) -- (-6.7775024065936558, 74.8015073668317285) -- (-6.7632780020601242, 74.7888931753447963) -- (-6.7453433188984535, 74.7784671952330626) -- (-6.7373027120785105, 74.7685994690670412) -- (-6.7348291929930602, 74.7625574220919731) -- (-6.7128747630187462, 74.6765020599347764) -- (-6.7066909653052234, 74.5982069212986545) -- (-6.7097823075645326, 74.5921802524453028) -- (-6.7218426611971305, 74.5768645170390556) -- (-6.7645147616028156, 74.5374732900253179) -- (-6.9339675168781403, 74.4238120934865179) -- (-6.9846802241040251, 74.3856093732402570) -- (-7.0137468563450778, 74.3937930887875751) -- (-7.0323004758755712, 74.4030742609821374) -- (-7.0415767290433751, 74.4128950917924357) -- (-7.0437407799443728, 74.4183533986949897) -- (-7.0375569822307851, 74.4691497783805261) -- (-7.0363202226880945, 74.4762624584561337) -- (-6.9973572877155190, 74.6502186745265561) -- (-6.9608700982181961, 74.7614734581643603) -- (-6.9587060473171967, 74.7685994690670412) -- (-6.9590155155015792, 74.7768241316668281) -- (-6.9624163259453375, 74.7834084471687390) -- (-6.9676728323005408, 74.7872518633570280) -- (-6.9744755663830169, 74.7894363368714750) -- (-7.0412672608589242, 74.7965787752546731) -- (-7.0483794631257188, 74.7965787752546731) -- (-7.0551821972080724, 74.7943782960279009) -- (-7.0638406272019409, 74.7839515698293553) -- (-7.0669319694613817, 74.7768241316668281) -- (-7.0715717658376329, 74.7636571971505361) -- (-7.0758998676395652, 74.7504925238251161) -- (-7.0805385508209762, 74.7351471215254577) -- (-7.0882685762616378, 74.7165054253556633) -- (-7.1018751576213681, 74.6912932589130634) -- (-7.1167173853288226, 74.6699249460014158) -- (-7.1365066512071111, 74.6436457851954600) -- (-7.1794893329922100, 74.5971095161421260) -- (-7.1940232057101241, 74.5845151081038296) -- (-7.2060813329529623, 74.5768645170390556) -- (-7.2190700911386427, 74.5741306596869578) -- (-7.2261811802106068, 74.5757674170504004) -- (-7.2373142424848211, 74.5834316876255485) -- (-7.2413328761024456, 74.5899994202646752) -- (-7.2459715592837908, 74.6009416365464517) -- (-7.2468988506420979, 74.6173598580372186) -- (-7.2357669015627160, 74.6781410027926000) -- (-7.2305103952074443, 74.7022499345640227) -- (-7.1797976879816838, 74.9294811654025210) -- (-7.1708320161932395, 74.9393696349386431) -- (-7.0907420953470890, 74.9707057372927324) -- (-7.0239515140660878, 74.9591662416753763) -- (-6.8931511123839799, 75.0637028274236684) -- (-6.8606825565042158, 75.0857404231464898) -- (-6.8371818988028821, 75.0978582540880382) -- (-6.8313064560788455, 75.0995082201339557) -- (-6.8232669624537348, 75.0978582540880382) -- (-6.8149180006441954, 75.0884925946366053) -- (-6.8112066088211458, 75.0741677132789391) -- (-6.8158452920025017, 75.0213246628295849) -- (-6.8146085324598671, 74.9894074120952183);
\draw[uk] (-6.8950068082955038, 75.2875816067446237) -- (-6.9018095423778592, 75.2853808811344294) -- (-6.9469551618691145, 75.2931165030230574) -- (-6.9720020472975124, 75.3013997151634698) -- (-6.9398429596023110, 75.3831804928181270) -- (-6.9336591618886656, 75.3942366762459102) -- (-6.9135593146311018, 75.4202451230086979) -- (-6.9045914164527824, 75.4296495299580840) -- (-6.8779994164920844, 75.4545443353571272) -- (-6.8724334419524205, 75.4589695557984754) -- (-6.8337799751642274, 75.4717086888353208) -- (-6.8248131901808744, 75.4700367183474867) -- (-6.8235764306381830, 75.4628450720732218) -- (-6.8773804801232510, 75.3074685431678006) -- (-6.8903681251141480, 75.2909017011591857) -- (-6.8950068082955038, 75.2875816067446237);
\draw[uk] (-6.6201088917560096, 74.8926535461730367) -- (-6.6479387644543717, 74.8102946762800798) -- (-6.7094739525750544, 74.8168818679460941) -- (-6.7153482821042623, 74.8196263689391401) -- (-6.7382311166316597, 74.8339037066244828) -- (-6.7493630657109760, 74.8410344790385267) -- (-6.7577120275204621, 74.8498124856747467) -- (-6.7611128379642835, 74.8564030754953649) -- (-6.7632780020601242, 74.8640954496411837) -- (-6.7672977488726485, 74.8860598371952335) -- (-6.7771929384093275, 74.9866588012657900) -- (-6.7768834702249441, 74.9949069102666215) -- (-6.7731731915968156, 75.0196763816080647) -- (-6.7669882806882633, 75.0422304051383549) -- (-6.6436095494573326, 75.2765388573250362) -- (-6.6300029680977275, 75.2903410523402812) -- (-6.5749621590699130, 75.3455887014666672) -- (-6.5697067659095385, 75.3505667103738404) -- (-6.4667351237308308, 75.4479020050150382) -- (-6.4064366951528937, 75.4949586523851366) -- (-6.3953058592684755, 75.5032637848129724) -- (-6.3578902652179359, 75.5287450086855188) -- (-6.3513958861250677, 75.5298558557390294) -- (-6.3390271775029374, 75.5226446126277864) -- (-6.3285130515975467, 75.4800112920444803) -- (-6.3278952284236780, 75.4722567205029122) -- (-6.3278952284236780, 75.4633930401195556) -- (-6.3297498111402932, 75.4545443353571272) -- (-6.3350074306904149, 75.4418240020565349) -- (-6.3479939624863482, 75.4263234002121976) -- (-6.3541788733948996, 75.4207907939216824) -- (-6.3640740629314587, 75.4075125810060740) -- (-6.3789162906389141, 75.3848366286851501) -- (-6.3965426188111651, 75.3560922423725543) -- (-6.4030358847091371, 75.3428252705841146) -- (-6.5075526414200793, 75.1292593258437336) -- (-6.5724886399844635, 74.9965705021248681) -- (-6.6201088917560096, 74.8926535461730367);
\draw[uk] (-6.3841739101891593, 75.5442524602054419) -- (-6.3925217588037464, 75.5442524602054419) -- (-6.3987066697122312, 75.5475837323260038) -- (-6.4008707206132307, 75.5536721574928549) -- (-6.3943774547152596, 75.5664291030577004) -- (-6.3847917333629614, 75.5836026732639539) -- (-6.3356252538643396, 75.6168792342150624) -- (-6.3291319879663686, 75.6190875388017361) -- (-6.3226387220684082, 75.6174263127579849) -- (-6.3176916838975776, 75.6118818333803375) -- (-6.3161443429755284, 75.6063377569595900) -- (-6.3158359879860528, 75.5980233957838266) -- (-6.3158359879860528, 75.5741772850843887) -- (-6.3183095070715014, 75.5669758199501160) -- (-6.3288225197819852, 75.5592124081484258) -- (-6.3479939624863482, 75.5509031804494953) -- (-6.3600543161189362, 75.5470211959309665) -- (-6.3841739101891593, 75.5442524602054419);
\draw[uk] (-6.2672873316613016, 75.6218609883801776) -- (-6.2762552298396201, 75.6013430428464090) -- (-6.2830579639219630, 75.6035668679760420) -- (-6.2873871789188698, 75.6074497342233087) -- (-6.3043945707223541, 75.6529289173210202) -- (-6.3065586216233527, 75.6595891190419252) -- (-6.2830579639219630, 75.7400929078758054) -- (-6.2805844448365145, 75.7484387748825583) -- (-6.2444056103286778, 75.7428847616749437) -- (-6.2410036866900800, 75.7367652157195437) -- (-6.2415536049745990, 75.7278867479797952) -- (-6.2672873316613016, 75.6218609883801776);
\draw[uk] (-6.2524451039537787, 75.7656533866854858) -- (-6.2592478380361225, 75.7623265789117966) -- (-6.2682157362145077, 75.7639889644045610) -- (-6.2734722425697873, 75.7678820090119700) -- (-6.2873871789188698, 75.8140215277136917) -- (-6.2846041916490361, 75.8301497315207769) -- (-6.2821295593687472, 75.8368254364714147) -- (-6.2682157362145077, 75.8601890024386023) -- (-6.2583205466778260, 75.8696478168019866) -- (-6.2499715848683302, 75.8701987157562030) -- (-6.2224511803544837, 75.8707636417308038) -- (-6.2159568012615489, 75.8674162159784942) -- (-6.2156473330772322, 75.8590733323983386) -- (-6.2175041421836532, 75.8507293596652090) -- (-6.2372934080619400, 75.7862156007997498) -- (-6.2434783189703698, 75.7745522609161526) -- (-6.2524451039537787, 75.7656533866854858);
\draw[uk] (-7.1297050303196636, 75.8351454424441158) -- (-7.1355804730437091, 75.8318155995208230) -- (-7.1448567262115583, 75.8323662249581787) -- (-7.1612451816461560, 75.8446031502101050) -- (-7.1649554602742951, 75.8490650807681135) -- (-7.1426915621155276, 75.8907891677579869) -- (-7.1176446766871306, 75.9102827979246086) -- (-7.1108419426047780, 75.9124996200001192) -- (-7.1043486767066941, 75.9097175787501897) -- (-7.1031119171640018, 75.9030354827824141) -- (-7.1123881703317959, 75.8735462637921643) -- (-7.1170257403182440, 75.8596221518843521) -- (-7.1256852835070870, 75.8401575008543176) -- (-7.1297050303196636, 75.8351454424441158);
\draw[uk] (-6.1711206499547426, 75.9938370610467331) -- (-6.1813241944809096, 75.9843628086860292) -- (-6.2165757376304489, 75.9927194311775622) -- (-6.2190492567157643, 75.9982936980858881) -- (-6.2202860162585765, 76.0066518501982102) -- (-6.2178124971730728, 76.0127931183008627) -- (-6.1980232312947852, 76.0406658516274234) -- (-6.1735941690402463, 76.0473600396366010) -- (-6.1337039427092961, 76.0507153866123815) -- (-6.1630800431347987, 76.0038663664465872) -- (-6.1711206499547426, 75.9938370610467331);
\draw[uk] (-6.8637738987636459, 76.1584462206108981) -- (-6.9005727828351668, 76.1461587413685947) -- (-6.9382978450700197, 76.1589972030208600) -- (-6.9447899977731415, 76.1612373696956126) -- (-6.9673633641161574, 76.1847007461606580) -- (-6.9686001236588471, 76.1930794043853439) -- (-6.9670550091267467, 76.2003403137725286) -- (-6.9481919214118060, 76.2187924608915353) -- (-6.9426259468721421, 76.2227088361655234) -- (-6.9293299468917038, 76.2277303833079003) -- (-6.9191252891707515, 76.2249347796612966) -- (-6.9073755169175097, 76.2187924608915353) -- (-6.8603730883198892, 76.1796659485985685) -- (-6.8569711646811689, 76.1746334956870896) -- (-6.8548071137802360, 76.1679421766566520) -- (-6.8578995692343856, 76.1606702835633200) -- (-6.8637738987636459, 76.1584462206108981);
\draw[uk] (-7.5020063881083416, 76.2612896762001355) -- (-7.5236524630931161, 76.2579250260375829) -- (-7.5409682098858886, 76.2612896762001355) -- (-7.5579756016893072, 76.2601681095962363) -- (-7.5650878039561107, 76.2590465595425258) -- (-7.5715810698541386, 76.2556981162834688) -- (-7.5765281080249007, 76.2523337136313586) -- (-7.6547623329595966, 76.1791128049883497) -- (-7.6609461306730626, 76.1668219863117741) -- (-7.6730053711108086, 76.1059482683596826) -- (-7.7088747374342059, 76.1087395011232104) -- (-7.7614420273765825, 76.1271694326693193) -- (-7.7821608110029832, 76.2182249453428824) -- (-7.7796861787226268, 76.2255043579541933) -- (-7.7626787869192739, 76.2456214605226137) -- (-7.7456725083108173, 76.2556981162834688) -- (-7.5295267926221880, 76.3267618658396003) -- (-7.5230335267242170, 76.3284386430926389) -- (-7.5168497290106382, 76.3250710185583472) -- (-7.5081901858218645, 76.3144471514677747) -- (-7.4955131222103688, 76.2914976565236458) -- (-7.4908744390290254, 76.2786206886988083) -- (-7.4905649708445079, 76.2702305835140066) -- (-7.5020063881083416, 76.2612896762001355);
\draw[uk] (-6.4382863146637703, 76.2473105409907106) -- (-6.3616005438460759, 76.1869255771360798) -- (-6.2907891025575218, 76.1310810918127174) -- (-6.2883155834720856, 76.1165591153618664) -- (-6.2876966471033180, 76.1003681194068662) -- (-6.2991369511721098, 76.0345383948991156) -- (-6.3040851025378482, 76.0194848549226521) -- (-6.3473761393125434, 75.9464715699767652) -- (-6.3616005438460759, 75.9320008139073934) -- (-6.4661173005569061, 75.8629692351414633) -- (-6.5013688437064339, 75.8457186083498698) -- (-6.5087894009627147, 75.8440524363249011) -- (-6.5233232736807505, 75.8490650807681135) -- (-6.5366192736610653, 75.8562912389677848) -- (-6.5434220077434206, 75.8663004400833216) -- (-6.5360003372922444, 75.8785448215054856) -- (-6.5298165395787233, 75.8824417202947501) -- (-6.5230138054962445, 75.8930214534883163) -- (-6.5165194264033648, 75.9063844743914871) -- (-6.5146637304918524, 75.9136160729107843) -- (-6.5149743118712653, 75.9236348155026235) -- (-6.5174478309565806, 75.9275340222263395) -- (-6.5236316286702243, 75.9320008139073934) -- (-6.5319805904797219, 75.9331015133170553) -- (-6.5483701591091483, 75.9320008139073934) -- (-6.5598104631780725, 75.9264173652461380) -- (-6.6609264094452012, 75.8651866836754465) -- (-6.6971041307581398, 75.8373760981176304) -- (-6.9738577432090354, 75.7456607176914218) -- (-7.0366296908724770, 75.7562054719603282) -- (-7.0409589058694504, 75.7612123408207196) -- (-7.0570378931195954, 75.7806568332724453) -- (-7.0644584503759331, 75.7912224404325343) -- (-7.0669319694613817, 75.7967799773801261) -- (-7.0867234617294717, 75.8457186083498698) -- (-7.0870318167189472, 75.8540600293913769) -- (-7.0839393612647328, 75.8880059157035305) -- (-7.0690971335572774, 75.9019191849718169) -- (-7.0570378931195954, 75.9063844743914871) -- (-7.0106544008907417, 75.9097175787501897) -- (-7.0026149072656967, 75.9080500062000141) -- (-6.9927197177290719, 75.9002517831682582) -- (-6.9886999709165467, 75.8946866930817805) -- (-6.9818972368342047, 75.8830067370800521) -- (-6.9373694405168065, 75.8646358209678198) -- (-6.7524555211651904, 75.9381200221081372) -- (-6.7094739525750544, 75.9587288495092992) -- (-6.7039079780354038, 75.9631814258916620) -- (-6.6998882312228147, 75.9676483035859889) -- (-6.6989598266695971, 75.9776741589277833) -- (-6.7005060543967376, 75.9849125318917658) -- (-6.7035985098509530, 75.9904862271094999) -- (-6.7097823075645326, 75.9938370610467331) -- (-6.7175134462001012, 75.9927194311775622) -- (-6.7669882806882633, 75.9704344682376700) -- (-6.7957465579398333, 75.9537250260061256) -- (-6.8081152665619076, 75.9481399841546221) -- (-6.8674864037815899, 75.9403536957143501) -- (-6.8851105055639161, 75.9414705570901987) -- (-6.8946962269160901, 75.9509254349658676) -- (-6.9002622014558765, 75.9637470420730523) -- (-6.9052103528216042, 75.9843628086860292) -- (-6.9021178973674004, 76.0016448757448728) -- (-6.8384186583455735, 76.0847490937895827) -- (-6.7716269638696751, 76.1143202642140579) -- (-6.7589487870631517, 76.1210209362474899) -- (-6.7147304589302683, 76.1478295288954712) -- (-6.7029795734821889, 76.1556531631595419) -- (-6.6933938521300114, 76.1640306322514107) -- (-6.6869005862319177, 76.1774252389834885) -- (-6.6853543585049016, 76.1847007461606580) -- (-6.6816440798767607, 76.2092901525502242) -- (-6.6822619030505628, 76.2176735335836497) -- (-6.6862827630581156, 76.2215878386657977) -- (-6.6988629787126639, 76.2224794026568873) -- (-6.7892521788469526, 76.1914076223749959) -- (-6.8105887856473437, 76.1757538011093658) -- (-6.8198661520100448, 76.1768741230373365) -- (-6.8260499497235658, 76.1796659485985685) -- (-7.0412672608589242, 76.3088516791131042) -- (-7.0551821972080724, 76.3233943156236876) -- (-7.0387937417735431, 76.4309575941080084) -- (-7.0353929313297865, 76.4371170899127037) -- (-6.8869662014754276, 76.5178926752228961) -- (-6.8356356710757522, 76.5375336399801114) -- (-6.8282140006246310, 76.5386593101753050) -- (-6.8090425579202138, 76.5324915699458188) -- (-6.7558574448038451, 76.5094760974351686) -- (-6.7425603316286216, 76.5032965880763243) -- (-6.7230794207397544, 76.4769317551981516) -- (-6.6748413457942926, 76.3973232906419355) -- (-6.6714394221556956, 76.3905982171075948) -- (-6.6507217517241930, 76.3317902931527215) -- (-6.6470103599011559, 76.3189206018485464) -- (-6.5193024136731967, 76.2680033131862842) -- (-6.4571494023786435, 76.2568196169284249) -- (-6.4497277319274655, 76.2551303251388362) -- (-6.4382863146637703, 76.2473105409907106);
\draw[uk] (-7.3350282651132952, 76.3905982171075948) -- (-7.4342874022740020, 76.3413099469080976) -- (-7.4426363640834881, 76.3435554932701734) -- (-7.4522220854356735, 76.3693195215303859) -- (-7.4571702368015353, 76.3905982171075948) -- (-7.4540777813473191, 76.3973232906419355) -- (-7.3625486696272739, 76.5004928445504930) -- (-7.3180208733099441, 76.5386593101753050) -- (-7.2867890769729824, 76.5644896414233074) -- (-7.2465904956526224, 76.5970715618232134) -- (-7.2410245211129602, 76.6004371152129124) -- (-7.2283463443064457, 76.6071808067111704) -- (-7.1974240161539349, 76.6105609688139708) -- (-7.1881477629860768, 76.6111152407673188) -- (-7.1850553075319281, 76.6049274489809875) -- (-7.1835090798047334, 76.5976277577639877) -- (-7.1850553075319281, 76.5903287688645378) -- (-7.2190700911386427, 76.5089225674199440) -- (-7.2283463443064457, 76.4915227729773761) -- (-7.2373142424848211, 76.4814147867563321) -- (-7.3186386964838688, 76.4018016243124265) -- (-7.3235857346546895, 76.3973232906419355) -- (-7.3350282651132952, 76.3905982171075948);
\draw[uk] (-6.8402743542570974, 76.9778053222584191) -- (-6.8563533415073659, 76.9761161592698215) -- (-6.8773804801232510, 76.9840256697500536) -- (-6.9144877191843035, 77.0128166798444909) -- (-6.9169612382696863, 77.0195971731528317) -- (-6.9178885296279935, 77.0365592015838700) -- (-6.9181979978124435, 77.0450324862647875) -- (-6.9163423019009196, 77.0529492020103675) -- (-6.9101573909924356, 77.0659440959899911) -- (-6.8557344051385449, 77.1219213099549279) -- (-6.8458403287967595, 77.1304224914346150) -- (-6.8322348606320613, 77.1264589563699019) -- (-6.8056406342814828, 77.1049564266776457) -- (-6.7988379001991390, 77.0936486176362479) -- (-6.7994568365679715, 77.0817664486135072) -- (-6.8068773938243066, 77.0297771886027220) -- (-6.8124433683639589, 77.0060510281615649) -- (-6.8164642283713901, 77.0003900638570116) -- (-6.8337799751642274, 76.9806288088041413) -- (-6.8402743542570974, 76.9778053222584191);
\draw[uk] (-7.0326088308650583, 77.1083588728453435) -- (-7.0415767290433751, 77.1083588728453435) -- (-7.0471427035829608, 77.1111872345155405) -- (-7.0820847785481185, 77.1332496412865396) -- (-7.1782525734495195, 77.2402579381527232) -- (-7.1825817884464245, 77.2470588503886546) -- (-7.1819628520775920, 77.2549824604046478) -- (-7.1417642707572222, 77.3094175061196438) -- (-7.1170257403182440, 77.3320994602814267) -- (-7.1068221957921320, 77.3406079699686302) -- (-7.0660057912979708, 77.3508312945192813) -- (-7.0421956654121409, 77.3530906214004972) -- (-6.9877726795582404, 77.3309619333012677) -- (-6.9803510091069967, 77.3275657921563067) -- (-6.9704558195703701, 77.3196202759085196) -- (-6.9621079709558504, 77.3077188797336277) -- (-6.9571587063951483, 77.2952437205417908) -- (-6.9565408832212903, 77.2878628908719065) -- (-6.9546851873097646, 77.2334576376816671) -- (-6.9630341491192604, 77.1694680968417686) -- (-6.9651993132151580, 77.1620974977150524) -- (-6.9760217941100349, 77.1439888852869302) -- (-6.9890094391008759, 77.1292879948697703) -- (-7.0115828054439460, 77.1145736698381654) -- (-7.0248788054243931, 77.1089188352631538) -- (-7.0326088308650583, 77.1083588728453435);
\draw[uk] (-8.3121651518138115, 77.1219213099549279) -- (-8.3214414049815488, 77.1219213099549279) -- (-8.3659692012988831, 77.1264589563699019) -- (-8.3829765931022315, 77.1287115875144309) -- (-8.4049321362714569, 77.1355045116191747) -- (-8.4151356807975013, 77.1434286632028545) -- (-8.4163724403403144, 77.1519159632222653) -- (-8.4157535039714162, 77.1615391791359855) -- (-8.4086424148995960, 77.1739922928314428) -- (-8.2883550259279719, 77.3559281538542791) -- (-8.2818617600300133, 77.3536666992814190) -- (-8.2583611023286885, 77.3320994602814267) -- (-8.2540307741367513, 77.3269899132437075) -- (-8.2132143696425910, 77.2062625507945057) -- (-8.2110503187414565, 77.1989046739695084) -- (-8.2762957854903476, 77.1304224914346150) -- (-8.2812428236611773, 77.1258846072626767) -- (-8.3121651518138115, 77.1219213099549279);
\draw[uk] (-7.2348396102044079, 77.3491174379082622) -- (-7.3260603669350441, 77.3320994602814267) -- (-7.3498704928207514, 77.3372113952514582) -- (-7.3486326200831522, 77.3468561871822118) -- (-7.3396658350997432, 77.3547902680698485) -- (-7.2976115578678602, 77.3780621132219437) -- (-7.2908088237855049, 77.3814768737649672) -- (-7.2843155578875445, 77.3803386030992471) -- (-7.2354574333783983, 77.3695514279481813) -- (-7.2292736356647547, 77.3661535518068888) -- (-7.2268001165793052, 77.3593398710489879) -- (-7.2302020402179705, 77.3530906214004972) -- (-7.2348396102044079, 77.3491174379082622);
\draw[uk] (-6.6114504617621428, 77.8118531854577782) -- (-6.6191816003977015, 77.8112735990272313) -- (-6.6445368408156416, 77.8158528643902656) -- (-6.6572150176221658, 77.8215608930447900) -- (-6.6751497007838916, 77.8329782475107379) -- (-6.6884468139591142, 77.8472534016943598) -- (-6.6992692948540498, 77.8638203294584059) -- (-6.7005060543967376, 77.8729690721103509) -- (-6.6989598266695971, 77.8809586760084187) -- (-6.6958673712153942, 77.8866716388762512) -- (-6.6856638266892263, 77.8952429243363582) -- (-6.6655650926265571, 77.8992449746220075) -- (-6.6309313726509425, 77.8952429243363582) -- (-6.6238191703841487, 77.8923850350007143) -- (-6.6055750190380973, 77.8809586760084187) -- (-6.6012458040411355, 77.8769555399295967) -- (-6.5910422595149676, 77.8592546009630269) -- (-6.5885687404295181, 77.8523983013730003) -- (-6.5876403358763014, 77.8449792294615008) -- (-6.5882592722451347, 77.8352685049277966) -- (-6.5938252467848546, 77.8227058395763720) -- (-6.5990817531401351, 77.8175609111804789) -- (-6.6114504617621428, 77.8118531854577782);
\draw[uk] (-8.0406691589129160, 77.9341201640739456) -- (-8.0496370570913012, 77.9243979436532186) -- (-8.0663349807102680, 77.9141130646109730) -- (-8.0731377147926224, 77.9118204092677473) -- (-8.0805582720489575, 77.9129667282167304) -- (-8.1099343724743296, 77.9312607993888093) -- (-8.1130268279286000, 77.9032492953026292) -- (-8.0808677402332751, 77.5542391067038324) -- (-8.0313929057451130, 77.4763323904434031) -- (-8.0320118421139455, 77.4678085787008115) -- (-8.0437616143672521, 77.4405290285340158) -- (-8.0487086525380853, 77.4359888483003687) -- (-8.0879799425001355, 77.4541675717231328) -- (-8.1120995365702910, 77.4592877747964650) -- (-8.1300342197320301, 77.4592877747964650) -- (-8.1637384219595202, 77.4467855169074824) -- (-8.1887864205828933, 77.4450694817464012) -- (-8.1965164460235442, 77.4467855169074824) -- (-8.2169257614656264, 77.4643899125807422) -- (-8.2228000909947667, 77.4768950859026262) -- (-8.2292944700876340, 77.4979244874183308) -- (-8.2333131037052496, 77.5121527384751090) -- (-8.2571232295910892, 77.6157102046153824) -- (-8.2586694573180957, 77.6242507372539734) -- (-8.2608346214141264, 77.6892055884320030) -- (-8.2654721914005727, 78.0262440160854709) -- (-8.2623808491411452, 78.0325513597043283) -- (-8.2512489000618281, 78.0417041314266839) -- (-8.2020824205632632, 78.0657560757580882) -- (-8.1659035860553590, 78.0543033813937654) -- (-8.1105521956482534, 78.0239500054910167) -- (-8.0950921447669408, 78.0119369621768612) -- (-8.0898345252167534, 78.0056313453192018) -- (-8.0422153866401143, 77.9426968506536326) -- (-8.0406691589129160, 77.9341201640739456);
\draw[uk] (-8.0916902211282746, 78.0606130766587540) -- (-8.1077703215733301, 78.0577489043318451) -- (-8.1414756369957288, 78.0623266832705411) -- (-8.1488961942519964, 78.0657560757580882) -- (-8.2132143696425910, 78.1047235458689073) -- (-8.2243463187219081, 78.1121703182134439) -- (-8.2428999382523322, 78.1494420775411811) -- (-8.2481564446076145, 78.1626272189274403) -- (-8.2487753809765021, 78.1706620797702243) -- (-8.2447545209689483, 78.1953398987062656) -- (-8.2422810018835673, 78.2027935805551095) -- (-8.2379517868866046, 78.2079649030499837) -- (-8.1906421164944039, 78.2452696998363990) -- (-8.1832193328483207, 78.2475725356679277) -- (-8.1269417642778148, 78.2355072033326167) -- (-8.0997308279482851, 78.2292014242113680) -- (-8.0641698166143065, 78.2205920270204587) -- (-8.0202598434709547, 78.1884529235491783) -- (-8.0177863243855167, 78.1815665789081891) -- (-8.0174779693960332, 78.1018576943721143) -- (-8.0199514884814143, 78.0938301773798713) -- (-8.0589133102590935, 78.0703508404974542) -- (-8.0647876397882321, 78.0674883598644413) -- (-8.0916902211282746, 78.0606130766587540);
\draw[uk] (-6.7091644843906177, 77.9158253985024913) -- (-6.7360659525357534, 77.9072517709671217) -- (-6.7462706102567607, 77.9078176908558078) -- (-6.7527649893496404, 77.9112544593068748) -- (-6.7552385084350233, 77.9169717784019866) -- (-6.7676072170571526, 77.9575735212283121) -- (-6.7679155720465722, 77.9661508197944357) -- (-6.7663704575143395, 78.0508745106584882) -- (-6.7604950147902922, 78.1259309377057747) -- (-6.7518365847964246, 78.1683781178626731) -- (-6.7499820020798085, 78.1758311935699481) -- (-6.7144209907459533, 78.2527448710773825) -- (-6.7100928889438212, 78.2579030898510126) -- (-6.6825713712350794, 78.2872010420112758) -- (-6.6569055494377825, 78.2607748928445233) -- (-6.6535047389940241, 78.2538819229679916) -- (-6.6531952708096433, 78.2452696998363990) -- (-6.6680386117119390, 77.9598655104080080) -- (-6.6729856498827695, 77.9449883871234448) -- (-6.6760781053371074, 77.9387069023267713) -- (-6.6862827630581156, 77.9289840898718751) -- (-6.7091644843906177, 77.9158253985024913);
\draw[uk] (-6.6433000812730167, 78.3165104163019947) -- (-6.6473209412804470, 78.3009937255039290) -- (-6.6550498535262594, 78.3021459883114801) -- (-6.6668018521692485, 78.3193823898152885) -- (-6.6683469667014821, 78.3268504484222490) -- (-6.6689659030702462, 78.3538856863551558) -- (-6.6593801817180713, 78.3987361180088556) -- (-6.6417549667407174, 78.4200431873692168) -- (-6.6364973471905406, 78.4246430755950854) -- (-6.6300029680977275, 78.4223348081901008) -- (-6.6278389171966712, 78.4154270061188612) -- (-6.6272210940228033, 78.4085198391346410) -- (-6.6433000812730167, 78.3165104163019947);
\draw[uk] (-6.8393459497038807, 78.4102579656347132) -- (-6.8313064560788455, 78.2372434005458786) -- (-6.8424384051581635, 78.1024415847080320) -- (-6.8436762778957609, 78.0714851388768523) -- (-6.8291424051778469, 77.8838140746687628) -- (-6.8207934433683510, 77.8695340497263544) -- (-6.5669226654447437, 77.7376735757091524) -- (-6.5530077290955955, 77.7319718954035181) -- (-6.5455860586444192, 77.7336800957597660) -- (-6.4936377050708742, 77.7616306828399928) -- (-6.4218978592291052, 77.7998595319462112) -- (-6.4014896569820632, 77.8078557739433876) -- (-6.3931406951725673, 77.8072906043324934) -- (-6.2886239384616260, 77.7747480932427067) -- (-6.2852220148229625, 77.7678974226981268) -- (-6.3022294066263234, 77.6703916627422615) -- (-6.3050123938961571, 77.6618459048536351) -- (-6.3121257093579146, 77.6510184294057808) -- (-6.4605513260172422, 77.4808749971482769) -- (-6.4747746173559886, 77.4678085787008115) -- (-6.4865255028040814, 77.4604272359737536) -- (-6.4933282368864926, 77.4581483308136285) -- (-6.5094083373316032, 77.4575694098477783) -- (-6.5223948691275346, 77.4320092571431644) -- (-6.5557907163654292, 77.3729658037968875) -- (-6.5601199313624550, 77.3672752631116367) -- (-6.5990817531401351, 77.3213191698401943) -- (-6.6117599299465262, 77.3150873564203920) -- (-6.6569055494377825, 77.2975013859658731) -- (-6.6773148648797314, 77.2906959338364885) -- (-6.6924665607717042, 77.2878628908719065) -- (-6.6977230671269066, 77.2929697929817081) -- (-6.7150399271146508, 77.3400462889284341) -- (-6.7178218011895776, 77.3542284815866168) -- (-6.7162766866574666, 77.3616014954467488) -- (-6.6689659030702462, 77.4706504243653598) -- (-6.6504122835398096, 77.5087320638209150) -- (-6.7131842312031287, 77.6054673507050268) -- (-6.7611128379642835, 77.5007838414811090) -- (-6.7651336979716490, 77.4962236631788670) -- (-6.7744099511395079, 77.4979244874183308) -- (-6.7867786597615156, 77.5047486368687686) -- (-6.8758342523961122, 77.5912378925577713) -- (-7.0292080204212350, 77.5644770572306044) -- (-7.0335372354182617, 77.5684615890929052) -- (-7.1767063457223772, 77.7815973319738987) -- (-7.1828901434358992, 77.7941531465362885) -- (-7.2125757120457754, 77.8564001328915225) -- (-7.2150492311312222, 77.8632567966505036) -- (-7.2144314079572984, 77.8723890154243037) -- (-7.1535140430104631, 77.9243979436532186) -- (-7.1485670048396974, 77.9284036017106416) -- (-7.1389812834874551, 77.9289840898718751) -- (-7.1225928280528690, 77.9158253985024913) -- (-7.0953807785284324, 77.8866716388762512) -- (-7.0641500953864469, 77.8615312607244761) -- (-7.0524003231331482, 77.8535437165358104) -- (-7.0459059440402703, 77.8512529036209884) -- (-7.0288985522369183, 77.8518184030279912) -- (-7.0245693372398792, 77.8569636144201382) -- (-7.0239515140660878, 77.8655315206025449) -- (-7.2657619383569854, 78.0766453811016987) -- (-7.2728741406238351, 78.0800919269890841) -- (-7.3177102919306511, 78.0302550956568268) -- (-7.3177102919306511, 78.0130848061145628) -- (-7.3075067474044832, 77.9678643372072315) -- (-7.3062699878617270, 77.9604318257593434) -- (-7.3065794560461761, 77.9507082675400227) -- (-7.3105992028586879, 77.9341201640739456) -- (-7.3136916583129032, 77.9289840898718751) -- (-7.3198754560265593, 77.9255444540297049) -- (-7.3279149496516585, 77.9266909818576039) -- (-7.3749162650543738, 77.9478482730300328) -- (-7.4463466427116272, 77.9821794976683123) -- (-7.4584069963442259, 77.9884666698978606) -- (-7.4692294772391588, 77.9970495486484765) -- (-7.4815981858612224, 78.0136494610799360) -- (-7.4837622367621641, 78.0205204560529353) -- (-7.4859274008580714, 78.0365462840135251) -- (-7.4856179326736889, 78.0703508404974542) -- (-7.5394230953537758, 78.1201997920375959) -- (-7.5437523103506825, 78.1253633758984734) -- (-7.5462258294361320, 78.1310949148990090) -- (-7.5539547416818866, 78.1580433342940211) -- (-7.5527179821391961, 78.1660777055314213) -- (-7.5112826412761340, 78.2544649271507069) -- (-7.3888323145984289, 78.4879844080401909) -- (-7.3724427459689261, 78.4885526404997620) -- (-7.3647127205282743, 78.4862589985984300) -- (-7.3551269991759654, 78.4776201046542781) -- (-7.3319358096590799, 78.4528515222211240) -- (-7.3229679114807071, 78.4413397210576875) -- (-7.3161651773984193, 78.4303974764639946) -- (-7.3087435069471764, 78.4108276821169596) -- (-7.3075067474044832, 78.3849420549446876) -- (-7.3062699878617270, 78.3774537263021500) -- (-7.2994661405844754, 78.3653840947686859) -- (-7.1949504970684295, 78.2877843046162809) -- (-7.1578432580075013, 78.2797369211539404) -- (-7.0972364744399545, 78.3193823898152885) -- (-7.1198098407830379, 78.4822255939114228) -- (-7.1470218903074096, 78.5484962769470343) -- (-7.1482586498501002, 78.5542601760410122) -- (-7.1457851307647742, 78.5634766922074874) -- (-7.1324880175894965, 78.5779077946551041) -- (-7.0638406272019409, 78.6523329440293963) -- (-7.0177666031575345, 78.6552198672857230) -- (-6.9874632113738571, 78.6263639420930787) -- (-6.8810918719462899, 78.5064189003382040) -- (-6.8733607333107294, 78.4943119398353701) -- (-6.8606825565042158, 78.4707072168576474) -- (-6.8551165819645536, 78.4580380055036528) -- (-6.8418194687893301, 78.4246430755950854) -- (-6.8393459497038807, 78.4102579656347132);
\draw[uk] (-8.0023262735040692, 78.6575353057281887) -- (-7.8761634418085338, 78.5640641796838395) -- (-7.8675050118145426, 78.5542601760410122) -- (-7.8656504290979283, 78.5456134556220604) -- (-7.9565606044491393, 78.3004248410431813) -- (-7.9636739199109075, 78.2889218865459640) -- (-7.9797529071610551, 78.2860490082474030) -- (-8.0524200443611207, 78.2768664618265859) -- (-8.1427123965384212, 78.2780183548900084) -- (-8.1501329537946887, 78.2803056490861024) -- (-8.1553905733448762, 78.2854802368661069) -- (-8.1569356878771089, 78.3262813682421211) -- (-8.1544621687916603, 78.3607878328031688) -- (-8.1795090542201798, 78.3745774621487641) -- (-8.2506299636931182, 78.4102579656347132) -- (-8.2994869750073548, 78.4016298787006320) -- (-8.3130924431721205, 78.3975989440614143) -- (-8.3232971008930630, 78.3987361180088556) -- (-8.3335017586140694, 78.4062286396578969) -- (-8.3894698590002701, 78.4476633243092181) -- (-8.3937990739972328, 78.4516969461804905) -- (-8.3972009976358297, 78.4574534199361295) -- (-8.3972009976358297, 78.4672447917627238) -- (-8.3393760881433412, 78.5727109494242910) -- (-8.3310282395286315, 78.5842429096190642) -- (-8.3186584177117187, 78.5905972493168150) -- (-8.2738222664049648, 78.5871272147929432) -- (-8.0984940684055253, 78.5934797219288868) -- (-8.0941637402137250, 78.5998182282675799) -- (-8.0647876397882321, 78.6361773688956447) -- (-8.0119119948563799, 78.6569637141922442) -- (-8.0023262735040692, 78.6575353057281887);
\draw[uk] (-8.0112930584875457, 78.6887175447539278) -- (-8.0199514884814143, 78.6881457156141835) -- (-8.0369588802847769, 78.6910340172053822) -- (-8.0363399439160759, 78.7002673470356768) -- (-8.0292277416492173, 78.7124075777638126) -- (-8.0134571093885469, 78.7355162587310389) -- (-8.0035630330468823, 78.7441848903926598) -- (-7.9859367048746313, 78.7540116609449541) -- (-7.9778972112495312, 78.7563301507416043) -- (-7.9704766539932610, 78.7551708968840387) -- (-7.9534692621898335, 78.7332150572124192) -- (-7.9540870853638239, 78.7233889304809509) -- (-7.9593447049138799, 78.7187561387719654) -- (-7.9649106794536637, 78.7152947335226258) -- (-8.0112930584875457, 78.6887175447539278);
\draw[uk] (-9.5236930187437885, 78.9149046411357489) -- (-9.5462652718920289, 78.8894179301938436) -- (-9.5539952973326816, 78.8899912963078975) -- (-9.5802800554988021, 78.9021602021833957) -- (-9.5867744345916819, 78.9056448432945814) -- (-9.5972863341072667, 78.9398283578619697) -- (-9.5941949918479477, 78.9467846142467380) -- (-9.5552331700702808, 78.9421359106192568) -- (-9.5351333228127153, 78.9334443597611539) -- (-9.5230751955699855, 78.9259177822615356) -- (-9.5202910951052591, 78.9212870598062040) -- (-9.5236930187437885, 78.9149046411357489);
\draw[uk] (-7.7994776709908518, 79.0622079034944392) -- (-7.8743077458970188, 79.0546662876272421) -- (-7.8789464290783640, 79.0593052631700175) -- (-7.8777096695355509, 79.0860142175524743) -- (-7.8743077458970188, 79.0929861559098129) -- (-7.8288526582213791, 79.1336370169489243) -- (-7.8229772154972652, 79.1365425407012140) -- (-7.8004049623491607, 79.1423518345718975) -- (-7.7917465323551580, 79.1429417844832699) -- (-7.7852521532622898, 79.1400381042684842) -- (-7.7858710896311782, 79.1301584856665130) -- (-7.7895813682593174, 79.0819484242790907) -- (-7.7911275959864588, 79.0743901891790415) -- (-7.7994776709908518, 79.0622079034944392);
\draw[uk] (-7.5721988930280624, 79.7374498922462323) -- (-7.6130164107171314, 79.7351155567531151) -- (-7.6467217261395311, 79.7392191436719173) -- (-7.6510509411365586, 79.7444807530425521) -- (-7.6686761561138459, 79.8247794993749693) -- (-7.6699129156565373, 79.8335753057036612) -- (-7.6696045606670520, 79.8417914328077529) -- (-7.6606366624886784, 79.8453201476289394) -- (-7.6368265366029613, 79.8388583552639233) -- (-7.6300238025206175, 79.8365229562744361) -- (-7.5805489680324554, 79.7913644688050425) -- (-7.5635415762289693, 79.7690855454977310) -- (-7.5601396525903724, 79.7456554395402719) -- (-7.5635415762289693, 79.7380445584866351) -- (-7.5721988930280624, 79.7374498922462323);
\draw[uk] (-6.9064471123642939, 80.0609157082875100) -- (-6.9379872636907978, 79.9797636810953492) -- (-6.9905556668280715, 79.9163248093257437) -- (-7.0397221463267599, 79.8652686559550915) -- (-7.0446068455827016, 79.8110579169462113) -- (-7.0447927491323048, 79.8057831679100218) -- (-7.0438643445790872, 79.8015640593091717) -- (-7.0412672608589242, 79.7980458555486649) -- (-7.0373710786811259, 79.7959375453980044) -- (-6.9965535609919360, 79.7794044659272998) -- (-6.9922877981048046, 79.7780034901217618) -- (-6.9874632113738571, 79.7790584393362110) -- (-6.9837529327457162, 79.7808117979286635) -- (-6.9746614699326184, 79.7889102751724977) -- (-6.9250007318947979, 79.8019101902996653) -- (-6.9113941505351919, 79.8171605429540989) -- (-6.8711966824096624, 79.8541162646846772) -- (-6.8659390628595522, 79.8588051351040349) -- (-6.8572817460605817, 79.8570499423319973) -- (-6.8439846328852374, 79.8494129057007882) -- (-6.8427478733425451, 79.8423869313192114) -- (-6.8554260501490578, 79.7696804639082728) -- (-6.8622287842314131, 79.7567895176020158) -- (-6.8708872142252799, 79.7456554395402719) -- (-6.9293299468917038, 79.6964672971408987) -- (-6.9728057740210954, 79.6817156984974559) -- (-7.0455341369410105, 79.6947074890129130) -- (-7.0873412849033306, 79.6437921844240861) -- (-7.0963080698867396, 79.5952498904344168) -- (-7.0987827021670959, 79.5876564318695898) -- (-7.1046581448912001, 79.5829677305019061) -- (-7.2326755593034768, 79.5151975302861587) -- (-7.2892625960583093, 79.5035213266644689) -- (-7.3483231518987138, 79.4889275868907816) -- (-7.3696597586990382, 79.4824986699823626) -- (-7.3684229991564019, 79.4749145622414090) -- (-7.3412120628268820, 79.4749145622414090) -- (-7.2323660911189593, 79.4941715884219917) -- (-7.1470218903074096, 79.5251281541537907) -- (-7.1358888280331847, 79.5321390485844546) -- (-7.1272303980393090, 79.5327152581314039) -- (-7.1201181957725810, 79.5303894176362292) -- (-7.1139343980589915, 79.5175397181227765) -- (-7.0731168803697324, 79.3798265701006898) -- (-7.0762104490189781, 79.3734217199452559) -- (-7.1089862466932638, 79.3297104693604780) -- (-7.1887655861600690, 79.2435313606748224) -- (-7.2067002693217397, 79.2022282639953374) -- (-7.2054635097790474, 79.1935064715264190) -- (-7.2067002693217397, 79.1853780732341761) -- (-7.2781306469789930, 79.1353773891214018) -- (-7.2849344942563663, 79.1324719105956262) -- (-7.3143094814868297, 79.1249253053757968) -- (-7.3220395069274913, 79.1237603346119016) -- (-7.3705881632523091, 79.1313235560113242) -- (-7.3783181886930276, 79.1330638743939971) -- (-7.4061480613912680, 79.1429417844832699) -- (-7.4163527191123322, 79.1516576856145804) -- (-7.4209902890987784, 79.1574681528209965) -- (-7.4494390981658443, 79.2225984254479698) -- (-7.4568596554222433, 79.2423813005365929) -- (-7.4636623895045870, 79.2825391324209789) -- (-7.4661359085899681, 79.3081600500950259) -- (-7.4652097304265714, 79.3262228099401767) -- (-7.4556228958794870, 79.3553475562236770) -- (-7.4522220854356735, 79.3617508581188531) -- (-7.4169705422862569, 79.4037341830360646) -- (-7.4129507954737335, 79.4083948541329079) -- (-7.4058397064018493, 79.4119031455631870) -- (-7.3798644164201130, 79.4119031455631870) -- (-7.3715154546106172, 79.4107336967906576) -- (-7.3637854291699654, 79.4072254602468632) -- (-7.3455412778237896, 79.4054808509505676) -- (-7.3399753032841257, 79.4078196024304788) -- (-7.3362650246560523, 79.4153948043559126) -- (-7.3418298860007409, 79.4206504467868655) -- (-7.3473958605404057, 79.4235650414350403) -- (-7.3557459355447952, 79.4253121728265228) -- (-7.4250100359113027, 79.4323137076578263) -- (-7.4342874022740020, 79.4328912480616935) -- (-7.4410901363563573, 79.4311439394954277) -- (-7.4463466427116272, 79.4258896591297940) -- (-7.5267449185472648, 79.3069922228654320) -- (-7.4871641604008863, 79.2150238803696851) -- (-7.4837622367621641, 79.2080594099963662) -- (-7.4794330217651916, 79.2039848881585300) -- (-7.4642813258733538, 79.1993222826190078) -- (-7.4543861363367947, 79.1900251565382689) -- (-7.4417079595302829, 79.1731630629591763) -- (-7.4358336300011993, 79.1603725093353603) -- (-7.4333601109156948, 79.1545618166540095) -- (-7.4191345931872563, 79.1022691444999566) -- (-7.4144970232008198, 79.0691755353737733) -- (-7.4163527191123322, 79.0604692314138191) -- (-7.4976771731113692, 78.9461962076285744) -- (-7.7487649637649678, 78.7412945415669867) -- (-7.7540214701203034, 78.7372619731623047) -- (-7.7617526087558746, 78.7355162587310389) -- (-7.7682447614589272, 78.7378341768416163) -- (-7.7744307855623207, 78.7430424713891881) -- (-7.9259488576759853, 78.9218585839440721) -- (-7.9293496681196753, 78.9288131721542783) -- (-7.9308958958468159, 78.9357684059089024) -- (-7.9333694149322653, 78.9618448059387248) -- (-7.9318231872051248, 78.9711282501615841) -- (-7.9240931617644623, 78.9717022418719807) -- (-7.9080130613192834, 78.9688031069143364) -- (-7.8950265295234203, 78.9606823693492572) -- (-7.8854396949763377, 78.9438718380149282) -- (-7.8656504290979283, 78.9357684059089024) -- (-7.8245245564192487, 78.9595366460868462) -- (-7.6034306893649468, 79.1028587785982751) -- (-7.6077599043619175, 79.1086654588934266) -- (-7.6581631434032298, 79.1690904407162321) -- (-7.7200066865135213, 79.2051364520026056) -- (-7.7323765083305025, 79.2115415641516734) -- (-7.7756675451050112, 79.2255053109932135) -- (-7.7843248619039827, 79.2249147098309692) -- (-7.8167934177836891, 79.2097994198678634) -- (-7.8254518477775576, 79.2103752443196782) -- (-7.8319451136755287, 79.2127079021639418) -- (-7.8761634418085338, 79.2318975055501937) -- (-7.9179093640508826, 79.2767189445952454) -- (-7.9191461235936425, 79.2837065778850558) -- (-7.8860586313451035, 79.3273741599771256) -- (-7.8542090118342278, 79.3530104456793168) -- (-7.8566825309196640, 79.3868238455146980) -- (-7.8968811122400346, 79.4585637518157029) -- (-7.9333694149322653, 79.5695303814947863) -- (-7.9352251108437803, 79.5841398782473419) -- (-7.9349156426593401, 79.5935008851700587) -- (-7.8975000486088005, 79.6929456148948532) -- (-7.8582287586467503, 79.7737859521857331) -- (-7.8542090118342278, 79.7802252305396280) -- (-7.8375110882151802, 79.8013318991791181) -- (-7.8279253668630737, 79.8112985506476491) -- (-7.8013322537074670, 79.8054370189496751) -- (-7.7936022282666828, 79.8036830815266143) -- (-7.7030004079049457, 79.7649971919628911) -- (-7.6900127629141046, 79.7579643978847912) -- (-7.6875392438287227, 79.7515122763432203) -- (-7.6900127629141046, 79.7439007979457131) -- (-7.7036193442738332, 79.7386244680718903) -- (-7.7212434460562127, 79.7058255367176827) -- (-7.6522877006791834, 79.5438197263021465) -- (-7.6454849665968387, 79.5333103989351713) -- (-7.6386822325144852, 79.5286251077319122) -- (-7.6365181816134191, 79.5374008058851416) -- (-7.6512991836010400, 79.6858964269892169) -- (-7.5754950631504192, 79.7223043704325818) -- (-7.5144819634415558, 79.6999006230922475) -- (-7.5081968649912882, 79.7251105827511424) -- (-7.5547885246679058, 79.7930716898046910) -- (-7.5786932721209315, 79.8570499423319973) -- (-7.5895168662107046, 79.8998939382527595) -- (-7.5870422339304273, 79.9098737706420934) -- (-7.5672518548571110, 79.9468606760426752) -- (-7.4284119595499556, 80.0461956064175695) -- (-7.4120235041154254, 80.0467948574759873) -- (-7.4052207700329493, 80.0450331154573718) -- (-7.3993453273089802, 80.0309152740087626) -- (-7.3028680642231274, 80.0685575005895203) -- (-7.2917372283387216, 80.0762021951352096) -- (-7.0746631080969413, 80.2759538524386187) -- (-7.0248788054243931, 80.3219894362339346) -- (-7.0050884263511444, 80.3414665113019169) -- (-6.9874632113738571, 80.3840051800381730) -- (-6.9822067050185881, 80.3887275279691096) -- (-6.9763312622945408, 80.3928672741424606) -- (-6.9682917686693742, 80.3934523031013129) -- (-6.9608700982181961, 80.3910973236381210) -- (-6.9512843768660098, 80.3845901393023325) -- (-6.9404618959710209, 80.3751462684493845) -- (-6.9277826059697230, 80.3585968620342328) -- (-6.8640844801429370, 80.2075511834276540) -- (-6.8637738987636459, 80.1975292804216195) -- (-6.9064471123642939, 80.0609157082875100);
\draw[uk] (-4.6234702597438675, 69.9057909760740870) -- (-4.6064628679405164, 69.9135191888834271) -- (-4.5987317293048902, 69.9135191888834271) -- (-4.5699745652482857, 69.9093860036540917) -- (-4.5266835284737112, 69.9073343321800849) -- (-4.4734984153574082, 69.9279425073538192) -- (-4.4561815553695405, 69.9361961893776112) -- (-4.2907485469067268, 70.0150950024327443) -- (-4.1553094620433413, 70.0259382351193551) -- (-4.1494340193192949, 70.0228339127926773) -- (-4.1104721975416263, 70.0171643343502978) -- (-4.0220355412757289, 70.0052896324368703) -- (-4.0149233390090586, 70.0063223058644439) -- (-3.9957518963045628, 70.0109713607926238) -- (-3.8535100773587669, 70.0821994401107418) -- (-3.7715666869909641, 70.1250750356605579) -- (-3.7434273461081511, 70.1297292271964068) -- (-3.7140512456827262, 70.1318015654420321) -- (-3.6988995497908861, 70.1318015654420321) -- (-3.6902411197970189, 70.1307653895181460) -- (-3.5752113503755831, 70.0393471137631707) -- (-3.4849189981982276, 69.9650704461122643) -- (-3.4747143404772207, 69.9578476444953026) -- (-3.4533777336768292, 69.9470137517094201) -- (-3.4382126794461283, 69.9697300696364266) -- (-3.4326600632453923, 69.9769242511134735) -- (-3.4304960123443275, 69.9836251760012544) -- (-3.4314233037026356, 69.9908503343762760) -- (-3.4799719600273979, 70.0904692553104809) -- (-3.5112026431693839, 70.1338609357117519) -- (-3.5226429472382979, 70.1483394347651910) -- (-3.5282100349728029, 70.1519585441835574) -- (-3.5359400604135209, 70.1607474375102242) -- (-3.5479993008512118, 70.1824586919020561) -- (-3.5517106926742597, 70.1948590286840641) -- (-3.5535663885857729, 70.2031387952387576) -- (-3.5479993008512118, 70.2134827422145804) -- (-3.5402692754104268, 70.2228036976849808) -- (-3.5297562626999435, 70.2305611366901559) -- (-3.4238947665402533, 70.2459963323877759) -- (-3.4166423017151493, 70.2489808932188566) -- (-3.4082944531005057, 70.2502235620223416) -- (-3.4003161851953743, 70.2477438330773936) -- (-3.3934511121981532, 70.2440122767275881) -- (-3.3889983325664002, 70.2387328825092681) -- (-3.3851021503886014, 70.2322095743495254) -- (-3.3728570064013752, 70.2045892867834453) -- (-3.3706306165854989, 70.1881401536276286) -- (-3.3278349515448995, 70.1690105970528748) -- (-3.2755760165920078, 70.0636215436610286) -- (-3.2597597433401662, 70.0480231298146947) -- (-3.2279568780154251, 70.0238686733459588) -- (-3.2087843221160433, 70.0166404947551086) -- (-3.1862120689679250, 70.0120059659669494) -- (-3.0906620972404482, 70.0300626581691290) -- (-3.0668519713546201, 70.0496826263451311) -- (-3.0628322245420976, 70.0543193664251760) -- (-3.0084092386882073, 70.1302407957631004) -- (-3.0065546559715912, 70.1374793812776574) -- (-3.0105744027841026, 70.1416096349908003) -- (-3.0189233645935984, 70.1431724900571254) -- (-3.0251071623071759, 70.1416096349908003) -- (-3.0782933886184414, 70.1204341386954155) -- (-3.0875707549811424, 70.1137085774599740) -- (-3.0909715654248320, 70.1080324710924003) -- (-3.0934450845102814, 70.1018323677996307) -- (-3.1766252344208419, 70.0821994401107418) -- (-3.1849741962303262, 70.0832349787917224) -- (-3.1988891325794189, 70.0889223175449558) -- (-3.2094032584849326, 70.0945970500882538) -- (-3.2882541933984655, 70.1560895557048667) -- (-3.2993861424778368, 70.1700472752232542) -- (-3.3471299588841088, 70.2624351753014338) -- (-3.3543657257856947, 70.2767248282311527) -- (-3.4592531764009977, 70.5130490231834273) -- (-3.4583258850426901, 70.5291379560525513) -- (-3.4555428977728582, 70.5359009257508092) -- (-3.4437920123246428, 70.5592700841583707) -- (-3.4209102909921523, 70.5998085405215647) -- (-3.3197954579198647, 70.7730993877632102) -- (-3.3157745979123674, 70.7788236250612641) -- (-3.3108275597415471, 70.7824718247502176) -- (-3.2752676616025327, 70.8059312623088175) -- (-3.2307398652852570, 70.8293826912302933) -- (-3.2270284734622203, 70.8340786972254364) -- (-3.2254822457350798, 70.8408529270094789) -- (-3.2288841693737345, 70.8465952433914197) -- (-3.2329039161862561, 70.8502484686182470) -- (-3.2397066502686105, 70.8528542644216088) -- (-3.2554772825293381, 70.8554620241384896) -- (-3.2721752061483182, 70.8559764520454394) -- (-3.3018596615632858, 70.8538849668787378) -- (-3.3315452301730391, 70.8523248378159707) -- (-3.3466969260650008, 70.8564927608142341) -- (-3.3637043178683625, 70.8653606589423646) -- (-3.3770003178487449, 70.8789282463350077) -- (-3.3847314564843050, 70.8883133178631795) -- (-3.3967906969219297, 70.9123191871614580) -- (-3.3998831523761428, 70.9222400241216206) -- (-3.4042123673731060, 70.9425985757332995) -- (-3.4060669500897323, 70.9577435905351734) -- (-3.4060669500897323, 70.9739238073328522) -- (-3.3986463928334532, 71.1778950811473976) -- (-3.3977179882802373, 71.1846990587737736) -- (-3.3921531269354808, 71.1977902232549837) -- (-3.3838030519310331, 71.2066948975907508) -- (-3.3674145964965034, 71.2161343536553915) -- (-3.2644440675127475, 71.2622540981802501) -- (-3.2248644225611551, 71.3535208212352217) -- (-3.1543613362621423, 71.5163946614920292) -- (-3.1364266531004814, 71.6258643089178832) -- (-3.1320985512984185, 71.7776838205692371) -- (-3.2579508016147871, 71.6495624602510048) -- (-3.2675365229669620, 71.6421831583883773) -- (-3.5090385922684408, 71.5106030669505088) -- (-3.5786132740141814, 71.5363760201444308) -- (-3.5912903376257854, 71.5421695237487114) -- (-3.5999487676196646, 71.5511153545684664) -- (-3.7632177251814682, 71.8188556544804300) -- (-3.7950673446923431, 71.8801339000771691) -- (-3.7981598001465575, 71.8933401142015214) -- (-3.7984681551359776, 71.9076103603024421) -- (-3.8225877492062001, 72.0160467409966856) -- (-3.8584571155296534, 72.0742936058478136) -- (-3.8884521523239126, 72.1214456624517908) -- (-3.9237036954733289, 72.1649262243746392) -- (-3.9595730617967271, 72.2073537103770207) -- (-4.0220355412757289, 72.2843405924051865) -- (-4.0405891608062889, 72.3172738551062935) -- (-4.0455361989771088, 72.3289644989124554) -- (-4.0446089076187999, 72.3358695186949632) -- (-3.9731785299615474, 72.5791414992635850) -- (-3.8173312428509862, 72.9120147346932299) -- (-3.7755864338034759, 73.0293736039256345) -- (-3.7638355483552699, 73.0449140173542162) -- (-3.6645764111945631, 73.1463633785029828) -- (-3.6587009684705269, 73.1501259660318937) -- (-3.6401473489400238, 73.1549483780047609) -- (-3.5718094267369183, 73.1705294905615204) -- (-3.5563493758556044, 73.1716067666851018) -- (-3.5504739331314350, 73.1689203735424201) -- (-3.5455257817657637, 73.1640816198392514) -- (-3.5436700858541839, 73.1565572045689123) -- (-3.5387230476834199, 73.1533415185057549) -- (-3.4914122640961898, 73.1356257351721979) -- (-3.4836822386555379, 73.1356257351721979) -- (-3.4546144932196428, 73.1404491771463796) -- (-3.3689608242235769, 73.1630044484523268) -- (-3.3674145964965034, 73.1705294905615204) -- (-3.3652505455954373, 73.2113700999761789) -- (-3.3742173305789138, 73.2205080682787326) -- (-3.3856587478426134, 73.2274928975674442) -- (-3.3963866071703603, 73.2293974325853725) -- (-3.4107056332711445, 73.2205080682787326) -- (-3.4162716078107969, 73.2183521690814558) -- (-3.4994517577213577, 73.2006214995655000) -- (-3.5068734281725336, 73.2006214995655000) -- (-3.7409538270227691, 73.2129781648152260) -- (-3.7585790420001239, 73.2151339142754125) -- (-3.7743496742607845, 73.2183521690814558) -- (-3.8352659260126467, 73.2420103539007101) -- (-3.9753425808624914, 73.2506195244083784) -- (-3.9855472385835546, 73.2135238695037884) -- (-3.9864745299418627, 73.2049168907313259) -- (-4.0251279967300659, 73.0272230207761055) -- (-4.0396607562530731, 73.0315088020972780) -- (-4.1064524507291038, 73.0384765986656816) -- (-4.1144919443541390, 73.0384765986656816) -- (-4.2437461183091045, 72.9762944308327661) -- (-4.3077548255153033, 72.8922072097492446) -- (-4.4011396331468733, 72.8194329147826949) -- (-4.4138178099532643, 72.8151588429990113) -- (-4.4298979103983189, 72.8151588429990113) -- (-4.5591520843532853, 72.8199758542735083) -- (-4.6039893488550003, 72.8333445314508907) -- (-4.8405421535958428, 72.9934431480368460) -- (-4.8454903049615705, 72.9982712506350850) -- (-4.8714655949433174, 73.0347348777019931) -- (-4.8767221012985873, 73.0454585386175808) -- (-4.8779588608412778, 73.0529721643341361) -- (-4.8751758735714583, 73.0706690439211002) -- (-4.8754853417558959, 73.0781851159955522) -- (-4.8773410376673532, 73.0846258362599173) -- (-4.8813596712850353, 73.0894461018716015) -- (-4.8903275694632864, 73.0905222502372851) -- (-4.8958935440029494, 73.0873073738452348) -- (-4.9240328848856842, 73.0422320163568628) -- (-4.9265064039711328, 73.0363392777139353) -- (-4.9305261507836553, 73.0213159811135455) -- (-4.9302166825992719, 73.0047053289885781) -- (-4.9292893912409639, 72.9982712506350850) -- (-4.9104274167209860, 72.9318273221414159) -- (-4.8532214435971408, 72.8263970758831078) -- (-4.8380686345104609, 72.7109424679834291) -- (-4.8371413431521528, 72.7034772919835461) -- (-4.8402337986063682, 72.6879897743959305) -- (-4.8488922286002349, 72.6666271843562583) -- (-4.8590968863213106, 72.6516907003805699) -- (-4.8711561267589349, 72.6447575669215837) -- (-4.8838331903704164, 72.6426176224928639) -- (-4.9005322271843044, 72.6452858387608700) -- (-5.0053573388847328, 72.6874477576348141) -- (-5.0161798197796541, 72.6949267698823718) -- (-5.0727679697294619, 72.7616970131633138) -- (-5.0749320206305839, 72.7675843039564256) -- (-5.0761687801732736, 72.7825431103097600) -- (-5.0780244760847983, 72.7889593093330944) -- (-5.1191514619583733, 72.8440403622327608) -- (-5.2106805736784185, 72.8815047334509956) -- (-5.2771627999700117, 72.9237952590975880) -- (-5.3402442158177683, 72.9843300076946093) -- (-5.3492110008011897, 72.9939873076261421) -- (-5.3591061903377470, 73.0009517300430559) -- (-5.3717843671442722, 73.0063110371335853) -- (-5.4014688225591732, 73.0143516580467065) -- (-5.4478523147879621, 72.9821961093821017) -- (-5.4772295284083610, 72.9580732543168438) -- (-5.4861963133917158, 72.9489662009472255) -- (-5.4942358070168833, 72.9393299360061178) -- (-5.5226846160839456, 72.8868557869266027) -- (-5.5236119074422536, 72.8777551695519890) -- (-5.5214467433463481, 72.8697306330998202) -- (-5.5044404647378915, 72.8194329147826949) -- (-5.4679521620456617, 72.7216233591941545) -- (-5.4546561620652234, 72.6917301776301059) -- (-5.4444515043442152, 72.6837094826813086) -- (-5.4379571252514030, 72.6805114304338105) -- (-5.4286808720836097, 72.6725072678075890) -- (-5.4215686698168062, 72.6612912979860539) -- (-5.4194046189158076, 72.6559557783055823) -- (-5.4054896825666701, 72.5578241771419812) -- (-5.4085821380208188, 72.5508994172395063) -- (-5.4147659357344615, 72.5498307298319816) -- (-5.4797030474937438, 72.5754084343977297) -- (-5.4908349965730707, 72.5828766629898183) -- (-5.5004207179252464, 72.5914148545837747) -- (-5.5084602115504149, 72.5999386258374813) -- (-5.5124799583629382, 72.6063397483349604) -- (-5.5220656797151131, 72.6383532771393163) -- (-5.5251581351693835, 72.6516907003805699) -- (-5.5254676033537660, 72.6586244529761132) -- (-5.5233024392578702, 72.6655569033496533) -- (-5.5152629456327578, 72.6762450033499903) -- (-5.5112431988202468, 72.6901175150796917) -- (-5.5180448197076828, 72.7333749991252745) -- (-5.5214467433463481, 72.7387293063566176) -- (-5.5715405142032761, 72.8114144105950487) -- (-5.6052458296256749, 72.8477882961662431) -- (-5.6491546895741722, 72.8932806220727230) -- (-5.6800781309216468, 72.9237952590975880) -- (-5.6921373713592720, 72.9382558858303867) -- (-5.6995590418104580, 72.9484358570326350) -- (-5.7128550417907746, 72.9693322242285802) -- (-5.7242953458596881, 72.9923683475945495) -- (-5.7552176740122656, 73.0577904600547754) -- (-5.7623309894738997, 73.0760331770888740) -- (-5.7657317999176563, 73.0905222502372851) -- (-5.7712977744573095, 73.1168220143538008) -- (-5.7726881548973514, 73.1419378351620253) -- (-5.7725345340000009, 73.1479720266371345) -- (-5.7644950403749666, 73.2468539833981964) -- (-5.7598563571936099, 73.2694402968561462) -- (-5.7465603572131725, 73.2855906986489742) -- (-5.7354284081338571, 73.2909697499310795) -- (-5.6828600049965710, 73.3146501595792301) -- (-5.6262729682416825, 73.3587891198985318) -- (-5.6250362086989902, 73.3679464633009815) -- (-5.5811273487504804, 73.5237584130157842) -- (-5.5786538296650319, 73.5307862447113223) -- (-5.5650472483053051, 73.5437282992773333) -- (-5.5369079074225676, 73.5691173504016973) -- (-5.4936168706479949, 73.6150203760822137) -- (-5.4194046189158076, 73.6939545485721936) -- (-5.4163121634616038, 73.6999112650861150) -- (-5.4144575807449877, 73.7074751468959590) -- (-5.3965217843882858, 73.7919180828699695) -- (-5.3869371762310072, 73.8726554649928744) -- (-5.3860087716777914, 73.8894694035010815) -- (-5.3829163162235885, 73.8954275789586887) -- (-5.2879852808648105, 74.0648171897745016) -- (-5.2839655340523555, 74.0702528003074860) -- (-5.2641751549791040, 74.0865560639738447) -- (-5.2471688763706492, 74.0957982663254882) -- (-5.2125362695899433, 74.1131994011242483) -- (-5.1837779923383751, 74.1186364719337547) -- (-5.1772847264404041, 74.1208176734858881) -- (-5.1667706005350125, 74.1278825280822673) -- (-5.1627508537225006, 74.1327840541684679) -- (-5.1361588537617369, 74.2160499885705889) -- (-5.1355399173929035, 74.2329371448549296) -- (-5.1451256387451458, 74.2667108168824086) -- (-5.1472896896461453, 74.2732515223140979) -- (-5.1868704477925904, 74.3643259703198396) -- (-5.2162465482180824, 74.4309225416517819) -- (-5.2193390036722969, 74.4374615109711470) -- (-5.2264512059390249, 74.4478432191346542) -- (-5.2304698395567719, 74.4527632232197902) -- (-5.2406744972777810, 74.4609560637263854) -- (-5.2647940913479365, 74.4740711477487309) -- (-5.3090124194808190, 74.4953849866145816) -- (-5.3584883671638890, 74.5287329270351222) -- (-5.4110556571063224, 74.5812373155582264) -- (-5.4277535807253026, 74.6003860164814512) -- (-5.4589853770622518, 74.6157202425885515) -- (-5.4722802638476598, 74.6299458616476272) -- (-5.4735181365853256, 74.6452859975843666) -- (-5.4336279102544420, 75.0951037626557678) -- (-5.4320827957222084, 75.1028062757394537) -- (-5.4280630489096851, 75.1083297961348535) -- (-5.3701145747823400, 75.1293724096548488) -- (-5.3404301193673724, 75.1359791751009993) -- (-5.3350489351824031, 75.1349970522375514) -- (-5.3315245601039312, 75.1330130041127831) -- (-5.3235462921987331, 75.1244167181114619) -- (-5.2895315085920194, 75.1226452057152301) -- (-5.2468594081861442, 75.0995082201339557) -- (-5.2388199145611640, 75.0978582540880382) -- (-5.0876101974369741, 75.0730634704380009) -- (-5.0350429074946090, 75.0681193540898306) -- (-4.9889699966450980, 75.0730634704380009) -- (-4.9973178452596869, 75.0829863679207676) -- (-5.0106149584349096, 75.0890378853201668) -- (-5.1510010814692482, 75.1215402629905782) -- (-5.3172790048670722, 75.1894574392415649) -- (-5.3546344863926700, 75.1828003905008018) -- (-5.4041293583890653, 75.2027891622706193) -- (-5.4172038325827581, 75.2927983946788260) -- (-5.4097331815556204, 75.3695665059317434) -- (-5.3947907663065173, 75.4012924486892757) -- (-5.3751796116134329, 75.4547654906680521) -- (-5.4181378031105583, 75.4397184285176223) -- (-5.4340141888875273, 75.3712382683749524) -- (-5.4526913730528115, 75.1927891593093847) -- (-5.4695017293574590, 75.1528220644648712) -- (-5.4863120856622292, 75.0979038656521567) -- (-5.5236119074422536, 74.9916135011073948) -- (-5.5424738819622217, 74.9641053050949893) -- (-5.5492777292396074, 74.9613576024390653) -- (-5.5563888183114365, 74.9608132206042086) -- (-5.5795800078284437, 74.9657662481045008) -- (-5.5947328169151804, 74.9712640482997159) -- (-5.6884270927311320, 75.0372862987627656) -- (-5.7589290658353027, 75.0890378853201668) -- (-5.7954273872816140, 75.0746137763957222) -- (-5.8159725125024329, 75.0347083563515866) -- (-5.8215763356689330, 74.9798748283080698) -- (-5.7975814194285320, 74.9212381220175843) -- (-5.7917059767044847, 74.8998075910581065) -- (-5.7935616726160095, 74.8910100101173697) -- (-5.7975814194285320, 74.8855159903163496) -- (-5.8050030898797074, 74.8844164426029977) -- (-5.8915840502340258, 74.9272768657456112) -- (-5.9036444038664913, 74.9333299376371258) -- (-5.9079725056685568, 74.9399138643481706) -- (-5.9429156938285548, 75.0257228946125139) -- (-5.9447713897400662, 75.0328874024739036) -- (-5.9450797447296084, 75.0411424761674226) -- (-5.9453892129139918, 75.0829863679207676) -- (-5.9107554929383763, 75.2345889666895999) -- (-5.9103847990340910, 75.2402193670655777) -- (-5.9094575076757838, 75.2445328442159962) -- (-5.8788446477075320, 75.2981905101151767) -- (-5.8704956858980477, 75.3124361143091647) -- (-5.7917059767044847, 75.4628450720732218) -- (-5.7870672935231404, 75.4683787337121146) -- (-5.6463728154992499, 75.6019059798843074) -- (-5.5372173756070753, 75.6945617264618420) -- (-5.5072679798040527, 75.7278867479797952) -- (-5.4803208706676685, 75.7578837265547378) -- (-5.4747548961280161, 75.7701106965073166) -- (-5.4775378833978481, 75.7767651369065049) -- (-5.4852690220334068, 75.7756666940486099) -- (-5.6018450191819840, 75.6928988556128814) -- (-5.7951067871481188, 75.5027015749870287) -- (-5.8319056712197632, 75.4661747858421847) -- (-5.8541684561834870, 75.4423698271903902) -- (-5.8859568499743924, 75.3967889430629583) -- (-5.9065509557711708, 75.3672659021967206) -- (-5.9186101962087951, 75.3397376676282846) -- (-5.9432852745379980, 75.2983058323735719) -- (-5.9623966047173544, 75.2790871954131688) -- (-6.0010489583105828, 75.2296324119023296) -- (-6.0087800969462641, 75.2285259223058347) -- (-6.0158911860180400, 75.2290722122955202) -- (-6.0372277928184195, 75.2362398985655005) -- (-6.0434115905320747, 75.2395597503758466) -- (-6.0465040459862784, 75.2450772616702466) -- (-6.0322796414526909, 75.0444380170995800) -- (-5.9889886046781182, 74.7351471215254577) -- (-6.0158911860180400, 74.7077308325252005) -- (-6.0607284505198091, 74.6523992475280664) -- (-6.0641292609635666, 74.6463821206050966) -- (-6.0706236400564348, 74.6102478855243589) -- (-6.0746422736740504, 74.6053177777636876) -- (-6.0879393868492730, 74.5741306596869578) -- (-6.1074202977381402, 74.5238060844409631) -- (-6.1840170129631646, 74.0750586849786714) -- (-6.1618432835920425, 74.0436243226209143) -- (-6.1590602963222096, 74.0381925478560419) -- (-6.1401983218022318, 73.9789881747528426) -- (-6.1389615622595395, 73.9719350164512548) -- (-6.1482378154273434, 73.9534794749370121) -- (-6.1618432835920425, 73.9306901988691578) -- (-6.1970948267415809, 73.8840484888178537) -- (-6.2292539144368391, 73.8629033440147396) -- (-6.3872663656433053, 73.8363464388214368) -- (-6.4036548210779696, 73.8357963153540595) -- (-6.4184970487853583, 73.8390561468168585) -- (-6.4314835805812907, 73.8450084748482709) -- (-6.4407598337492056, 73.8542236867487674) -- (-6.4429249978451137, 73.8612660766659985) -- (-6.4453996301254026, 73.8759012414395073) -- (-6.4512750728495059, 73.9692206313925027) -- (-6.4518928960233755, 73.9947345461788757) -- (-6.4509644914701596, 74.0197191667428882) -- (-6.4494193769379926, 74.0278659955516218) -- (-6.4429249978451137, 74.0398179325324151) -- (-6.4110753783342380, 74.0811212590433144) -- (-6.3758249483796066, 74.1083147839343042) -- (-6.3687116329179068, 74.1186364719337547) -- (-6.3662381138325239, 74.1257167914048267) -- (-6.3634562397575998, 74.1425704160141521) -- (-6.3653108224742150, 74.2721724833904915) -- (-6.3470666711280401, 74.4167199935936310) -- (-6.3108889498152223, 74.5604411717730784) -- (-6.2425510276120626, 74.7515921258135450) -- (-6.2471885975984982, 74.7543342899647314) -- (-6.2867682425501021, 74.7965787752546731) -- (-6.3142897602589114, 74.8404771164892395) -- (-6.3164538111599775, 74.8470669033283258) -- (-6.3192991373446219, 75.0060219743750309) -- (-6.3198557347985762, 75.0106451437579835) -- (-6.3205971226072251, 75.0301293045689022) -- (-6.3183707327913492, 75.0545692757330301) -- (-6.3139179531596517, 75.0615105170241748) -- (-6.3055701045450769, 75.0694336569746099) -- (-6.2949936397248729, 75.0806725622947226) -- (-6.2713081916687221, 75.1623385079454778) -- (-6.2134832821761004, 75.2528088054121156) -- (-6.2100813585375692, 75.2588737761223285) -- (-6.2011145735540927, 75.3035989136673578) -- (-6.2072994844626450, 75.3052672448029625) -- (-6.2147200417189241, 75.3041596597587812) -- (-6.2311096103483621, 75.2936632548336746) -- (-6.2623402934904719, 75.2704744724191954) -- (-6.3001889203601049, 75.1969629491123612) -- (-6.3020446162716288, 75.1933193093224759) -- (-6.3096510770775014, 75.1787623677424506) -- (-6.3120622572481384, 75.1754530950869650) -- (-6.3165161500747331, 75.1748019773152123) -- (-6.3202264287029299, 75.1767791691545142) -- (-6.3222669149691475, 75.1800884991955058) -- (-6.3228235124231027, 75.2082206094649308) -- (-6.3226387220684082, 75.2125223511253154) -- (-6.3183707327913492, 75.2251012696381594) -- (-6.2808927998260575, 75.3152064013244171) -- (-6.2700714321260307, 75.3328845102641509) -- (-6.2354377121504161, 75.3815243928580401) -- (-6.1930750799290584, 75.4550922438318423) -- (-6.1460726513314361, 75.5536721574928549) -- (-6.1435991322459875, 75.5619836819236497) -- (-6.1355596386208200, 75.5913685672743583) -- (-6.1358691068053366, 75.5996822031688822) -- (-6.1411256131605381, 75.6030019224156433) -- (-6.1479294604378003, 75.6002311545907730) -- (-6.1584424731483516, 75.5930272299667934) -- (-6.1881269285633191, 75.5669758199501160) -- (-6.2048259653771405, 75.5481323090455561) -- (-6.2314179653379043, 75.5475837323260038) -- (-6.2301812057951462, 75.7334396685059517) -- (-6.2298728508056715, 75.7501008575064532) -- (-6.2045153839978493, 75.8774288825845389) -- (-6.1995683458270188, 75.8919053024445702) -- (-6.1516397390659421, 75.9754393901416023) -- (-6.1510208026971647, 75.9921696515899043) -- (-6.1497840431544173, 75.9999612417001771) -- (-6.1460726513314361, 76.0072177894717953) -- (-6.0526889568947757, 76.0732468296887276) -- (-6.0426702027233583, 76.0863040215585329) -- (-6.0341353373642157, 76.0946818924678183) -- (-5.9350609905582026, 76.1010352240627128) -- (-5.9263413348443539, 76.0993634547592279) -- (-5.9094575076757838, 76.0879755039044170) -- (-5.8349970134790698, 76.1014873314475295) -- (-5.8285037475810313, 76.0998175615367103) -- (-5.8118058239620618, 76.1003681194068662) -- (-5.8003644066983640, 76.1065169561445742) -- (-5.7904692171616725, 76.1148870048646415) -- (-5.7811929639938784, 76.1254830188709235) -- (-5.7168747886034197, 76.1992196460416409) -- (-5.7001768649844395, 76.2204668576961240) -- (-5.6859535736457500, 76.2450678368317938) -- (-5.6423519554918293, 76.3446762755315689) -- (-5.6398784364064465, 76.3513993003689535) -- (-5.6432803600450336, 76.3558749271357442) -- (-5.6507020304962214, 76.3547519635679066) -- (-5.6559574236565826, 76.3497058887304121) -- (-5.6633802073026107, 76.3390785765488147) -- (-5.6726564604705265, 76.3194747638104900) -- (-5.6816232454538795, 76.2887017208308293) -- (-5.6946108904447197, 76.2551303251388362) -- (-5.7787194449084973, 76.1528461331773059) -- (-5.7920154448888681, 76.1372163293295756) -- (-5.8046936216953258, 76.1327515481493577) -- (-5.8207726089454725, 76.1305142304903484) -- (-5.8278848112122095, 76.1310810918127174) -- (-5.9760020728821752, 76.1777711970194673) -- (-6.0097686140244209, 76.1864870429139387) -- (-5.9858471686478865, 76.2719363220133602) -- (-5.9706342470361342, 76.2595237700202659) -- (-5.9485551392321963, 76.2605285394290320) -- (-5.9433609717918037, 76.2611970506828811) -- (-5.9359393013405599, 76.2645537909558442) -- (-5.8476751902855151, 76.3306838090449276) -- (-5.8433459752885542, 76.3357124959464244) -- (-5.8402535198343406, 76.3418783823519647) -- (-5.8393262284760317, 76.3508287777075765) -- (-5.8476751902855151, 76.3508287777075765) -- (-5.9137243837578772, 76.3492603664642786) -- (-5.9172498720313236, 76.3465772084303467) -- (-5.9250422363868518, 76.3378409702961847) -- (-5.9374721707288289, 76.3287691660521972) -- (-5.9439665498217087, 76.3244100062521795) -- (-5.9515730106275129, 76.3210506210061510) -- (-5.9571389851671768, 76.3210506210061510) -- (-5.9653031566219656, 76.3244100062521795) -- (-5.9847840675107671, 76.3425476096735309) -- (-5.9799594807798204, 76.3549737324362070) -- (-5.9923905283167054, 76.3967705776762784) -- (-5.9163225806730031, 76.5453975128433513) -- (-5.8356159498478917, 76.6577695158433556) -- (-5.7623309894738997, 76.7545366634625879) -- (-5.7150190926918958, 76.8204359745895999) -- (-5.7116182822481392, 76.8266413814261142) -- (-5.7014136245270750, 76.8615829065419405) -- (-5.6995590418104580, 76.8694804629396486) -- (-5.7379019272192950, 76.8339631573403778) -- (-5.7762459258230487, 76.7877559251623438) -- (-5.8266480516695198, 76.7207728242819940) -- (-5.9704360983425229, 76.5560699158655353) -- (-6.1936940162978242, 76.3110801469967299) -- (-6.2258519907982413, 76.2780668245892031) -- (-6.2360566485192486, 76.2685571039610863) -- (-6.3195462666141937, 76.2171201133945004) -- (-6.3337706711477235, 76.2478641810162969) -- (-6.3424279879466816, 76.2568196169284249) -- (-6.4460174532993282, 76.3060530900534388) -- (-6.4862160346196980, 76.3144471514677747) -- (-6.5502247418257644, 76.3368331625080998) -- (-6.5629029186322869, 76.3441078191527112) -- (-6.6018647404099555, 76.3698720347708928) -- (-6.6154713217695607, 76.3838737389948648) -- (-6.6745307644150023, 76.4539351386769823) -- (-6.6872100544164246, 76.4701996276281903) -- (-6.6918487375977795, 76.4825396249599407) -- (-6.6915381562184884, 76.4903978016543249) -- (-6.6884468139591142, 76.5066721256414013) -- (-6.6800978521495615, 76.5178926752228961) -- (-6.6711299539713131, 76.5268800085328991) -- (-6.6652545112471993, 76.5307961354840671) -- (-6.6510312199085755, 76.5347266105577972) -- (-6.5484937237439960, 76.5453975128433513) -- (-6.3442836838582073, 76.5981981156545828) -- (-6.2880061152877023, 76.5931378560310350) -- (-6.2478064207724238, 76.5965012124025577) -- (-6.1850344731090487, 76.6038169768389992) -- (-6.1726657644870411, 76.6094342313778185) -- (-6.1748309285828809, 76.6167330357835539) -- (-6.1797790799487426, 76.6212243343020276) -- (-6.3891209483599223, 76.6566420769075876) -- (-6.4144761887779289, 76.6571987098348586) -- (-6.4228262637823219, 76.6549600582321631) -- (-6.4336487446773205, 76.6487666648311574) -- (-6.4484909723847768, 76.6358431269394771) -- (-6.4857217760806201, 76.6159460743692335) -- (-6.5033469910578976, 76.5960542394348636) -- (-6.5107675483141110, 76.5926828132533899) -- (-6.5891876767983506, 76.5802096367113450) -- (-6.8922227078307747, 76.5920113773393894) -- (-6.9120130869039578, 76.6010074942255272) -- (-6.9293299468917038, 76.6217968928186650) -- (-6.9361326809741826, 76.6330344814092257) -- (-6.9407713641555260, 76.6538326610607044) -- (-6.9410797191449456, 76.6633907676109345) -- (-6.9407713641555260, 76.6718244388535197) -- (-6.9382978450700197, 76.6785744808680505) -- (-6.8950068082955038, 76.7297680207286703) -- (-6.8909859482879492, 76.7342822287196071) -- (-6.8350167347069863, 76.7500354354115615) -- (-6.6423727899147078, 76.7956457896143689) -- (-6.6275294490122896, 76.7984624937798799) -- (-6.6219634744726248, 76.7950901670301391) -- (-6.6126872213048324, 76.7860709950185054) -- (-6.5734170445376794, 76.7477778287045709) -- (-6.5316711222953385, 76.7196444481777178) -- (-6.5159004900346646, 76.7179590056336451) -- (-6.5075526414200793, 76.7280843791235583) -- (-6.4549853514776450, 76.8136732887331704) -- (-6.4627153769183634, 76.8243814915259975) -- (-6.5004404391532287, 76.8429717992658681) -- (-6.5118818564169292, 76.8514396819129217) -- (-6.5193024136731967, 76.8621390169506498) -- (-6.5217770459536091, 76.8677754466101533) -- (-6.5220865141379356, 76.8773626026771382) -- (-6.5211581095847206, 76.8858201190352446) -- (-6.5171383627722648, 76.8920308842189399) -- (-6.5121902114065255, 76.8954075289447445) -- (-6.4407609469441134, 76.9159528747341739) -- (-6.4045821124362670, 76.9219345532100505) -- (-6.3953058592684755, 76.9208031774513188) -- (-6.3820087460931294, 76.9151628004059233) -- (-6.3653108224742150, 76.9157193043446767) -- (-6.3059396852544634, 76.9670754076979478) -- (-6.3016104702576259, 76.9755571786141957) -- (-6.3040851025378482, 76.9812020565139647) -- (-6.3145992284433623, 76.9902282310015806) -- (-6.3885031251861308, 77.0258189220008376) -- (-6.3971604419850916, 77.0263782739403950) -- (-6.4203527446968716, 77.0235530804345672) -- (-6.4766303132675125, 77.0094327616314303) -- (-6.5291976032098891, 76.9970067011163053) -- (-6.5440409441122531, 76.9953171096325235) -- (-6.5533171972801121, 76.9970067011163053) -- (-6.5870225127024451, 77.0105654626024290) -- (-6.5922790190576572, 77.0145229315549784) -- (-6.4896168450634439, 77.2442274424633979) -- (-6.4809595282644841, 77.2549824604046478) -- (-6.4633332000921682, 77.2657532919887586) -- (-6.4500383133067585, 77.2702995780852859) -- (-6.4158376261503749, 77.2633802719104210) -- (-6.4021085933508290, 77.2664388903361612) -- (-6.3874511559980665, 77.2674570969769547) -- (-6.3841115712742251, 77.2650758616339175) -- (-6.3820710850080067, 77.2620254598055567) -- (-6.3774324018266508, 77.2497822426500136) -- (-6.3700107313754746, 77.2358378772322709) -- (-6.3421196329572762, 77.2204339347586313) -- (-6.3136708238901464, 77.2011757737428184) -- (-6.3019210516369055, 77.1938041222112759) -- (-6.2883155834720856, 77.1870079983432760) -- (-6.2821295593687472, 77.1847515919776441) -- (-6.2669789766718154, 77.1875828118090936) -- (-6.2107002949062791, 77.2017364237720329) -- (-6.1961675353832621, 77.2062625507945057) -- (-6.1649357390462560, 77.2209947276062678) -- (-6.1491662199805477, 77.2300658748121265) -- (-6.1451453599731289, 77.2351545959995036) -- (-6.1473105240690362, 77.2419550491241012) -- (-6.1556594858785205, 77.2413941007775406) -- (-6.1754498649517702, 77.2328967523825298) -- (-6.1853450544883959, 77.2243860835362170) -- (-6.1995683458270188, 77.2215555246134784) -- (-6.2518272807798549, 77.2113639430410075) -- (-6.2688346725833393, 77.2119246686192611) -- (-6.2938815580117380, 77.2260971139872368) -- (-6.4271554787793477, 77.3355121439750803) -- (-6.4389052510325904, 77.3434429910895034) -- (-6.4423071746711891, 77.3485556936053200) -- (-6.4444712255721877, 77.3559281538542791) -- (-6.4438522892034227, 77.3638774893418741) -- (-6.4407598337492056, 77.3814768737649672) -- (-6.4327203401241047, 77.4019053109249455) -- (-6.4252997828678380, 77.4126971077596551) -- (-6.3727324929253921, 77.4734903309147143) -- (-6.3358734963288752, 77.4691670862501240) -- (-6.2928295888238077, 77.4794078280742724) -- (-6.2887475030964080, 77.4800789981151894) -- (-6.2611035339476455, 77.4725777844541454) -- (-6.2521979746842078, 77.4643980959174456) -- (-6.2468179036941462, 77.4538239189080286) -- (-6.2088445989947560, 77.4416682066806459) -- (-6.1887458649320850, 77.4314489410065079) -- (-6.1680270813056861, 77.4234924915251668) -- (-6.1596792326910430, 77.4217912983293246) -- (-6.1426718408876786, 77.4223535875138253) -- (-6.0109441478472636, 77.4587088450946055) -- (-6.0121809073898991, 77.4666690061288250) -- (-6.0155828310286195, 77.4723506725873960) -- (-6.0199109328306850, 77.4763323904434031) -- (-6.0344448055486000, 77.4831545883667161) -- (-6.0598000459665373, 77.4842944097877506) -- (-6.0749528550533416, 77.4808749971482769) -- (-6.1021649045778341, 77.4717900570291960) -- (-6.1785423204059855, 77.4803122720269215) -- (-6.2779873611162857, 77.5612957285922135) -- (-6.2852220148229625, 77.5653989096493603) -- (-6.2876343081884416, 77.5688057641837361) -- (-6.2891181970007040, 77.5729093536050414) -- (-6.2885627127416575, 77.5787158119481148) -- (-6.2852220148229625, 77.7234047867673041) -- (-6.2479923243220270, 77.7332181387205310) -- (-6.2429829472363174, 77.7390369509720784) -- (-6.1901061891096241, 77.7640194180086297) -- (-6.1784176425762976, 77.7681336423531775) -- (-6.1747073639481558, 77.7667553624573458) -- (-6.1680270813056861, 77.7510065550720384) -- (-6.1654299975854565, 77.7472464695718202) -- (-6.1166965509059441, 77.7313929228142229) -- (-6.1015459682090141, 77.7159963468268842) -- (-6.0919602468568259, 77.7085906871438539) -- (-6.0666027800490161, 77.6954683636010657) -- (-6.0427937673580834, 77.6886269441029640) -- (-6.0356815650912896, 77.6920557843351673) -- (-6.0180563501140689, 77.7120040147617033) -- (-6.0158911860180400, 77.7182830527881805) -- (-6.1366728335287855, 77.8105994757359412) -- (-6.2017947356428955, 77.8242968713562533) -- (-6.2326057443046361, 77.9196947578665231) -- (-6.2092286512380372, 77.9224363604472074) -- (-6.2044051777019309, 77.9238092574939714) -- (-6.1454548281575114, 77.9787374314494457) -- (-6.1142230318204955, 78.0027713245744536) -- (-6.0811366527669293, 78.0359793958814691) -- (-6.0768074377699568, 78.0422854937043553) -- (-6.0700047036876130, 78.0548869029101695) -- (-6.0653660205062581, 78.0892504819203310) -- (-6.0650576655167834, 78.0989940151430915) -- (-6.0672217164177811, 78.1047235458689073) -- (-6.0755706782272672, 78.1075874431465422) -- (-6.0805188295930055, 78.1041561439987930) -- (-6.1262833854530276, 78.0651910298042537) -- (-6.2701326578460002, 78.0022046907076145) -- (-6.4021085933508290, 77.9461352109778289) -- (-6.4225156824030947, 77.9392709998390103) -- (-6.4321025169501258, 77.9398371603490290) -- (-6.4438522892034227, 77.9472820522952503) -- (-6.4753935537246976, 77.9827459808553840) -- (-6.4778670728102039, 77.9884666698978606) -- (-6.5233232736807505, 78.1677957806223702) -- (-6.5310532991214130, 78.2027935805551095) -- (-6.5360003372922444, 78.2263308267580584) -- (-6.5350730459339355, 78.2521618802306307) -- (-6.5335268182068500, 78.2613579504758263) -- (-6.5134269709491610, 78.3682579365240457) -- (-6.5118818564169292, 78.3763148036444335) -- (-6.5007499073376120, 78.4200431873692168) -- (-6.4828141109811011, 78.4315517251743159) -- (-6.4760113768986907, 78.4344301713006331) -- (-6.4685908196423432, 78.4338436970883919) -- (-6.4308657574074886, 78.4142895795182113) -- (-6.4011801887976132, 78.3780376897051951) -- (-6.4005612524288473, 78.3607878328031688) -- (-6.2957361407283532, 78.2889218865459640) -- (-6.2901701661887008, 78.2860490082474030) -- (-6.2815117361948216, 78.2872010420112758) -- (-6.2561564957768256, 78.3205349345356439) -- (-6.1568962454212111, 78.3280031076613312) -- (-6.1355596386208200, 78.3349028774349563) -- (-6.1321588281771957, 78.3412340872635724) -- (-6.1423623727032970, 78.3676740491156920) -- (-6.1466915877002037, 78.3717013080834732) -- (-6.1618432835920425, 78.3774537263021500) -- (-6.2824401407480392, 78.3774537263021500) -- (-6.2935709766324575, 78.3694112310126627) -- (-6.2966634320866604, 78.3642308620824650) -- (-6.3034661661691374, 78.3619244498016343) -- (-6.3108889498152223, 78.3642308620824650) -- (-6.3164538111599775, 78.3688273346428446) -- (-6.3402639370457514, 78.3924166315511854) -- (-6.4679718832735231, 78.5548309859528331) -- (-6.4692086428163353, 78.5623370211317820) -- (-6.4760113768986907, 78.9282249066499162) -- (-6.4750851987352780, 78.9375041857836806) -- (-6.4605513260172422, 79.0024433134704509) -- (-6.4556031746515030, 79.0070790551998527) -- (-6.4197338083281732, 79.0273794655592638) -- (-6.4064366951528937, 79.0331802951893394) -- (-6.3640740629314587, 79.0372434412762459) -- (-6.3414383576736322, 79.0030154570645493) -- (-6.3349450917756620, 78.9984906305865593) -- (-6.3291932136863380, 78.9929242370359930) -- (-6.3265972431610837, 78.9891014959097220) -- (-6.3256688386078013, 78.9845775098326612) -- (-6.3056313302650455, 78.8749434527229738) -- (-6.3025388748108320, 78.8633667167323722) -- (-6.2796560402833093, 78.8378975221600484) -- (-6.2502799398579389, 78.8193793370201945) -- (-6.2419320912432958, 78.8199521676504844) -- (-6.2366755848880144, 78.8240058505668912) -- (-6.2141022185449328, 78.9288131721542783) -- (-6.2125548776228943, 78.9369304560155456) -- (-6.2122465226334089, 78.9548871511860710) -- (-6.2147200417189241, 78.9618448059387248) -- (-6.2311096103483621, 78.9832957286560884) -- (-6.2357471803347986, 78.9885190199878195) -- (-6.2623402934904719, 79.0012802467164335) -- (-6.2724826122966029, 79.0266943414215177) -- (-6.2828731735672685, 79.0496598598907667) -- (-6.2836145613759848, 79.0545388243382376) -- (-6.2832427542767242, 79.0663854789705596) -- (-6.2823154629184170, 79.0716062085729163) -- (-6.2833663189114377, 79.1005295418812153) -- (-6.2734722425697873, 79.1196903995346332) -- (-6.2651232807603039, 79.1307336976637998) -- (-6.2462613062401910, 79.1504922987177082) -- (-6.0752612100428829, 79.0152048244641350) -- (-6.0495965014404387, 79.0976275063241729) -- (-6.0452672864435879, 79.1028587785982751) -- (-6.0390823755350356, 79.1063360415072765) -- (-5.9429156938285548, 79.1446823730209132) -- (-5.8080955453339227, 79.1028587785982751) -- (-5.7079080036199441, 79.0372434412762459) -- (-5.6905911436321972, 79.0163681074655386) -- (-5.6803864859111899, 78.9983799668197122) -- (-5.6754383345454498, 79.0343438574298318) -- (-5.7768637489969725, 79.2022282639953374) -- (-5.7855221789908411, 79.2121174020112591) -- (-5.7920154448888681, 79.2156164975234134) -- (-5.8584976711803929, 79.2499397608376910) -- (-6.0090884519357521, 79.3757426802320794) -- (-6.0301155905516257, 79.4002262749803691) -- (-6.0647481973323991, 79.4509820727650720) -- (-6.0715509314147420, 79.4632446514860078) -- (-6.0727876909574343, 79.4708254735990636) -- (-6.0712414632303036, 79.4795796745521983) -- (-6.0659849568750905, 79.4947643203638279) -- (-6.0591822227927370, 79.5076101028425768) -- (-6.0542351846217821, 79.5128554158775387) -- (-6.0409380714465604, 79.5175397181227765) -- (-6.0325902228319830, 79.5157757116603960) -- (-6.0205298691994509, 79.5081882254998220) -- (-5.9936272878593959, 79.4860104419511799) -- (-5.9800218196946977, 79.4702497411647357) -- (-5.9215790870281619, 79.4457263975671708) -- (-5.9119933656759756, 79.4445732233195088) -- (-5.8971511379685193, 79.4504064931898881) -- (-5.8878737716058867, 79.4579860129829711) -- (-5.8832362016194386, 79.4643981138917752) -- (-5.8792153416118849, 79.4702497411647357) -- (-5.8773607588952697, 79.4778333754315014) -- (-5.8708674929973084, 79.5566739476323193) -- (-5.8714853161712890, 79.6227300148301680) -- (-5.8897283543225019, 79.6233258733733322) -- (-5.8971511379685193, 79.6274212135625703) -- (-6.0072327560242273, 79.8206733150659176) -- (-6.0100168564889564, 79.8347513958688921) -- (-6.0103252114783761, 79.8435462665281364) -- (-6.0069232878398440, 79.8517634921316954) -- (-6.0025940728428155, 79.8564543267350615) -- (-5.9964102751292279, 79.8599816231777169) -- (-5.9790945283363897, 79.8576307777254613) -- (-5.9735285537967364, 79.8541162646846772) -- (-5.7926332680627937, 79.8382797776334030) -- (-5.6457538791304831, 79.8676049552453549) -- (-5.6500830941273881, 79.8728925285137024) -- (-5.6575047645785652, 79.8764058138603872) -- (-5.7066712440773086, 79.9157435165764554) -- (-5.7493422312880975, 79.9956323676055234) -- (-5.7604752935623100, 80.0479721986796591) -- (-5.7539809144694418, 80.0797420243147116) -- (-5.6878081563623661, 80.3840051800381730) -- (-5.5712321592138005, 80.6290225289354652) -- (-5.5641199569469961, 80.6325716856442085) -- (-5.3124143431194160, 80.5869449702908724) -- (-5.3077756599380610, 80.5827965370134649) -- (-5.2338717631953049, 80.4910292963308791) -- (-5.2060418904970742, 80.4182724060659382) -- (-5.2341812313798091, 80.3875447856629535) -- (-5.2518064463570973, 80.3544590224587409) -- (-5.3000445213025014, 80.2629729158630738) -- (-5.3028286217672411, 80.2535296027250808) -- (-5.3037559131255501, 80.2458666836353274) -- (-5.2950974831316726, 80.2441002261997909) -- (-5.2780900913283197, 80.2564769260219606) -- (-5.1964561691447653, 80.3414665113019169) -- (-5.1250257914875128, 80.4159187686718013) -- (-5.1191514619583733, 80.4301098203071518) -- (-5.1172957660468610, 80.4383749932819114) -- (-5.1185325255894849, 80.4649987984302868) -- (-5.1222428042176249, 80.4803685117817764) -- (-5.1203882215010088, 80.4898427957618452) -- (-5.1151306019509537, 80.5046338679043600) -- (-5.1080195128790571, 80.5170689110552331) -- (-5.0965780956152917, 80.5247598202514183) -- (-5.0817358679078461, 80.5283210510503551) -- (-5.0718395651762469, 80.5271339551605649) -- (-5.0167987561484306, 80.5105506946728440) -- (-4.9827839725416494, 80.4963442844041595) -- (-4.9435137957744955, 80.4756403414295391) -- (-4.9339269612274128, 80.4655822656274466) -- (-4.9302166825992719, 80.4508010881957034) -- (-4.7613828636927122, 80.4377896108639732) -- (-4.5424541607343718, 80.4839276167266888) -- (-4.3451715327606291, 80.4904285881107739) -- (-4.2839446996294752, 80.4939881746709460) -- (-4.2137521947097545, 80.5081941346782344) -- (-4.2078767519857081, 80.5117375257746346) -- (-4.1525264747735102, 80.5638403973518251) -- (-4.1345917916117161, 80.5851725265051897) -- (-4.1290258170721188, 80.5893210924222245) -- (-4.0746028312182183, 80.6213208231067853) -- (-3.9809085554022765, 80.6260604040661804) -- (-3.9555533149842033, 80.5987922084505186) -- (-3.7554865865457865, 80.5703654211455387) -- (-3.7459008651936001, 80.5691776535778530) -- (-3.7412632952071516, 80.5744982325361008) -- (-3.7384803079373321, 80.5822100162892383) -- (-3.7310586374861439, 80.6059048716326174) -- (-3.7301302329329391, 80.6136049796419201) -- (-3.7344594479299009, 80.6189458113743456) -- (-3.7551771183614031, 80.6379138809618752) -- (-3.7335321565715929, 80.7085061972159536) -- (-3.5359400604135209, 80.6847615151570636) -- (-3.3680335328652693, 80.6806225661340193) -- (-3.3615391537723895, 80.6764688992096808) -- (-3.3572099387754948, 80.6711239307722678) -- (-3.3547364196900458, 80.6503712468283283) -- (-3.4014293801032851, 80.2694546057791030) -- (-3.4066858864585536, 80.2541284010377609) -- (-3.4617278086812213, 80.1150478370817609) -- (-3.4753332768460439, 80.0891484918498691) -- (-3.5650066926545749, 79.9592028885697914) -- (-3.5718094267369183, 79.9515701860167667) -- (-3.7152880052254162, 79.8934274157072508) -- (-3.7329132202027600, 79.8934274157072508) -- (-3.7400265356644611, 79.8910903072906962) -- (-3.7449735738352921, 79.8869804706077531) -- (-3.7647639529084866, 79.8676049552453549) -- (-3.8074360533143614, 79.8230251071114765) -- (-3.8318651155689007, 79.7878551866457428) -- (-3.8365026855553377, 79.7714357118273938) -- (-3.8470156982658321, 79.7439007979457131) -- (-3.8544373687170750, 79.7310143040282782) -- (-3.8776296714289789, 79.6947138116023268) -- (-3.9085508863865823, 79.6572429049517154) -- (-3.9131895695679377, 79.6519874414765496) -- (-3.9697777195177348, 79.6028462261387943) -- (-4.1015054125581498, 79.4988547193281647) -- (-4.2802344210013352, 79.3524356199883982) -- (-4.4493788212871870, 79.2068931444416791) -- (-4.4657672767217163, 79.1574681528209965) -- (-4.4441212017370759, 79.1225953819776606) -- (-4.4428844421943170, 79.1051713599942303) -- (-4.4694775553498571, 79.0337693854824579) -- (-4.5715219061703225, 79.0007060283176941) -- (-4.5863641338777228, 78.9983799668197122) -- (-4.7233483332734636, 79.0389961980580296) -- (-4.8519846840544503, 79.0912300442983422) -- (-4.8606420008535434, 79.1016941511812917) -- (-4.8736296458443835, 79.1086654588934266) -- (-4.8893991649100803, 79.1121429915330623) -- (-4.8881624053673898, 79.0958880115561414) -- (-4.8470365326887119, 79.0227403785384297) -- (-4.8374508113365362, 79.0140415595306678) -- (-4.7839562300358507, 78.9890931449686349) -- (-4.7681855977751795, 78.9844585161940245) -- (-4.6194505129313450, 78.9450340138083391) -- (-4.5008952552365304, 78.9223195609685462) -- (-4.3708362413630724, 78.9079513435861344) -- (-4.3133208000549681, 78.9241822707680001) -- (-4.2864193319098201, 78.9340347524461095) -- (-4.2765241423731393, 78.9438718380149282) -- (-4.2687941169325434, 78.9560494978604908) -- (-4.2598262187541600, 78.9867841454206001) -- (-4.2527140164874213, 78.9995429885396447) -- (-4.2428188269507974, 79.0099775836453802) -- (-4.2326141692297226, 79.0175147003124465) -- (-4.2171530051535662, 79.0198413561393949) -- (-4.2023107774460549, 79.0146137917283085) -- (-4.2001467265450563, 78.9989541673869411) -- (-4.2054032329002711, 78.9832957286560884) -- (-4.2477669783165917, 78.9079513435861344) -- (-4.2558064719416366, 78.8957960933529989) -- (-4.3031172555288668, 78.8268973061211398) -- (-4.3906266204362669, 78.7199127496861735) -- (-4.3992850504302572, 78.7100904452371708) -- (-4.4178375567657868, 78.6893039340711908) -- (-4.4280422144867941, 78.6806270876838454) -- (-4.4404109231088134, 78.6742817524038287) -- (-4.4540175044685419, 78.6696665442027125) -- (-4.4661379706261357, 78.6800553202054971) -- (-4.4761567247975522, 78.6828247945182824) -- (-4.4781983242586012, 78.6911442281801214) -- (-4.4756001273435970, 78.7011969817232284) -- (-4.4707766538074241, 78.7081311304940812) -- (-4.4710237830769968, 78.7285795780914128) -- (-4.4744245935208085, 78.7424535809582551) -- (-4.4874133517065564, 78.7493915449146158) -- (-4.5477117802844944, 78.7384063848860904) -- (-4.5616256034388005, 78.7337872300184358) -- (-4.7830278254826348, 78.6309764876382076) -- (-4.9163017462501246, 78.4799155563438262) -- (-4.9379478212348324, 78.4246430755950854) -- (-4.9283609866877489, 78.4154270061188612) -- (-4.9144460503386007, 78.4200431873692168) -- (-4.8739391140287553, 78.4551607275441540) -- (-4.8352867604355376, 78.4925884589748364) -- (-4.8204434195331736, 78.5156461421667160) -- (-4.8006530404598555, 78.5346580838887434) -- (-4.7415924846195852, 78.5911662589169424) -- (-4.7304605355401357, 78.5998182282675799) -- (-4.7063409414700450, 78.6113543541991930) -- (-4.6658328919652376, 78.6246374270179302) -- (-4.5427012900038770, 78.6120523113908405) -- (-4.5047290984993955, 78.5652184087513348) -- (-4.5767772993306277, 78.4730024572745890) -- (-4.6525368919848544, 78.3987361180088556) -- (-4.6874789669499446, 78.3441090755168972) -- (-4.7075777010127364, 78.3015626142698409) -- (-4.7205653460036441, 78.2694033668647364) -- (-4.7143815482899871, 78.2556164770638816) -- (-4.7035579542002148, 78.2470040552756814) -- (-4.6899524860353941, 78.2406807758888192) -- (-4.6717083346893391, 78.2401123475955131) -- (-4.6522274238005386, 78.2475725356679277) -- (-4.6296551706523639, 78.2625096059622649) -- (-4.4988536557753491, 78.3814815094438444) -- (-4.4954528453315916, 78.3878166447957199) -- (-4.1790861920818880, 78.5284039246857049) -- (-4.0514116417012778, 78.5675165399711943) -- (-4.0541946289711106, 78.5611807605090462) -- (-4.0548124521449118, 78.5519479213786553) -- (-4.0504832371480726, 78.5473402294713594) -- (-4.0347126048873339, 78.5438887968651613) -- (-4.0053376176567497, 78.5513771333867936) -- (-3.9994621749327028, 78.5536748409291761) -- (-3.9209207082034974, 78.6078999771103071) -- (-3.9032943800313129, 78.6292498664446242) -- (-3.8998935695876229, 78.6344485501543176) -- (-3.8927813673208180, 78.6483051546191803) -- (-3.8890699754977143, 78.6633075849884449) -- (-3.8903067350405278, 78.6708099251861768) -- (-3.8930897223103598, 78.6777412657876738) -- (-3.8946359500374337, 78.6858314116482660) -- (-3.8933991904947431, 78.6939224299992048) -- (-3.8828861777841377, 78.7037426175490396) -- (-3.7177626375058286, 78.7326283245402578) -- (-3.6617923107298340, 78.7378341768416163) -- (-3.6546812216580040, 78.7343573265375198) -- (-3.6512792980193400, 78.7280074450093394) -- (-3.6469500830224337, 78.7152947335226258) -- (-3.6302521594034642, 78.6939224299992048) -- (-3.6135542357844184, 78.6806270876838454) -- (-3.5881978821715714, 78.6667811438437070) -- (-3.4768783913782761, 78.6252066947472343) -- (-3.4295676077911130, 78.6125113879632522) -- (-3.3862776842114459, 78.6055860695918227) -- (-3.3785465455758752, 78.6067430144340022) -- (-3.3646316092266715, 78.6119422165219248) -- (-3.3278349515448995, 78.6304071761402241) -- (-3.3034058892903042, 78.6431055109099333) -- (-3.2755760165920078, 78.6621497677702024) -- (-3.2545488779761222, 78.6777412657876738) -- (-3.2261011821039127, 78.6985391306146340) -- (-3.1756979430625338, 78.6962224264142236) -- (-3.1064327295011211, 78.6893039340711908) -- (-3.0971553631384210, 78.6881457156141835) -- (-2.9453278228403157, 78.6592625778825436) -- (-2.8609109133870638, 78.6419480053664586) -- (-2.7941192189111557, 78.6252066947472343) -- (-2.7211437267216021, 78.6182802048734573) -- (-2.6719772472229266, 78.6142397014217238) -- (-2.6630093490446649, 78.6142397014217238) -- (-2.4335676333756631, 78.6188681214481164) -- (-2.4187254056682623, 78.6217516800281544) -- (-2.4125404947597771, 78.6257946655855022) -- (-2.3825465711604266, 78.6644654200729008) -- (-2.3726513816238568, 78.6742817524038287) -- (-2.3599732048173996, 78.6788993236323506) -- (-2.3104983703291815, 78.6829433103395104) -- (-2.2511283463044602, 78.6713961745566763) -- (-2.1481567041258076, 78.6379062274664165) -- (-2.1354796405141920, 78.6229088740979023) -- (-2.0705425287548995, 78.5208224408341380) -- (-2.0281787833385114, 78.4321216031600841) -- (-1.9830331638473770, 78.2808888581973861) -- (-1.9749936702222088, 78.2504418768389201) -- (-1.9740663788639008, 78.1827170131383156) -- (-2.0659049587684630, 78.0606130766587540) -- (-2.1806252600053817, 77.9238339575202730) -- (-2.1951591327234188, 77.8986791191314012) -- (-2.2016523986213903, 77.8855421782983512) -- (-2.2866882444434751, 77.6897698735253641) -- (-2.2950372062529589, 77.6601413819816031) -- (-2.2996758894342482, 77.6322283778426652) -- (-2.3098794339603490, 77.5519573678209753) -- (-2.3083332062333413, 77.5365994141385357) -- (-2.3012221171613887, 77.5172604975327459) -- (-2.3302887494024316, 77.3860137782561566) -- (-2.4122321397703019, 77.2436501583295865) -- (-2.4382063165570069, 77.1909726694444629) -- (-2.4450090506394848, 77.1655025152008704) -- (-2.4462458101821087, 77.1315407102645736) -- (-2.4453185188238007, 77.1247501787536009) -- (-2.4375884933832053, 77.1049564266776457) -- (-2.4326403420173550, 77.0840340338981349) -- (-2.4338771015601686, 77.0755563811724329) -- (-2.4425355315539798, 77.0438992693500353) -- (-2.4709832274262573, 76.9732928410167858) -- (-2.5776651482331161, 76.8333910230154657) -- (-2.5872508695852914, 76.8232515718519835) -- (-2.6138439827408297, 76.8040941881658199) -- (-2.6605358299591604, 76.7781821272723874) -- (-2.6886751708418952, 76.7517215912861701) -- (-2.7585593207721413, 76.6639474500749571) -- (-2.7415519289687778, 76.6257320850646835) -- (-2.7251946429915925, 76.6323242424226692) -- (-2.7251634735341259, 76.6161787227800914) -- (-2.7260907648924335, 76.6071808067111704) -- (-2.7703090930253156, 76.4595554419603900) -- (-2.8132917748104251, 76.3877986319882467) -- (-2.8222585597938905, 76.3777185598737987) -- (-2.8401932429555616, 76.3581209040741840) -- (-2.9397618483007188, 76.2747175432711799) -- (-3.0288185541302792, 76.2070485697578448) -- (-3.2230087266496299, 76.1567571023330601) -- (-3.2935106997538011, 76.1411264948038990) -- (-3.3800927733028949, 76.1377671579567732) -- (-3.3893701396656608, 76.1388869204194094) -- (-3.3974096332907626, 76.1388869204194094) -- (-3.4178178355378153, 76.1327515481493577) -- (-3.4295676077911130, 76.1260498426983361) -- (-3.6491152471183410, 75.9431368564903977) -- (-3.6342719062159881, 75.9202990973689822) -- (-3.6206664380511562, 75.9242001379838740) -- (-3.5544936799440805, 75.9459221242712204) -- (-3.5486182372199102, 75.9487054886021866) -- (-3.3346376856273769, 76.0568441461314109) -- (-3.2641345993283086, 76.1143202642140579) -- (-3.2496018398053024, 76.1277221986061363) -- (-3.2437263970811334, 76.1305142304903484) -- (-3.2291936375581161, 76.1333043546553654) -- (-3.2109494862120731, 76.1316318760091093) -- (-3.1948693857668959, 76.1277221986061363) -- (-3.1339531340150342, 76.1093062001959169) -- (-3.1268409317482968, 76.1065169561445742) -- (-3.1215844253930265, 76.0942358530667491) -- (-3.1160184508533630, 76.0657771148349582) -- (-3.1147816913105495, 76.0579627180921705) -- (-3.1166373872220627, 76.0501491237828731) -- (-3.1203476658503355, 76.0429025364016979) -- (-3.1274598681170067, 76.0217048591281497) -- (-3.1339531340150342, 75.9754393901416023) -- (-3.1317890831140347, 75.9537250260061256) -- (-3.1215844253930265, 75.9219669381709821) -- (-3.1175646785803814, 75.9169655298542949) -- (-3.0850961227007967, 75.8930214534883163) -- (-3.0789112117922439, 75.8902381198589637) -- (-3.0399505032096070, 75.8774288825845389) -- (-3.0316004282052145, 75.8757620168361910) -- (-3.0102649345997201, 75.8774288825845389) -- (-2.9781058469044619, 75.8746461186842112) -- (-2.9709936446376686, 75.8724303981006472) -- (-2.9428543037549897, 75.8613046888194589) -- (-2.9375977973996532, 75.8579576786974883) -- (-2.9317223546756064, 75.8535112499756963) -- (-2.8785372415593158, 75.7956652354578750) -- (-2.8745163815518842, 75.7906721122316327) -- (-2.8695693433810536, 75.7778796023977890) -- (-2.8751353179207171, 75.7639889644045610) -- (-2.8794645329176234, 75.7589839135043093) -- (-2.9759417960033963, 75.6618146430269434) -- (-2.9811983023587438, 75.6579270106393551) -- (-2.9867642768983966, 75.6546049001204182) -- (-3.1036497422312892, 75.6030019224156433) -- (-3.1392096403703036, 75.5963626283731287) -- (-3.1552897408154252, 75.5952508127020053) -- (-3.1692046771645739, 75.6007940834236365) -- (-3.1812639176021977, 75.6085617276964115) -- (-3.2004353603065487, 75.6251956509007073) -- (-3.2063108030305956, 75.6290816152216649) -- (-3.2146597648402135, 75.6318534311916721) -- (-3.2468188525354704, 75.6379485854201619) -- (-3.2635167761544395, 75.6385117926673161) -- (-3.2916550038422776, 75.6324026141784742) -- (-3.3129916106426016, 75.6174263127579849) -- (-3.4904849727378813, 75.4556541031512751) -- (-3.4982149981785438, 75.4445890250303535) -- (-3.5078007195308420, 75.4152605797890487) -- (-3.5096564154423655, 75.4080760815257776) -- (-3.5112026431693839, 75.3997653681752951) -- (-3.7196172202224447, 75.2776460421400913) -- (-3.7254926629465470, 75.2743404373521940) -- (-3.7765137251617835, 75.2517019803087379) -- (-3.7839353956129709, 75.2505951712317227) -- (-3.7953768128766709, 75.2544621179999638) -- (-3.8306283560262098, 75.2677137564723608) -- (-3.8933991904947431, 75.2980771763409678) -- (-3.9756520490469969, 75.3361926485243316) -- (-3.9880207576690032, 75.3389558395133321) -- (-4.1407755893252931, 75.3422801563531692) -- (-4.1602576134090690, 75.3400639290550203) -- (-4.1488150829504624, 75.2909017011591857) -- (-4.1469605002338454, 75.2842735842629054) -- (-4.1426312852368179, 75.2804071168730076) -- (-4.0814055653005736, 75.2439705485070647) -- (-4.0764585271296863, 75.2412127764353471) -- (-4.0588321989575578, 75.2395597503758466) -- (-3.8457789387231962, 75.2136286091327690) -- (-3.6231410703315685, 75.1733627884947850) -- (-3.4376082146111213, 75.1265037036003207) -- (-3.4147253800836004, 75.1193462947609447) -- (-3.3986463928334532, 75.1176820154424547) -- (-3.3596845710557739, 75.1292593258437336) -- (-3.2672270547825790, 75.1689560615035930) -- (-3.2554772825293381, 75.1750143840278611) -- (-3.2397066502686105, 75.1860404921477965) -- (-3.2353774352716393, 75.1899041751617148) -- (-3.2233181948340799, 75.2070071887063278) -- (-3.1902318157804479, 75.2721261795270493) -- (-3.1930136898553720, 75.2859275776059320) -- (-3.1896128794116825, 75.2931165030230574) -- (-3.1546708044465257, 75.3312295002582175) -- (-3.1395191085546865, 75.3428252705841146) -- (-3.1104524763136441, 75.3494584684928128) -- (-3.0971553631384210, 75.3516749683311531) -- (-2.9580059996468289, 75.3483502426881842) -- (-2.9289393674057860, 75.3384087635287187) -- (-2.8949256969939010, 75.3157513224359292) -- (-2.7854619021121847, 75.2351492067231646) -- (-2.6441473745246302, 75.1441402109920631) -- (-2.5161299601124218, 75.0681193540898306) -- (-2.3714146220811103, 75.0031578886833472) -- (-2.3151359403156846, 74.9663106651550351) -- (-2.3074059148750337, 74.9393696349386431) -- (-2.3024588767040806, 74.9272768657456112) -- (-2.2944182698841269, 74.9124380793739988) -- (-2.2505094099356278, 74.8454243449434244) -- (-2.2415415117573771, 74.8234696258717520) -- (-2.2273182204186202, 74.7943782960279009) -- (-2.2041259177067722, 74.7477504970838282) -- (-2.1970137154400455, 74.7378867237212745) -- (-2.0609568074026017, 74.5516981931634319) -- (-2.0257052642531850, 74.5063137219934077) -- (-2.0210676942667485, 74.5019451895641112) -- (-2.0130270874466722, 74.4997531457777882) -- (-2.0056065301903927, 74.5008491598481157) -- (-1.9978753915548328, 74.5117786731195508) -- (-1.9446902784385307, 74.4882685676559078) -- (-1.8812993944062566, 74.4500337815478019) -- (-1.8234755981086761, 74.4058039997088656) -- (-1.8188369149273207, 74.4014254593089248) -- (-1.7603941822607743, 74.2144189166392891) -- (-1.7594668909024662, 74.2068070515732217) -- (-1.7514273972773649, 74.0854659408952756) -- (-1.7539009163628136, 74.0680730313376046) -- (-1.7585384863492617, 74.0533993529979142) -- (-1.7647233972578138, 74.0403558231050170) -- (-1.7733818272516813, 74.0305824486363377) -- (-1.7804940295184744, 74.0191677322835915) -- (-1.7795656249652698, 74.0115751550833068) -- (-1.7430784354679474, 73.8265833061373655) -- (-1.6926751964265689, 73.5869327673803326) -- (-1.6864902855180832, 73.5588534305478419) -- (-1.6530944382801238, 73.4423195265179345) -- (-1.6484568682936871, 73.4288297549304332) -- (-1.5726972756393947, 73.2672985253827136) -- (-1.5377552006742492, 73.2226640275505503) -- (-1.5195110493281951, 73.1286356608712111) -- (-1.4978660875382623, 73.0197118922284005) -- (-1.4521015316783064, 72.8269246601520734) -- (-1.4443703930426799, 72.8119572939607451) -- (-1.2971815359260424, 72.5930101108907309) -- (-1.1815328301358405, 72.5306429055222139) -- (-1.1620519192469063, 72.5226522539116729) -- (-0.9749728357994516, 72.4406488377966440) -- (-0.6286456547976094, 72.2678734508632914) -- (-0.5804075798521400, 72.2110710599693419) -- (-0.5739143139541739, 72.2004600821317695) -- (-0.5092866703792065, 72.0711071645794021) -- (-0.4854765444933709, 72.0202771129790733) -- (-0.4693975572432248, 71.9858862685494216) -- (-0.4378562927218889, 71.8727327773651012) -- (-0.4387846972750390, 71.8658686780655387) -- (-0.4372384695479640, 71.8595281485935402) -- (-0.4251781159154341, 71.8399868119149403) -- (-0.2922136633323309, 71.6859191563855234) -- (-0.2232278616928625, 71.6428070816415783) -- (-0.1688349321014064, 71.6232326778386863) -- (-0.1301814653132736, 71.6058468002304664) -- (-0.0918385799044304, 71.5805872021452387) -- (-0.0871998967230799, 71.5763654304125083) -- (-0.0834896180949425, 71.5674167102995966) -- (-0.0906018203616758, 71.5563757460654273) -- (-0.1252344271424450, 71.5416354201330051) -- (-0.1515180721136637, 71.5363760201444308) -- (-0.1573935148377060, 71.5337455058998444) -- (-0.1793479448118931, 71.5221734303997039) -- (-0.1935723493454231, 71.5100824313736041) -- (-0.2257314370408099, 71.4590826237402865) -- (-0.2322247029387771, 71.4470013717319148) -- (-0.2374812092939264, 71.4233512798309391) -- (-0.2402641965637549, 71.4060248991744970) -- (-0.2405736647482648, 71.3981486489460480) -- (-0.2393369052054469, 71.3839711273188726) -- (-0.2362444497512342, 71.3713685214215730) -- (-0.1904798938912105, 71.2077453462655114) -- (-0.1861506788942433, 71.1957046918495990) -- (-0.1419323507613581, 71.0988583886781385) -- (-0.1255438953268292, 71.0737599090657142) -- (-0.1107005544244690, 71.0533673068176057) -- (-0.0599889603936167, 70.9843827475215789) -- (-0.0145338727179131, 70.9196300460718589) -- (0.0680284540187213, 70.8007191163788576) -- (0.0714303776572539, 70.7955091921866710) -- (0.1406955912186671, 70.6882206717465067) -- (0.1570840466533230, 70.6544098059912926) -- (0.1663602998212445, 70.6309974020416149) -- (0.1709989830025958, 70.6122878421707014) -- (0.1716179193713618, 70.6050055136751240) -- (0.1691432870909483, 70.5883734139175090) -- (0.1635773125512891, 70.5753834346802336) -- (0.1598670339232784, 70.5696608907569356) -- (0.1564662234795239, 70.5644662512405461) -- (0.1314193380509972, 70.5369274820326666) -- (0.1255438953268292, 70.5343302644546242) -- (0.1267806548694567, 70.5410936802789905) -- (0.1329644525831032, 70.5530343883524722) -- (0.1372936675801338, 70.5577157898197385) -- (0.1512086039292173, 70.5800511985455614) -- (0.1533737680253112, 70.5873312177865557) -- (0.1546105275679399, 70.5956542031661343) -- (0.1539915911991104, 70.6029206945929246) -- (0.1453342744000173, 70.6190429869249243) -- (0.1215241485143085, 70.6393250082780071) -- (0.0992602503557369, 70.6533667346977552) -- (0.0751406562855179, 70.6658545361688226) -- (0.0491664794986827, 70.6767726865789712) -- (0.0330863790535051, 70.6783443207557980) -- (0.0188619745199753, 70.6773009319876309) -- (0.0123687086220086, 70.6757443040992257) -- (0.0000000000000000, 70.6699844868843456) -- (-0.0386523535932270, 70.6512806336064756) -- (-0.0575154413081074, 70.6466105145025693) -- (-0.0794698712824219, 70.6455675466293798) -- (-0.0878188330919098, 70.6471235773822883) -- (-0.0952405035430264, 70.6492095682236965) -- (-0.1178127566912653, 70.6627249084351519) -- (-0.1444058698468036, 70.6809462966790250) -- (-0.1580113380116309, 70.6934234897334761) -- (-0.1910988302601045, 70.7444582056409814) -- (-0.2575810565516244, 70.8314716472460333) -- (-0.2776797906142909, 70.8465952433914197) -- (-0.3033456124116470, 70.8575385311550150) -- (-0.3271557382973569, 70.8596301136075084) -- (-0.3977200503163374, 70.8481571359552191) -- (-0.4240649210074647, 70.8406445670952962) -- (-0.4704484132363815, 70.8237542353450351) -- (-0.6113299080046455, 70.8022800971797608) -- (-0.6243164398006422, 70.8080064605518373) -- (-0.6335938061633436, 70.8163528384395846) -- (-0.6431795275155235, 70.8314716472460333) -- (-0.6524557806833191, 70.8408529270094789) -- (-0.6744102106576332, 70.8465952433914197) -- (-0.6898713747338542, 70.8471114905027406) -- (-0.7037863110830647, 70.8455496187291516) -- (-0.7285237283270810, 70.8382456522173669) -- (-0.7504792714963008, 70.8267757970774738) -- (-0.7606828160224056, 70.8205188533550540) -- (-0.8002635741688455, 70.7902864820111120) -- (-0.7950070678135066, 70.7777920311856121) -- (-0.7715064101121811, 70.7647576023278617) -- (-0.5785518839406195, 70.7631973666480434) -- (-0.5620398638711787, 70.7653877062857362) -- (-0.5097197031984381, 70.7835166103031099) -- (-0.4343931433632579, 70.8057229949866098) -- (-0.3401433832882583, 70.8163528384395846) -- (-0.3314849532943875, 70.8147915780421187) -- (-0.3203530042150701, 70.8069763598173978) -- (-0.2943777142333298, 70.7736132766982564) -- (-0.2689122675194016, 70.7309138150998393) -- (-0.2566537651933165, 70.7033167256023489) -- (-0.2479953351994446, 70.6866638276976857) -- (-0.2306784752115761, 70.6611685690140661) -- (-0.1304909334976578, 70.5696608907569356) -- (-0.1122467821516071, 70.5530343883524722) -- (0.0299939235994823, 70.4346779244966399) -- (0.0949310353587063, 70.3911073938729572) -- (0.1366758444061465, 70.3693367582789904) -- (0.1431691103041126, 70.3682974569615141) -- (0.1558472871105051, 70.3714154020339748) -- (0.1635773125512891, 70.3693367582789904) -- (0.1876969066215077, 70.3418810415686835) -- (0.2415009561066362, 70.2616041144971462) -- (0.2622197397331015, 70.2289984123453053) -- (0.2755157397132909, 70.1938350567671421) -- (0.2838647015229055, 70.1700472752232542) -- (0.3005626251418186, 70.1302407957631004) -- (0.3772495091545371, 69.9258773660591402) -- (0.3787957368816757, 69.9197045295248216) -- (0.3939474327735136, 69.8347086760996802) -- (0.3967304200432163, 69.8146324587123246) -- (0.3982766477703538, 69.7858149272041430) -- (0.3970398882277264, 69.7631791486277564) -- (0.3942569009577697, 69.7462144307137777) -- (0.3803419646085605, 69.6655249051343901) -- (0.3747759900689013, 69.6526710979832018) -- (0.3664270282595404, 69.6434319215424296) -- (0.3608610537198813, 69.6408657776273259) -- (0.3079842955931252, 69.6121095761614015) -- (0.2269681965835654, 69.5402748607897649) -- (0.1929534129766570, 69.5054034501518032) -- (0.0358693663234609, 69.3358948426925963) -- (0.0020482786307402, 69.2725921170303280) -- (0.0000411882117426, 69.2665417338518949) -- (0.0073047849858889, 69.2628911173748634) -- (0.0312317963368890, 69.2689007396126044) -- (0.0643181753905207, 69.2842323046720878) -- (0.0766868840125926, 69.2873078076256945) -- (0.0837990862793891, 69.2878139610608486) -- (0.0961677949014610, 69.2852574590861536) -- (0.1360569080373793, 69.2627659984136557) -- (0.1641962489201186, 69.2464139108173242) -- (0.1750187298151163, 69.2392563325316956) -- (0.1839866279933078, 69.2310878055998131) -- (0.1910988302600411, 69.2213790010648182) -- (0.1960458684308697, 69.2106453393662377) -- (0.2081051088684950, 69.1897173070986042) -- (0.2424293606599123, 69.1330482762214302) -- (0.2603640438214529, 69.1167228628151520) -- (0.2656205501768549, 69.1131489116362729) -- (0.2714959929007703, 69.1116227769214078) -- (0.4044604454840004, 69.0840802550957847) -- (0.4217773054717430, 69.0861254184089404) -- (0.4758908231413180, 69.1616313342245519) -- (0.4792927467798502, 69.1672528804072755) -- (0.4820757340498068, 69.1733802490334000) -- (0.4864038358518058, 69.1881749952192990) -- (0.4879500635789433, 69.2040113119485909) -- (0.4935160381186015, 69.2372201000575558) -- (0.4956812022145061, 69.2448833746804411) -- (0.5439192771599765, 69.3614725184342120) -- (0.5501041880684008, 69.3732363013267843) -- (0.6020525416420709, 69.4234246947996496) -- (0.6076185161817301, 69.4259838081092653) -- (0.6147307184485270, 69.4275178406829383) -- (0.7427481328606746, 69.4444144437299400) -- (0.9771379998954198, 69.4290500590842896) -- (0.9827039744350143, 69.4280230454047995) -- (0.9898161767018112, 69.4244368990553369) -- (1.0003291894124882, 69.4106100890224837) -- (1.0102243789489878, 69.4019059656465771) -- (1.0148630621303387, 69.4003724282949150) -- (1.0488767325422148, 69.3978141410943437) -- (1.0794895925104007, 69.3962806828377694) -- (1.1073194652086935, 69.3983339017007950) -- (1.1196881738307629, 69.4013990840352335) -- (1.3178992063576676, 69.3758103795675254) -- (1.4273630012392502, 69.3548240349250875) -- (1.4536466462105950, 69.3440723186799062) -- (1.5055961129791142, 69.3220834520313076) -- (1.5281694793221963, 69.3087807513099960) -- (1.5510512006546202, 69.2924191852816023) -- (1.5943422374293945, 69.2576565292556836) -- (1.6348502869340809, 69.2203546709333324) -- (1.6738121087116264, 69.1891989104921663) -- (1.6889638046036548, 69.1779659764996921) -- (1.7041155004954938, 69.1667346303480173) -- (1.8315150917338443, 69.0672619753601964) -- (1.8445016235297766, 69.0545047036360273) -- (1.8649109389717267, 69.0249548234167349) -- (1.8772796475938676, 69.0040609416655997) -- (1.8952143307554046, 68.9658568114286510) -- (1.9004708371108077, 68.9480408462298726) -- (1.9159320011870855, 68.8941003898197692) -- (1.9412883547999329, 68.8035929025571704) -- (1.9440713420698981, 68.7908924510977755) -- (1.9443796970593732, 68.7842787011083630) -- (1.9469578564660901, 68.6243058139888404) -- (1.9474721525133989, 68.4906611719611220) -- (1.9261366589081046, 68.4097206000096207) -- (1.9292280011673448, 68.3763526941937982) -- (1.9221157989005517, 68.3556280740321824) -- (1.8765694292425614, 68.2549869684710302) -- (1.8145076999304151, 68.0172806883955445) -- (1.8120341808449660, 67.9941352082780810) -- (1.8089417253907623, 67.9780410958039312) -- (1.7684336758860759, 67.8187609539338325) -- (1.7656506886161771, 67.8127511170775961) -- (1.7585384863493838, 67.8022051763543914) -- (1.5476503902110639, 67.6102266708342796) -- (1.4848784425476116, 67.5366667391777611) -- (1.4790029998234422, 67.5351686053373896) -- (1.4669437593858179, 67.5626778560032761) -- (1.4258167735121761, 67.6332616545138308) -- (1.4183962162559078, 67.6432751621855175) -- (1.4078820903505722, 67.6507870137118630) -- (1.3027475104656394, 67.7109287849785062) -- (1.2965625995572096, 67.7104198984202981) -- (1.2916155613863232, 67.7064138828201720) -- (1.2897598654749216, 67.6999018272089614) -- (1.3451101426871308, 67.5756928766521838) -- (1.4227243180579614, 67.4501790228880225) -- (1.4292186971508964, 67.4496843755640754) -- (1.4347846716905601, 67.4471860033731474) -- (1.4400411780458966, 67.4436876932097249) -- (1.4418968739572979, 67.4297049647580025) -- (1.4320016844206063, 67.4032316600687693) -- (1.4251989503383848, 67.3937405014521005) -- (1.4094283180777130, 67.3747777145370605) -- (1.3621175344906162, 67.3203799731199126) -- (1.3451101426871308, 67.3044184685408737) -- (1.3355244213349553, 67.2974326020670475) -- (1.3151162190877690, 67.2859703431433189) -- (1.2603837650495626, 67.2630327717691330) -- (1.2375009305219637, 67.2565583076190023) -- (1.2115267537351360, 67.2515754061319910) -- (1.1688546533293172, 67.2515754061319910) -- (1.1546302487957971, 67.2555588316590871) -- (1.1512294383520958, 67.2605419795235946) -- (1.1104119206629062, 67.3238729280694059) -- (1.0983526802252805, 67.3513225910027273) -- (1.0896942502314102, 67.3483206703677695) -- (1.0426929348285683, 67.3183920196263159) -- (1.0331072134763897, 67.3114049410148994) -- (0.9596975752728921, 67.2570499753882132) -- (0.8061390168929511, 67.1728588226424392) -- (0.7838751187343164, 67.1648985462374100) -- (0.7789269673685185, 67.1599213082286468) -- (0.7795459037372858, 67.1539461607774939) -- (0.8444819023016682, 67.1096445960936165) -- (0.8506857375237332, 67.1105742005954937) -- (0.9607484312659217, 67.1459752077238079) -- (0.9715720253557606, 67.1514587374292233) -- (0.9796115189808019, 67.1594158946899569) -- (1.0012564807707964, 67.1847101745919275) -- (1.0036687741361441, 67.1868024032892635) -- (1.0070083588600534, 67.1886006518106598) -- (1.0229025557553801, 67.1977664261285810) -- (1.0309420493804851, 67.1972607752650930) -- (1.0371269602889102, 67.1942771411333553) -- (1.0460937452725121, 67.1868095746892635) -- (1.0541332388973641, 67.1708745054650933) -- (1.0631011370757446, 66.9714569173409302) -- (1.0618643775331795, 66.9639956433262569) -- (1.0581529857101375, 66.9575433504953281) -- (1.0130073662189443, 66.9128598629892082) -- (0.9743550126256537, 66.8766340026394488) -- (0.9137471158631435, 66.8310134949673937) -- (0.8995238245245826, 66.8211004087658864) -- (0.8788050408981174, 66.8106835185726879) -- (0.8673647368293216, 66.8067219904453111) -- (0.8537581554695909, 66.8057280699330249) -- (0.7575903605681930, 66.8250626416226226) -- (0.7291426646959119, 66.8334928490063476) -- (0.7223399306136885, 66.8329876900702260) -- (0.5176356321888212, 66.7730211627996084) -- (0.5027934044813029, 66.7641085812010715) -- (0.4261065204687736, 66.6794354440459500) -- (0.4306427897184102, 66.6756976867182942) -- (0.4375468245375693, 66.6729933541254383) -- (0.4588834313378949, 66.6729933541254383) -- (0.4941349744874320, 66.6814059423381309) -- (0.5037206958396108, 66.6878487020634338) -- (0.5077404426521315, 66.6922944341221324) -- (0.5163988726460031, 66.7096306679277404) -- (0.5210375558273533, 66.7229971449646939) -- (0.5244383662711078, 66.7284377677685541) -- (0.5330967962649789, 66.7358648442716174) -- (0.5445382135287427, 66.7398371660530501) -- (0.5967960352867324, 66.7462703163836011) -- (0.6110204398202624, 66.7462703163836011) -- (0.6549292997688915, 66.7462703163836011) -- (0.6629699065887116, 66.7452771386557515) -- (0.7764534482830731, 66.7155730321964597) -- (0.7878948655469006, 66.7096306679277404) -- (0.7971711187147595, 66.7022059995263419) -- (0.8015003337115998, 66.6982355251418966) -- (0.8073757764357690, 66.6878487020634338) -- (0.8089220041628429, 66.6819083736304350) -- (0.8092303591523849, 66.6754660868443239) -- (0.8073757764357690, 66.6680445526065171) -- (0.8045927891658124, 66.6621056777296275) -- (0.8005730423532916, 66.6576633849868330) -- (0.7891316250894646, 66.6536950955043039) -- (0.7690317778320193, 66.6522132706901687) -- (0.7616112205757452, 66.6527173018069448) -- (0.7412019051335372, 66.6591435284161236) -- (0.7177012474323374, 66.6626097704581042) -- (0.7124447410769987, 66.6601356493341086) -- (0.6212250975414622, 66.6091916245765532) -- (0.6193694016298136, 66.6037465526389951) -- (0.6218429207152602, 66.5983143057285503) -- (0.6286456547977362, 66.5889173079279004) -- (0.6437973506895742, 66.5716211402652078) -- (0.6481265656864155, 66.5676588449857434) -- (0.6543114765948407, 66.5651876238140829) -- (0.6898713747338542, 66.5592562890372932) -- (0.7838751187343164, 66.5493764395880163) -- (1.0074413916790956, 66.4841807954047681) -- (1.0142441257615085, 66.4836920306548507) -- (1.0924772375011127, 66.4925755679256696) -- (1.1122676165744914, 66.4965217491939597) -- (1.1268003760975673, 66.5039400236196201) -- (1.1503010337989465, 66.5217089568854192) -- (1.1719471087835318, 66.5310981597852589) -- (1.1830790578628481, 66.5340698558170374) -- (1.1908090833036331, 66.5350462178021047) -- (1.3268659913410770, 66.5513580052493410) -- (1.5423938838557274, 66.5676588449857434) -- (1.5890857310740583, 66.4669069720840184) -- (1.5708415797280042, 66.2362367812010717) -- (1.5683680606425554, 66.2135967331713431) -- (1.5655850733726677, 66.1978521654659602) -- (1.5566171751942841, 66.1703060712299163) -- (1.5476503902110639, 66.1530865390623291) -- (1.5238402643251567, 66.1206401089484501) -- (1.5117799106925691, 66.1078714413800128) -- (1.5068328725216713, 66.1044288268468279) -- (1.4054085712651903, 66.0602132166544322) -- (1.3713937876582862, 66.0587421933842620) -- (1.3587167240468594, 66.0567726197653116) -- (1.2248238669105480, 66.0106268240058256) -- (1.2099805260079948, 66.0017852613958951) -- (1.1639065019637116, 65.9679297613880067) -- (1.1465907551707510, 65.9527341663998925) -- (1.1144316674754269, 65.9188867099165918) -- (1.1070099970243736, 65.9095821413288547) -- (1.0977337438565147, 65.8933987497774751) -- (1.0896942502314102, 65.8747841520405615) -- (1.0828915161489328, 65.8546923607341057) -- (1.0794895925104007, 65.8400010281062293) -- (1.0776338965988166, 65.8248134241957956) -- (1.0797990606949106, 65.7841766201810714) -- (1.0853650352345687, 65.7606808195106538) -- (0.9536373421939672, 65.7494321541390292) -- (0.9063265586068694, 65.7650901703268147) -- (0.8831342558951480, 65.7680198869066146) -- (0.8698382559147046, 65.7670526988310229) -- (0.7974805868990791, 65.7132394599423435) -- (0.7857297014509323, 65.7005204513681349) -- (0.7597555246640965, 65.6677717549241891) -- (0.7517160310390554, 65.6594617182638842) -- (0.7371821583210165, 65.6506564314216376) -- (0.6314286420674389, 65.6125486800546156) -- (0.5550512262392304, 65.5954590014055867) -- (0.5040301640241207, 65.5881354466814201) -- (0.4796011017693922, 65.5832540278875058) -- (0.4458968995419615, 65.5715341898992676) -- (0.4140461668361802, 65.5573752568213592) -- (0.3889992814076524, 65.5437159802671090) -- (0.3419979660050011, 65.4944305123412676) -- (0.3299387255671866, 65.4817596054916038) -- (0.3194245996619199, 65.4671335343707881) -- (0.3082926505826026, 65.4437288435427860) -- (0.3017993846846365, 65.4334887675441905) -- (0.2934504228750853, 65.4256972734102362) -- (0.2826279419802780, 65.4208255547255959) -- (0.2696402969893755, 65.4198484390629886) -- (0.2535613097392296, 65.4222851101162348) -- (0.2396463733900203, 65.4261797244478629) -- (0.1372936675801338, 65.4573731597652824) -- (0.1094637948819042, 65.4720080152972486) -- (0.0633897708373708, 65.4949256604311643) -- (-0.1051345798847471, 65.5466377558062874) -- (-0.1536832362095684, 65.5607948262920388) -- (-0.1747092616305430, 65.5661704823034484) -- (-0.2124343238655255, 65.5739753911054635) -- (-0.2294417156689475, 65.5754359286818129) -- (-0.2838647015228420, 65.5749525950620580) -- (-0.6425605911466928, 65.5202893349233335) -- (-0.7096617538071067, 65.5061392859214493) -- (-0.7517160310389931, 65.4954085205024228) -- (-0.7919146123593576, 65.4827374619299576) -- (-0.8305669659525213, 65.4685939172940294) -- (-0.8549960282071241, 65.4456836578067112) -- (-0.8534498004800490, 65.4349608310788398) -- (-0.8534498004800490, 65.4286165843104044) -- (-0.8562327877498787, 65.4144824053011149) -- (-0.8642722813749198, 65.4042582708359816) -- (-0.8689109645562697, 65.4013275530914484) -- (-0.8741674709116084, 65.3988932606231259) -- (-0.8865361795336169, 65.3974341229481411) -- (-0.8933389136159671, 65.4008470025809174) -- (-1.0055856957675104, 65.4724907398971112) -- (-1.0133168344032004, 65.4798039277365405) -- (-1.0139346575770618, 65.4866349593902157) -- (-1.0114611384916790, 65.4924885775309065) -- (-0.9956905062310112, 65.5100519498169547) -- (-0.9913624044288863, 65.5134554876674571) -- (-0.9675522785431765, 65.5202893349233335) -- (-0.9638408867201345, 65.5246839536685428) -- (-0.9613673676346254, 65.5300449359704231) -- (-0.9626041271773164, 65.5368801719565255) -- (-0.9706447339973893, 65.5441991289576009) -- (-1.0318693407387296, 65.5978991603348192) -- (-1.0351566053018448, 65.5986639225679085) -- (-1.0454759220984613, 65.6003411503548790) -- (-1.2183306010124542, 65.6091410078138466) -- (-1.2854306504778377, 65.6062029278910757) -- (-1.2906882700280151, 65.6037607624407570) -- (-1.2937796122873220, 65.5983966944901766) -- (-1.2903788018436877, 65.5930171574814693) -- (-1.2764638654944840, 65.5832540278875058) -- (-1.2446142459836651, 65.5495736921238006) -- (-1.2427585500721525, 65.5441991289576009) -- (-1.2421407268981606, 65.5378586886984635) -- (-1.2439953096147764, 65.5315187408269821) -- (-1.2548177905097764, 65.5061392859214493) -- (-1.2572924227899998, 65.5007798728689465) -- (-1.2610027014181393, 65.4963865403763208) -- (-1.2656413845995507, 65.4924885775309065) -- (-1.2708978909548203, 65.4905326428528269) -- (-1.2767733336788669, 65.4890718676173833) -- (-1.4820954552776571, 65.5134554876674571) -- (-1.5674407692841252, 65.4773655405438575) -- (-1.7730723590672979, 65.3954804625724933) -- (-1.7968824849530045, 65.4052352001063184) -- (-1.8519232939808863, 65.4188713350837503) -- (-1.8616637494253201, 65.4202098126144449) -- (-1.8751155966927346, 65.4213079770003105) -- (-1.8939775712127802, 65.4178942427882504) -- (-2.0219949856250561, 65.3935408782702297) -- (-2.1531048554915357, 65.3755203772482787) -- (-2.2369039417709295, 65.3794127636598006) -- (-2.2492726503929372, 65.3842820142824905) -- (-2.2570026758335882, 65.3920818361470708) -- (-2.2594773081138784, 65.3979286835206466) -- (-2.2619508271993269, 65.4101059894166070) -- (-2.2665883971857745, 65.4130229892677448) -- (-2.2730827762787102, 65.4125423699812529) -- (-2.2993664212498652, 65.3803893960034230) -- (-2.3021494085196972, 65.3760025321741978) -- (-2.3148275853262099, 65.3540945302821115) -- (-2.3169916362272094, 65.3487328113085226) -- (-2.3151359403156846, 65.3428877288136931) -- (-2.1982504749827365, 65.3010344246690266) -- (-2.1854487335414881, 65.2893620016481862) -- (-2.1808111635549845, 65.2946129846418017) -- (-2.1776563691859590, 65.2966570708380232) -- (-2.1739460905578190, 65.2954940198146119) -- (-2.1722762981958974, 65.2925689495251618) -- (-2.1713490068375894, 65.2890747661383131) -- (-2.1546510832186767, 65.2470459185542353) -- (-2.1800063236365488, 65.1755993305509946) -- (-2.1846450068179051, 65.1721958353887061) -- (-2.1911382727158761, 65.1707424846206180) -- (-2.2835957889891376, 65.1532518313302234) -- (-2.2903985230716040, 65.1527569991643674) -- (-2.3621383689132509, 65.1906526610036821) -- (-2.3754343688936888, 65.2013435008493332) -- (-2.3874936093312460, 65.2076701405799923) -- (-2.4459363419977920, 65.2237041335910419) -- (-2.6583717790582280, 65.2509339457420481) -- (-2.6648650449561893, 65.2509339457420481) -- (-2.7063003858192496, 65.2392616665088667) -- (-2.7236172458070511, 65.2241993947771448) -- (-2.7434076248803021, 65.1751061123114681) -- (-2.7468084353239361, 65.1605347777938135) -- (-2.7449538526073205, 65.1498492649032954) -- (-2.7369132457874330, 65.1343104605807497) -- (-2.6991892967472864, 65.1080945375144324) -- (-2.7100184568117793, 65.0762810505712821) -- (-2.7331506469985656, 65.0887041029643143) -- (-2.7364991372816490, 65.1111545810916397) -- (-2.7769815833034537, 65.1704294316950410) -- (-2.8531808879464013, 65.2343992915863566) -- (-2.9632636191970056, 65.3063930170393121) -- (-3.0062451877872083, 65.3302402954084300) -- (-3.0291269091196988, 65.3414313430011191) -- (-3.0850961227007967, 65.3682182694179517) -- (-3.1426115640089014, 65.3886710760628489) -- (-3.1880666516845415, 65.4028007912450136) -- (-3.2090937903004266, 65.4076714354610260) -- (-3.2359963716404705, 65.4115653533563233) -- (-3.2557867507136544, 65.4091290019191689) -- (-3.2619705484273100, 65.4066944770681715) -- (-3.2777411806879706, 65.3940231395070128) -- (-3.2944391043070174, 65.3755203772482787) -- (-3.3151567747385089, 65.3643123916242672) -- (-3.4447204168778587, 65.3394790205249762) -- (-3.5294467945155605, 65.3331544467216219) -- (-3.5359400604135209, 65.3321783784611227) -- (-3.6113901848833634, 65.3083305350224492) -- (-3.6225221339626801, 65.3039615494674450) -- (-3.7975408637776695, 65.2062021626287276) -- (-3.8262980278343290, 65.1867692455409440) -- (-3.8501081537201691, 65.1615088007737313) -- (-3.8572203559868958, 65.1522779065596183) -- (-3.8896889118666031, 65.0843118818493451) -- (-3.8964916459489580, 65.0658667744355341) -- (-3.9533881508882973, 64.8957505788590510) -- (-3.9555533149842033, 64.8734774882111083) -- (-4.0653265780504260, 64.5353451770300097) -- (-4.0909912866527485, 64.5165458133637060) -- (-4.1262428298022868, 64.4982277655359582) -- (-4.1373747788815471, 64.4929338300398030) -- (-4.1509802470463688, 64.4924446959056041) -- (-4.2183897646962016, 64.5025679364608635) -- (-4.2335425737830708, 64.5059437831630476) -- (-4.2449839910467713, 64.5112368204834468) -- (-4.2672467760104285, 64.5285929453733047) -- (-4.2743600914721966, 64.5372645709274337) -- (-4.2802344210013352, 64.5473933256863575) -- (-4.2879644464420537, 64.5710209405783218) -- (-4.3151764959664920, 64.6202317651352303) -- (-4.3179594832363124, 64.6250561354619748) -- (-4.3869163418081278, 64.6766982736245239) -- (-4.3937190758906057, 64.6800812739589901) -- (-4.3992850504302572, 64.6820179565475684) -- (-4.4067056076865354, 64.6820179565475684) -- (-4.4258770503908993, 64.6791268574023093) -- (-4.4367006444806725, 64.6747756232511790) -- (-4.4583456062706057, 64.6655993247162115) -- (-4.4657672767217163, 64.6569124223224350) -- (-4.4685502639915482, 64.6520861827114572) -- (-4.4781370985387543, 64.6453243623832350) -- (-4.4840114280678920, 64.6429124160785733) -- (-4.4901952257814814, 64.6424206518478854) -- (-4.5115318325817384, 64.6477381172298919) -- (-4.5211175539340465, 64.6549763310645460) -- (-4.5767772993306277, 64.7076198341122790) -- (-4.6370757279086332, 64.7772132895219954) -- (-4.6475887406191827, 64.7902683020169263) -- (-4.8757936967453830, 64.7660868348199728) -- (-4.9265064039711328, 64.7477270419181394) -- (-4.9357826571389918, 64.7409573490514987) -- (-4.9436807750107636, 64.7322503015647897) -- (-4.9546457448538783, 64.7192045048140443) -- (-4.9574287321236987, 64.7143746064927115) -- (-4.9719626048416146, 64.7056842872072338) -- (-4.9902067561877903, 64.6998866144414109) -- (-5.0211279711453933, 64.6940893506174319) -- (-5.0576162738376231, 64.6936111902418958) -- (-5.1710998155319849, 64.6960246244769479) -- (-5.2082070545929806, 64.6998866144414109) -- (-5.3015918622245497, 64.6747756232511790) -- (-5.4073442652832870, 64.5363118101335260) -- (-5.5134072497212454, 64.4698054753436054) -- (-5.6194702341592047, 64.4312755357522917) -- (-5.6321484109657174, 64.3759279886904778) -- (-5.6497736259430056, 64.2816832808849199) -- (-5.6451360559565575, 64.2778425878488235) -- (-5.6423519554918293, 64.2730385132538089) -- (-5.6268919046105035, 64.2422917969863647) -- (-5.6256551450676904, 64.2360390436442970) -- (-5.6278191959688115, 64.2211595347523456) -- (-5.6321484109657174, 64.2086698039156687) -- (-5.6389511450481393, 64.2000309549585779) -- (-5.6618339795755936, 64.1745863472970086) -- (-5.7431584335746297, 64.1429142427316918) -- (-5.7465603572131725, 64.1385929634644754) -- (-5.7626393444634427, 64.1055000151182526) -- (-5.7675863826342733, 64.0944760061940997) -- (-5.7808834958094977, 64.0580430265064962) -- (-5.7873778749023650, 64.0585175132348610) -- (-5.7917059767044847, 64.0623480149780704) -- (-5.8204642539559295, 64.0891958440905682) -- (-5.8303583302976483, 64.1026290737280675) -- (-5.8507676457397313, 64.1366828297612130) -- (-5.8523138734668727, 64.1424409980829893) -- (-5.8520044052824218, 64.1486728456659563) -- (-5.8455111393844605, 64.1669012818189088) -- (-5.8452016712000789, 64.1750597762546846) -- (-5.8479846584699091, 64.1870709206493615) -- (-5.8739599484515121, 64.2384391042072167) -- (-5.8884915947797438, 64.2552622378412934) -- (-5.9373486060939822, 64.2932202758961324) -- (-6.0897950827607970, 64.3513917990250803) -- (-6.1030910827412468, 64.3547614556023717) -- (-6.1108222213768055, 64.3552497728654771) -- (-6.1238098663675782, 64.3533207633725794) -- (-6.1352512836314119, 64.3494767318617136) -- (-6.1568962454212111, 64.3393722860020745) -- (-6.1643168026774910, 64.3311957941812977) -- (-6.1686460176743845, 64.3187032376679468) -- (-6.1692649540432178, 64.3110047358599388) -- (-6.1649357390462560, 64.2850483594020261) -- (-6.1646262708618735, 64.2778425878488235) -- (-6.1664819667733868, 64.2629416999781995) -- (-6.1714290049442173, 64.2514089392796990) -- (-6.1803969031226016, 64.2379653115043965) -- (-6.1884363967477034, 64.2297988671155622) -- (-6.1974042949259642, 64.2235591625123305) -- (-6.2147200417189241, 64.2177953326448403) -- (-6.2938815580117380, 64.1985808646729197) -- (-6.3003737107148012, 64.1971446263474093) -- (-6.3081048493504941, 64.1981073007783039) -- (-6.3204735579725027, 64.2019442835137966) -- (-6.3312960388673671, 64.2077070062627797) -- (-6.3591270247606282, 64.2278727899096822) -- (-6.3646929993002912, 64.2379653115043965) -- (-6.3659297588429808, 64.2442043764775264) -- (-6.3659297588429808, 64.2514089392796990) -- (-6.3563429242958991, 64.3586195871764488) -- (-6.3405734052300700, 64.3999950286828522) -- (-6.3319149752361987, 64.4072132195494760) -- (-6.1742131054089571, 64.4905246123462632) -- (-6.1355596386208200, 64.5059437831630476) -- (-6.1179333104486346, 64.5107614567778285) -- (-6.1043289554788336, 64.5073854125037371) -- (-6.0990713359285893, 64.5045004443015841) -- (-6.0935053613889369, 64.4943769661222035) -- (-6.0904129059347882, 64.4823348400672103) -- (-6.0879393868492730, 64.4775188296009532) -- (-6.0814461209513127, 64.4751100629241733) -- (-6.0551624759800911, 64.4683627698676389) -- (-6.0477408055289139, 64.4688396228472698) -- (-6.0421748309892500, 64.4707713391687065) -- (-5.8643731139044286, 64.6096163672751658) -- (-5.7372841040453695, 64.7288790692283271) -- (-5.7332632440379498, 64.7332192520316738) -- (-5.6237994491561665, 64.8676679592052778) -- (-5.6203975255175136, 64.8725068814181469) -- (-5.6151421323572626, 64.8836495105632451) -- (-5.5931865891881039, 64.9451424177201346) -- (-5.5925687660141801, 64.9514493060771230) -- (-5.5962790446423201, 64.9630806474754934) -- (-5.6098845128070192, 64.9960282061152554) -- (-5.6092655764382533, 65.0037911212256319) -- (-5.5894751973650685, 65.0736363069290746) -- (-5.5814357037399667, 65.0818852544549742) -- (-5.4722802638476598, 65.1381913888565833) -- (-5.4518720616004845, 65.1411126953723141) -- (-5.4255884166293953, 65.1411126953723141) -- (-5.3448828989991242, 65.1595607664002046) -- (-5.3275649258164819, 65.1653910182537714) -- (-5.3226178876455172, 65.1683132952642268) -- (-5.3117954067506510, 65.1814307246090436) -- (-5.3006634576713356, 65.2037736704941011) -- (-5.2895315085920194, 65.2504525275821834) -- (-5.2805636104136342, 65.2961701287888872) -- (-5.2329444718369960, 65.3326742967857825) -- (-5.1779025496143287, 65.3798949415455155) -- (-5.0814263997234521, 65.4822406321505639) -- (-5.0740047292722759, 65.4915106043362272) -- (-5.0690576911013228, 65.5027237608700119) -- (-5.0644190079199669, 65.5329802756113367) -- (-5.0622549570189674, 65.5573752568213592) -- (-5.0619454888345841, 65.5725131221464608) -- (-5.0610181974762769, 65.6247703470167352) -- (-5.0613265524657631, 65.6320971908916562) -- (-5.0631822483772764, 65.6379612505380408) -- (-5.0576162738376231, 65.7518786860067621) -- (-5.0396815906759524, 65.8983005431187649) -- (-5.0356618438634291, 65.9076161754998537) -- (-5.0285485284016724, 65.9085991524791552) -- (-4.9809293898250333, 65.9056502571504410) -- (-4.9320723785107958, 65.8983005431187649) -- (-4.9249601762439923, 65.8968324444621203) -- (-4.9113547080792932, 65.8855649628138451) -- (-4.8717750631277568, 65.8649731083978196) -- (-4.8510573926961431, 65.8620399029235415) -- (-4.8306480772541809, 65.8640047685097585) -- (-4.8142585086246203, 65.8723339262943313) -- (-4.7907589641182051, 65.8880155867705071) -- (-4.7214926373619512, 65.9576377688060944) -- (-4.7180918269181280, 65.9610724088341414) -- (-4.7134531437367837, 65.9664621722266133) -- (-4.6884062583082526, 66.0052217657773355) -- (-4.6853138028540373, 66.0116110461968475) -- (-4.6979919796605616, 66.0759293666958456) -- (-4.7085061055659532, 66.2101496357920922) -- (-4.7047947137429045, 66.2150687910755096) -- (-4.6850043346697214, 66.2268816491153132) -- (-4.6738723855903377, 66.2313021690628716) -- (-4.5826527420547425, 66.2559215146239495) -- (-4.5325589711977470, 66.2588689646550648) -- (-4.4357733531223653, 66.2642926941390300) -- (-4.3235254577759754, 66.2770859778136412) -- (-4.2217916883349229, 66.3036947367857152) -- (-4.1191306275354949, 66.2874344664632105) -- (-4.0433721480761093, 66.2662531247415103) -- (-4.0282193389893735, 66.2637928450863569) -- (-4.0041008581140591, 66.2702024903651647) -- (-3.9935867322086667, 66.2756127950401037) -- (-3.9793623276751360, 66.2854593403659464) -- (-3.9691587831490351, 66.2903972457350079) -- (-3.9079330632126577, 66.2766002322841672) -- (-3.9005113927615476, 66.2746253698086321) -- (-3.8321734705583759, 66.2495271743504475) -- (-3.8269169642031620, 66.2465658863893765) -- (-3.8232066855750229, 66.2431173891441460) -- (-3.8108379769530147, 66.2219617782211429) -- (-3.8034163065018385, 66.2130954247172667) -- (-3.7944484083234542, 66.2067044559863263) -- (-3.7817713447119057, 66.2017858073034660) -- (-3.7737307378919627, 66.2007992794976872) -- (-3.7517763079176469, 66.2007992794976872) -- (-3.6750894239050576, 66.2007992794976872) -- (-3.3711259883194842, 66.2450914852639983) -- (-3.3500988497035995, 66.3224096059332169) -- (-3.3544280647005711, 66.3337378168496770) -- (-3.3553553560588787, 66.3401529891066559) -- (-3.3544280647005711, 66.3633164089757202) -- (-3.3473158624337662, 66.4244702304198285) -- (-3.2950569274808088, 66.5449243194109528) -- (-3.1756979430625338, 66.6690367826504513) -- (-3.0841688313424882, 66.7482567084293947) -- (-2.9536778978447087, 66.8816080896458800) -- (-2.9382167337684861, 66.9084019856040157) -- (-2.8420489388670847, 67.0484754521217212) -- (-2.7514471185053480, 67.1688794824181628) -- (-2.7375321821561327, 67.1793265658116781) -- (-2.7279464608039574, 67.1843193444020557) -- (-2.7038268667338676, 67.1912810667905518) -- (-2.6787799813053366, 67.1962620344757795) -- (-2.6732140067656727, 67.1987508406321865) -- (-2.6627009940551902, 67.2042379614206453) -- (-2.6546603872351802, 67.2131992687140496) -- (-2.6494038808798441, 67.2246655281554126) -- (-2.6481671213372198, 67.2331375824490038) -- (-2.6490955258904254, 67.2396101648115518) -- (-2.6509501086070411, 67.2455934604467700) -- (-2.6571350195155374, 67.2555588316590871) -- (-2.6614631213174684, 67.2595424541430873) -- (-2.6679575004104037, 67.2620332154859710) -- (-2.7390784098832732, 67.2087247672006356) -- (-2.8708061029237451, 67.0852709295565575) -- (-2.9267753165047865, 67.0251150768597483) -- (-2.9666644296406965, 66.9729446058172613) -- (-2.9774880237305372, 66.9585391403763026) -- (-3.0043894918756844, 66.9272553946914144) -- (-3.0220158200478129, 66.9118646361660012) -- (-3.1701319685228704, 66.8478742454303756) -- (-3.2202257393797988, 66.8364613909275391) -- (-3.2492923716207862, 66.8349762138119132) -- (-3.2891814847567740, 66.8384500001369588) -- (-3.3021691297476692, 66.8419221505694026) -- (-3.3630853814995301, 66.8126714936782946) -- (-3.4682210745793154, 66.7497402933805688) -- (-3.4753332768460439, 66.7408160131840731) -- (-3.4818265427440145, 66.7299209892324399) -- (-3.5084196558995520, 66.6838807138651077) -- (-3.5180053772518503, 66.6556792159243514) -- (-3.5161496813403375, 66.6492514834070704) -- (-3.5155318581664128, 66.6428117955982913) -- (-3.5161496813403375, 66.6274875414197396) -- (-3.5195516049789353, 66.6131434915995442) -- (-3.5220251240643732, 66.6067191356047061) -- (-3.5248081113342051, 66.6012742299393352) -- (-3.5362495285979043, 66.5879260594354037) -- (-3.5461447181345958, 66.5815054833358317) -- (-3.6382916530284444, 66.5548178568413533) -- (-3.6441682089475100, 66.5533253885943168) -- (-3.7248737265777252, 66.5513580052493410) -- (-3.7397170674800773, 66.5513580052493410) -- (-3.7885740787943143, 66.5533253885943168) -- (-3.8031068383173219, 66.5548178568413533) -- (-3.8417591919106844, 66.5622290183073346) -- (-3.8835040009580513, 66.5716211402652078) -- (-3.9434929613516720, 66.5854559992878308) -- (-3.9632833404248666, 66.5992914425268765) -- (-4.0875904762090034, 66.7210131556819874) -- (-4.1738619683788700, 66.8126714936782946) -- (-4.1786631780168069, 66.8270634645383694) -- (-4.1868496133698327, 66.8468798187572162) -- (-4.2097324478972311, 66.8880525327806055) -- (-4.2697214082908523, 66.9803941890208137) -- (-4.2749779146461213, 66.9833681344428129) -- (-4.3968104181499115, 66.9689751134392282) -- (-4.4144367463220968, 66.9635039708347222) -- (-4.6757280815018616, 66.8434074553308051) -- (-4.6778932455977689, 66.8295284179543927) -- (-4.6816035242258982, 66.8250626416226226) -- (-4.6880979033187771, 66.8240684955771087) -- (-4.6945911692168050, 66.8260567998959232) -- (-4.7230388650889594, 66.8404386582772645) -- (-4.7620018000615341, 66.8687046730493364) -- (-4.7790080786699898, 66.9505874660119389) -- (-4.7790080786699898, 66.9580492888821937) -- (-4.7768440277690578, 66.9644998342269417) -- (-4.7316972950829594, 66.9972861528421220) -- (-4.7109796246514009, 67.0052281175407813) -- (-4.6880979033187771, 66.9938055289947840) -- (-4.6537736515276675, 66.9863564966231451) -- (-4.6469709174451914, 66.9858503808250418) -- (-4.6336738042699794, 66.9878426712444650) -- (-4.5823432738702925, 67.0072208862656851) -- (-4.5396711734645399, 67.0285970511234552) -- (-4.5362703630207948, 67.0330597804429544) -- (-4.5257562371154041, 67.0569290145104446) -- (-4.5239016543987773, 67.0638940693739585) -- (-4.5251384139414785, 67.0708454033559605) -- (-4.5308546697937020, 67.0796683554210773) -- (-4.5359597816415036, 67.0842735748531283) -- (-4.6726356260476472, 67.0877563123405878) -- (-4.7446838268788918, 67.0837811690886525) -- (-4.9382561762243187, 67.1917884731040687) -- (-5.0922488806183299, 67.1863021997152288) -- (-5.1522378410118845, 67.1768383592982303) -- (-5.1717187519007508, 67.1728588226424392) -- (-5.1955288777864572, 67.1653914475288332) -- (-5.2051145991387662, 67.1589230299291984) -- (-5.2125362695899433, 67.1494624027273943) -- (-5.2165549032075580, 67.1360233746588051) -- (-5.2171738395763905, 67.1061609134530244) -- (-5.2211935863889130, 67.0653816725034488) -- (-5.2416017886360891, 67.0360499829447321) -- (-5.2468594081861442, 67.0335661923434714) -- (-5.3247830517415471, 67.0032497082092817) -- (-5.3318941408133202, 67.0047361938736117) -- (-5.3374612285479470, 67.0077128250751599) -- (-5.3513750517021892, 67.0186453735163781) -- (-5.3742578862297092, 67.0266037545641922) -- (-5.3816795566808970, 67.0280906704028183) -- (-5.3881739357736986, 67.0270940222224993) -- (-5.4122924166489472, 67.0191499058511369) -- (-5.4172394548199101, 67.0166648462231649) -- (-5.4438325679755044, 66.9942973860550381) -- (-5.5007290729148544, 66.9356962721378324) -- (-5.5718510955826215, 66.9575433504953281) -- (-5.6132853232507749, 66.9759182768384136) -- (-5.6231816259823750, 66.9823845564843197) -- (-5.5896611009146699, 67.0688653847494720) -- (-5.5332599677093075, 67.0816916115189343) -- (-5.4602221366050099, 67.1195949656390098) -- (-5.4481617829723445, 67.1240770632546884) -- (-5.4113651252906401, 67.1489695988052517) -- (-5.4181678593731162, 67.1658843518337392) -- (-5.4373393020774783, 67.2087247672006356) -- (-5.4435230997910002, 67.1838137778852200) -- (-5.4441420361598327, 67.1693706155867005) -- (-5.4463072002557293, 67.1629018787429146) -- (-5.4540372256965135, 67.1534407846923784) -- (-5.4592937320516599, 67.1499534176307975) -- (-5.4694983897727356, 67.1449771020736961) -- (-5.5704896582102430, 67.1322375568386036) -- (-5.5779113286614193, 67.1304459397519935) -- (-5.6383945475941841, 67.1340274220534212) -- (-5.7787194449084973, 67.1409991376661850) -- (-5.8204642539559295, 67.1663898059230746) -- (-5.8408724562030496, 67.1793265658116781) -- (-5.7453235976704828, 67.2366141986994137) -- (-5.7341916485911666, 67.2416087334860322) -- (-5.7276972694982859, 67.2425954777711894) -- (-5.7041966117968954, 67.2421021041181177) -- (-5.6909006118165806, 67.2445941204939430) -- (-5.6859535736457500, 67.2475778239156057) -- (-5.6822421818227120, 67.2520670434937813) -- (-5.6794591945528810, 67.2665250352394963) -- (-5.6794591945528810, 67.2740017876111693) -- (-5.6936824858915047, 67.3607943753450371) -- (-5.6949203586291031, 67.3677857413200343) -- (-5.6995590418104580, 67.3862544206520937) -- (-5.7017230927114468, 67.3922468244458202) -- (-5.7075985354356149, 67.4022303619255467) -- (-5.7174926117773444, 67.4092253280942941) -- (-5.7682053190030951, 67.4287051354032769) -- (-5.7756269894542145, 67.4301994892647372) -- (-5.8034568621525118, 67.4312011354870151) -- (-5.8541684561834870, 67.4312011354870151) -- (-5.8322140262092379, 67.5136594173687143) -- (-5.6794591945528810, 67.5917064952526658) -- (-5.6738932200132286, 67.5947175128040101) -- (-5.6568858282097327, 67.6197416132382045) -- (-5.6547206641138370, 67.6257514413753853) -- (-5.6513198536701461, 67.6573077116578645) -- (-5.3928115057601467, 67.6893707008003673) -- (-5.3792060375954476, 67.6918766120080733) -- (-5.3748768225984191, 67.6953875449767111) -- (-5.3074661917536776, 67.7806243754315147) -- (-5.2978804704015028, 67.7961821274428758) -- (-5.2833477108783740, 67.8272958403983779) -- (-5.2731430531573551, 67.8514000965285362) -- (-5.2573724208967594, 67.8684759172561769) -- (-5.1980023968719067, 67.8996176892433567) -- (-5.1729555114434422, 67.9066561030265632) -- (-5.1664611323506300, 67.9076637085730255) -- (-5.1503821451003597, 67.9066561030265632) -- (-5.1101835637800006, 67.9021283658082666) -- (-5.0396815906759524, 67.9006252063466178) -- (-5.0093770856973103, 67.9106721286610195) -- (-5.0041205793420307, 67.9131849605188478) -- (-4.7777713191273659, 68.0666078118066480) -- (-4.7106712696618596, 68.1144563449177554) -- (-4.6683064110506853, 68.1643406399642942) -- (-4.6120288424801679, 68.2440356796499685) -- (-4.5984233743153471, 68.2702773160492029) -- (-4.5622434266127270, 68.3561428950285261) -- (-4.5585331479845861, 68.3702863066003346) -- (-4.5551323375407620, 68.3849472284180848) -- (-4.5192629712174330, 68.5453576582304862) -- (-4.5208080857495423, 68.5823317340377230) -- (-4.5424541607343718, 68.6685318665495288) -- (-4.5906922356797768, 68.7512712006053306) -- (-4.5934763361445166, 68.7568510331322926) -- (-4.5947130956872080, 68.7634478790983934) -- (-4.5931668679601323, 68.7710802775375498) -- (-4.5712113247909096, 68.8468020366763369) -- (-4.5121507689505718, 68.9673819847360505) -- (-4.5232827180299555, 68.9811300284225268) -- (-4.5981139061309646, 69.0794870908271434) -- (-4.6154296529239254, 69.1116227769214078) -- (-4.6179042852042151, 69.1172407868506866) -- (-4.6191410447469616, 69.1238694105172016) -- (-4.6191410447469616, 69.1315217593910205) -- (-4.6046082852239003, 69.2612351470883567) -- (-4.6030609443017845, 69.2694067757186076) -- (-4.6008968934008516, 69.2750359548185628) -- (-4.5937846911339921, 69.2862826267371759) -- (-4.5845084379661998, 69.2939506379747456) -- (-4.6015158297696841, 69.3307806649122256) -- (-4.7125258523784757, 69.3317916879684333) -- (-4.7972522300161664, 69.3143998347294286) -- (-4.8049822554568280, 69.3123635154263980) -- (-4.9138282271647427, 69.2760609904987916) -- (-4.9432043275901796, 69.2535721342544832) -- (-4.9840207320843408, 69.2213790010648182) -- (-5.0096865538816937, 69.1743892701618392) -- (-5.0532870588406507, 69.1075458820168507) -- (-5.2629383954364135, 69.0861254184089404) -- (-5.2774722681543276, 69.0851028301760408) -- (-5.2848928254106626, 69.0866302917755064) -- (-5.2944785467627833, 69.0937741345661465) -- (-5.2997684489653354, 69.1000500830588749) -- (-5.2901504449608394, 69.1330482762214302) -- (-5.2777817363387101, 69.1560230482889295) -- (-5.2564451295384522, 69.1866455808239351) -- (-5.1775941946247981, 69.2801465424649336) -- (-5.1417248283013892, 69.3210578219485427) -- (-5.1052365256091701, 69.3481632146339706) -- (-5.0681292865481069, 69.3573809363414853) -- (-5.0542143501989587, 69.3604463791972137) -- (-5.0390637675019612, 69.3701722068182249) -- (-5.0158714647901226, 69.3891095785689487) -- (-4.9676333898447078, 69.4336653313910688) -- (-4.8584779499524107, 69.5371895345384843) -- (-4.8507468113168519, 69.5464311010533436) -- (-4.8476554690575453, 69.5520670468217759) -- (-4.8430167858761877, 69.5654106197705033) -- (-4.8368318749677037, 69.6044125893672145) -- (-4.8359045836093957, 69.6131389777924312) -- (-4.8204434195331736, 69.6958378984793541) -- (-4.6713988665048900, 69.8697254615356513) -- (-4.6234702597438675, 69.9057909760740870);
\draw[uk] (-3.2486745484469948, 80.8611780104319706) -- (-3.2721752061483182, 80.8570142307703605) -- (-3.2817609275004935, 80.8582066106394564) -- (-3.2925834083954930, 80.8671233079357421) -- (-3.3068078129290246, 80.8825935897474579) -- (-3.3173208256395079, 80.9004382022113191) -- (-3.3807117096717820, 81.0457599264109945) -- (-3.3770003178487449, 81.0517200362183701) -- (-3.3717438114934093, 81.0564850346822254) -- (-3.3507166728775233, 81.0749769871880233) -- (-3.3445328751639458, 81.0779592302898919) -- (-3.3114453829154726, 81.0827259154472273) -- (-3.2461999161665713, 81.0904735062722608) -- (-3.2273379416466019, 81.0892945661938853) -- (-3.2217719671069394, 81.0862968357950109) -- (-3.2131135371130721, 81.0755674145635084) -- (-3.2035278157607627, 81.0588762348867533) -- (-3.2022910562180718, 81.0505245418629130) -- (-3.2437263970811334, 80.8653441646511908) -- (-3.2486745484469948, 80.8611780104319706);
\draw[uk] (-3.5971657803497656, 80.9545979865548304) -- (-3.6611744875559649, 80.9545979865548304) -- (-3.6689056261915911, 80.9557919356269764) -- (-3.6744716007312550, 80.9599612621867522) -- (-3.8194952937519853, 81.1603027712100982) -- (-3.8216604578478925, 81.1674694605232219) -- (-3.8216604578478925, 81.1764234220542562) -- (-3.8049625342288458, 81.2307874079982071) -- (-3.7984681551359776, 81.2445433433882869) -- (-3.7826986360702133, 81.2588937233666542) -- (-3.7282756502163230, 81.2798173749156234) -- (-3.7131239543244172, 81.2810165739412724) -- (-3.5990214762613566, 81.1943388924482292) -- (-3.5937638567112344, 81.1895650240727917) -- (-3.5786132740141814, 81.1662720979793164) -- (-3.4914122640961898, 81.0159493986511734) -- (-3.4895565681846650, 81.0087973411955033) -- (-3.4920312004650889, 80.9784120053732579) -- (-3.4972877068202348, 80.9742566502540342) -- (-3.5180053772518503, 80.9665040403150158) -- (-3.5408870985843519, 80.9623343460101665) -- (-3.5971657803497656, 80.9545979865548304);
\draw[uk] (-3.1092157167709527, 81.3324626230667747) -- (-3.0677803759078359, 81.2600753647304543) -- (-3.1441577917360428, 81.1764234220542562) -- (-3.1939420944085879, 81.2074815216528521) -- (-3.3061888765601912, 81.3067275932085209) -- (-3.3488620901609072, 81.3079272280381673) -- (-3.4032839628199012, 81.2989552056768758) -- (-3.4626551000395294, 81.2822157923834965) -- (-3.4728586445657514, 81.2720478107101627) -- (-3.4728586445657514, 81.2630822857808255) -- (-3.5313024904270733, 81.2451351232380432) -- (-3.5492371735888106, 81.2433447346860049) -- (-3.5566577308450804, 81.2457419714844917) -- (-3.6772545880010870, 81.3204976405656339) -- (-3.6883865370804019, 81.3294722733852637) -- (-3.7242559034038005, 81.3635862327917323) -- (-3.7415716501966938, 81.4204885712463238) -- (-3.7437368142926011, 81.4276804829760721) -- (-3.7493027888323205, 81.4588441598571933) -- (-3.7493027888323205, 81.4678334232082193) -- (-3.7307491693017609, 81.6671596172083980) -- (-3.7295124097590695, 81.6749707789955863) -- (-3.6930241070668504, 81.7170752271658642) -- (-3.6488057789339021, 81.7399464317251585) -- (-3.6426219812203806, 81.7429555865438999) -- (-3.6055147421593183, 81.7531884961893667) -- (-3.5832508440006179, 81.7585925988269508) -- (-3.5650066926545749, 81.7585925988269508) -- (-3.5575861353982949, 81.7555849740950293) -- (-3.4252395059891132, 81.7002409629850348) -- (-3.4209102909921523, 81.6948238044903690) -- (-3.3377290278866836, 81.5878459111601728) -- (-3.3343282174428710, 81.5812501816963760) -- (-3.3349460406168516, 81.5578337160855114) -- (-3.2808325229472772, 81.4336714000531998) -- (-3.1750790066936996, 81.3845545822285743) -- (-3.1092157167709527, 81.3324626230667747);
\draw[uk] (-2.9849085809868727, 81.8609985272843375) -- (-2.9898556191576477, 81.8356906734819205) -- (-2.9941859473495729, 81.8411029832108881) -- (-2.9963499982505826, 81.8483359231357781) -- (-2.9985140491515150, 81.8646093728059157) -- (-2.9972772896088911, 81.8923454228224017) -- (-2.9951132387078916, 81.9013875556234012) -- (-2.9883105046254146, 81.9152709581542808) -- (-2.8862672669999667, 82.0668183446421438) -- (-2.8448319261369175, 82.0940340410109570) -- (-2.8213312684355940, 82.0982527739735559) -- (-2.6877467662886905, 82.1170048914119945) -- (-2.6797072726636451, 82.1176211164040382) -- (-2.6741424113188894, 82.1151954407041842) -- (-2.6716677790385432, 82.1097455823239102) -- (-2.6685764367792366, 82.0946304829398912) -- (-2.6608452981436654, 82.0505024454413388) -- (-2.6605358299591604, 82.0432475001051387) -- (-2.7780402316610084, 81.9677579071777984) -- (-2.8618393179402686, 81.9617235508431179) -- (-2.8689504070122323, 81.9617235508431179) -- (-2.9171895951526006, 81.9484462266955234) -- (-2.9221366333234315, 81.9430086664189332) -- (-2.9849085809868727, 81.8609985272843375);
\draw[uk] (-3.2171332839255951, 82.0770962321330444) -- (-3.1722960194238805, 82.0178771150625892) -- (-3.1760074112469172, 81.9774243353985952) -- (-3.2032183475763798, 81.9321542951680755) -- (-3.1982713094056159, 81.9828470429580989) -- (-3.1982713094056159, 81.9919024397678982) -- (-3.2026005244024551, 81.9973414285298361) -- (-3.2690827506941034, 82.0517120349277462) -- (-3.2935106997538011, 82.0607758848431104) -- (-3.3012418383893620, 82.0613894646007509) -- (-3.3083540406561545, 82.0595639710193154) -- (-3.3133010788269854, 82.0541356245975066) -- (-3.3108275597415471, 82.0468824975115751) -- (-3.3139189020009101, 82.0281481886243853) -- (-3.3262887238178260, 82.0094164594424626) -- (-3.3309262938043398, 82.0039917998626322) -- (-3.3386574324398990, 82.0027808180665687) -- (-3.3432961156212548, 82.0082227827039532) -- (-3.3711259883194842, 82.0547339197543693) -- (-3.4215281141660761, 82.1400003211643508) -- (-3.4236932782619842, 82.1472626080438602) -- (-3.4246205696202918, 82.1557238343527132) -- (-3.4193640632649447, 82.1557238343527132) -- (-3.3940077096520977, 82.1508874842141239) -- (-3.2251738907456042, 82.0867769914898986) -- (-3.2171332839255951, 82.0770962321330444);
% \draw[uk] (-1.4428241653156062, 84.7267600730597366) -- (-1.4323111526051118, 84.6816540809161751) -- (-1.2956353081988459, 84.6440739017741208) -- (-1.1592689319771414, 84.6271747678962214) -- (-1.1546302487957971, 84.6209154874526490) -- (-1.1552491851645628, 84.6115320790189571) -- (-1.1867893364910547, 84.4271727892801778) -- (-1.2257511582687226, 84.2426456851788998) -- (-1.2959447763833514, 84.2034268250630618) -- (-1.2977993590999675, 84.2115122281748967) -- (-1.3213000168012907, 84.2289496328856018) -- (-1.3302679149795524, 84.2314282241817409) -- (-1.3345971299765136, 84.2245887712381460) -- (-1.3364528258881045, 84.2146243049849375) -- (-1.3336698386182060, 84.1978220081582691) -- (-1.3197549022690580, 84.1412244507682203) -- (-1.3141889277294054, 84.1287916669237461) -- (-1.2885231059321749, 84.0610561497668130) -- (-1.2535810309669519, 83.9648610674157680) -- (-1.2529632077931605, 83.9561732737314372) -- (-1.2563640182367837, 83.9481160493638612) -- (-1.2764638654944840, 83.9208365752541852) -- (-1.2835760677612216, 83.9233047461218717) -- (-1.2947080168405378, 83.9332318291103121) -- (-1.3030569786500221, 83.9326024192080240) -- (-1.3216094849856739, 83.9102975742389390) -- (-1.3494393576840367, 83.8557769592699742) -- (-1.3587167240467370, 83.6770098450229085) -- (-1.4029350521796189, 83.3213895104429128) -- (-1.4054085712650677, 83.3127776716785036) -- (-1.4122113053474112, 83.2967894664649151) -- (-1.4548845189480712, 83.3109363610825113) -- (-1.5272421879637548, 83.3865854268038760) -- (-1.5340449220461092, 83.4099812386627661) -- (-1.5337354538617265, 83.4192117747798534) -- (-1.4824049234620400, 83.6201867287594496) -- (-1.4805915289569733, 83.6269854842251590) -- (-1.4765294807380045, 83.6368546923342109) -- (-1.4722002657410318, 83.6436572176904463) -- (-1.4653975316586885, 83.6467430158347014) -- (-1.4598315571190246, 83.6417897949024081) -- (-1.4561212784908848, 83.6467430158347014) -- (-1.4536466462104725, 83.6553786056621362) -- (-1.4119018371630283, 83.8353321628010804) -- (-1.4060275076338338, 83.8650638277826488) -- (-1.4038623435379267, 83.8848857006385202) -- (-1.4029350521796189, 83.9034724673359307) -- (-1.4044812799067596, 83.9183506355761892) -- (-1.4304554566935987, 84.1175938995216228) -- (-1.4375676589603255, 84.1654761411942189) -- (-1.5621831497338807, 84.1978220081582691) -- (-1.5699142883695738, 84.1959583113395809) -- (-1.5742435033664686, 84.1891064979934498) -- (-1.5699142883695738, 84.1835184562378060) -- (-1.5445579347565928, 84.1561510601606386) -- (-1.5179648216010546, 84.1393621962343872) -- (-1.5062150493477469, 84.0927407147622432) -- (-1.5120893788770187, 84.0560927030049072) -- (-1.6125863887753149, 83.9636173390054665) -- (-1.6181523633149784, 83.9667166512551688) -- (-1.6592793491885534, 84.0008311578985456) -- (-1.6923657282421860, 84.0312626405800103) -- (-1.7000957536828376, 84.0436766181761925) -- (-1.7007146900516146, 84.0523569253824263) -- (-1.6994779305089123, 84.0604435259544687) -- (-1.6917467918733531, 84.0753536532348988) -- (-1.7687431440703254, 84.1082726856715368) -- (-1.8342980790035430, 84.1281605720817254) -- (-1.8423375726286442, 84.1312691025287478) -- (-1.8531600535235220, 84.1418377517068734) -- (-1.8639836476134184, 84.1611213034596091) -- (-1.8680033944259415, 84.1766678087575713) -- (-1.8847013180449770, 84.2513496488400335) -- (-1.8843918498604713, 84.2606891258746060) -- (-1.8828456221333976, 84.2706621176043029) -- (-1.8556346858039341, 84.2962084220475703) -- (-1.8426470408130273, 84.3036871885863377) -- (-1.8126531172137430, 84.3130311144660425) -- (-1.7774015740642042, 84.3148977994126625) -- (-1.7696704354286334, 84.3111667207377735) -- (-1.6416530210163018, 84.3304770555482435) -- (-1.6094939333210998, 84.3385765295760876) -- (-1.5884667947050917, 84.3448251165025908) -- (-1.4981755557227676, 84.3866038287467859) -- (-1.4697267466555275, 84.4234179953853356) -- (-1.6094939333210998, 84.6578480474752979) -- (-1.6722658809845523, 84.6785242904319233) -- (-1.7294718541082743, 84.6916847465880807) -- (-1.7347283604635437, 84.6860423274233938) -- (-1.7421500309147313, 84.6860423274233938) -- (-1.7919343335872766, 84.6885364844681732) -- (-1.7940994976831726, 84.6941813519062521) -- (-1.7922438017716600, 84.7048317880415169) -- (-1.7733818272516813, 84.7643828903420768) -- (-1.7220512968519390, 84.8277423151089636) -- (-1.5764075542674674, 84.9841926220619683) -- (-1.5702237565539567, 84.9885968546442996) -- (-1.4493174312135664, 85.0559366429210399) -- (-1.4425146971311562, 85.0515283369331314) -- (-1.4406601144145403, 85.0433334498970055) -- (-1.4332384439634196, 85.0030591021699422) -- (-1.4428241653156062, 84.7267600730597366);
% \draw[uk] (-1.1459718188019183, 84.7374175246867338) -- (-1.2297709050812455, 84.7142294617493405) -- (-1.2551272586940927, 84.7161064345982879) -- (-1.2690421950433075, 84.7230078835088989) -- (-1.2721346504975113, 84.7298941905268066) -- (-1.3027475104657060, 84.8082999726445905) -- (-1.3163529786304602, 84.9615484280369628) -- (-1.3175897381731509, 84.9791660084397336) -- (-1.3151162190877135, 85.0263392503466235) -- (-1.3129521681867145, 85.0464970746683662) -- (-1.3104775359063019, 85.0559366429210399) -- (-1.2356463478052919, 85.2583145604564834) -- (-1.2229681709989018, 85.2677839094111647) -- (-1.2161654369164248, 85.2709391174237794) -- (-1.2075070069225460, 85.2715824511231659) -- (-1.1765857919649430, 85.2696729260146640) -- (-1.1094846293045348, 85.2557826389776352) -- (-1.1032997183961093, 85.2450513609574188) -- (-1.0937139970439307, 85.0987484847217246) -- (-1.0940234652283136, 85.0893010982864979) -- (-1.1329852870059862, 84.7907236419935089) -- (-1.1345315147329933, 84.7806951519793586) -- (-1.1391701979144158, 84.7537237546459892) -- (-1.1416437169998535, 84.7443209906912642) -- (-1.1459718188019183, 84.7374175246867338);
% \draw[uk] (-0.9125103563204525, 85.1744012521710943) -- (-0.9295177481238109, 85.1390916850790802) -- (-1.0414561752859406, 85.1416013967917706) -- (-1.0671208838882655, 85.1598867813821840) -- (-1.0748520225238283, 85.1724942751112337) -- (-1.0683576434309563, 85.2254973238158584) -- (-1.0451664539140770, 85.4144250970963697) -- (-0.9830134426193987, 85.5233257522280468) -- (-0.9777558230692813, 85.5284083470419034) -- (-0.8973586604285509, 85.5195322876307671) -- (-0.8602525345623991, 85.4979910791012685) -- (-0.8546865600227399, 85.4929288521703086) -- (-0.8438640791278692, 85.4675865330874984) -- (-0.8429356745746565, 85.4587361293476135) -- (-0.8500478768413897, 85.4118877363376754) -- (-0.9125103563204525, 85.1744012521710943);

% Ireland
\draw[ire] (-10.7494430638176048, 69.8996067239403658) -- (-10.7571742024532959, 69.8980505351157717) -- (-10.7853135433359739, 69.9006268881833392) -- (-10.7930446819716046, 69.9016600497518823) -- (-10.8023209351393934, 69.9083675814116390) -- (-10.8088142010373662, 69.9186711321011956) -- (-10.8264405292095507, 69.9769242511134735) -- (-10.8255132378512435, 69.9846594231233041) -- (-10.7652148092731839, 70.0099367691679788) -- (-10.7584120751908952, 70.0104605643115434) -- (-10.7543901019885695, 70.0068479343935337) -- (-10.7333640765674581, 69.9547459963016252) -- (-10.7407868602135412, 69.9083675814116390) -- (-10.7494430638176048, 69.8996067239403658);
\draw[ire] (-11.0920621925812526, 71.3907930301858329) -- (-11.0657763212202251, 71.3514160245539131) -- (-11.0608281698543642, 71.3393513787853379) -- (-11.0472227016896554, 71.2622540981802501) -- (-11.0472227016896554, 71.2533412209617865) -- (-11.0496962207750364, 71.2119340796510585) -- (-11.0660857894046085, 71.1417735706074268) -- (-11.0688698898693385, 71.1354914650971040) -- (-11.0769082702994766, 71.1260617146948420) -- (-11.0818575348601769, 71.1229287639051790) -- (-11.0889686239320078, 71.1213632773712021) -- (-11.1081411798314011, 71.1281604661181603) -- (-11.1937948488273769, 71.1846990587737736) -- (-11.2002881147253461, 71.2040773237312266) -- (-11.2018332292575575, 71.2108967767758969) -- (-11.2018332292575575, 71.2192729181948181) -- (-11.3014040609923985, 71.2795486897320956) -- (-11.3258320100521850, 71.2816515556713881) -- (-11.4260195517660215, 71.2963278850113085) -- (-11.4309665899369843, 71.3083991219529452) -- (-11.4275646662982329, 71.3136451103675455) -- (-11.3508777822856537, 71.3713685214215730) -- (-11.3428382886605643, 71.3724211809376499) -- (-11.3088246182486785, 71.3729409013581488) -- (-11.2710984428189480, 71.3587679440587834) -- (-11.2650003611132927, 71.3556256745030595) -- (-11.2089465447191383, 71.3965625292858874) -- (-11.2021438106368709, 71.3981486489460480) -- (-11.1031919152707541, 71.3970956433926460) -- (-11.0963891811882753, 71.3949896747842132) -- (-11.0920621925812526, 71.3907930301858329);
\draw[ire] (-9.3919642125084657, 73.1608656211570718) -- (-9.3990775279701673, 73.1597884945264241) -- (-9.4052624388787205, 73.1624726368677472) -- (-9.4092810724963361, 73.1673132239520783) -- (-9.4117545915817811, 73.1732159395586592) -- (-9.4108273002234739, 73.1812719774602556) -- (-9.4071159084004368, 73.1866437801439531) -- (-9.4000048193284744, 73.1861137405402928) -- (-9.3938199084199887, 73.1818174585988146) -- (-9.3913463893346751, 73.1753686474158371) -- (-9.3898001616074662, 73.1683749832978663) -- (-9.3919642125084657, 73.1608656211570718);
\draw[ire] (-9.4828755010547194, 73.1990136914992036) -- (-9.4918445124278783, 73.1984661543801138) -- (-9.5212194996584643, 73.2022312758866462) -- (-9.5270938291876011, 73.2038410856980590) -- (-9.5314230441845762, 73.2097601335207315) -- (-9.5338965632698880, 73.2156660878831218) -- (-9.5329692719115808, 73.2468539833981964) -- (-9.5181281573990333, 73.2845118342353885) -- (-9.5144167655759855, 73.2915024455164428) -- (-9.5051393992133537, 73.2990399970498459) -- (-9.4964809692194176, 73.2984936887451539) -- (-9.4921517542225118, 73.2952701608636659) -- (-9.4547361601719757, 73.2635321723774098) -- (-9.4516448179126122, 73.2581530926177038) -- (-9.4448420838302560, 73.2215860404049579) -- (-9.4460777301779864, 73.2129781648152260) -- (-9.4504069451750112, 73.2102922701693899) -- (-9.4828755010547194, 73.1990136914992036);
\draw[ire] (-8.2447545209689483, 73.1780552643971305) -- (-8.2865004432113576, 73.0411719854131150) -- (-8.3022699622771867, 72.9934431480368460) -- (-8.3298582716804948, 72.9397406628976483) -- (-8.4083329467151451, 72.8103421308553180) -- (-8.6152012960410111, 72.7098715758086769) -- (-8.7042568886756762, 72.7494140227861550) -- (-8.7101323313997767, 72.7558274826312470) -- (-8.7160066609288585, 72.7579551601936885) -- (-8.7308500018312802, 72.7542115639699034) -- (-8.8223802267462332, 72.6938560986459379) -- (-8.8248537458316161, 72.6874477576348141) -- (-8.8097009367448109, 72.6340738002136135) -- (-8.8013519749353168, 72.6212828739032119) -- (-8.7444554699960442, 72.5642218299825430) -- (-8.7379622040980731, 72.5610152620585325) -- (-8.6578733964468313, 72.5492896764626494) -- (-8.6297351687590496, 72.4869709600998249) -- (-8.7203358759258887, 72.4092567141278209) -- (-8.8508279226183877, 72.3677718164255168) -- (-8.9578193116096205, 72.3199290132569104) -- (-9.0966569805270208, 72.2370688139292838) -- (-9.0830526255571655, 72.1951560903659697) -- (-9.0799612832978571, 72.1887887214968629) -- (-8.9785358688463468, 72.0631635581097356) -- (-8.9677133879513455, 72.0509884924155841) -- (-8.9559636156981615, 72.0414495724894550) -- (-8.9392656920791929, 72.0335089879938550) -- (-8.8403137967129961, 71.9234648105225745) -- (-8.7945492408529731, 71.9134333705635953) -- (-8.7698118236089471, 71.9039093591162839) -- (-8.7670288363391258, 71.8975657377167607) -- (-8.7661004317859099, 71.8912226320808543) -- (-8.7602261022566488, 71.8352300916679809) -- (-8.7599166340722672, 71.7935270953576747) -- (-8.7580609381607424, 71.7803206601960682) -- (-8.7518771404471547, 71.7687120138182735) -- (-8.7401262549990726, 71.7576406714824628) -- (-8.7058031164027501, 71.7328521791006466) -- (-8.7002371418630950, 71.7328521791006466) -- (-8.6247870173932526, 71.7349539774515819) -- (-8.5743837783518853, 71.7391824278839465) -- (-8.5681999806382976, 71.7391824278839465) -- (-8.4912036284413244, 71.6769597813185442) -- (-8.4874933498131853, 71.6706345727760947) -- (-8.4859471220860438, 71.6643231549286952) -- (-8.4871838816288001, 71.6500840255570637) -- (-8.4865649452599676, 71.6374536407771529) -- (-8.4825463116423538, 71.6321877905362498) -- (-8.4714143625630385, 71.6253296331827300) -- (-8.4151356807975013, 71.5958409156538522) -- (-8.1600281433313810, 71.5695273260646161) -- (-8.1411661688113437, 71.5684720110599670) -- (-8.1117900683858526, 71.5863707173507180) -- (-8.1049873343034999, 71.5921527652666043) -- (-7.9547060217325241, 71.7824370353836230) -- (-7.9497578703667955, 71.8341711810092391) -- (-7.9509957431043832, 71.8405096505689897) -- (-7.9537776171793064, 71.8463257538275570) -- (-7.9877924007862227, 71.9039093591162839) -- (-7.9714028321566603, 71.9943439693719114) -- (-7.8573003540935344, 72.1341699132984928) -- (-7.8328724050338359, 72.1479594092299834) -- (-7.8266886073203805, 72.1490073198631734) -- (-7.8205036964118291, 72.1474211436751460) -- (-7.7639166596569291, 72.1182725565363683) -- (-7.7110399015301114, 72.0753557818916022) -- (-7.7002174206351119, 72.0647499265752458) -- (-7.6544528647750907, 72.0107552880336925) -- (-7.6476501306927345, 71.9985731638512618) -- (-7.6260040557080275, 71.9091935671282698) -- (-7.6272408152507865, 71.9007431920575613) -- (-7.4970593499373903, 71.7033228611207534) -- (-7.4089310486610884, 71.4790625924865424) -- (-7.3832663400587659, 71.4412149122176885) -- (-7.3749162650543738, 71.4296565156330843) -- (-7.3693514037096186, 71.4259761932989647) -- (-7.3560542905342743, 71.4354421204430992) -- (-7.3535807714488906, 71.4396411600454684) -- (-7.3109075578481750, 71.4517212755582847) -- (-7.1773241688963028, 71.4659110778357132) -- (-7.0864139935450234, 71.5437453400720358) -- (-7.0867234617294717, 71.5495242363596020) -- (-7.0857950571762576, 71.5553206085579632) -- (-7.0839393612647328, 71.5611003604397524) -- (-7.0799207276471190, 71.5668975909155165) -- (-7.0743536399125579, 71.5716379988097628) -- (-7.0684793103832959, 71.5742697881845231) -- (-7.0564189567507638, 71.5742697881845231) -- (-7.0004497431697894, 71.5611003604397524) -- (-6.9893177940904714, 71.5584709087092108) -- (-6.9825150600079962, 71.5542654852893634) -- (-6.9769501986632401, 71.5484843444543088) -- (-6.9763646581417023, 71.5444821208588735) -- (-6.9806604772913809, 71.5405804815122082) -- (-6.9794237177486895, 71.5326906737113859) -- (-6.9472646300534988, 71.4969257008535379) -- (-6.8690315183137685, 71.4254580117634816) -- (-6.8554260501490578, 71.4133668226387215) -- (-6.8408921774310238, 71.4097023925659187) -- (-6.8282140006246310, 71.4149532828921707) -- (-6.8223396710954383, 71.4117936304023431) -- (-6.7963643811136905, 71.3608729384031903) -- (-6.7941992170177832, 71.3545732407957445) -- (-6.7938908620283085, 71.3477412683183445) -- (-6.7960549129293071, 71.3409231158211270) -- (-6.7997663047523549, 71.3351577156114729) -- (-6.8136801279066512, 71.3230823536174228) -- (-6.8328526838059185, 71.3083991219529452) -- (-6.8446035692540699, 71.3031553731343024) -- (-6.8517157715209303, 71.3031553731343024) -- (-6.8600636201355059, 71.3041938983133008) -- (-6.9330402255198322, 71.3325358457178140) -- (-7.0004497431697894, 71.3640154185780204) -- (-7.0264250331514022, 71.3734606252291428) -- (-7.0468332353985774, 71.3802948426597510) -- (-7.0622943994749328, 71.3829183124756668) -- (-7.0694066017416048, 71.3818655117927250) -- (-7.0749725762812572, 71.3787090847363572) -- (-7.0786839681043601, 71.3734606252291428) -- (-7.1031119171640018, 71.2680141339324962) -- (-7.1043486767066941, 71.2601517528869266) -- (-7.1031119171640018, 71.1993723225242690) -- (-7.1015656894368622, 71.1846990587737736) -- (-7.0950713103440499, 71.1663754780495026) -- (-7.0823942467325018, 71.1443909490748183) -- (-7.0706433612844188, 71.1307773901676796) -- (-7.0632216908332399, 71.1224125831827081) -- (-7.0542549058497661, 71.1140335991387786) -- (-7.0415767290433751, 71.1088028393503180) -- (-7.0016865027124799, 71.0988583886781385) -- (-6.9630341491192604, 71.0418659810919451) -- (-6.9463373386951899, 70.9462562699793722) -- (-6.9540673641358408, 70.8924981779765631) -- (-6.9585435208606059, 70.8438319710830626) -- (-6.9587060473171967, 70.8273050515272047) -- (-6.9475907961615224, 70.7555579097113707) -- (-6.9423164786876361, 70.7371636286682275) -- (-6.9175801746385197, 70.6700125825298784) -- (-6.8452213924279937, 70.5852468669326214) -- (-6.8399637728778053, 70.5816209015880531) -- (-6.8170820515453148, 70.5769399488158626) -- (-6.7920351661167837, 70.5717448288287272) -- (-6.7694629129686756, 70.5353717589802045) -- (-6.7654420529611237, 70.5068169875455908) -- (-6.7648242297873216, 70.4912315810455397) -- (-6.7657515211456278, 70.4647739186031146) -- (-6.7669882806882633, 70.4585457022841126) -- (-6.8458403287967595, 70.2041759219058719) -- (-6.8563533415073659, 70.1979682612166158) -- (-6.9293299468917038, 70.1441939441979798) -- (-6.9290204787073213, 70.1400635552504212) -- (-6.9250007318947979, 70.1224913523402194) -- (-6.8798539992087004, 70.0579341682725385) -- (-6.8022398238377910, 69.9557798655069121) -- (-6.7187502057427935, 69.6989146094331460) -- (-6.7165850416468853, 69.6845329838755561) -- (-6.6937033203143965, 69.3870713413527369) -- (-6.7564752679777689, 69.2413036541295241) -- (-6.7988379001991390, 69.1937984197920883) -- (-6.8099698492785228, 69.1795080902115131) -- (-6.8127539497432510, 69.1733802490334000) -- (-6.8322348606320613, 69.1233643005900973) -- (-6.8408921774310238, 69.0973453786200338) -- (-6.8442941010696856, 69.0820369801922709) -- (-6.8467676201550685, 69.0580760195545054) -- (-6.8436762778957609, 69.0071109485597987) -- (-6.8919143528411642, 68.9292082823256749) -- (-6.8980981505547438, 68.9174953815842599) -- (-6.9175801746385197, 68.8773075668921138) -- (-6.9219082764405835, 68.8640790745559457) -- (-6.9265458464270306, 68.8330625294095881) -- (-6.9271647827957974, 68.8086732842151321) -- (-6.9243817955259654, 68.7878379131582562) -- (-6.9203620487134412, 68.7756410485359027) -- (-6.9141782509998650, 68.7654848204855824) -- (-6.9120130869039578, 68.7593887795313492) -- (-6.9030463019206039, 68.7182627459041839) -- (-6.9018095423778592, 68.6883203902424100) -- (-6.9021178973674004, 68.6720859137368365) -- (-6.9055198210059876, 68.6583892751152405) -- (-6.9141782509998650, 68.6411439874466112) -- (-6.9355137446052817, 68.6026134681249999) -- (-6.9515927318555510, 68.5762624476289346) -- (-7.0820847785481185, 68.3758522998655991) -- (-7.1909307502559088, 68.3455251557043226) -- (-7.1986607756965704, 68.3485646282572787) -- (-7.2079370288644302, 68.3425003495384544) -- (-7.2345301420200911, 68.3091623103469914) -- (-7.2329839142928831, 68.3015884960221769) -- (-7.1374350557603270, 68.1814851122378940) -- (-7.1337247771321870, 68.1784646888210091) -- (-7.1043486767066941, 68.1658650896099232) -- (-7.0505435140267174, 68.1184828004280547) -- (-7.0462142990298124, 68.1144563449177554) -- (-7.0372475140464017, 68.0988367023256558) -- (-7.0347739949608981, 68.0933026430750772) -- (-7.0412672608589242, 68.0726512136970996) -- (-7.0462142990298124, 68.0605648659149125) -- (-7.0774472085616704, 67.9906129868094240) -- (-7.0811574871898095, 67.9850863991554633) -- (-7.0870318167189472, 67.9835798725649312) -- (-7.0938345508013025, 67.9825713122974520) -- (-7.1096051830619631, 67.9840778199269948) -- (-7.1225928280528690, 67.9886083777069814) -- (-7.1247568789539253, 67.9941352082780810) -- (-7.1986607756965704, 68.0132593175026869) -- (-7.3724427459689261, 68.0117395816880332) -- (-7.5579756016893072, 68.0565413215739170) -- (-7.5901346893846311, 68.0026887843979324) -- (-7.6018844616378161, 67.9750301463348165) -- (-7.6062136766347770, 67.9709964117711678) -- (-7.6482679538666600, 67.9589380596158605) -- (-7.6838278520056749, 67.9649606728969786) -- (-7.6955787374537667, 68.0313661352906820) -- (-7.6992890160818970, 68.0424387633437533) -- (-7.7694826341965246, 68.1648542310654619) -- (-7.7864889128051038, 68.1759453363018366) -- (-7.7880440460914189, 68.1749760964448654) -- (-7.7911275959864588, 68.1638433852892121) -- (-7.7902003046281507, 68.1572940673332681) -- (-7.7872236214443262, 68.1524345983798838) -- (-7.7799967601020494, 68.1421631817982387) -- (-7.7632977232881073, 68.0867445890259404) -- (-7.7595874446599682, 68.0655982057382829) -- (-7.7636071914724916, 67.9679857269539553) -- (-7.7905097728125332, 67.9187192574316327) -- (-7.7982409114480937, 67.9101746370920978) -- (-7.8328724050338359, 67.9041434757697147) -- (-7.9046122508756618, 67.8971052825276331) -- (-7.9101782254153150, 67.9001277771103418) -- (-7.9114149849579496, 67.9066561030265632) -- (-7.9707850089828041, 67.9358094209628831) -- (-8.2030108251164773, 67.9066561030265632) -- (-8.2766041404798898, 67.8920825138834516) -- (-8.4367806425874239, 67.8453733534494035) -- (-8.4828546666318303, 67.7811337072268429) -- (-8.4624464643846551, 67.7781156925872068) -- (-8.4241024657809120, 67.7711029944779568) -- (-8.4024575039911120, 67.7660881081700666) -- (-8.4002934530901019, 67.7600677690880531) -- (-8.3996745167212801, 67.7525330498954190) -- (-8.4435833766697783, 67.6548028787354525) -- (-8.4472947684928155, 67.6492867510322213) -- (-8.5858251956155840, 67.5541854527607200) -- (-8.5997401319647881, 67.5521791071695077) -- (-8.6962173950505726, 67.5591871265194612) -- (-8.7027117741434399, 67.5611792373666020) -- (-8.7092039268465715, 67.5641891115035094) -- (-8.7181718250247560, 67.5711837665617594) -- (-8.7400772744231237, 67.6224923066222345) -- (-8.7401262549990726, 67.5942075599455592) -- (-8.7404357231834542, 67.5711837665617594) -- (-8.7429092422689028, 67.5566853501865978) -- (-8.7821805322309636, 67.4666750045869890) -- (-8.9259674657089398, 67.3667848840230192) -- (-9.0020376397425661, 67.3238729280694059) -- (-9.0907826509978840, 67.2904617027180336) -- (-9.0972781432857275, 67.2894618060339127) -- (-9.1653054841094175, 67.2969406877624294) -- (-9.1716128464577213, 67.3118035706743711) -- (-9.1743969469225153, 67.3135974285591061) -- (-9.1769940306426872, 67.3165836890180742) -- (-9.1775495149017363, 67.3249702320809433) -- (-9.1578838136582981, 67.3607943753450371) -- (-9.1529367754874116, 67.3647975812828150) -- (-9.1359293836840489, 67.3702798527832840) -- (-9.1071722196273210, 67.3917525331053469) -- (-9.1009873087189597, 67.4022303619255467) -- (-9.1399491304965039, 67.4561761891286693) -- (-9.2246755081342631, 67.4651781740144969) -- (-9.3128049226054603, 67.4701761137030331) -- (-9.3316657839305286, 67.4711764585830593) -- (-9.3351712346955544, 67.4698864478143179) -- (-9.3678457316332846, 67.4576854444450618) -- (-9.3743389975312450, 67.4551868237729622) -- (-9.3836152506990480, 67.4491897312816207) -- (-9.3823517744785665, 67.4473370906322032) -- (-9.3805239084396739, 67.4446895073639325) -- (-9.3724833016197309, 67.4426858915301040) -- (-9.3502194034610895, 67.4401877366510973) -- (-9.3196076566877455, 67.4416984905905110) -- (-9.3038370244270858, 67.4401877366510973) -- (-9.2908482662414027, 67.4371968872250847) -- (-9.2852834048966475, 67.4332044653399407) -- (-9.2781712026298546, 67.4247113366761965) -- (-9.2443423225725994, 67.3410292179531353) -- (-9.2361792643128382, 67.3111050722565381) -- (-9.2363651678624414, 67.3066107658931116) -- (-9.2361180385928670, 67.2341224291009638) -- (-9.2487951022044186, 67.2057281557816424) -- (-9.2540527217545403, 67.1942771411333553) -- (-9.2837382903644148, 67.1658843518337392) -- (-9.2939418348905161, 67.1599213082286468) -- (-9.3706276057082079, 67.1365160994671726) -- (-9.4890604119630826, 67.0628969788987632) -- (-9.5082307414726586, 67.0325694765424203) -- (-9.5261665378292921, 67.0067289504539616) -- (-9.6783035463117262, 66.8989688918590701) -- (-9.6838684076564139, 66.8969805791367662) -- (-9.7262321530727487, 66.9034405412140529) -- (-9.8882632378968953, 66.8875488188724461) -- (-10.0008217146226315, 66.8741533246747366) -- (-10.1436802435475233, 66.8533232781045115) -- (-10.2008862166712451, 66.7992884034792809) -- (-10.2704608984170420, 66.7393325919223344) -- (-10.2794276834004492, 66.7328999933543940) -- (-10.4114659578201554, 66.7150703918049999) -- (-10.4269260087016047, 66.7135874426084143) -- (-10.4399136536924342, 66.7155730321964597) -- (-10.5249494995144506, 66.8225835461587394) -- (-10.7293443297549462, 66.7844125760904177) -- (-10.8020103537601049, 66.7140775970876803) -- (-10.8020103537601049, 66.7091262821796391) -- (-10.8063395687569415, 66.7051697256787293) -- (-10.8168548078573075, 66.6997181929601766) -- (-10.8737513127966476, 66.6819083736304350) -- (-10.9071460468397543, 66.6734957331871954) -- (-10.9176768706688545, 66.6717142612322675) -- (-10.9287921218245287, 66.6705189155366611) -- (-10.9278648304662198, 66.7195300975418490) -- (-10.9102396154888783, 66.7502306669872070) -- (-10.8811718700529259, 66.7710341623827901) -- (-10.8165442264780189, 66.8131622490965924) -- (-10.8007747074121987, 66.8201063114505871) -- (-10.7754194669942471, 66.8349762138119132) -- (-10.7652148092731839, 66.8414187241397002) -- (-10.7463517215582964, 66.8548087924241514) -- (-10.7420225065613355, 66.8587851772685298) -- (-10.6798683820718754, 66.9277611433625879) -- (-10.6777054443657864, 66.9639956433262569) -- (-10.7769645815264798, 66.9103780724370552) -- (-10.8929216423062343, 66.8543177840688969) -- (-10.9151866536597097, 66.8438966121531166) -- (-10.9420881218048471, 66.8389391269550686) -- (-10.9495109054509303, 66.8399334559265554) -- (-10.9550757667956820, 66.8424130836182826) -- (-10.9590955136082080, 66.8463906438787063) -- (-10.9590955136082080, 66.8528322787750255) -- (-10.8861189082237573, 66.9173304945175289) -- (-10.8483949591838016, 66.9451336564661972) -- (-10.7011437631522313, 67.0216350419952960) -- (-10.6944645937046658, 67.0243242279979370) -- (-10.6316303071264695, 67.0504692889973910) -- (-10.6034909662437258, 67.0549350180132251) -- (-10.5920506621748114, 67.0594135243373017) -- (-10.5425747144917406, 67.0827838328029742) -- (-10.5323711699656393, 67.0907485426251924) -- (-10.5215486890707624, 67.1036731725411073) -- (-10.5196929931592500, 67.1086469399829184) -- (-10.5107262081757611, 67.1479714560891239) -- (-10.5215486890707624, 67.1693706155867005) -- (-10.6149334967023297, 67.2147039765414434) -- (-10.6294673694202437, 67.1609070529688665) -- (-10.6325587116796854, 67.1559282691890047) -- (-10.7166661529484202, 67.1035782494617479) -- (-10.7201916412218683, 67.1014881790198245) -- (-10.7543299894634945, 67.0895523534680223) -- (-10.7613798528154110, 67.0871618190010537) -- (-10.8842632123122343, 67.0494723644009696) -- (-10.9164245263974298, 67.0425085542273678) -- (-11.0543360171513001, 66.9982824330288196) -- (-11.0543360171513001, 66.9918130914159633) -- (-11.0568095362368037, 66.9858503808250418) -- (-11.0602103466805630, 66.9813884724737960) -- (-11.0725790553025707, 66.9709526829958435) -- (-11.1607084697738017, 66.9376873254482518) -- (-11.1730760652008208, 66.9337052679124724) -- (-11.1804977356520094, 66.9332137798090940) -- (-11.2794496310182613, 66.9342092697280862) -- (-11.2766655305535206, 66.9679791948493630) -- (-11.1990502419877291, 67.0241184654240243) -- (-11.1186530793469753, 67.0852709295565575) -- (-11.1106146989167804, 67.1400010936070544) -- (-11.1226728261594978, 67.1439790087782313) -- (-11.1276220907201342, 67.1474661174505059) -- (-11.1297850284263582, 67.1534407846923784) -- (-11.1282399138940598, 67.1599213082286468) -- (-11.1059749025405843, 67.1773438820708577) -- (-11.1007183961853180, 67.1813236401405334) -- (-11.0268144994425477, 67.2286476235564692) -- (-10.8833381473438653, 67.3059123125111540) -- (-10.6783232675397102, 67.4207069477765799) -- (-10.6634810398321171, 67.4301994892647372) -- (-10.6616253439206066, 67.4341917620444917) -- (-10.6684280780030836, 67.4366897307584594) -- (-10.7336735447519072, 67.4291978555110632) -- (-10.7850040751515266, 67.4142179952403922) -- (-10.8091236692217496, 67.4057268701467223) -- (-11.0642300934930837, 67.3034183985780032) -- (-11.1891561656460183, 67.2525748326778086) -- (-11.2454348474113566, 67.2186855797343554) -- (-11.3354177314043945, 67.2789860764391392) -- (-11.3239763141406637, 67.3263674755242505) -- (-11.3264498332260892, 67.3413309944588292) -- (-11.3283055291376122, 67.3478266486731201) -- (-11.3310885164074460, 67.3528155088268079) -- (-11.3548986422932163, 67.3867612593704308) -- (-11.3604635036379733, 67.3892559588665989) -- (-11.3678851740891620, 67.3902445139191570) -- (-11.3886028445205856, 67.3862544206520937) -- (-11.4049924131501914, 67.3782730303456674) -- (-11.4090121599627352, 67.3742691542422136) -- (-11.5088690827890261, 67.2737684651176551) -- (-11.5258965121006529, 67.3812616906572259) -- (-11.5459974725533066, 67.4461841581323682) -- (-11.4254017285922291, 67.6427793253595269) -- (-11.4096299831366377, 67.6517914119565802) -- (-11.2698627964711449, 67.7104198984202981) -- (-11.2451242660321551, 67.7199467175693144) -- (-11.2333744937789248, 67.7244641856327263) -- (-11.1953410765544952, 67.7339800358504363) -- (-11.1171079648147426, 67.7545426003032816) -- (-10.9105468572835136, 67.8860527131430018) -- (-10.8623087823380420, 67.9327855809641790) -- (-10.8681842250620893, 67.9363070716297699) -- (-10.9219893877421885, 67.9398269072726322) -- (-10.9584776904344050, 67.9388188974170077) -- (-10.9900178417608423, 67.9378109002362436) -- (-11.0057873608265382, 67.9363070716297699) -- (-11.0679414853161209, 67.9257593483203266) -- (-11.1597811784154928, 67.9101746370920978) -- (-11.1919391529158521, 67.9016327361090930) -- (-11.2355385446799474, 67.8875574080065149) -- (-11.3375817823054152, 67.8579244569063604) -- (-11.4380799053985633, 67.8599384564394938) -- (-11.4993056253349764, 67.8689731502004321) -- (-11.5391947384708171, 67.8659662940206658) -- (-11.5568199534481177, 67.8518972236267217) -- (-11.6180456733845272, 67.8308108945731050) -- (-11.6263946351940231, 67.8313223523081490) -- (-11.6322700779180934, 67.8338308918786055) -- (-11.6502047610797987, 67.8483938990350168) -- (-11.6551517992506497, 67.8584342796559383) -- (-11.6591715460631953, 67.8976171631837389) -- (-11.6604094188008176, 67.9272647885158989) -- (-11.6517509888069171, 67.9835798725649312) -- (-11.6492774697214898, 67.9891063563166398) -- (-11.6449482547244259, 67.9936371984157546) -- (-11.5323920043885515, 68.0797048428676561) -- (-11.4269468431243268, 68.1482122896708375) -- (-11.4108678558742600, 68.1552725795093153) -- (-11.3592278572901666, 68.1729124169803526) -- (-11.3205743905020206, 68.1845165412332079) -- (-11.3131538332457406, 68.1850157396553698) -- (-11.3082067950748861, 68.1814851122378940) -- (-11.3066605673477678, 68.1749343504227738) -- (-11.3066605673477678, 68.1598146474611042) -- (-11.3069689223371554, 68.1532511427411123) -- (-11.0784556112215125, 68.0822229812552564) -- (-11.0716528771391705, 68.0807001235226892) -- (-11.0432040680719847, 68.0802033880978001) -- (-11.0026960185672316, 68.0802033880978001) -- (-10.8502506550953228, 68.1043838382082356) -- (-10.8434479210129702, 68.1069009393384164) -- (-10.8449941487401098, 68.1124359385894280) -- (-10.8530325291702479, 68.1220124635820952) -- (-10.8749869591444313, 68.1421631817982387) -- (-10.8892124768729932, 68.1441843385524066) -- (-10.8950879195970387, 68.1426748162956386) -- (-10.9492014372665452, 68.1829935514730465) -- (-10.9535306522634510, 68.3520899063704945) -- (-10.8678758700726110, 68.4668773589997954) -- (-10.7760372901681709, 68.5944972364283387) -- (-10.7707796706179941, 68.6538205997135549) -- (-10.7689239747064143, 68.6588914436524931) -- (-10.7485157724592959, 68.6801954325646022) -- (-10.7441887838522749, 68.6842651145258287) -- (-10.7188302038495546, 68.6974491687418123) -- (-10.7052247356847232, 68.7005020713433652) -- (-10.4290911727975004, 68.7167441631638809) -- (-10.4250725391798866, 68.7142059468538520) -- (-10.4209782083084743, 68.7092430446528937) -- (-10.4168237649121149, 68.6974637843941736) -- (-10.3879641869239272, 68.7010045051769396) -- (-10.2664422647994265, 68.7263915844026343) -- (-10.0008217146226315, 68.8096923266772222) -- (-9.8165256184449436, 68.8737443546395269) -- (-9.8505392888568828, 68.9185158176141925) -- (-9.9596936155542828, 68.9801234568346615) -- (-10.0499870809264795, 68.9928541334453627) -- (-10.0558625236505268, 68.9719686796520222) -- (-10.0917307767790820, 68.9159711639776873) -- (-10.1198701176616925, 68.8768040200285867) -- (-10.1625433312624107, 68.8183336720807972) -- (-10.1823337103356621, 68.7954561855499662) -- (-10.1919183184929292, 68.7888566891440405) -- (-10.2045964952993859, 68.7842787011083630) -- (-10.2188220130278804, 68.7827570310122809) -- (-10.2877766452100126, 68.7781812222324191) -- (-10.5537066635710559, 68.7847944660767467) -- (-10.6356500539389813, 68.8234149939715820) -- (-10.7914984542443957, 68.7578694130092458) -- (-10.7998474160538915, 68.7497373635953437) -- (-10.8032482264977023, 68.7446607381760941) -- (-10.8054122773986325, 68.7380636192105499) -- (-10.8054122773986325, 68.7314670462597093) -- (-10.7995368346746670, 68.7202985327743079) -- (-10.8013925305861793, 68.7136906385094193) -- (-10.8066501501363685, 68.7106500216282825) -- (-10.9040547045804495, 68.6822302475800939) -- (-11.0051695376527476, 68.6629600384357701) -- (-11.0209390567185004, 68.6624432350401577) -- (-11.0500068021544600, 68.6670142307361715) -- (-11.0568095362368037, 68.6690340989614754) -- (-11.0611387512337789, 68.6731032112846123) -- (-11.0586652321483285, 68.6786775747481215) -- (-11.0379464485218612, 68.7005020713433652) -- (-11.0329994103509748, 68.7035423011836457) -- (-10.9628069054312540, 68.7334885481425175) -- (-10.9334296918109786, 68.7370454923832170) -- (-10.9222988559265008, 68.7426371249208188) -- (-10.8369535419200993, 68.8249356109987360) -- (-10.8286045801105484, 68.8330625294095881) -- (-10.6477104075715250, 69.0208676943848332) -- (-10.5595821062952240, 69.1213181790328832) -- (-10.4702170454761774, 69.2791214540396680) -- (-10.4581589182334618, 69.3532914072850417) -- (-10.5317533467917706, 69.3537979816557026) -- (-10.5394833722325014, 69.3548240349250875) -- (-10.5435031190449564, 69.3594202532293593) -- (-10.5459766381304600, 69.3650558196637803) -- (-10.5475217526626928, 69.3722247501283107) -- (-10.5456660567511680, 69.3788746918183108) -- (-10.3313771501690379, 69.7467235229782432) -- (-10.3156076311032034, 69.7647197243942259) -- (-10.3088048970207957, 69.7662732921091902) -- (-10.3010748715801324, 69.7652418975057600) -- (-10.2766458093255384, 69.7528958330104274) -- (-10.2383029239166916, 69.7241007722709298) -- (-10.2120192789455384, 69.6989146094331460) -- (-10.1903720907658553, 69.6860701798429290) -- (-10.0960599917759772, 69.7050831922962004) -- (-10.0203972470786962, 69.7576003673948293) -- (-10.0567898150088340, 69.7791307449112139) -- (-10.0669944727298404, 69.7853055833806053) -- (-10.0700858149891506, 69.7904491805561662) -- (-10.0673028277193168, 69.7966249024752727) -- (-10.0604989804420644, 69.7981661296013982) -- (-9.9977281459735874, 69.7904491805561662) -- (-9.9974197909841003, 69.7827329841360324) -- (-9.9949462718986624, 69.7765583464260004) -- (-9.9566022732949104, 69.7467235229782432) -- (-9.9497995392125560, 69.7482637784259225) -- (-9.9460881473895189, 69.7534049682574278) -- (-9.9532014628510961, 69.9774348308201866) -- (-9.9745369564565802, 69.9831164130848293) -- (-10.0552424740868496, 69.9913610040819947) -- (-10.0635925490911777, 69.9913610040819947) -- (-10.0703963963684284, 69.9893072034307693) -- (-10.0802904727101552, 69.9825909424166355) -- (-10.1306937117515918, 69.9660895992555822) -- (-10.2055237866576931, 69.9480475197658365) -- (-10.2438677852614362, 69.9444377422045704) -- (-10.2744806452296302, 69.9428936614814916) -- (-10.3527137569692940, 69.9413496109242487) -- (-10.5073242845372299, 69.9212481573016049) -- (-10.5153637781622749, 69.9197045295248216) -- (-10.5503058531274316, 69.9042346655119928) -- (-10.5725697512859966, 69.9011499593186016) -- (-10.6956401275274438, 69.9181479450199674) -- (-10.7067720766068266, 69.9253671193349504) -- (-10.7132653425047994, 69.9356728862852606) -- (-10.7148104570368972, 69.9423833048948893) -- (-10.7135748106891793, 69.9578476444953026) -- (-10.6981136466128248, 70.0424520851757961) -- (-10.6736856975532604, 70.0801432760909506) -- (-10.6668829634707834, 70.0785835273447475) -- (-10.6616253439206066, 70.0806526643577001) -- (-10.6572961289236989, 70.0858173776821047) -- (-10.6396709139464125, 70.1100893615713687) -- (-10.6378152180348877, 70.1168164743941844) -- (-10.6365773452973009, 70.1912509830869311) -- (-10.6458547116600002, 70.1990053199158979) -- (-10.7401668106498107, 70.2041759219058719) -- (-10.7478979492855053, 70.2036508399143457) -- (-10.7574836706376828, 70.1958941847115767) -- (-10.7976811387631422, 70.1581495598162803) -- (-10.8020103537601049, 70.1473030417099466) -- (-10.8051039224092271, 70.1235274196565683) -- (-10.8103615419594039, 70.1116515926808574) -- (-10.8428278714491722, 70.0662014424396062) -- (-10.8471570864462006, 70.0620602625712081) -- (-10.8867389577875535, 70.0341872965920658) -- (-10.8944678700333650, 70.0331672646114640) -- (-11.0227947526300234, 70.0713633517388104) -- (-11.0258883212791456, 70.0760013650241405) -- (-11.0255788530947640, 70.0847818051322236) -- (-11.0064697493051575, 70.1384095997432837) -- (-11.0018299529288921, 70.1439948619349565) -- (-10.9578598672604226, 70.1715958037086978) -- (-10.9303383495516719, 70.1767646097646889) -- (-10.9192052872774479, 70.1834806546785046) -- (-10.9096217923149528, 70.1912509830869311) -- (-10.8975614386824873, 70.2052130622075055) -- (-10.8923038191323123, 70.2171048408301175) -- (-10.8938500468594395, 70.2238261832096384) -- (-10.9062198686763558, 70.2548795522331488) -- (-10.9161139450180151, 70.2621295824179697) -- (-10.9244629068275003, 70.2621295824179697) -- (-11.0002224994817812, 70.2471178582307090) -- (-11.0443784886999197, 70.2177231282648222) -- (-11.0454927968027388, 70.2087230546835031) -- (-11.0466059917107060, 70.2040697880302531) -- (-11.0486464779769111, 70.2009659008070912) -- (-11.0864939916517162, 70.1778013898088773) -- (-11.0945357116665697, 70.1778013898088773) -- (-11.1464829520453286, 70.1902140263060517) -- (-11.3276877059635979, 70.2445320385968728) -- (-11.3292328204959212, 70.2517662126042381) -- (-11.3564448700203382, 70.4928013827421438) -- (-11.3530429463816951, 70.4985033900525337) -- (-11.3391268968376266, 70.5104544625082923) -- (-11.1721487738425118, 70.6044908539190459) -- (-11.1325691288910757, 70.6138450864991114) -- (-11.1257663948086201, 70.6154004899073442) -- (-10.9940375885733630, 70.6237264451928581) -- (-10.9086933877618044, 70.5966815391523426) -- (-10.8969413891187479, 70.5930550468777938) -- (-10.8882851855146843, 70.5920127887389555) -- (-10.8032482264977023, 70.5940842201660672) -- (-10.7970633155891509, 70.5956542031661343) -- (-10.7927341005922433, 70.6008509019274300) -- (-10.7951864689744319, 70.6082189381722571) -- (-10.8691115164203946, 70.6070903880003016) -- (-10.8861189082237573, 70.6081178733178945) -- (-10.9086933877618044, 70.6122878421707014) -- (-10.9312656409099791, 70.6237264451928581) -- (-10.9791964740608048, 70.6544098059912926) -- (-11.0258883212791456, 70.6845768432334296) -- (-11.0351645744469415, 70.6923948894855130) -- (-11.0367108021740119, 70.6981132530037826) -- (-11.0478427512533290, 70.7746579444904000) -- (-11.0277417908008566, 70.9008385104452117) -- (-11.0172298912852682, 70.9081483004535755) -- (-10.9671361204282842, 70.9211933918345210) -- (-10.9609512095197310, 70.9227548896486439) -- (-10.7803665051650768, 70.9488652536670799) -- (-10.7250151147579729, 70.9473032364465865) -- (-10.7200680765871432, 70.9425985757332995) -- (-10.7138831656785900, 70.9399898003009639) -- (-10.6978041784284450, 70.9389429313643944) -- (-10.6730656479895210, 70.9493821735510721) -- (-10.6511112180152043, 70.9697496464424376) -- (-10.6526574457423440, 70.9775810811981103) -- (-10.6665723820915606, 70.9796758639141245) -- (-10.6733762293688113, 70.9822878391743757) -- (-10.7058447852485177, 71.0010988848353861) -- (-10.7092455956921544, 71.0068397659582899) -- (-10.7024428616097964, 71.0449956954097814) -- (-10.6981136466128248, 71.0497046168997741) -- (-10.6885268120657422, 71.0564974800227418) -- (-10.6746141021064762, 71.0596277781349812) -- (-10.6622431670945321, 71.0648703190354496) -- (-10.6436906607589350, 71.0910129563238087) -- (-10.6492555221036920, 71.1496240836511618) -- (-10.6508039762206383, 71.1569570899930284) -- (-10.7952076196776279, 71.1606097145308496) -- (-10.8264405292095507, 71.1590415906814968) -- (-10.8697315659841269, 71.1564256077272148) -- (-10.8768426550559560, 71.1548575675709571) -- (-10.8817896932269100, 71.1517083893622555) -- (-10.8861189082237573, 71.1475247576031791) -- (-10.8926144005116559, 71.1360096161208730) -- (-10.9015800723001028, 71.1276442509538214) -- (-10.9291027032038848, 71.1124664065542902) -- (-10.9553863481749758, 71.1046196127829262) -- (-11.0169193099058553, 71.0889151988513248) -- (-11.0317637640031823, 71.0868325615696364) -- (-11.0614482194181605, 71.1004253093305181) -- (-11.0657763212202251, 71.1046196127829262) -- (-11.0682520666954129, 71.1103680751420768) -- (-11.0348551062624871, 71.2601517528869266) -- (-11.0299080680916699, 71.2638224014734334) -- (-11.0166098417215377, 71.2680141339324962) -- (-10.9814217506817844, 71.2640318884060946) -- (-10.9773396649543820, 71.2640318884060946) -- (-10.9287921218245287, 71.2449482293388030) -- (-10.9226072109159755, 71.2418105229508285) -- (-10.9201336918306637, 71.2360392071618946) -- (-10.9179696409295399, 71.2213722806247347) -- (-10.9216788063627721, 71.2161343536553915) -- (-10.9232250340899011, 71.2108967767758969) -- (-10.9158044768336318, 71.1925538721734341) -- (-10.9124025531950366, 71.1873178709169281) -- (-10.9065282236658323, 71.1846990587737736) -- (-10.9003433127574105, 71.1857492126738265) -- (-10.8926144005116559, 71.1946544045494818) -- (-10.8923038191323123, 71.2035587225726943) -- (-10.8957057427709643, 71.2459991909606600) -- (-10.8997243763885763, 71.2591006013972645) -- (-10.9415315243508342, 71.3174183603467071) -- (-10.9686200092405457, 71.3526781400088765) -- (-11.0605209280598533, 71.4706453684192837) -- (-11.0636122703191599, 71.5653213079232700) -- (-11.0500068021544600, 71.5900566997911056) -- (-11.0749913486680480, 71.6965761169534801) -- (-11.0770318349343206, 71.6997321163030250) -- (-11.1387518134096215, 71.7703057022004600) -- (-11.1464829520453286, 71.7713637468999792) -- (-11.1511216352266818, 71.7671316541233750) -- (-11.2250255319694290, 71.6369169894593085) -- (-11.2117284187940598, 71.5932083999914823) -- (-11.2036889251690805, 71.5363760201444308) -- (-11.2036889251690805, 71.5290073946204075) -- (-11.2064730256338212, 71.5232281207367180) -- (-11.2194584442348564, 71.5190245921698136) -- (-11.2281179874236656, 71.5190245921698136) -- (-11.2497640624084188, 71.5237469502868493) -- (-11.2618221896510331, 71.5290073946204075) -- (-11.2704806196449354, 71.5379498258328965) -- (-11.2738825432836887, 71.5432093316771756) -- (-11.2754276578158983, 71.5674167102995966) -- (-11.2568751514802923, 71.7914104195251639) -- (-11.2228603678733876, 71.8431486133042654) -- (-11.2036889251690805, 71.8663916901498823) -- (-11.1418453820588788, 71.9261084838680773) -- (-11.1353510029660008, 71.9271686429853219) -- (-11.1294755602418416, 71.9261084838680773) -- (-11.0048600694682310, 71.8500107143534308) -- (-11.0030043735567062, 71.8452666926151551) -- (-11.0141363226360216, 71.8209730746332156) -- (-11.0268144994425477, 71.7982660732033366) -- (-11.0472227016896554, 71.7750451703475818) -- (-11.0540254357720773, 71.7729441922559346) -- (-11.0679414853161209, 71.7692476718384569) -- (-11.0769082702994766, 71.7697700369222531) -- (-11.0861845234672671, 71.7766389880028299) -- (-11.0905137384642956, 71.7819145855472556) -- (-11.0954630030248786, 71.7940477228030289) -- (-11.0997899916320986, 71.7977454176251655) -- (-11.1059749025405843, 71.8014413873127211) -- (-11.1127787498178385, 71.7988038342523254) -- (-11.1140166225554360, 71.7919310328554019) -- (-11.1130882180022184, 71.7829727875736410) -- (-11.1090673579947996, 71.7692476718384569) -- (-11.1031919152707541, 71.7597564320121677) -- (-11.0902064966696088, 71.7539467745791342) -- (-11.0432040680719847, 71.7476153289479157) -- (-11.0184655376330625, 71.7476153289479157) -- (-11.0125900949090134, 71.7497308178629964) -- (-11.0082608799119850, 71.7544671381707104) -- (-10.9884727272286078, 71.7829727875736410) -- (-10.9671361204282842, 71.8663916901498823) -- (-10.9615690326936566, 71.9462082799770002) -- (-10.9631141472259568, 71.9541399780156041) -- (-10.9628069054312540, 71.9620705786137904) -- (-10.9609512095197310, 71.9678976562878034) -- (-10.9575492858812034, 71.9731862013948671) -- (-10.9522927795258660, 71.9774142484139503) -- (-10.9127131345743820, 71.9890539891242156) -- (-10.9055998191126129, 71.9906388501231760) -- (-10.8570522759828396, 71.9784751051964378) -- (-10.8431384528285868, 71.9731862013948671) -- (-10.6724478248155403, 71.9488527243225775) -- (-10.5725697512859966, 71.9493743891589759) -- (-10.3932218064741040, 71.9567827872499919) -- (-10.3146803397448981, 71.9197630570891135) -- (-10.2546902661564356, 71.8764322999124374) -- (-10.2522167470709977, 71.8706139524460781) -- (-10.2392302152750005, 71.8046016276755807) -- (-10.1863534571482397, 71.6922475002019439) -- (-10.1654432039976523, 71.6579268909071061) -- (-10.1690354839655299, 71.6811869907893993) -- (-10.1650168503479161, 71.7360115432704220) -- (-10.0861648022394181, 71.8859377387429674) -- (-10.0815283454478770, 71.8901762784843044) -- (-10.0716320427163435, 71.8970425166270672) -- (-10.0648281954390271, 71.8996852946847866) -- (-10.0481313850149565, 71.8991468380418723) -- (-9.9485616664748342, 71.8933401142015214) -- (-9.9411411092185631, 71.8917591315490796) -- (-9.9278451092382376, 71.8869973520919245) -- (-9.9228958446776030, 71.8832973285897765) -- (-9.7389092166842399, 71.8547702371746624) -- (-9.5957412195750358, 71.8431486133042654) -- (-9.4377287683684887, 71.8732539334233138) -- (-9.4306176792967253, 71.8748516598203366) -- (-9.4745265392452378, 71.9673606328756108) -- (-9.4838027924130284, 71.9742470006579680) -- (-9.4949358546872524, 71.9800597511850668) -- (-9.5555415250597573, 71.9996343085572335) -- (-9.5641999550536241, 72.0001734594255964) -- (-9.5784243595872898, 71.9969900466212778) -- (-9.6031617768313033, 71.9874710651616141) -- (-9.6260446113588340, 71.9837643677524710) -- (-9.6473801049641850, 71.9964661581068128) -- (-9.6498547372445405, 72.0038656952466596) -- (-9.6504736736134280, 72.0107552880336925) -- (-9.6495463822551208, 72.0186935549623684) -- (-9.6421235986089808, 72.0298001011026372) -- (-9.6303738263557292, 72.0430488423164519) -- (-9.3891823384335424, 72.2057663453803826) -- (-9.3458913016589680, 72.2333578662223914) -- (-9.3296698254605825, 72.2431993354333457) -- (-9.2933217853267855, 72.2487550597967356) -- (-9.2221953098793215, 72.2551273350755991) -- (-9.2039589508976114, 72.3433147694501741) -- (-9.2017926736067395, 72.3523510578962288) -- (-9.1294372309809191, 72.4459695915053743) -- (-9.0948023978104064, 72.4811068204994058) -- (-9.1152117132524886, 72.5620841033277486) -- (-9.3403242139244096, 72.5445028613112441) -- (-9.5286400569147425, 72.5231796959839272) -- (-9.5561615746234985, 72.5162580228120390) -- (-9.6031617768313033, 72.5317113290462601) -- (-9.6625318008561578, 72.5514270510926309) -- (-9.6878881544690607, 72.5786155964930231) -- (-9.7389092166842399, 72.5898023544292954) -- (-9.7831286580121546, 72.6079372312868401) -- (-9.7865294684557771, 72.6122133479472325) -- (-9.7970447075562088, 72.6735776446362109) -- (-9.7958068348185421, 72.6805114304338105) -- (-9.7920976693853099, 72.6858486380379389) -- (-9.7339621785134085, 72.7381869351453076) -- (-9.7240681021716817, 72.7461997976560610) -- (-9.6504736736134280, 72.8044669940286440) -- (-9.5997598531927046, 72.8231796597146115) -- (-9.5935771686740239, 72.8253245889970913) -- (-9.4995723114787900, 72.8520641665601261) -- (-9.4915339310485862, 72.8526073345407639) -- (-9.4519542860970489, 72.8697306330998202) -- (-9.3158962648647083, 72.9436108583795715) -- (-9.2995066962352695, 72.9580732543168438) -- (-9.2571440640138469, 73.0164999536781352) -- (-9.2552883681023221, 73.0223911832315338) -- (-9.2574546453931923, 73.0293736039256345) -- (-9.2790984939881636, 73.0771091390674457) -- (-9.2831204671904910, 73.0819424596881362) -- (-9.2902304430674789, 73.0846258362599173) -- (-9.2979615817030385, 73.0851551814243550) -- (-9.3196076566877455, 73.0808664305519642) -- (-9.3322858334942715, 73.0760331770888740) -- (-9.3307384925722214, 73.0599418921013921) -- (-9.3270293271389875, 73.0454585386175808) -- (-9.3301206693983634, 73.0384765986656816) -- (-9.3353771757535782, 73.0363392777139353) -- (-9.3474375293861662, 73.0417010336518899) -- (-9.3925820356823895, 73.0690620010951193) -- (-9.4139186424827805, 73.1017985569249760) -- (-9.4157743383943053, 73.1157455003914265) -- (-9.4173205661214467, 73.1490489888420399) -- (-9.3508372266348925, 73.1397841235962716) -- (-9.3342684336252866, 73.1561066524122481) -- (-9.3540521335290450, 73.1771616492613077) -- (-9.3812552774942315, 73.2038023879537718) -- (-9.4151565152203798, 73.2586855622335094) -- (-9.4139186424827805, 73.2656738395632345) -- (-9.2589997599253699, 73.4795251518532382) -- (-9.1077900428013017, 73.5437282992773333) -- (-8.9302977939008077, 73.6209729475302197) -- (-8.8849128375044479, 73.6886875508181163) -- (-8.8453443245018999, 73.6497715903827697) -- (-8.7939047010013240, 73.6134622787773907) -- (-8.7637326662167041, 73.6549600297269080) -- (-8.7488948912888684, 73.6990719614014722) -- (-8.6756923073381085, 73.7406157950421175) -- (-8.6786600849626918, 73.6956116340564193) -- (-8.6989391566004333, 73.6627413964993849) -- (-8.6949817487027907, 73.6108733002490823) -- (-8.6504673107243750, 73.6246964560075696) -- (-8.5851784293741211, 73.5763106831975620) -- (-8.5747912076881647, 73.5512613405013127) -- (-8.5678671353607818, 73.5184525032026528) -- (-8.5739507455326542, 73.4831097967542632) -- (-8.5771678788166135, 73.4584905609109597) -- (-8.5759300060789592, 73.4509374455358710) -- (-8.5678905124539160, 73.4487751115566709) -- (-8.5641802338257715, 73.4530998400342838) -- (-8.5357314247585876, 73.5437282992773333) -- (-8.5354219565742060, 73.5523730899320327) -- (-8.5456266142952160, 73.5939488497813556) -- (-8.5700545633549119, 73.6842250008486133) -- (-8.5777857019904715, 73.6934190702732366) -- (-8.5839706128989697, 73.6966592575797677) -- (-8.5901544106124916, 73.6950391469154908) -- (-8.5941741574251367, 73.6890828065616574) -- (-8.5957203851522657, 73.6723280759866412) -- (-8.5951014487834438, 73.6463699072620415) -- (-8.5841887991008790, 73.6056858788204238) -- (-8.5891347240769242, 73.6005007508284450) -- (-8.6138654621515141, 73.6160591250924909) -- (-8.6311767561647983, 73.6368020179966152) -- (-8.6697567520890555, 73.6592807518077706) -- (-8.6737130467917378, 73.6800430547999099) -- (-8.6440363837411578, 73.7163837178206052) -- (-8.6123815733391851, 73.7458122964178528) -- (-8.5700545633549119, 73.7772969032471764) -- (-8.5382049438440379, 73.7967963252113037) -- (-8.5242900074948924, 73.8016612267734615) -- (-8.5007904629884070, 73.7892100312030834) -- (-8.4921320329945296, 73.7794684604792792) -- (-8.3808114290063926, 73.5167487054233391) -- (-8.3771011503782518, 73.5021795638237023) -- (-8.3780295549314694, 73.4854498090821835) -- (-8.3829765931022315, 73.4541810599792910) -- (-8.3919433780857098, 73.4363807093562571) -- (-8.3984377571785878, 73.4336870434179474) -- (-8.4197743639788456, 73.4293789256548735) -- (-8.4324525407853681, 73.4148262498761284) -- (-8.4584256043773003, 73.3867958183405023) -- (-8.4788349198193185, 73.3727979876209417) -- (-8.5007904629884070, 73.3426389311529903) -- (-8.5527388165620746, 73.1818174585988146) -- (-8.5530471715515510, 73.1748232117288353) -- (-8.5508842338454603, 73.1683749832978663) -- (-8.5307843865877580, 73.1791326452825786) -- (-8.5094477797873687, 73.1930935309989223) -- (-8.4695586666515137, 73.2339460544896212) -- (-8.4333798321436770, 73.2780544577291835) -- (-8.4197743639788456, 73.2920351447709919) -- (-8.3733908717501251, 73.3232538550116999) -- (-8.3681332521998772, 73.3264921153748901) -- (-8.3563834799466949, 73.3324069900771320) -- (-8.3242243922513683, 73.3447982976663155) -- (-8.3171133031795517, 73.3453313642115603) -- (-8.3044340131781205, 73.3507145730784629) -- (-8.2960850513686353, 73.3604021916643774) -- (-8.2899012536551808, 73.3722511579031362) -- (-8.2889739622968719, 73.3803434692123631) -- (-8.2979407472801565, 73.5162144550237997) -- (-8.2997953299967762, 73.5313186528247371) -- (-8.3653513781249558, 73.6171873701927097) -- (-8.3771011503782518, 73.6301594832153938) -- (-8.3987472253629711, 73.6458367050812370) -- (-8.4043131999026244, 73.6577295458183272) -- (-8.4058583144348571, 73.6647540137781505) -- (-8.4052404912609315, 73.6739457631126129) -- (-8.4018396808171740, 73.6896318844035676) -- (-8.3622589226707404, 73.8016612267734615) -- (-8.2280565973500366, 73.8325522273712949) -- (-8.1869307246713685, 73.9171323523400616) -- (-8.2364066723543736, 74.0023411030171303) -- (-8.2308395846198668, 74.0066813410418121) -- (-8.2147605973695992, 74.0083080157941708) -- (-8.2085756864612343, 74.0072190001687460) -- (-8.0604595379862349, 73.9529421908469828) -- (-8.0047997925895853, 73.9176674338723387) -- (-7.9955235394217263, 73.9089816062712543) -- (-7.8501892650217053, 73.7962464839022516) -- (-7.7435084574098196, 73.7529402757640327) -- (-7.7240275465209525, 73.7366918599164620) -- (-7.7162964078853831, 73.7264212132107133) -- (-7.7209350910667380, 73.6950391469154908) -- (-7.7407243569450817, 73.6793519263246566) -- (-7.8671966568251230, 73.6069170575288041) -- (-7.9528503258210002, 73.5610188485555767) -- (-7.9695482494400469, 73.5512904945650376) -- (-7.9967602989644071, 73.5243082590083077) -- (-8.0122203498458529, 73.5037956317465131) -- (-8.0734538621464296, 73.4052055254939120) -- (-8.1556989283343526, 73.3560981568168415) -- (-8.2447545209689483, 73.1780552643971305);

\end{tikzpicture}
\caption{UK National Grid}
\end{figure}

For delivery to most consumers, electric energy is transferred, at a
substation, from the transmission system to the grid supply point of a distribution
system.  Distribution networks are also three-phase AC power systems, but
typically operate at lower voltages and differ in their general
structure or topology from transmission networks.  Transmission networks are
typically highly interconnected, providing multiple paths for power flow.
Whereas distribution networks, in rural areas, typically consist of long radial
feeders (usually overhead lines) or, in urban areas, consist of many ring
circuits.  Three-phase transformers, that step the voltage down to levels
more convenient for general use (typically from 11kV or 33kV to 400V), are
spaced along the branches/rings. All three-phases at 400V may be provided for
industrial and commercial loads or individual phases at 230V supply typical
domestic and other commercial loads. Splitting of phases is usually planned so
that each is loaded equally. This produces a balanced, symmetrical system that
may be analysed, as explained in Section \ref{sec:pf_form}, as a
\textit{single} phase circuit.  Figure \ref{fig:powersystem} illustrates the
basic structure of a typical national electric power system.

\section{Electricity Markets}
The UK was the first large country to privatise its electricity supply industry
when it did so in the early 1990s.  The approach has been used as a model
by other countries and the market structures implemented in the UK have used
most of the main concepts available in national electricity market design.

The England and Wales Electricity Pool was created in 1990 to break up the
vertically integrated Central Electricity Generating Board (CEGB) and
gradually introduce competition in generation and retail supply.  Early
adoption of electricity markets by the UK has lead to the country hosting many
of the main European power and gas exchanges and the UK boasts a high degree
of consumer switching compared to other European countries, an important factor
in a competitive marketplace. The Pool has since been replaced by trading arrangements in which market outcomes
are not centrally determined, but arise largely from bilateral agreements
between producers and suppliers.

\subsection{The England and Wales Electricity Pool}
\label{sec:thepool}
The Electric Lighting Act 1882 initiated the development of the UK's
electricity supply industry by permitting persons, companies and local authorities to set
up supply systems, principally at the time for the purposes of street lighting and
trams.  The Central Electricity Board started operating the first grid of
interconnected regional networks (synchronised at 132kV, 50Hz) in 1933.
This began operation as a national system five years later and was
nationalised in 1947.  Over 600 electricity companies were merged in the
process and the British Electricity Authority was created.  It was later
dissolved and replaced with the CEGB and the Electricity Council under The
Electricity Act 1957.  The CEGB was responsible for planning the network and
generating sufficient electricity until the beginning of privatisation.

The UK electricity supply industry was privatised and The England and Wales
Electricity Pool created in March 1990.  Control of the transmission
system was transferred from the CEGB to The National Grid Company, which was
originally owned by twelve regional electricity companies and has since become publicly listed.  The
Pool was a multilateral contractual arrangement between generators and suppliers and did
not itself buy or sell electricity.  Competition in generation was introduced
gradually, by first entitling customers with consumption greater than or equal
to 1MW (approximately 45\% of the non-domestic market \cite{decc:dukes09}) to
purchase electricity form any listed supplier.  This limit was lowered in April
1994 to included customers with peak loads of 100kW or more.  Finally, between
September 1998 and March 1999 the market was opened to all customers.

Scheduling of generation was on a merit order basis (cheapest first) at a day
ahead stage and set a wholesale electricity price for each half-hour period of
the schedule day.  Forecasts of total demand in MW, based on historic data and
adjusted for factors such as the weather, for each settlement period were used
by generating companies and organisations with interconnects to the England
and Wales grid to formulate bids that had to be submitted to the grid operator
by 10AM on the day before the schedule day.

%\begin{figure}
\centering
\begin{tikzpicture}[thick,yscale=0.85]
  \tikzstyle{incr}=[above,sloped,text width=30mm,text centered]

  \coordinate (orig) at (0,0);
  \coordinate (y) at (0,9);
  \coordinate (x) at (10,0);

  \coordinate (p0) at (0,1);
  \coordinate (p1) at (4,1.8);
  \coordinate (p2) at (7.5,4.2);
  \coordinate (p3) at (9.5,8);

  \draw[axis] (y) node[above]{\pounds/hour} -- (orig) -- (x) node[right]{MW};

  \foreach \pnt in {p0,p1,p2,p3}
    \node[circle,draw,minimum width=4pt,inner sep=0pt,fill=black] at (\pnt) {};

%   \draw[line width=1.5pt] (p0) node[left]{$C_{noload}$} --
%   node[incr]{{\small $1^{st}$ incremental price (\pounds/MWh)}} (p1) --
%   node[incr]{{\small $2^{nd}$ incremental price (\pounds/MWh)}} (p2) --
%   node[incr]{{\small $3^{rd}$ incremental price (\pounds/MWh)}} (p3);
  \draw[line width=1.5pt] (p0) node[left]{$C_{noload}$} --
  node[incr]{$C_1$~(\pounds/MWh)} (p1) --
  node[incr]{$C_2$~(\pounds/MWh)} (p2) --
  node[incr]{$C_3$~(\pounds/MWh)} (p3);

  \draw[help lines] (p1 |- x) coordinate (cp1) -- (p1);
  \draw[help lines] (p2 |- x) coordinate (cp2) -- (p2);
  \draw[help lines] (p3 |- x) coordinate (cp3) -- (p3);

  \coordinate (c1) at (intersection of orig--x and p1--cp1);
  \coordinate (c2) at (intersection of orig--x and p2--cp2);
  \coordinate (c3) at (intersection of orig--x and p3--cp3);
  \coordinate (off) at (0,3mm);

%  \fill[red] (c1) circle (2pt);
  \draw[<->,thin] ($(orig)-(off)$) -- node[below]{$P_1$} ($(c1)-(off)$);
  \draw[<->,thin] ($(c1)-(off)$) -- node[below]{$P_2$} ($(c2)-(off)$);
  \draw[<->,thin] ($(c2)-(off)$) -- node[below]{$P_3$} ($(c3)-(off)$);

  \foreach \pnt in {orig,c1,c2,c3}
    \draw[-,thin,shorten <= 4pt] (\pnt) -- ($(\pnt)-2*(off)$);

\end{tikzpicture}
\caption{Pool bid structure.}
\label{fig:poolbids}
\end{figure}


\begin{figure}
\centering
\begin{Large}
\begin{tikzpicture}[thick,yscale=0.85]
%  \draw[step=1.0cm,color=gray] (0,0) grid (8,5);
%  \node at (4.5,9.5) {Active power cost function};
  \coordinate (y) at (0,9);
  \coordinate (x) at (10,0);
  \coordinate (p0) at (0,0.1);
  \coordinate (p1) at (3.2,0.4);
  \coordinate (p2) at (5.2,1.0);
  \coordinate (p3) at (7,2.5);
  \coordinate (p4) at (8.6,5);
  \coordinate (pn) at (9.5,8.5);

  \foreach \pnt in {p0,p1,p2,p3,pn}
    \node[circle,draw,minimum width=4pt,inner sep=0pt,fill=black] at (\pnt) {};

%   \draw[axis] (y) -- node[anchor=south,rotate=90,minimum size=15mm]
%   {Cost \$/MWh} (0,0) -- node[anchor=north,minimum size=15mm] {Set-point (MW)}
%   (x);
  \draw[axis] (y) node[above]{$c$} -- (0,0) -- (x) node[right]{$p$};
%  \coordinate (offprca) at ($0.8*(y)$);
%  \coordinate (offprcb) at ($0.533*(y)$);
%  \coordinate (offqtya) at ($.6*(x)$);
%  \coordinate (offqtyb) at ($.9*(x)$);
%  \draw[] let \p1=(offprca), \p2=(offqtya) in
%  (\p1) node[left] {$\alpha$} -- (\p2);
  \draw[line width=1.5pt] (p0) -- (p1) -- (p2) -- (p3);
  \draw[line width=1.5pt] (p3) -- ($(p3)!0.6!(p4)$);
  \draw[dotted] ($(p3)!0.6!(p4)$) -- ($(p3)!1.5!(p4)$);
  \draw[dashed,line width=1.5pt,shorten >=15pt] ($(p3)!0.6!(p4)$) -- (p4);

  \draw[line width=1.5pt] ($(p4)!0.4!(pn)$) -- (pn);
  \draw[dotted] ($(p4)!-0.5!(pn)$) -- ($(p4)!0.4!(pn)$);
  \draw[dashed,line width=1.5pt,shorten >=15pt] ($(p4)!0.4!(pn)$) -- (p4);

  \draw[help lines] (p1 |- x) node[below]{$p_1$} -- (p1) -- (y |- p1)
  node[left]{$c_1$};
  \draw[help lines] (p2 |- x) node[below]{$p_2$} -- (p2) -- (y |- p2)
  node[left]{$c_2$};
  \draw[help lines] (p3 |- x) node[below]{$p_3$} -- (p3) -- (y |- p3)
  node[left]{$c_3$};
  \draw[help lines] (pn |- x) node[below]{$p_n$} -- (pn) -- (y |- pn)
  node[left]{$c_n$};

  \foreach \x in {2,3.5,5,6.5}
    \draw[->,line width=2pt] (\x,6.2) -- (\x,5);
  \node at (4.25,6.2) {$y$};
\end{tikzpicture}
\end{Large}
\caption{Piecewise linear active power cost function with constrained cost
variable minimsation illustrated.}
\label{fig:ccv_function}
\end{figure}

Figure \ref{fig:poolbids} graphically depicts four of the five price parameters
that made up a bid.  A start-up price would also be stated, representing the
cost of turning on the generator from cold.  The no-load price $c_{noload}$
represents the cost in pounds of keeping the generator running regardless of output. Three
incremental prices $c_1$, $c_2$ and $c_3$ specify the cost in \pounds/MWh of
generation between set-points $p_1$, $p_2$ and $p_3$.

A settlement algorithm would determine an unconstrained schedule
(with no account being taken for the physical limitations of the transmission
system), meeting the forecast demand and requirements for reserve while minimising cost.
Cheapest bids up to the marginal point would get accepted first and the bid
price from the marginal generator would generally determine the system marginal
price for each settlement period.  The system marginal price would form the
basis of the prices paid by consumers and paid to generators, which would be
adjusted such that that the costs of transmission are covered by the market and that the
availability of capacity is encouraged at certain times.

Variations in demand and changes in plant availability got adjusted for by
the grid operator between day close and physical delivery, producing a
constrained schedule. Generators having submitted bids would be instructed to
increase or reduce production appropriately.  Alternatively, the grid operator
could instruct large customers with contracts to curtail their demand to do so
or instruct generators contracted to provide ancillary services to adjust
production.

\subsection{British Electricity Transmission and Trading Arrangements}
%\subsection{New Electricity Trading Arragements}
\label{sec:betta}
Concerns over exploitation of market power in The England and Wales Electricity
Pool and its effectiveness in reducing consumer electricity prices prompted the
introduction of New Electricity Trading Arrangements (NETA) in March 2001
\cite{martoccia:2005}.  The aim was to improve efficiency and provide greater
choice to participants.  Control of the Scottish transmission system was
included with the introduction of the nationwide British Electricity
Transmission and Trading Arrangements (BETTA) in April 2005 under The Energy
Act 2004.  While The Pool operated a single daily auction and dispatched plant
centrally, under the new arrangements participants became self-dispatching and
market positions became determined through continuous bilateral trading
between generators, suppliers, traders and consumers.

The majority of power is traded under the BETTA through long-term contracts
that are customised to the requirements of each party \cite{kirschen:book}.
These instruments suit participants responsible for large power plants or those
purchasing large volumes of power for many customers.  Sizeable amounts of time and
effort are required for these long-term contracts to be formed and this results
in a high associated transaction cost.  However, they reduce risk for large
players and may include a degree of flexibility.

Electric power is also traded directly between participants through
over-the-counter contracts that are usually of a standardised form.  Such contracts
typically concern smaller volumes of power and have much lower associated
transaction costs.  Often they are used by participants to refine their market
position ahead of delivery time \cite{kirschen:book}.

Trading facilities, such as power exchanges, provide a means for participants
to fine-tune their positions further, through short-term transactions for
relatively small quantities of energy.  Modern exchanges are computerised and
accept anonymous offers and bids submitted electronically \cite{kirschen:book}.
A submitted offer/bid will be paired with any outstanding bids/offers in the
system with compatible price and quantity values.  The details are then
displayed for traders to observe and to use in subsequent trading.

All bilateral trading must be completed before ``gate-closure'' which is a
point in time, before delivery time, that gives the system operator an
opportunity to balance supply and demand and mitigate potential breaches of
system limits.  In keeping with the UK's free market philosophy, a competitive
spot market \cite{schweppe:spot} is used in the balancing process.  A
generator that is not fully loaded may offer a price at which it is willing to
increase its output by a specified quantity, stating the rate at which it is
capable of doing so.  Certain loads may also offer demand reductions at a
price which can typically be implemented very quickly.  Longer-term contracts
for balancing services are also struck between the system operator and
generators/suppliers in order to avoid the price volatility often associated
with spot markets.

\section{Electricity Market Simulation}
Previous sections have identified the importance of electricity to
modern societies and explained how the majority of electricity supply in the UK
is trusted to unadministered bilateral trade. Electricity supply involves
technology, money, people, natural resources and the environment.  These aspects are all changing and the discipline must be constantly researched to ensure that systems such as electricity markets are fit for purpose.  The value of electricity to society
means that it is not feasible to experiment with radical changes to trading
arrangements on real systems.  A practical alternative is to study abstract
mathematical models (with sets of simplifying approximations and assumptions)
and find analytical solutions, where possible, by simulating them using
computer programs.

Game theory is the branch of applied mathematics in which behaviour in
strategic situations is captured mathematically.  A common approach to doing
this is to model the system and players as a mathematical optimisation problem.  Optimal
power flow is a classic optimisation problem in the field of electric
power engineering and variants are widely used to research electricity
markets.  In this thesis, optimal power flow forms one part of an
\textit{agent-based} simulation, which is an alternative approach to the
mathematics of games.

\subsection{Agent-Based Simulation}
Social systems, such as electricity markets, are inherently complex and involve
interactions between different types of individuals and between individuals
and collective entities, such as organisations or groups, the behaviour of which
is itself the product of individual interactions.  This complexity
drives classical monolithic equilibrium models to their limits.  The models are
often highly stylised and limited to small numbers of players with strong
constraining assumptions made on their behaviour.

Agent-based simulation involves modelling simultaneous operations and
interactions between adaptive agents and assessing their effect on the system
as a whole.  Macro-level system properties arise from agent interactions, even
those with simple behavioural rules, that could not be deduced by simply
aggregating the agent's properties. % Game of Life

Following \citeA{tesfatsi:handbook}, the objectives of agent-based modelling
research fall roughly into four strands: empirical, normative, heuristic and
methodological. The \textit{empirical} objectives are to understand how and why macro-level
regularities have evolved from micro-level interactions when little or no
top-down control is present.  Research with the \textit{normative} goals aims
to relate agent-based models to an ideal standard or optimal design.  The objective being
to evaluate proposed designs for social policy, institutions or processes in
their ability to produce socially desirable system performance.  The
\textit{heuristic} strand aims to generate theories on the fundamental causal
mechanisms in social systems that can be observed, even in simple systems, when there are
alternative initial conditions.  This thesis aims to provide
\textit{methodological} advancement.  Improvements in the tools and methods
available aid research with the former objectives.

\subsection{Optimal Power Flow}
\label{sec:opf}
Nationalised electricity supply industries were for many years planned,
operated and controlled centrally.  A system operator would determine which
generators must operate and the required output of the operating units such
that demand and reserve requirements were met and the overall cost of
production was minimised.  In Electric Power Engineering, this is termed the
\textit{unit commitment} and \textit{economic dispatch} problem.

In 1962 a unit commitment formulation was published with power system
constraints incorporated \cite{carpentier:opf}.  \textit{Optimal power flow} is
this integration of the economic and the power flow aspects of power systems into a
mathematical optimisation problem.  The ability to use optimal power flow to
solve centralised power system operation problems and determine prices in power
pool markets has led to it being one of the most widely studied subjects in
the power systems community.

\subsubsection{Power Flow Formulation}
\label{sec:pf_form}
Optimal power flow derives its name from the \textit{power flow} (or load flow)
steady-state power system analysis technique.  Given sets of generator data,
load data and a nodal admittance matrix, a power flow study determines the
voltage
\begin{equation}
V_i = \vert V_i \vert \angle\delta_i = \vert
V_i\vert(\cos\delta_i + j\sin\delta_i)
\end{equation}
at each node $i$ in the power system from which branch flows may be calculated
\cite{grainger:psa}.

\paragraph{Nodal Admittance Matrix}
The nodal admittance matrix describes the electrical network and its
formulation is dependant upon the transmission line, transformer and shunt
models employed.  Following \citeA[p.11]{pserc:mp_manual}, a branch in a power
system nodal representation is typically modelled as a medium length
transmission line in series with a regulating transformer at the ``from'' end.
A nominal-$\pi$ model with total series admittance $y_s = 1/(r_s+jx_s)$ and
total shunt capacitance $b_c$ represents the transmission line.  The
transformer is assumed to be ideal, phase-shifting and tap-changing, with the
ratio between primary winding voltage $v_{f}$ and secondary winding voltage
$N = \tau e^{j\theta_{ph}}$ where $\tau$ is the tap ratio and $\theta_{ph}$ is
the phase shift angle. Figure \ref{fig:branch_model} diagrams this
conventional branch model.  From Kirchhoff's Current Law the current in the
series impedance is
\begin{equation}
\label{eq:iseries}
i_s = \frac{b_c}{2}v_t - i_t
\end{equation}
and from Kirchhoff's Voltage Law the voltage across the secondary winding of
the transformer is
\begin{equation}
\frac{v_{f}}{N} = v_t + \frac{i_s}{y_s}
\end{equation}
Substituting $i_s$ from equation (\ref{eq:iseries}), gives
\begin{equation}
\label{eq:vfrom}
\frac{v_{f}}{N} = v_t - \frac{i_t}{y_s} + v_t\frac{b_c}{2y_s}
\end{equation}
and rearranging in terms if $i_t$, gives
\begin{equation}
\label{eq:ito}
i_t = v_s \left( \frac{-y_s}{\tau e^{\theta_{ph}}} \right) +
v_r \left( y_s + \frac{b_c}{2} \right)
\end{equation}
The current through the secondary winding of the transformer is
\begin{equation}
N^*i_f = i_s + \frac{b_c}{2}\frac{v_{f}}{N}
\end{equation}
Substituting $i_s$ from equation(\ref{eq:iseries}) again, gives
\begin{equation}
N^*i_f = \frac{b_c}{2}v_t - i_t + \frac{b_c}{2}\frac{v_{f}}{N}
\end{equation}
and substituting $\frac{v_{f}}{N}$ from equation (\ref{eq:vfrom}) and
rearranging, gives
\begin{equation}
\label{eq:ifrom}
i_s = v_s \left( \frac{1}{\tau^2} \left(y_s + \frac{b_c}{2}\right) \right) +
v_r \left(\frac{y_s}{\tau e^{-j\theta}}\right)
\end{equation}

%\begin{figure}
\centering
\begin{tikzpicture}[thick]
%   \def\capacitor(#1,#2){%
%     \draw (#1,#2+3cm) -- (#1,#2+5mm);
% %    \draw (#1-1cm,#2+5mm) -- (#1+1cm,#2+5mm);
% %     \draw[linewidth=1pt] (-1cm,-5mm) -- (1cm,-5mm);
% %     \draw[linewidth=1pt] (0,-1cm) -- (0,-5mm);
%   }
%
%   \capacitor(0,0);

%   \def\terminal(#1,#2){#3}{%
%     \node[circle,draw,minimum width=5pt,inner sep=0pt] (#3) at (#1,#2) {};
%   }
%   \terminal(5,5);

  \coordinate (vf0) at (0,0);
  \coordinate (vf+) at (0,6);
  \coordinate (vt0) at (13,0);
  \coordinate (vt+) at (13,6);

  \coordinate (trx) at (2,3);
  \coordinate (vf0p) at ($(trx)-(0.5,3)$);
  \coordinate (vf+p) at ($(trx)+(-0.5,3)$);
  \coordinate (vf0s) at ($(trx)-(-0.5,3)$);
  \coordinate (vf+s) at ($(trx)+(0.5,3)$);

  \coordinate (vl0) at (4,0);
  \coordinate (vl+) at (4,6);
  % resistor
  \coordinate (res) at (7.5,6);
  \coordinate (rin) at ($(res) -(7mm,0)$);
  \coordinate (rout) at ($(res) + (7mm,0)$);
  \coordinate (rh) at (0,6pt);
  % inductor
  \coordinate (ind) at (9.5,6);
  \coordinate (lin) at ($(ind)-(7mm,0)$);
  \coordinate (lout) at ($(ind)+(7mm,0)$);

  \coordinate (capw) at (7mm,0);
  \coordinate (bc1) at (5.8,3);
  \coordinate (bc1in) at ($(bc1) + (0,2mm)$);
  \coordinate (bc1out) at ($(bc1) - (0,2mm)$);
  \coordinate (bc2) at (11.2,3);
  \coordinate (bc2in) at ($(bc2) + (0,2mm)$);
  \coordinate (bc2out) at ($(bc2) - (0,2mm)$);

  \node[circle,draw,minimum width=4pt,inner sep=0pt] (nvf0) at (vf0) {};
  \node[circle,draw,minimum width=4pt,inner sep=0pt] (nvf+) at (vf+) {};
  \node[circle,draw,minimum width=4pt,inner sep=0pt] (nvl0) at (vl0) {};
  \node[circle,draw,minimum width=4pt,inner sep=0pt] (nvl+) at (vl+) {};
  \node[circle,draw,minimum width=4pt,inner sep=0pt] (nvt0) at (vt0) {};
  \node[circle,draw,minimum width=4pt,inner sep=0pt] (nvt+) at (vt+) {};

  \draw (nvl0) -- (nvt0);
  % transformer
  \draw (nvf0) -- (vf0p);
  \draw (vf0s) -- (nvl0);
  \draw (nvf+) -- (vf+p);
  \draw (vf+s) -- (nvl+);
  \draw[decorate,decoration={coil,amplitude=5pt,segment length=5.5pt,
  pre length=10mm,post length=10mm}] (vf+p) -- (vf0p);
  \draw[decorate,decoration={coil,amplitude=5pt,segment length=5.5pt,
  pre length=10mm,post length=10mm}] (vf0s) -- (vf+s);
  \coordinate (cor1) at ($(trx)-(1mm,0)$);
  \coordinate (cor2) at ($(trx)+(1mm,0)$);
  \coordinate (corh) at (0,1.8);
  \draw[line width=1.5pt] ($(cor1)-(corh)$) -- ($(cor1)+(corh)$);
  \draw[line width=1.5pt] ($(cor2)-(corh)$) -- ($(cor2)+(corh)$);

  % resistor
  \filldraw[fill=white,drop shadow] ($(rin)-(rh)$) rectangle ($(rout)+(rh)$);
  \draw ($(res) + (0,5mm)$) node {$r_s$};
  \draw (nvl+) -- (rin);

  \draw (rout) -- (lin);
  % inductor
  \draw[decorate,decoration={coil,amplitude=6pt,segment length=6pt}]
  (lin) -- (lout);
  \draw ($(ind) +(0,5mm)$) node {$x_s$};
  \draw (lout) -- (nvt+);
  % capacitor 1
  \draw (bc1in) -- (bc1in |- vf+s) node[circle,draw,minimum width=2pt,
  inner sep=0pt,fill=black]{};
  \draw (bc1out) -- (bc1out |- vf0s) node[circle,draw,minimum width=2pt,
  inner sep=0pt,fill=black]{};
  \draw[line width=1.2pt] ($(bc1in)-(capw)$) -- ($(bc1in)+(capw)$);
  \draw[line width=1.2pt] ($(bc1out)-(capw)$) -- ($(bc1out)+(capw)$);
%   \draw[line width=1.2pt] ($(bc1out)$) .. controls ($(bc1out)+(capw)+(-2mm,0)$)
%   .. ($(bc1out)+(capw)+(0,-2mm)$); \draw ($(bc1) + (capw)$) node[anchor=west] {$\dfrac{b_c}{2}$};
  % capacitor 2
  \draw (bc2in) -- (bc2in |- vt+) node[circle,draw,minimum width=2pt,
  inner sep=0pt,fill=black]{};
  \draw (bc2out) -- (bc2out |- vt0) node[circle,draw,minimum width=2pt,
  inner sep=0pt,fill=black]{};
  \draw[line width=1.2pt] ($(bc2in)-(capw)$) -- ($(bc2in)+(capw)$);
  \draw[line width=1.2pt] ($(bc2out)-(capw)$) -- ($(bc2out)+(capw)$);
  \draw ($(bc2) - (capw)$) node[anchor=east] {$\dfrac{b_c}{2}$};

  \draw[->,shorten >=5pt,shorten <=5pt,thin] (vf0) node[above right] {$-$} --
  node[fill=white] {$v_f$} (vf+) node[below right] {$+$};
  \draw[->,shorten >=5pt,shorten <=5pt,thin] (vl0) node[above left] {$-$} --
  node[fill=white] {$\dfrac{v_f}{N}$} (vl+) node[below left] {$+$};
  \draw[->,shorten >=5pt,shorten <=5pt,thin] (vt0) node[above left] {$-$} --
  node[fill=white] {$v_t$} (vt+) node[below left] {$+$};
  \node at ($(trx)-(0,2.4)$) {$N:1$};
  \draw[->,thin] ($(vf+)+(rh)$) -- node[above]{$i_f$} ++(12mm,0);
  \draw[->,thin] ($(vf+s)+(rh)$) -- node[above]{$N^*i_f$} ++(12mm,0);
  \draw[->,thin] ($(vt+)+(rh)$) -- node[above]{$i_t$} ++(-12mm,0);
  \draw[->,thin] ($(lout)+(rh)$) -- node[above]{$i_l$} ++(8mm,0);

\end{tikzpicture}
\caption{Nominal-$\pi$ transmission line model in series with a phase
shifting transformer model.}
\label{fig:branch_model}
\end{figure}

Combining equations (\ref{eq:ito}) and (\ref{eq:ifrom}), the \textit{from}
and \textit{to} end complex current injections for branch $l$ are
\begin{equation}
\label{eq:ybranch}
\begin{bmatrix}
i_f^l\\
i_t^l
\end{bmatrix}
=
\begin{bmatrix}
y_{ff}^l& y_{ft}^l\\
y_{tf}^l& y_{tt}^l
\end{bmatrix}
\begin{bmatrix}
v_f^l\\
v_t^l
\end{bmatrix}
\end{equation}
where
\begin{eqnarray}
\label{eq:yff}
y_{ff}^l& =& \frac{1}{\tau^2} \left(y_s + \frac{b_c}{2}\right)\\
\label{eq:yft}
y_{ft}^l& =& \frac{y_s}{\tau e^{-j\theta_{ph}}}\\
\label{eq:ytf}
y_{tf}^l& =& \frac{-y_s}{\tau e^{j\theta_{ph}}}\\
\label{eq:ytt}
y_{tt}^l& =& y_s + \frac{b_c}{2}
\end{eqnarray}
Let $Y_{ff}$, $Y_{ft}$, $Y_{tf}$ and $Y_{tt}$ be $n_l \times 1$ vectors where
the $l^{th}$ element of each corresponds to $y_{ff}^l$, $y_{ft}^l$,
$y_{tf}^l$ and $y_{tt}^l$, respectively.  Furthermore, let $C_f$ and $C_t$ be the
$n_l \times n_b$ branch-bus connection matrices, where $C_{f_{i,j}} = 1$ and
$C_{t_{i,k}} = 1$ if branch $i$ connects from bus $j$ to bus $k$.  The
$n_l \times n_b$ branch admittance matrices are
\begin{eqnarray}
Y_f& =& \diag(Y_{ff})C_f + \diag(Y_{ft})C_t\\
Y_t& =& \diag(Y_{tf})C_f + \diag(Y_{tt})C_t
\end{eqnarray}
and the
$n_b \times n_b$ nodal admittance matrix is
\begin{eqnarray}
Y_{bus}& =& C_f^\mathsf{T} Y_f + C_t^\mathsf{T} Y_t .
\end{eqnarray}
% and it relates the complex bus voltages to the nodal current injections
% \begin{eqnarray}
% I_{bus}& =& Y_{bus}V
% \end{eqnarray}
% The complex bus power injections are expressed as a non-linear function of $V$
% \begin{eqnarray}
% S_{bus}(V)& =& \diag(V)I_{bus}^* \nonumber \\
% \label{eq:sbus}
% &= & \diag(V)Y_{bus}^*V^*
% \end{eqnarray}
% The net complex power injection (generation - load) at each bus must equal the
% sum of complex power flows on each branch connected to the bus.  Hence the AC
% power balance equations are
% \begin{equation}
% \label{eq:mismatch}
% S_{bus}(V) + S_d - S_g = 0
% \end{equation}

\paragraph{Power Balance}
% Importantly, the relationship between nodal voltages and power entering the
% network is non-linear.
For a network of $n_b$ nodes, the current injected at
node $i$ is
\begin{equation}
I_i = \sum_{j=1}^{n_b} Y_{ij} V_j
\end{equation}
where $Y_{ij} = \vert Y_{ij}\vert \angle\theta_{ij}$ is the $(i,j)^{th}$ element
if the $Y_{bus}$ matrix.  Hence, the apparent power entering
the network at bus $i$ is
\begin{equation}
S_i = P_i+Q_i = V_iI_i^* = \sum_{n=1}^{n_b} \vert Y_{ij}V_iV_j \vert \angle
(\delta_i - \delta_j - \theta_{ij})
\end{equation}
Converting to polar coordinates and separating the real and imaginary parts,
the active power
\begin{equation}
P_i = \sum_{n=1}^{n_b} \vert Y_{ij}V_iV_j \vert \cos(\delta_i - \delta_j -
\theta_{ij})
\end{equation}
and the reactive power entering the network
\begin{equation}
Q_i = \sum_{n=1}^{n_b} \vert Y_{ij}V_iV_j \vert \sin(\delta_i - \delta_j -
\theta_{ij})
\end{equation}
at bus $i$ are non-linear functions of $V_i$, as indicated by the presence of
the sine and cosine terms.  Kirchoff's Current Law requires that the net
complex power injection (generation - load) at each bus equals the sum of
complex power flows on each branch connected to the bus.  The power balance
equations
\begin{equation}
\label{eq:p_balance}
P_g^i - P_d^i = P^i
\end{equation}
and
\begin{equation}
\label{eq:q_balance}
Q_g^i - Q_d^i = Q^i
\end{equation}
where the subscripts $g$ and $d$ indicate generation and demand
respectively, form a key non-linear constraint in the optimal power flow
problem.

\subsubsection{Optimal Power Flow Formulation}
Optimal power flow is a mathematical optimisation problem in which the complex
power balance equations (\ref{eq:p_balance}) and (\ref{eq:q_balance}) form one
of the constraints. Mathematical optimisation problems have the general form
\begin{equation}
\min_x f(x)
\end{equation}
subject to
\begin{eqnarray}
\label{eq:equality}
g(x)& =& 0\\
\label{eq:inequality}
h(x)& \leq& 0
\end{eqnarray}
where $x$ is the optimisation variable, $f$ is the objective function and
equations (\ref{eq:equality}) and (\ref{eq:inequality}) are sets of equality
and inequality constraints on $x$, respectively.  Typical inequality
constraints are bus voltage magnitude contingency state limits, generator
output limits and branch power or current flow limits.  The optimisation
variable $x$ may be made up of generator set-points, bus voltages, transformer
tap settings etc.  If the optimisation variable $x$ is empty then the
formulation reduces to the general power flow problem described in above.

A common objective of optimal power flow is total system cost minimisation.
For and network of $n_g$ generators the objective function is
\begin{equation}
\label{eq:objfunc}
\min_{\theta, V_m, P_g} \sum_{k=1}^{n_g} C^k_P(P_g^k) + C_Q^k(Q_g^k)
\end{equation}
where $C_P^k$ and $C_Q^k$ are cost functions (typically quadratic) of the
set-points $P_g^k$ and $Q_g^k$ for generator $k$, respectively. Alternative
objectives may be to minimise losses, maximise the voltage stability margin or
minimise deviation of an optimisation variable from a particular schedule
\cite[\S18]{kallrath:2009}.

\subsubsection{Nodal Marginal Prices}
Many solution methods for optimal power flow have been developed
since Carpentier introduced the problem and a review of the main techniques
can be found in \citeA{momoh:part1,momoh:part2}.  One of the most robust
strategies is to solve the Lagrangian function
\begin{equation}
\mathcal{L}(x) = f(x) + \lambda^\mathsf{T}g(x) + \mu^\mathsf{T}h(x),
\end{equation}
where $\lambda$ and $\mu$ are the Lagrangian multipliers, using an Interior
Point Method.  When solved, the Lagrangian multiplier for a constraint gives
the rate of change of the objective function value with respect to the
constraint variable.  If the objective function is equation (\ref{eq:objfunc}), the Lagrangian
multipliers $\lambda^i_P$ and $\lambda^i_Q$ for the power balance constraint at
each bus $i$, given by equations (\ref{eq:p_balance}) and (\ref{eq:q_balance}),
are the nodal marginal prices and can be interpreted as the increase in the
total system cost for and additional injection at $i$ of 1MW or 1MVAr,
respectively.  For a case in which none of the inequality constraints $h(x)$
(such as branch power flow or bus voltage limits) are binding, the nodal
marginal prices are uniform across all buses and equal the cost of the
marginal generating unit.  When the constraints \textit{are} binding, the nodal
marginal prices are elevated for buses at which adjustments to power injection
are required for the constraints to be satisfied.  Nodal marginal prices are
commonly used in agent-based electricity market simulation to determine the
revenue for generating units as they reflect the increased value of production in
constrained areas of the power system.

\section{Reinforcement Learning}
\label{sec:rl}
% The problem of learning how best to interact with an environment so as to
% maximise some long-term reward is a general one.  Reinforcement learning is a
% term from the field of machine learning that is typically applied to
% understanding, automating and solving this problem through adaptive
% computational approaches.  Unlike many machine learning techinques, the
% algorithms are not instructed as to which actions to take, but must learn to
% maximise the long-term reward through trial-and-error.

% Reinforcement learning starts with an interactive, goal-seeking individual
% agent existing in an environment.  The agent requires the ability to;
% \begin{itemize}
%   \item Sense aspects of its environment,
%   \item Perform actions that influence the state of its environment and
%   \item be assigned rewards as a response to their chosen action.
% \end{itemize}
% An agent follows a particular \textit{policy} when mapping the perceived state
% of its environment to an action choice.  Reinforcement learning methods adjust
% the agent's policy.
Reinforcement learning is learning from reward by mapping situations to actions
when interacting with an uncertain environment \cite{suttonbarto:1998}.  An
agent learns \textit{what} to do in order to achieve a task through
trial-and-error using a numerical reward or a penalty signal without being
instructed \textit{how} to achieve it.  In challenging cases, actions may not
yield immediate reward or may affect the next situation and all subsequent
rewards.  A compromise must be made between exploitation of past experiences
and exploration of the environment through new action choices.  A
reinforcement learning agent must be able to:
\begin{itemize}
  \item Sense aspects of its environment,
  \item Take actions that influence its environment and,
  \item Have an explicit goal or set of goals relating to the state of its
  environment.
\end{itemize}
In the classical model of agent-environment interaction, at each time step $t$
in a sequence of discrete time steps $t = 1,2,3\dotsc$ an agent receives as
input some form of the environment's state $s_t \in \mathscr{S}$, where
$\mathscr{S}$ is the set of possible states.  From a set of actions
$\mathscr{A}(s_t)$ available to the agent that state and the agent selects
an action $a_t$ and performs it upon its environment.  The environment enters a
new state $s_{t+1}$ in the next time step and the agent receives a scalar
numerical reward $r_{t+1} \in \mathbb{R}$ in part as a result of its action.
The agent then learns from the state representation, the
chosen action $a_t$ and the reinforcement signal $r_{t+1}$ before beginning
its next interaction.  Figure \ref{fig:seq_rl} diagrams the classical
agent-environment interaction event sequence in reinforcement learning.

%\begin{figure}
  \centering
  \begin{sequencediagram}
    \newthread{agt}{:Agent}
    \newinst[3]{env}{:Environment}
    \begin{sdblock}{Interaction}{}

      \begin{call}{agt}{getState()}{env}{$s_t$}
      \end{call}

      \begin{callself}{agt}{chooseAction($s_t$)}{$a_t$}
      \end{callself}

      \begin{call}{agt}{performAction($a_t$)}{env}{$r_{t+1}$}
        \begin{callself}{env}{changeState($a_t$)}{$s_{t+1}$}
        \end{callself}
      \end{call}

      \begin{callself}{agt}{learn($s_t$,$a_t$,$r_{t+1}$)}{}
      \end{callself}

    \end{sdblock}
  \end{sequencediagram}
  \caption{Sequence diagram for the basic reinforcement learning model.}
  \label{fig:seq_rl}
\end{figure}


%\subsection{Markov Decision Processes}
For a finite number of states $\mathscr{S}$, if all states are Markov, the
agent interacts with a finite Markov decision process (MDP).  Informally,
for a state to be Markov it must retain all relevant information about the
complete sequence of positions leading up to the state, such that all future
states and expected rewards can be predicted as well as would be possible given
a complete history.  A particular MDP is defined for a discrete set of time
steps by a state set $\mathscr{S}$, an action set $\mathscr{A}$, a set of state
transition probabilities $\mathscr{P}$ and a set of expected reward values
$\mathscr{R}$.
% Given a state $s$ and an action $a$, the probability of
% transitioning to each possible next state $s^\prime$ is
% \begin{equation}
% \mathscr{P}^a_{ss^\prime} = \Pr \bigl\lbrace s_{t+1} = s^\prime \vert s_t=s,
% a_t=a \bigr\rbrace .
% \end{equation}
% Given the next state $s^\prime$, the expected value of the next reward is
% \begin{equation}
% \mathscr{R}^a_{ss^\prime} = E \bigl\lbrace r_{t+1} \vert s_t=s, a_t=a,
% s_{t+1}=s^\prime \bigr\rbrace .
% \end{equation}
In practice not all state signals are Markov, but should provide a good basis
for predicting subsequent states, future rewards and selecting actions.

If the state transition probabilities and expected reward values are not known,
only the states and actions, then samples from the MDP must be taken and a value
function approximated iteratively based on new experiences generated by
performing actions.

\subsection{Value Function Methods}
\label{sec:valuebased}
% Value-based methods attempt to find the optimal policy by
% approximating a \textit{value-function} which returns the total reward an
% agent can expect to accumulate, given an initial state and following the
% current policy thereafter.  The policy is adjusted using the reinforcing
% signal and information from previous action choices.
Any method that can optimise control of a MDP may be considered a reinforcement
learning method.  All search for an optimal policy $\pi^*$ that maps state
$s \in \mathscr{S}$ and action $a \in \mathscr{A}$ to the probability
$\pi^*(s,a)$ of taking $a$ in $s$ and maximises the sum of rewards over the
agents lifetime.

Each state $s$ under policy $\pi$ may be associated with a \textit{value}
$V^\pi(s)$ equal to the expected return from following policy $\pi$ from state
$s$.  Most reinforcement learning methods are based on estimating the
state-value function
\begin{equation}
\label{eq:statevalue}
V^\pi(s) = E \Bigg\lbrace \sum^\infty_{t=0} \gamma^t r_t \Bigg\vert s_0 = s
\Bigg\rbrace
\end{equation}
where $\gamma$ is a discount factor, with $0\leq \gamma \leq 1$.
Performing certain actions may result in no state change, creating a loop and
causing the value of that action to be infinite for certain policies.
The discount factor $\gamma$ prevents values from going unbounded and
represents reduced trust in the reward $r_t$ as discrete time $t$
increases.  Many reinforcement learning methods estimate the action-value
function
\begin{equation}
\label{eq:actionvalue}
Q^\pi(s,a) = E \Bigg\lbrace \sum^\infty_{t=0} \gamma^t r_t \Bigg\vert s_0 = s,
a_0 = a \Bigg\rbrace
\end{equation}
which defines the value of taking action $a$ in state $s$ under fixed policy
$\pi$.

\subsubsection{Temporal-Difference Learning}
Temporal Difference (TD) learning is a fundamental concept in reinforcement
learning. TD methods do not attempt to estimate the state transition probabilities and
expected rewards of the finite MDP, but estimate the value function directly.
They learn to \textit{predict} the expected value of total reward returned by
the state-value function (\ref{eq:statevalue}).  For an exploratory policy $\pi$
and a non-terminal state $s$, an estimate of $V^\pi(s_t)$ at any given time step
$t$ is updated using the estimate at the next time step $V^\pi(s_{t+1})$ and the
observed reward $r_{t+1}$
\begin{equation}
V^\pi(s_t) \leftarrow V^\pi(s_t) + \alpha \bigl[r_{t+1} + \gamma
V^\pi(s_{t+1}) - V^\pi(s_t) \bigr]
\end{equation}
where $\alpha$ is the learning rate, with $0 \leq \alpha \leq 1$, which controls
how much attention is paid to new data when updating $V^\pi$.  TD learning
evaluates a particular policy and offers strong convergence guarantees, but does
not learn better policies.

\subsubsection{Sarsa}
\label{sec:sarsa}
% The SARSA algorithm is an on-policy Temporal Difference control method where
% the policy is typically represented by a $M \times N$ table, where $M$ and $N$
% are arbitrary positive numbers equal to the total number of feasible states and
% actions respectively.  The action-value update for agent $j$ is defined by
% \begin{equation}
% Q_j(s_{jt},a_{jt}) + \alpha [r_{jt+1} + \gamma Q_j(s_{jt+1},a_{jt+1}) -
% Q_j(s_{jt},a_{jt})]
% \end{equation}
% While the Q-learning method (See Section \ref{sec:qlearning}, below) updates
% action-values using a greedy policy, which is a different policy to that being
% followed, SARSA uses the discounted future reward of the next state-action
% observation following the original policy.
Sarsa (or modified Q-learning) is an on-policy TD control method that
approximates the state-action value function in (\ref{eq:actionvalue}).
Recall that the state-action value function for an agent returns the total
expected reward for following a particular policy for selecting actions as a
function of future states.  The function is updated according to the rule
\begin{equation}
\label{eq:sarsa}
Q(s_t,a_t) \leftarrow Q(s_t,a_t) + \alpha \bigl[r_{t+1} + \gamma
Q(s_{t+1},a_{t+1}) - Q(s_t,a_t)\bigr]
\end{equation}
This update also uses the action from the next time step $a_{t+1}$ and the
requirement to transition through state-action-reward-state-action for each
time step derives the algorithm's name.  Sarsa is referred to as an on-policy
method since it learns the same policy that it follows.

\subsubsection{Q-Learning}
\label{sec:qlearning}
% This section describes the original off-policy Temporal Difference Q-learning
% method developed by Watkins \cite{watkins:1989}.  The action-value function,
% $Q(s,a)$, returns values from a $M \times N$ matrix where $M$ and $N$ are
% arbitrary positive numbers equal to the total number of feasible states and
% actions, respectively.  Each value represents the \textit{quality} of taking a
% particular action, $a$, in state $s$.  Actions are selected using either the
% $\epsilon$-greedy or softmax (See Section \ref{sec:rotherev}, above) methods.
% The $\epsilon$-greedy method either selects the action (or one of the actions)
% with the highest estimated value or it selects an action at random, uniformly,
% independently of the estimated values with (typically small) probability
% $\epsilon$.
%
% Agent $j$ will observe a reward, $r_{jt}$, and a new state, $s_{jt+1}$,
% after taking action $a_{jt}$ at step $t$ when in state $s_{jt}$.  The
% state-action value, $Q_j(s_{jt},a_{jt})$, is updated according to the
% maximum value of available actions in state $s_{t+1}$ and becomes
% \begin{equation}
% \label{eq:qlearning}
% Q_j(s_{jt},a_{jt}) + \alpha [r_{jt+1} + \gamma\max_{a} Q_j(s_{jt+1},a_{jt}) -
% Q_j(s_{jt},a_{jt})]
% \end{equation}
% where $\alpha$ and $\gamma$ are the learning rate, $0\leq\alpha\leq1$, and
% discount factor, $0\leq\gamma\leq1$, respectively.  The learning rate
% determines the extent to which new rewards will override the effect of older
% rewards.  The discount factor allows the balance between maximising immediate
% rewards and future rewards to be set.
Q-learning is an off-policy TD method that does not estimate the finite
MDP directly, but iteratively approximates a state-action value
function which returns the value of taking action $a$ in state $s$ and
following an \textit{optimal} policy thereafter. The same theorems used in defining the TD error also apply for
state-action values.
\begin{equation}
\label{eq:qlearning}
Q(s_t,a_t) \leftarrow Q(s_t,a_t) + \alpha \bigl[r_{t+1} + \gamma\max_a
Q(s_{t+1},a)-Q(s_t,a_t) \bigr]
\end{equation}
The method is off-policy since the update function is independent of the policy
being followed and only requires that all state-action pairs be continually
updated.

\subsubsection{Eligibility Traces}
With the TD methods described above, only the value for the immediately
preceding state or state-action pair is updated at each time step.  However,
the prediction $V(s_{t+1})$ also provides information concerning earlier
predictions and TD methods can be extended to update a set of values at each step.  An eligibility trace $e(s)$ represents how eligible the state $s$ is to receive
credit or blame for the TD error:
\begin{equation}
\delta = r_{t+1} + \gamma V^\pi(s_{t+1}) - V^\pi(s_t)
\end{equation}
When extended with eligibility traces TD methods update values for all states
\begin{equation}
%V(s) \leftarrow V(s) + \gamma \delta e(s)
\Delta V_t(s) = \alpha \delta_t e_t(s)
\end{equation}
For the current state $e(s) \leftarrow e(s) + 1$ and for all states the
$e(s) \leftarrow \gamma \lambda e(s)$ where $\lambda$ is the eligibility trace
attenuated factor from which the extended TD methods TD($\lambda$),
Q($\lambda$) and Sarsa($\lambda$) derive their names. For $\lambda = 0$ only
the preceding value is updated is updated, as in the unextended definitions,
and for $\lambda = 1$ all preceding state-values or state-action values are
updated equally.

\subsubsection{Action Selection}
% TODO: Epsilon-greedy.
Action selection may be accomplished using a form of the \textit{softmax}
method \cite{suttonbarto:1998} using the Gibbs, or Boltzmann, distribution to select
action $k$ for the $(t+1)^{th}$ interaction with probability
\begin{equation}
p_{jk}(t+1) = \frac{e^{q_{jk}(t+1)/\tau}}{\sum_{l=0}^K e^{q_{jl}(t+1)/\tau}}
\end{equation}
where $\tau$ is the \textit{temperature} parameter.  This parameter may be
lowered in value over the course of an experiment since high values give all
actions similar probability and encourage exploration of the action space,
while low values promote exploitation of past experience.

% \subsubsection{Q($\lambda$)}
% \label{sec:qlambda}
% In the Q-learning formulation, described in equation \ref{eq:qlearning}, only
% the quality associated with the previous state, $s_{jt}$, is updated.  However,
% the preceding states can also, in general, be said to be associated with the
% reward $r_{jt+1}$.  Eligibility traces are a mechanism for representing this
% effect and in algorithms such as Q($\lambda$), they are what the $\lambda$
% refers to.  The eligibility trace for a state $e(s)$ represents how eligible
% the state $s$ is to receive credit or blame for the error.  The term ``trace''
% refers to fact that only recently visited states become eligible.  The
% eligibility value for the current state is increased while for all other
% states it is attenuated by a factor $\lambda$.
%
% The off-policy nature of Q-learning requires special care to be taken when
% implementing eligibility traces.  While the algorithm may learn a greedy
% policy, in which the action with the maximum value would always be taken,
% typically a policy with some degree of exploration will be followed when
% choosing actions.  If an exploratory (pseudo-random) step is taken the
% preceding states can no longer be considered eligible for credit or blame.
% Setting $\lambda$ to $0$ for non-greedy actions removes much of the benefit of
% using eligibility traces if exploratory actions are frequent.  A solution to
% this has been developed, but requires a very complex implementation
% \cite{peng:1996}.  A na\"ive approach can be taken, where the effect of
% exploratory actions is ignored, but the results of this are unexplored.

\subsection{Policy Gradient Methods}
\label{sec:policygradient}
% The value-function methods defined in Section \ref{sec:valuebased} typically
% rely upon discretisation of the sensor and action spaces so the associated
% values may be stored in tables.  The memory requirements for this restrict the
% application of these methods to small environments.  Many environments,
% particularly from real applications, exhibit continuous sensor and/or action
% spaces and require generalisation techniques to be employed to provide a more
% compact policy representation.
%
% Policy-gradient methods use only the reward signal and search directly in the
% space of the policy parameters.  The agent's policy function approximator is
% updated according to the gradient of expected reward with respect to its
% parameters.
Value function based methods have been successfully applied with discrete lookup
table parameterisation to a great many problems.  However, the number of
discrete states required increases exponentially as the dimensions of the state space
increase and if all possibly relevant situations are to be covered then these
methods become subject to Bellman's Curse of Dimensionality
\cite{bellman:1961}.  Value function based methods can be used in conjunction
with function approximators, artificial neural networks are popular, to work
with continuous state and action space.  However, when used with value
function approximation they have been shown to offer poor convergence and even
divergence characteristics, even in simple systems \cite{peters:enac}.

These convergence problems have motivated research into policy gradient methods
which make small incremental changes to the parameters $\theta$ of a policy
function approximator.  With artificial neural networks the parameters are
the weights of the network connections.  Policy gradient methods update
$\theta$ in the direction of the gradient of some policy performance measure
$Y$ with respect to the parameters
\begin{equation}
\theta_{i+1} = \theta_i + \alpha \frac{\partial Y}{\partial \theta_i}
\end{equation}
where $\alpha$ is a positive definite step size learning rate.

As well as working with continuous state and actions space, policy gradient
methods offer strong convergence guarantees, do not require all states to be
continually updated and although uncertainty in state data can degrade policy
performance, the techniques need not be altered.

Policy gradient methods are differentiated largely by the techniques used to
obtain an estimate of the policy gradient $\partial Y / \partial \theta$.
The most successful real-world robotics results have been yielded using
Williams' \textsc{Reinforce} likelihood ratio methods \cite{williams:reinforce}
and natural policy gradient methods such as Natural Actor-Critic
\cite{peters:enac}.

\subsubsection{Artificial Neural Networks}
This subsection provides a very brief introduction to the theory of artificial
neural networks, concentrating on the aspects utilised in reinforcement
learning.  These mathematical models mimic aspects of biological neural
networks, such as the human brain, and are widely used in supervised learning
applications.  A wealth of literature is available that covers the field
in great depth \cite{bishop96ann,fausett94}.

In reinforcement learning, the most widely used type of artificial neural
network is the multi-layer feedforward network (or multi-layer perceptron).
This model consists of an input layer and an output layer of artificial neurons,
plus any number of optional hidden layers.  Weighted connections link the
neurons, but unlike architectures such as the recurrent neural network, only
neurons from adjacent layers connect.  Most commonly, a fully connected scheme
is used in which all neurons from one layer are connected to all neurons in the
next.  Figure \label{fig:perceptron} diagrams a fully connected three layer
feedforward network.

%
% \newcommand{\neuron}[4]{
% \node[minimum size=20pt,inner sep=1pt,rotate=-90,name=s,shape=circle
% split,draw,#4] (#1) at (#2,#3)
% {\begin{sideways}\tiny$\hspace{0.5mm}f$\end{sideways}\nodepart{lower}\begin{sideways}\tiny$\sum$\end{sideways}};
% }

\def\layersep{3.5cm}
% \begin{figure}
% \centering
% \begin{tikzpicture}[shorten >=1pt,->,draw=black!50, node distance=\layersep]
%     \tikzstyle{every pin edge}=[<-,shorten <=1pt]
%     \tikzstyle{neuron}=[circle,draw=black!25,minimum size=20pt,inner sep=0pt]
%     \tikzstyle{input neuron}=[neuron];
%     \tikzstyle{output neuron}=[neuron];
%     \tikzstyle{hidden neuron}=[neuron];
%     \tikzstyle{annot} = [text width=4em, text centered]
%
%     % Input layer nodes.
%     \foreach \name / \y in {1,...,4}
% %        \node[input neuron, pin=left:Input \y] (I-\name) at (0,-\y) {\tiny$f$};
%         \neuron{I-\name}{0}{-\y}{pin=south:Input \y};
%
%     % Hidden layer nodes.
%     \foreach \name / \y in {1,...,5} {
%         \path[yshift=0.5cm]
%             node[minimum size=20pt,inner sep=1pt,rotate=-90,name=s,shape=circle
%             split,draw] (H-\name) at (\layersep,-\y cm)
%             {\begin{sideways}\tiny$\hspace{0.5mm}g$\end{sideways}\nodepart{lower}\begin{sideways}\tiny$\sum$\end{sideways}};
%
% %             node[hidden neuron] (H-\name) at (\layersep,-\y cm)
% %             {\scriptsize$\sum$};
% %        \draw[-,yshift=0.5cm] (\layersep-1cm,-\y-1) cos (\layersep,-\y) sin
% %        (\layersep+1cm,-\y+1);
%     }
%
%     % Output layer nodes.
% %    \node[output neuron,pin={[pin edge={->}]right:Output}, right of=H-3] (O){};
%     \node[minimum size=20pt,inner sep=1pt,rotate=-90,name=s,shape=circle
%     split,draw,pin={[pin edge={->}]north:Output},above of=H-3] (O)
%     {\begin{sideways}\tiny$\hspace{0.5mm}h$\end{sideways}\nodepart{lower}\begin{sideways}\tiny$\sum$\end{sideways}};
%
%
%     % Input layer - hidden layer connections.
%     \foreach \source in {1,...,4}
%         \foreach \dest in {1,...,5}
%             \path (I-\source) edge (H-\dest);
%
%     % Hidden layer - output layer connections.
%     \foreach \source in {1,...,5}
%         \path (H-\source) edge (O);
%
%     % Annotate the layers
%     \node[annot,above of=H-1, node distance=1cm] (hl) {Hidden Layer};
%     \node[annot,left of=hl] {Input Layer};
%     \node[annot,right of=hl] {Output Layer};
% \end{tikzpicture}
% \caption{Multi-layer feed-forward perceptron with bias nodes.}
% \label{fig:perceptron}
% \end{figure}

\begin{figure}
\centering
\begin{scriptsize}
\begin{tikzpicture}[shorten >=1pt,->,draw=black, node distance=\layersep]
	\tikzstyle{neuron}=[minimum size=30pt,inner sep=1pt,rotate=-90,name=s,
  		shape=circle split,draw,drop shadow,fill=white];
	\tikzstyle{every pin edge}=[<-,shorten <=1pt];
    \tikzstyle{annot} = [text width=4em, text centered];
    \tikzstyle{conn} = [draw=black];

% Input layer bias node.
\node[neuron] (I-0) at (0, 0)
{\begin{sideways}\hspace{1.0mm}1\end{sideways}};

% Input layer neurons.
\node[neuron,pin=south:\small Input 1] (I-1) at (0,-1.4 cm)
{\begin{sideways}\hspace{1.0mm}$f$\end{sideways}};
\node[neuron,pin=south:\small Input 2] (I-2) at (0,-2.8 cm)
{\begin{sideways}\hspace{1.0mm}$f$\end{sideways}};
\node[neuron,pin=south:\small Input 3] (I-3) at (0,-4.2 cm)
{\begin{sideways}\hspace{1.0mm}$f$\end{sideways}};

% Hidden layer bias node.
\node[neuron,xshift=-7mm] (H-0) at (\layersep, 0)
{\begin{sideways}\hspace{1.0mm}1\end{sideways}};

% Hidden layer neurons.
\node[neuron] (H-1) at (\layersep,-0.7 cm)
{\begin{sideways}$\hspace{1.0mm}g$\end{sideways}\nodepart{lower}\begin{sideways}$\sum$\end{sideways}};
\node[neuron] (H-2) at (\layersep,-2.1 cm)
{\begin{sideways}$\hspace{1.0mm}g$\end{sideways}\nodepart{lower}\begin{sideways}$\sum$\end{sideways}};
\node[neuron] (H-3) at (\layersep,-3.5 cm)
{\begin{sideways}$\hspace{1.0mm}g$\end{sideways}\nodepart{lower}\begin{sideways}$\sum$\end{sideways}};
\node[neuron] (H-4) at (\layersep,-4.9 cm)
{\begin{sideways}$\hspace{1.0mm}g$\end{sideways}\nodepart{lower}\begin{sideways}$\sum$\end{sideways}};

% Output layer neurons.
\node[neuron,xshift=-7mm,pin={[pin edge={->}]north:\small Output 1},above
of=H-2] (O-1)
{\begin{sideways}$\hspace{1.0mm}h$\end{sideways}\nodepart{lower}\begin{sideways}$\sum$\end{sideways}};

\node[neuron,xshift=-7mm,pin={[pin edge={->}]north:\small Output 2},above
of=H-3] (O-2)
{\begin{sideways}$\hspace{1.0mm}h$\end{sideways}\nodepart{lower}\begin{sideways}$\sum$\end{sideways}};


% Layer annotations.
\node[annot,above of=H-0, node distance=1.2cm] (hl) {\small Hidden Layer};
\node[annot,left of=hl] {\small Input Layer};
\node[annot,right of=hl] {\small Output Layer};


% Input layer - hidden layer connections.
\foreach \source in {0,...,3}
    \foreach \dest in {0,...,4}
        \path[conn] (I-\source) edge (H-\dest);

% Hidden layer - output layer connections.
\foreach \source in {0,...,4}
    \foreach \dest in {1,...,2}
    	\path[conn] (H-\source) edge (O-\dest);

\end{tikzpicture}
\end{scriptsize}
\caption{Multi-layer feed-forward perceptron with bias nodes.}
\label{fig:perceptron}
\end{figure}

% \begin{tikzpicture}
%   \node[rotate=-90,name=s,shape=circle split,draw] (neuron) at (2,5)
%   {\begin{sideways}$~g$\end{sideways}\nodepart{lower}\begin{sideways}$\sum$\end{sideways}};
% \end{tikzpicture}

% \begin{tikzpicture}
%   \matrix (network)
%     [matrix of nodes,%
%      nodes in empty cells,
%      nodes={outer sep=0pt,circle,minimum size=4pt,draw},
%      column sep={1cm,between origins},
%      row sep={1cm,between origins}]
%   {
%                   &                 &                & \\
%                   &                 &                & \\
%     |[draw=none]| & |[xshift=1mm]| & |[xshift=-1mm]|   \\
%   };
%   \foreach \a in {1,...,4}{
%     \draw (network-3-2) -- (network-2-\a);
%     \draw (network-3-3) -- (network-2-\a);
%     \draw [-stealth] ([yshift=5mm]network-1-\a.north) -- (network-1-\a);
%     \foreach \b in {1,...,4}
%       \draw (network-1-\a) -- (network-2-\b);
%   }
%   \draw [stealth-] ([yshift=-5mm]network-3-2.south) -- (network-3-2);
%   \draw [stealth-] ([yshift=-5mm]network-3-3.south) -- (network-3-3);
% \end{tikzpicture}


\citeA{mcculloch43} conceived of an artificial neuron $j$ that
computes a function $g$ as a weighted sum of all $n$ inputs
\begin{equation}
y_j(x) = g \left(\sum_{i=0}^n w_ix_i\right)
\end{equation}
where $(w_0 \dotsc w_n)$ are weights applied to the inputs $(x_0 \dotsc x_n)$.
In an multi-layer neural network the output $y_j$ forms part of the input
to the neurons in any following layer.  The activation function $g$ is
typically either:
\begin{itemize}
  \item Linear, where the output is simply the weighted sum of inputs,
  \item A threshold function, with an output of either 0 or 1,
  \item Sigmoidal, such that the output is in the range between 0 and 1,
  \item A hyperbolic tangent, where the output ranges between -1 and 1.
\end{itemize}
The paramemters of these functions can be adjusted along with the
connection weights to tune the transfer function between input and output that
the network provides.  To simplify this process a \textit{bias} node that always outputs 1
may be added to a layer and connected to all neurons in the following layer.
This can be shown to allow the activation function parameters to be removed and
for network adjustment to be purely in terms of connection weights.

The output is obtained during the network's \textit{execution} phase by
presenting an input to the input layer that propagates through.  It can be
shown that a suitably configured feedforward network with one hidden layer can
approximate any non-linear function [ref].
% However, this requires the weights
% to be adjusted in the \textit{training} or \textit{learning} phase and it was
% not until the back-propagation algorithm was proposed by \citeA{werbos74}, 30
% years after the inital period of neural network popularity in the 1950s, that
% practical methods became available.  The original back-propagation method is
% the most commonly used, but methods such as RProp \cite{riedmiller93} have been
% introduced to overcome some of its limitations.  In supervised learning, the
% error between the output and the expected output is propagated back though the
% network while the weights of the network are adjusted in reduce the size of the
% error.

% \subsubsection{REINFORCE}
% \label{sec:reinforce}
% REINFORCE is an associative reinforcement learning method that determines
% a policy by modifying the parameters of a policy function approximator, rather
% than approximating a value function \cite{williams:reinforce}.  Commonly,
% feedforward artificial neural networks are used to represent the policy, where
% the input is a representation of the state and the output is action selection
% probabilities.  In learning, a policy gradient approach is taken where the
% weights of the network are adjusted in the direction of the gradient of
% expected reinforcement.
%
% Defining the network, let $\mathbf{x}^i$ denote the vector of inputs to the
% $i$th unit and $y_i$ denote output of the unit.  In the input layer of the
% network the elements $x_j$ of $\mathbf{x}^i$ are normalised sensor values from
% the environment and in the output layer, or in any hidden layers, they are
% outputs from the $j$ unit in the preceding layer.  Let $\mathbf{w}^i$ denote
% the vector of the weights, $w_{ij}$, on the connections to the $i$th unit.  The
% output of the $i$th unit is dependant on the vector of inputs, $\mathbf{x}^i$,
% and the associated weights, $\mathbf{w}^i$.
%
% For each interaction of the agent with the environment, each parameter $w_{ij}$
% is incremented by
% \begin{equation}
% \label{eq:reinforce}
% \Delta w_{ij} = \alpha_{ij}(r - b_{ij})\frac{\partial\ln\rho_i}{\partial
% w_{ij}}
% \end{equation}
% where $\alpha_{ij}$ is the \textit{learning factor}, $b_{ij}$ is the
% \textit{reinforcement baseline} and $\rho_i$ is the performance of the policy
% (e.g., the average reward per interaction).
%
% \subsubsection{ENAC}
% \label{sec:enac}
% ToDo: Episodic Natural Actor Critic\cite{peters:enac}.


\subsection{Roth-Erev Method}
\label{sec:rotherev}
The reinforcement learning method formulated by Alvin~E.~Roth and Ido~Erev is
based on empirical results obtained from observing how humans learn decision
making strategies in games against multiple strategic players
\cite{roth:games,roth:aer}.  It learns a stateless policy in which each action
$a$ is associated with a value $q$ for the propensity of its selection.  In
time period $t$, if agent $j$ performs action $a^\prime$ and receives a reward
$r_{ja^\prime}(t)$ then the propensity value for action $a$ at time $t+1$ is
\begin{equation}
q_{ja}(t+1) =
\begin{cases}
(1-\phi)q_{ia}(t) + r_{ja^\prime}(t)(1-\epsilon), & \text{$a = a^\prime$} \\
(1-\phi)q_{ia}(t) + r_{ja^\prime}(t)(\frac{\epsilon}{A-1}), & \text{$a \ne
a^\prime$}
\end{cases}
\end{equation}
where $A$ is the total number of feasible actions, $\phi$ is the
\textit{recency} parameter and $\epsilon$ is the \textit{experimentation} parameter.  The recency (forgetting) parameter
degrades the propensities for all actions and prevents propensity values from
going unbounded.  It is intended to represent the tendency for players to forget
older action choices and to prioritise more recent experience.  The
experimentation parameter prevents the probability of choosing an action from
going to zero and encourages exploration of the action space.

Erev and Roth proposed action selection according to a discrete probability
distribution function, where action $k$ is selected for interaction $t+1$ with
probability
\begin{equation}
\label{eq:re_prob}
p_{jk}(t+1) = \frac{q_{jk}(t+1)}{\sum_{l=0}^K q_{jl}(t+1)}
\end{equation}
Since $\sum_{l=0}^K q_{jl}(t+1)$ increases with $t$, a reward $r_{jk}(t)$ for
performing action $k$ will have a greater effect on the probability
$p_{jk}(t+1)$ during early interactions while $t$ is small.  This is intended
to represent Psychology's Power Law of Practice in which it is qualitatively
stated that, with practice, learning occurs at a decaying exponential rate and
that a learning curve will eventually flatten out.

\subsubsection{Modified Roth-Erev Method}
\label{sec:variant}
Two shortcomings of the basic Roth-Erev algorithm have been identified and a
modified formulation proposed \cite{nicolaisen:2001}.  The two issues are that
\begin{itemize}
  \item the values by which propensities are updated can be zero or very small
  for certain combinations of the experimentation parameter $\epsilon$ and
  the total number of feasible actions $A$ and
  \item all propensity values are decreased by the same amount when the reward,
  $r_{jk^\prime}(t)$ is zero.
\end{itemize}
Under the variant algorithm, the propensity for agent $j$ to select action $a$
for interaction $t+1$ is:
\begin{equation}
q_{ja}(t+1) =
\begin{cases}
(1-\phi)q_{ia}(t) + r_{ja^\prime}(t)(1-\epsilon), & \text{$a = a^\prime$} \\
(1-\phi)q_{ia}(t) + q_{ja}(t)(\frac{\epsilon}{A-1}), & \text{$a \ne
a^\prime$}
\end{cases}
\end{equation}
As with the original Roth-Erev algorithm, the propensity for selection of the
action that the reward is associated with is adjusted by the experimentation
parameter.  All other action propensities are adjusted by a small proportion of
their current value.

\section{Summary}
The combination of electricity markets and electric power systems presents a
complex dynamic environment to participants.  Network power flows are
non-linear functions of the bus voltages and thus one party's generation or
consumption decisions affect all other parties.  Substantial modifications to
the design of the UK's electricity trading arrangements have been required
since the their introduction two decades ago.  Major changes to the UK power
systems will be required if its ambitious greenhouse gas emission reduction
commitments are to be met and further modifications will be likely.

The main electricity trading mechanisms can be modelled using well established
mathematical optimisation formulations.  Robust techniques exist for computing
solutions to these problems and they provide price data that reflects network
topology and conditions.

The combination of non-linear optimisation problems and participant behavioural
models is beyond the capabilities of conventional equilibrium approaches to
market simulation when analysing large systems.  An alternative is to take a
``bottom-up'' approach to modelling them and examine the system dynamics that
result from interactions between goal driven individuals.

Reinforcement learning is an unsupervised machine learning technique that can
be used to model the dynamic behaviour of these individuals.  Traditional
methods associated a \textit{value} with each state and the available actions,
but are limited to realtively small discrete problems.  Methods that search
directly in the space of the parameters of an action selection policy can
operate in continuous state and actions space, have good convergence properties
and have potential for application in electricity market simulation.
